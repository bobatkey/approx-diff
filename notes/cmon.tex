\subsection{Commutative monoids}
\label{sec:cmon}

\begin{definition}[Commutative monoid]
A \emph{commutative monoid} $X = (X, \bullet, \varepsilon)$ is a set $X$ equipped with distinguished element
$\varepsilon \in X$ called the \emph{unit} and associative binary operation $\bullet: X^2 \to X$ satisfying
$\varepsilon \bullet x = x$ and $x \bullet \varepsilon = x$ for any $x \in X$.
\end{definition}

A commutative monoid homomorphism from $X$ to $Y$ is any function $f: X \to Y$ preserving $\varepsilon$ and
$\bullet$.

\subsubsection{Category of commutative monoids}

\begin{definition}[Category $\CMon$]
\label{def:cmon:cmon}
Define $\CMon$ to be the category which has as objects $X$ all commutative monoids and as morphisms $f: X \to
Y$ all commutative monoid homomorphisms.
\end{definition}

$\CMon$ is complete and cocomplete, inheriting all limits and colimits from $\Set$. $\CMon$ is also monoidal
closed. \todo{But the monoidal product is not the Cartesian product?}

\subsubsection{Example of commutative monoid}

\begin{definition}[Bounded semilattice]
\label{def:cmon-enriched:bounded-semilattice}
A \emph{bounded semilattice} $X = (X, \bullet, \varepsilon)$ is a commutative monoid where $\bullet$ is
\emph{idempotent}, i.e.~satisfies $x \bullet x = x$.
\end{definition}

\noindent The idempotence of $\bullet$ (together with commutativity and associativity) induces a partial order
$\le_\bullet$ on $X$, with $x \le_{\bullet} y \iff x \bullet y = x$. With respect to this partial order, $x
\bullet y$ is the greatest lower bound (meet) of $x$ and $y$ and $\varepsilon$ is the top element; with
respect to the opposite order, $x \bullet y$ is the least upper bound (join) of $x$ and $y$ and $\varepsilon$
is the bottom element. This therefore provides an algebraic characterisation of the usual (dual)
order-theoretic notions of bounded meet semilattices $(X, \meet, \top)$ and bounded join semilattices $(X,
\join, \bot)$.

\begin{definition}[Category $\SemiLat$]
Define $\SemiLat$ to be the category which has as objects all bounded semilattices and as morphisms all
bounded semilattice homomorphisms.
\end{definition}
