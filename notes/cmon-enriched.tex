\subsection{$\CMon$-enriched categories}
\label{sec:cmon-enriched}

Recall that if a category $\cat{C}$ is $\CMon$-enriched, then:
\begin{enumerate}
\item Every hom-object $\Hom{\cat{C}}{X}{Y}$ is a commutative monoid of morphisms; we write $\zero_{X,Y}$
(zero morphism) for the unit, and $+_{X,Y}$ (addition of morphisms) for the binary operation (omitting the
indices where implied by the context).
\item Composition is \emph{bilinear}, i.e.~given by a family of morphisms $\Hom{\cat{C}}{Y}{Z} \tensor
\Hom{\cat{C}}{X}{Y} \to \Hom{\cat{C}}{X}{Z}$ in $\CMon$ that preserve the additive structure in
$\Hom{\cat{C}}{Y}{Z}$ and $\Hom{\cat{C}}{X}{Y}$ separately:

\begin{salign*}
f \comp \zero = f = \zero \comp f
\end{salign*}
\begin{salign*}
(f + g) \comp h &= (f \comp h) + (g \comp h) \\
h \comp (f + g) &= (h \comp f) + (h \comp g)
\end{salign*}
\end{enumerate}

Because $\CMon$ is monoidal closed, it is enriched over itself; every hom-object $\Hom{\CMon}{X}{Y}$ is a
commutative monoid with:

\begin{enumerate}
\item unit $0_{X,Y} = \const(\varepsilon_Y)$, the constant homomorphism sending every element of $X$ to
$\varepsilon_Y$;
\item binary operation $+_{X,Y}$ given by pointwise addition of homomorphisms $(f + g)(x) = f(x) + g(x)$.
\end{enumerate}

Suppose $F, G: \cat{D} \to \cat{C}$ are functors where $\cat{C}$ is $\CMon$-enriched. Then the hom-object
$\Hom{\Func{\cat{C}}{\cat{D}}}{F}{G}$ of natural transformations between $F$ and $G$ has a zero $0_{F,G}$ and
addition $\eta + \mu$ given component-wise as $(0_{F,G})_X = 0_{F(X),G(X)}$ and $(\eta + \mu)_X = \eta_X +
\mu_X$.

\begin{proposition}
If $C$ is $\CMon$-enriched then any functor category $\Func{\cat{D}}{\cat{C}}$ is $\CMon$-enriched.
\end{proposition}

\subsubsection{$\CMon$-enriched presheaves}

In the specific setting of a $\CMon$-enriched presheaf $F: \cat{C}^\op \to \CMon$, enrichment means that $F$
preserves the additive structure of morphisms:
\begin{itemize}
\item $F(0_{X,Y}) = 0_{F(Y),F(X)}$
\item $F(f + g) = F(f) + F(g)$
\end{itemize}
It then follows automatically that $F$ also preserves bilinear composition, i.e.:
\begin{itemize}
\item $F(f) \comp F(0) = F(f) = F(0) \comp F(f)$
\item $F(f + g) \comp F(h) = F(h \comp f) + F(h \comp g)$
\item $F(h) \comp F(f + g) = F(f \comp h) + F(g \comp h)$
\end{itemize}
in addition to the usual enriched functor properties from \defref{cmon-enriched:enriched-functor}.

\subsubsection{Enriched Yoneda embedding}

Suppose a monoidal category $\mathscr{V}$ and a $\mathscr{V}$-enriched category $\cat{C}$, again restricted to
the case where $\mathscr{V}$ is concrete.

\begin{definition}[Hom-functor]
Define the contravariant \emph{hom-functor} $\Hom{\cat{C}}{-}{X}: \cat{C}^\op \to \mathscr{V}$ sending
\begin{itemize}
\item any object $Y$ in $\cat{C}$ to the hom-object $\Hom{\cat{C}}{Y}{X}$ in $\mathscr{V}$;
\item any morphism $f: Y \to Z$ in $\cat{C}$ to the function $\Hom{\cat{C}}{f}{X} = (- \comp f):
\Hom{\cat{C}}{Z}{X} \to \Hom{\cat{C}}{Y}{X}$ on the underlying hom-sets.
\end{itemize}
\end{definition}

\begin{definition}[Yoneda embedding]
The \emph{Yoneda embedding} for $\cat{C}$ is the functor $\Yoneda: \cat{C} \to
\Func{\cat{C}^\op}{\mathscr{V}}$ sending
\begin{itemize}
\item any object $X$ to the hom-functor $\Hom{\cat{C}}{-}{X}: \cat{C}^\op \to \mathscr{V}$;
\item any morphism $f: X \to Y$ to the $\mathscr{V}$-natural transformation $\Hom{\cat{C}}{-}{X} \naturalto
\Hom{\cat{C}}{-}{Y}$ where $\Yoneda(f)_Z = (f \comp -): \Hom{\cat{C}}{Z}{X} \to \Hom{\cat{C}}{Z}{Y}$.
\end{itemize}
\end{definition}

\noindent Now suppose $\mathscr{V}$ is monoidal closed. Then the hom-functor $\Hom{\cat{C}}{-}{X}: \cat{C}^\op
\to \mathscr{V}$ is itself $\mathscr{V}$-enriched, providing a family of morphisms $\Hom{\cat{C}}{-}{X}_{Y,Z}:
\Hom{\cat{C}}{Y}{Z} \to \Hom{\mathscr{V}}{\Hom{\cat{C}}{Z}{X}}{\Hom{\cat{C}}{Y}{X}}$ in $\mathscr{V}$ for any
$Y, Z$ in $\cat{C}$. Moreover there exists an enriched Yoneda embedding, namely a functor
$\Yoneda_{\mathscr{V}}: \cat{C} \to \PSh_{\mathscr{V}}(\cat{C})$ which is itself $\mathscr{V}$-enriched,
sending objects $X$ to the $\mathscr{V}$-enriched hom-functor $\Hom{\cat{C}}{-}{X}$ and providing a family of
morphisms $(\Yoneda_{\mathscr{V}})_{X,Y}: \Hom{\cat{C}}{X}{Y} \to
\Hom{\PSh_{\mathscr{V}}(\cat{C})}{\Hom{\cat{C}}{-}{X}}{\Hom{\cat{C}}{-}{Y}}$ acting concretely at each
component as post-composition with $f: X \to Y$.

The $\CMon$-enriched Yoneda embedding preserves biproducts:

\begin{proposition}
\end{proposition}

The \emph{enriched Yoneda lemma} states that, for any $\mathscr{V}$-enriched presheaf $F$ and any object $X
\in \cat{C}$, there is an isomorphism between $F(X)$ and the hom-object (also living in $\mathscr{V}$) of
$\mathscr{V}$-natural transformations from the $\mathscr{V}$-enriched presheaf $\Hom{\cat{C}}{-}{X}$ to $F$:

\begin{lemma}
Suppose $\cat{C}$ is $\mathscr{V}$-enriched with $\mathscr{V}$ monoidal closed. For any $F: \cat{C}^\op \to
\mathscr{V}$ and any object $X$ in $\cat{C}$:
\[\Hom{\PSh_{\mathscr{V}}{(\cat{C})}}{\Hom{\cat{C}}{-}{X}}{F} \iso F(X)\]
\end{lemma}


% The following diagram commutes because precomposition with the identity is the identity on morphisms:
%
% \begin{center}
% \begin{tikzcd}[column sep=1.8cm]
% I \arrow[d, equals] \arrow[r, "\id_X"] & \Hom{\cat{C}}{X}{X} \arrow[d, "\Hom{\cat{C}}{-}{W}_{X,X}"] \\
% I \arrow[r, "\id_{\Hom{\cat{C}}{X}{W}}"'] & \Hom{\mathscr{V}}{\Hom{\cat{C}}{X}{W}}{\Hom{\cat{C}}{X}{W}}\
% \end{tikzcd}
% \end{center}
%
% \noindent and the following commutes by the associativity of composition:
%
% \begin{center}
% \begin{tikzcd}[column sep=2.7cm]
%    \Hom{\cat{C}}{Z}{Y} \tensor \Hom{\cat{C}}{Y}{X} \arrow[r, "\comp_{X,Y,Z}"] \arrow[d,
%    "\Hom{\cat{C}}{-}{W}_{Y,Z}\,\tensor\,\Hom{\cat{C}}{-}{W}_{X,Y}"'] & \Hom{\cat{C}}{Z}{X} \arrow[d,
%    "\Hom{\cat{C}}{-}{W}_{X,Z}"] \\
%    \Hom{\mathscr{V}}{\Hom{\cat{C}}{Y}{W}}{\Hom{\cat{C}}{Z}{W}} \tensor
%    \Hom{\mathscr{V}}{\Hom{\cat{C}}{X}{W}}{\Hom{\cat{C}}{Y}{W}} \arrow[r,
%    "\comp_{\Hom{\cat{C}}{X}{W},\Hom{\cat{C}}{Y}{W},\Hom{\cat{C}}{Z}{W}}"'] &
%    \Hom{\mathscr{V}}{\Hom{\cat{C}}{X}{W}}{\Hom{\cat{C}}{Z}{W}}
% \end{tikzcd}
% \end{center}
