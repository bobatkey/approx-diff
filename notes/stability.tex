\subsection{Stable Functions}
\label{sec:stable-functions}

\begin{definition}
  Let $f : X \to Y$ be a monotone function between posets $X$ and
  $Y$. The function $f$ is \emph{stable} if for all $x \in X$ and
  $y \leq f(x)$, there exists an $x_0 \leq x$ such that $y \leq f(x_0)$ which
  is minimal: for any other $x'_0 \leq x$ such that $y \leq f(x'_0)$ then
  $x_0 \leq x'_0$.
\end{definition}

The intuition behind stability is that for stable functions, we have a
way of determining what minimal approximation of the input is required
for any approximation of the output. The ordering on $X$ and $Y$
describes an ordering of information content between
elements. Monotonicity means that increases in information content on
the input are not lost on the output. Stability means that, if we use
$f$ to derive a result $f(x)$ from $x$, and then we identify some
approximation $y$ of $f(x)$, then there is a unique minimal
approximation of $x$ that gets us as close as possible to $y$.

We can rephrase stability in terms of Galois connections / adjunctions
between principal downsets.

\newcommand{\downset}[1]{\mathop{\downarrow}(#1)}

\begin{definition}
  For a poset $X$ and element $x \in X$, the \emph{principal
    downset of $x$} is $\downset{x} = \{ x' \in X \mid x' \leq x \}$.
\end{definition}

\begin{lemma}
  If $f : X \to Y$ is monotonic, then for all $x \in X$, the
  restriction of $f$ to $\downset{x}$ is a monotone function
  $f_x : \downset{x} \to \downset{f(x)}$.
\end{lemma}

\begin{proof}
  $f_x$ is well defined: for any $x' \leq x$,
  $f_x(x') = f(x') \leq f(x)$ by monotonicity, so
  $f_x(x') \in \downset{f(x)}$. The function $f_x$ is monotone because
  $f$ is.
\end{proof}

\begin{lemma}
  A monotone function $f : X \to Y$ is stable if and only if for all
  $x \in X$, the restriction of $f_x : \downset{x} \to \downset{f(x)}$
  has a left adjoint.
\end{lemma}

\begin{proof}
  If $f$ is stable, then define a left adjoint
  $g_x : \downset{f(x)} \to \downset{x}$ by setting $g_x(y)$ to be the
  minimal $x_0$ required by stability. (FIXME: this is monotone?). Let
  $x' \leq x$ and $y \leq f(x)$. Then if $g_x(y) \leq x'$, we have
  $y \leq f(g_x(y)) \leq f(x')$ by monotonicity of $f$ and the first
  part of stability. In the other direction, if we have
  $y \leq f(x')$, then by uniqueness we have $g_x(y) \leq x'$.

  TODO: left adjoint implies stability.
\end{proof}



\begin{enumerate}
\item Posets where the principal downsets are meet-semilattices. And
  posets where the principal downsets are lattices.
\item Stable functions on these.
\item Full and faithful functor into Fam(LatGal), copying the idea
  from automatic differentiation: we formally delineate points and
  their tangent bundles.
\item
  \begin{enumerate}
  \item Posets that are meet semilattices in their downsets, with
    monotone functions form a Tangent Category
  \item To get ``reverse derivatives'', we further assume stability.
  \end{enumerate}
\end{enumerate}
