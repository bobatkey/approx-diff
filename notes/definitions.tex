\section{Definitions}

\subsection{Indexed families of objects}

For any set $I$ and any category $C$ write $\Fam(I,C)$ for the category where:
\begin{itemize}
\item objects are the $I$-indexed families of objects of $C$;
\item morphisms from $X$ to $Y$ are families of morphisms $f_i: X_i \to Y_i$ in $C$ for any $i \in I$.
\end{itemize}

\noindent Equivalently $\Fam(I,C)$ is the functor category $\Func{I}{C}$ where we interpret $I$ as a discrete
category. $\Fam(-,C): \Set^{\op} \to \Cat$ is then the functor which sends any set $I$ to $\Fam(I,C)$ and any
function $f: I \to J$ to the following \emph{reindexing functor}.

\begin{definition}[Reindexing]
For any $f: I \to J$ the \emph{reindexing} functor $\reindex{-}{f}: \Fam(J,C) \to \Fam(I,C)$ sends any
$J$-indexed family $X$ to the $I$-indexed family $X[f]$ where $X[f]_i = X_{f(i)}$ for any $i \in I$ and
similarly for morphisms.
\end{definition}

\subsection{Category of families}

The \emph{Grothendieck construction} $\Grothendieck{D}F$ for a functor $F: D \to \Cat$ is the category where:
\begin{itemize}
\item objects are pairs $(I, X)$ of an object $I$ of $D$ and an object $X$ of $FI$;
\item morphisms from $(I, X)$ to $(J, Y)$ are morphisms $f: I \to J$ in $D$ paired with a morphism $(Ff)X
\to Y$ in $FJ$.
\end{itemize}

\noindent When $F$ is contravariant, a morphism from $(I, X)$ to $(J, Y)$ is a morphism $f: I \to J$ in $D$
paired with a morphism $X \to (Ff)Y$ in $FI$. Writing $\Fam(C)$ for the Grothendieck construction
$\Grothendieck{\Set^{\op}}\Fam(-,C)$ we then have that in $\Fam(C)$:
\begin{itemize}
\item objects are pairs $(I, X)$ of a set $I$ and an indexed family $X$ in $\Fam(I,C)$;
\item morphisms from $(I, X)$ to $(J, Y)$ are functions $f: I \to J$ paired with a morphism $X \to Y[f]$ in
$\Fam(I,C)$.
\end{itemize}

\todo{Some properties of $\Fam(C)$.}

\subsection{Category of bounded lattices and Galois connections}

Define the category $\LatGal$ where:
\begin{itemize}
\item objects $L$ are the bounded lattices;
\item morphisms $f \vdash g: L \to L'$ are meet-semilattice homomorphisms $f: L \to L'$ paired with
join-semilattice homomorphisms $g: L' \to L$ satisfying $y \leq f(x) \iff g(y) \leq x$ for any $x \in L$ and any
$y \in L'$.
\end{itemize}

\noindent $\LatGal$ has finite biproducts. \todo{Tie this to pre-additive structure.}

\subsection{Category of commutative monoids}

$\CMon$ is the category whose objects are the commutative monoids $M$ and morphisms $M \to M'$ are the
commutative monoid homomorphisms. $\CMon$ is symmetric closed monoidal, with the trivial one-element monoid
$\One$ as terminal object and tensor product $M \tensor M'$ given by the direct product of $M$ and $M'$. In
fact $\CMon$ is complete and cocomplete, inheriting all limits and colimits from $\Set$.

\subsection{$\CMon$-enriched category}

A category is $\CMon$-enriched if every $\Hom{A}{B}$ is a commutative monoid with the \emph{zero} morphism
$\zero_{A,B}$ as the unit and addition of morphisms $+_{A,B}$ as the commutative monoid operation.

\subsection{Category $\Func{\LatGal^\op}{\CMon}$}

A presheaf $F: \LatGal^\op \to \CMon$ assigns to every
bounded lattice $L$ a commutative monoid $FL$ and to every Galois connection $f: L \to L'$ .
$\Func{\LatGal^\op}{\CMon}$ is the category of presheaves of commutative monoids on $\LatGal$.

\subsection{Biproducts}

Suppose $C$ a $\CMon$-enriched category. For any objects $A, B$, the \emph{biproduct} of $A$ and $B$ is an
object $A \oplus B$ of $C$ together with morphisms

\begin{center}
\begin{tikzcd}
   A \arrow[r, "i_A", shift left] &
   A \oplus B \arrow[l, "p_A", shift left] \arrow[r, "p_B"', shift right] &
   B \arrow[l, "i_B"', shift right]
\end{tikzcd}
\end{center}

\noindent satisfying

\begin{minipage}[t]{0.45\textwidth}
\begin{center}
\begin{salign*}
   p_A \after i_A &= \id_A \\
   p_B \after i_A &= \zero_{A,B}
\end{salign*}
\end{center}
\end{minipage}%
\begin{minipage}[t]{0.45\textwidth}
\begin{center}
\begin{salign*}
   p_B \after i_B &= \id_B \\
   p_A \after i_B &= \zero_{B,A}
\end{salign*}
\end{center}
\end{minipage}

\noindent and

\begin{salign*}
i_A \after p_A + i_B \after p_B &= \id_{A \oplus B}
\end{salign*}
