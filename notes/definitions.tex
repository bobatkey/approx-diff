\section{Definitions}

\subsection{Category of families}

\begin{definition}[$\Fam(I,C)$]
For any set $I$ and any category $C$ write $\Fam(I,C)$ for the category where:
\begin{itemize}
\item objects are the $I$-indexed families of objects of $C$;
\item morphisms from $X$ to $Y$ are families of morphisms $f_i: X_i \to Y_i$ in $C$ for any $i \in I$.
\end{itemize}
\end{definition}

\noindent Equivalently $\Fam(I,C)$ is the functor category $\Func{I}{C}$ where we interpret $I$ as a discrete
category.

\begin{definition}[Reindexing]
For any $f: I \to J$ the \emph{reindexing} functor $\reindex{-}{f}: \Fam(J,C) \to \Fam(I,C)$ sends any
$J$-indexed family $X$ to the $I$-indexed family $X[f]$ where $X[f]_i = X_{f(i)}$ for any $i \in I$ and
similarly for morphisms.
\end{definition}

\begin{definition}[$\Fam(-,C)$ functor]
$\Fam(-,C): \Set^{\op} \to \Cat$ is then the functor which sends any set $I$ to $\Fam(I,C)$ and any function
$f: I \to J$ to the functor $\reindex{-}{f}: \Fam(J,C) \to \Fam(I,C)$.
\end{definition}

\begin{definition}[Grothendieck construction for $F$]
Suppose a functor $F: D \to \Cat$. The \emph{Grothendieck construction} $\Grothendieck{D}F$ for $F$ is the
category where:
\begin{itemize}
\item objects are pairs $(I, X)$ of an object $I$ of $D$ and an object $X$ of $F(I)$;
\item morphisms from $(I, X)$ to $(J, Y)$ are morphisms $f: I \to J$ in $D$ paired with a morphism $F(f)(X)
\to Y$ in $F(J)$.
\end{itemize}
\end{definition}

\noindent When $F$ is contravariant, a morphism from $(I, X)$ to $(J, Y)$ is a morphism $f: I \to J$ in $D$
paired with a morphism $X \to F(f)(Y)$ in $F(I)$.

\begin{definition}[Category of families]
For any category $C$ define $\Fam(C)$ to be the Grothendieck construction
$\Grothendieck{\Set^{\op}}\Fam(-,C)$.
\end{definition}

In $\Fam(C)$ we then have that:
\begin{itemize}
\item objects are pairs $(I, X)$ of a set $I$ and an indexed family $X$ in $\Fam(I,C)$;
\item morphisms from $(I, X)$ to $(J, Y)$ are functions $f: I \to J$ paired with a morphism $X \to Y[f]$ in
$\Fam(I,C)$.
\end{itemize}

\subsection{$\CMon$-enriched categories}

\begin{definition}[Commutative monoid]
A \emph{commutative monoid} $A = (A, \varepsilon, \bullet)$ is a set $A$ equipped with distinguished element
$\varepsilon \in A$ called the \emph{unit} and associative binary operation $\bullet$ satisfying $\varepsilon
\bullet x = x$ and $x \bullet \varepsilon = x$ for any $x \in A$.
\end{definition}

A commutative monoid homomorphism from $A$ to $B$ is any function $f: A \to B$ preserving $\varepsilon$ and
$\bullet$.

\begin{definition}[Category of commutative monoids]
Define $\CMon$ to be the category whose objects are the commutative monoids $A$ and morphisms $A \to B$ are
the commutative monoid homomorphisms.
\end{definition}

$\CMon$ is symmetric closed monoidal, with the trivial one-element monoid $\One$ as terminal object, monoidal
product $A \tensor B$ given by the direct product of $A$ and $B$, and braiding given by the isomorphism $A
\times B \iso B \times A$. $\CMon$ is complete and cocomplete, inheriting all limits and colimits from $\Set$.

\subsection{$\CMon$-enriched category}

We note that if a category $C$ is $\CMon$-enriched, then:
\begin{enumerate}
\item Every hom-object $\Hom{C}{A}{B}$ is a commutative monoid of morphisms; we write $\zero_{A,B}$ (zero
morphism) for the unit, and $+_{A,B}$ (addition of morphisms) for the binary operation.
\item Composition is \emph{biadditive}, i.e.~given by a family of morphisms $\Hom{C}{B}{C} \tensor
\Hom{C}{A}{B} \to \Hom{C}{A}{C}$ in $\CMon$ that preserve the additive structure in $\Hom{C}{B}{C}$ and
$\Hom{C}{A}{B}$ separately:

\begin{salign*}
f \comp \zero = f = \zero \comp f
\end{salign*}
\begin{salign*}
(f + g) \comp h &= (f \comp h) + (g \comp h) \\
h \comp (f + g) &= (h \comp f) + (h \comp g)
\end{salign*}
\end{enumerate}

$\CMon$ is closed monoidal, and so enriched over itself. Thus every hom-object $\CMon(A,B)$ is a commutative
monoid of commutative monoid homomorphisms where:
\begin{enumerate}
\item the unit $0$ is the constant homomorphism sending every element of $A$ to $\varepsilon_B$;
\item addition of morphisms $+$ is defined pointwise so that $(f + g)(a) = f(a) + g(a)$.
\end{enumerate}

\begin{proposition}
If $C$ is $\CMon$-enriched then for any category $D$ the functor category $\Func{D}{C}$ is $\CMon$-enriched.
\end{proposition}

\subsection{Biproducts}

\begin{definition}[Zero object]
A \emph{zero} object is an object which is both terminal and initial.
\end{definition}

Suppose $C$ a $\CMon$-enriched category.

\begin{definition}[Biproduct]
For any objects $A, B$, the \emph{biproduct} of $A$ and $B$ is an object $A \oplus B$ of $C$ together with
morphisms

\begin{center}
\begin{tikzcd}
   A \arrow[r, "i_A", shift left] &
   A \oplus B \arrow[l, "p_A", shift left] \arrow[r, "p_B"', shift right] &
   B \arrow[l, "i_B"', shift right]
\end{tikzcd}
\end{center}

\noindent satisfying

\begin{minipage}[t]{0.45\textwidth}
\begin{center}
\begin{salign*}
   p_A \comp i_A &= \id_A \\
   p_B \comp i_A &= \zero_{A,B}
\end{salign*}
\end{center}
\end{minipage}%
\begin{minipage}[t]{0.45\textwidth}
\begin{center}
\begin{salign*}
   p_B \comp i_B &= \id_B \\
   p_A \comp i_B &= \zero_{B,A}
\end{salign*}
\end{center}
\end{minipage}

\noindent and

\begin{salign*}
i_A \comp p_A + i_B \comp p_B &= \id_{A \oplus B}
\end{salign*}
\end{definition}

If $C$ has biproducts then it has products given by $(A \oplus B, p_A, p_B)$ and coproducts given by $(A
\oplus B, i_A, i_B)$.

\begin{proposition}
\item
\begin{itemize}
\item If $C$ has products (or coproducts) then they are biproducts.
\item If $C$ has a terminal (or initial) object then it is a zero object.
\end{itemize}
\end{proposition}

\noindent We say that $C$ is \emph{semi-additive} if it has finite biproducts.

\subsection{Category of bounded lattices and Galois connections}

Define the category $\LatGal$ where:
\begin{itemize}
\item objects $L$ are the bounded lattices;
\item morphisms $f \vdash g: L \to L'$ are meet-semilattice homomorphisms $f: L \to L'$ paired with
join-semilattice homomorphisms $g: L' \to L$ satisfying $y \leq f(x) \iff g(y) \leq x$ for any $x \in L$ and any
$y \in L'$.
\end{itemize}

\noindent $\LatGal$ has finite biproducts.

\subsection{Category $\Func{\LatGal^\op}{\CMon}$}

A presheaf $F: \LatGal^\op \to \CMon$ assigns to every bounded lattice $L$ a commutative monoid $F(L)$ and to
every Galois connection $f: L \to L'$ a commutative monoid homomorphism $F(f): F(L) \to F(L')$.
$\Func{\LatGal^\op}{\CMon}$ is the category of presheaves of commutative monoids on $\LatGal$.
