\subsection{Predicate systems}
\label{sec:predicate-system}

A \emph{predicate system} over a category $\cat{C}$ consists of the following data:
\begin{itemize}
\item For every object $X$ of $\cat{C}$, a partial order $(\Pred(X), \sqsubseteq)$ of predicates on $X$.
\item For every morphism $f: X \to Y$ of $\cat{C}$, a Galois connection $\reindex{-}{f} \dashv
\directImage{-}{f} : \Pred(X) \to \Pred(Y)$. The left adjoint $\reindex{-}{f}: \Pred(Y) \to \Pred(X)$ computes
the pullback (reindexing) of a predicate along $f$; the right adjoint $\directImage{-}{f}: \Pred(X) \to
\Pred(Y)$ computes the pushforward (existential image) of a predicate along $f$, with the adjointness property
meaning that
\[\directImage{P}{f} \sqsubseteq Q \iff P \sqsubseteq \reindex{Q}{f} \]
\item For every fiber $\Pred(X)$ the following structure:
   \begin{itemize}
   \item $\Pred(X)$ is a Heyting algebra with universal quantification, i.e.~closed under top element $\top$,
   meets $\meet$, joins $\join$, residual $\Rightarrow$, and unary connective $\forall$, satisfying the usual
   laws;
   \item each operation is preserved by reindexing.
   \end{itemize}
\end{itemize}
% These operations satisfy standard laws such as monotonicity, closure under reindexing, etc.
