\subsection{Predicate systems}
\label{sec:predicate-system}

A \emph{predicate system} over a category $\cat{C}$ consists of the following data:
\begin{enumerate}
\item for every object $X$ of $\cat{C}$, a partial order $(\Pred(X), \sqsubseteq)$ of predicates on $X$
forming a Heyting algebra with universal quantification (i.e.~closed under the null connective $\top$ and
binary connectives $\meet$, $\join$ and $\implies$ and unary connective $\forall$);
\item for every morphism $f: X \to Y$ of $\cat{C}$, a Galois connection $\reindex{-}{f} \dashv \directImage{-}{f} : \Pred(X) \to \Pred(Y)$.
\end{enumerate}
\noindent The left adjoint $\reindex{-}{f}: \Pred(Y) \to \Pred(X)$ computes the reindexing or pullback of a
predicate along $f$; the right adjoint $\directImage{-}{f}: \Pred(X) \to \Pred(Y)$ computes the pushforward or
existential image of a predicate along $f$, with the adjointness property meaning that
\[\directImage{P}{f} \sqsubseteq Q \iff P \sqsubseteq \reindex{Q}{f} \]

% These operations satisfy standard laws such as monotonicity, closure under reindexing, etc.
