\subsection{Higher-order language}

We introduce a standard total functional language with a (reasonably) expressive type
system~\cite{crole94,pitts01,santocanale02}, over a set $\PrimTy$ of primitive types $\rho$ and sets
$\PrimOp^\rho_{\rho_1,\ldots,\rho_n}$ of primitive operations.

\subsubsection{Syntax}

\begin{figure}
  \begin{mathpar}
  \small
  \inferrule*
  {
    \strut
  }
  {
    \Pol(+,\alpha,\alpha)
  }
  \and
  \inferrule*
  {
    \alpha \neq \beta
  }
  {
    \Pol(p,\alpha,\beta)
  }
  \and
  \inferrule*
  {
    \strut
  }
  {
    \Pol(p,\alpha,\tyZero)
  }
  \and
  \inferrule*
  {
    \Pol(p,\alpha,\sigma)
    \\
    \Pol(p,\alpha,\tau)
  }
  {
    \Pol(p,\alpha,\sigma \tySum \tau)
  }
  \and
  \inferrule*
  {
    \strut
  }
  {
    \Pol(p,\alpha,\tyUnit)
  }
  \and
  \inferrule*
  {
    \Pol(p,\alpha,\sigma)
    \\
    \Pol(p,\alpha,\tau)
  }
  {
    \Pol(p,\alpha,\sigma \tyProd \tau)
  }
  \and
  \inferrule*
  {
    \Pol(\neg p,\alpha,\sigma)
    \\
    \Pol(p,\alpha,\tau)
  }
  {
    \Pol(p,\alpha,\sigma \tyFun \tau)
  }
  \and
  \inferrule*
  {
    \strut
  }
  {
    \Pol(p,\alpha,\mu\alpha.\tau)
  }
  \and
  \inferrule*
  {
    \alpha \neq \beta
    \\
    \Pol(p,\alpha, \tau)
  }
  {
    \Pol(p,\alpha,\mu\beta.\tau)
  }
  \end{mathpar}
\caption{Polarity checking}
\end{figure}

\begin{figure}
  \begin{subfigure}[t]{0.48\linewidth}
  \small
  \[
  \begin{array}{lllll}
    & \textit{Types}
    \\
    &
    \sigma, \tau
    & ::= &
    \rho
    &
    \text{primitive type}
    \\
    && \mid &
    \sigma \tySum \tau
    &
    \text{sum}
    \\
    && \mid &
    \tyUnit
    &
    \text{unit}
    \\
    && \mid &
    \sigma \tyProd \tau
    &
    \text{product}
    \\
    && \mid &
    \sigma \tyFun \tau
    &
    \text{function}
    \\
    && \mid &
    \tyList\;\tau
    &
    \text{list}
    \\
    && \mid &
    \tyLift\;\tau
    &
    \text{lifting}
  \end{array}
  \]
  \end{subfigure}%
  \begin{subfigure}[t]{0.48\linewidth}
  \small
  \[
  \begin{array}{lllll}
    & \textit{Terms}
    \\
    &
    t, s
    & ::= &
    x
    &
    \text{variable}
    \\
    && \mid &
    \phi(\vec t)
    &
    \text{primitive op}
    \\
    && \mid &
    \tmInl{t} \mid \tmInr{t}
    &
    \text{injection}
    \\
    && \mid &
    \tmCase{s}{x}{t_1}{y}{t_2}
    &
    \text{case}
    \\
    && \mid &
    \tmUnit
    &
    \text{unit}
    \\
    && \mid &
    \tmPair{s}{t}
    &
    \text{pair}
    \\
    && \mid &
    \tmFst{t} \mid \tmSnd{t}
    &
    \text{projection}
    \\
    && \mid &
    \tmFun{x}{t}
    &
    \text{function}
    \\
    && \mid &
    \tmApp{s}{t}
    &
    \text{application}
    \\
    && \mid &
    \tmNil
    &
    \text{nil}
    \\
    && \mid &
    \tmCons{s}{t}
    &
    \text{cons}
    \\
    && \mid &
    \tmFoldList{s_1}{s_2}{t}
    &
    \text{fold}
    \\
    && \mid &
    \tmReturn{t}
    &
    \text{return}
    \\
    && \mid &
    \tmBind{s}{t}
    &
    \text{bind}
  \end{array}
  \]
  \end{subfigure}
  \caption{Syntax of types and terms}
  \label{fig:syntax}
\end{figure}

\begin{figure}
\begin{subfigure}{\linewidth}
  \begin{mathpar}
  \small
  \inferrule*
  {
    \alpha: \kType \in \Delta
  }
  {
    \Delta \vdash \alpha: \kType
  }
  \and
  \inferrule*
  {
    \strut
  }
  {
    \Delta \vdash \tyZero: \kType
  }
  \and
  \inferrule*
  {
    \Delta \vdash \sigma: \kType
    \\
    \Delta \vdash \tau: \kType
  }
  {
    \Delta \vdash \sigma \tySum \tau: \kType
  }
  \and
  \inferrule*
  {
    \strut
  }
  {
    \Delta \vdash \tyUnit: \kType
  }
  \and
  \inferrule*
  {
    \Delta \vdash \sigma: \kType
    \\
    \Delta \vdash \tau: \kType
  }
  {
    \Delta \vdash \sigma \tyProd \tau: \kType
  }
  \and
  \inferrule*
  {
    \Delta \vdash \sigma: \kType
    \\
    \Delta \vdash \tau: \kType
  }
  {
    \Delta \vdash \sigma \tyFun \tau: \kType
  }
  \and
  \inferrule*
  {
    \Delta, \alpha: \kType \vdash \tau: \kType
    \\
    \Pol(+,\alpha,\tau)
  }
  {
    \Delta \vdash \mu\alpha.\tau: \kType
  }
  \and
  \inferrule*
  {
    \Delta \vdash \tau: \kType
  }
  {
    \Delta \vdash \tyLift\;\tau: \kType
  }
  \end{mathpar}
  \caption{Well-kinded types}
\end{subfigure}
\begin{subfigure}{\linewidth}
  \begin{mathpar}
    \small
    \inferrule*
    {
      x : \tau \in \Gamma
    }
    {
      \Gamma \vdash x: \tau
    }
    \and
    \inferrule*
    {
      \Gamma \vdash t : \sigma
    }
    {
      \Gamma \vdash \tmInl{t}: \sigma \tySum \tau
    }
    \and
    \inferrule*
    {
      \Gamma \vdash t : \tau
    }
    {
      \Gamma \vdash \tmInr{t}: \sigma \tySum \tau
    }
    \and
    \inferrule*
    {
      \Gamma \vdash s : \sigma \tySum \tau
      \\
      \Gamma, x: \sigma \vdash t_1 : \tau'
      \\
      \Gamma, y : \tau \vdash t_2 : \tau'
    }
    {
      \Gamma \vdash \tmCase{s}{x}{t_1}{y}{t_2}: \tau'
    }
    \and
    \inferrule*
    {
      \strut
    }
    {
      \Gamma \vdash \tmUnit : \tyUnit
    }
    \and
    \inferrule*
    {
      \Gamma \vdash s : \sigma
      \\
      \Gamma \vdash t : \tau
    }
    {
      \Gamma \vdash \tmPair{s}{t}: \sigma \tyProd \tau
    }
    \and
    \inferrule*
    {
      \Gamma \vdash t : \sigma \tyProd \tau
    }
    {
      \Gamma \vdash \tmFst{t}: \sigma
    }
    \and
    \inferrule*
    {
      \Gamma \vdash t : \sigma \tyProd \tau
    }
    {
      \Gamma \vdash \tmSnd{t}: \tau
    }
    \and
    \inferrule*
    {
      \Gamma, x: \sigma \vdash t : \tau
    }
    {
      \Gamma \vdash \tmFun{x}{t}: \sigma \tyFun \tau
    }
    \and
    \inferrule*
    {
      \Gamma \vdash s: \sigma \tyFun \tau
      \\
      \Gamma \vdash t : \sigma
    }
    {
      \Gamma \vdash \tmApp{s}{t}: \tau
    }
    \and
    \inferrule*
    {
      \Gamma \vdash t : \subst{\tau}{\mu \alpha.\tau}{\alpha}
    }
    {
      \Gamma \vdash \tmRoll{t}: \mu\alpha.\tau
    }
    \and
    \inferrule*
    {
      \Gamma \vdash s : \subst{\sigma}{\tau}{\alpha} \tyFun \tau
      \\
      \Gamma \vdash t : \mu\alpha.\sigma
    }
    {
      \Gamma \vdash \tmFold{s}{t} : \tau
    }
    \and
    \inferrule*
    {
      \Gamma \vdash t : \tau
    }
    {
      \Gamma \vdash \tmReturn{t} : \tyLift\;\tau
    }
    \and
    \inferrule*
    {
      \Gamma \vdash s : \tyLift\;\sigma
      \\
      \Gamma \vdash t : \sigma \tyFun \tyLift\;\tau
    }
    {
      \Gamma \vdash \tmBind{s}{t} : \tyLift\;\tau
    }
  \end{mathpar}
  \caption{Well-typed terms (all types well-kinded)}
\end{subfigure}
\caption{Kinding and typing rules}
\label{fig:typing}
\end{figure}


\figrefTwo{syntax}{typing} give the syntax and typing rules of the higher-order language.

\subsubsection{Semantics}

\begin{figure}
\begin{subfigure}{\linewidth}
  \begin{mathpar}
  \small
    \inferrule*
    {
      \strut
    }
    {
      \cat{C} \in \Ob(\muPoly_{\cat{C}})
    }
    \and
    \inferrule*
    {
      \strut
    }
    {
      \One \in \Ob(\muPoly_{\cat{C}})
    }
    \and
    \inferrule*
    {
      \cat{D} \in \Ob(\muPoly_{\cat{C}})
      \\
      \cat{D}' \in \Ob(\muPoly_{\cat{C}})
    }
    {
      \cat{D} \times \cat{D}' \in \Ob(\muPoly_{\cat{C}})
    }
  \end{mathpar}
  \caption{Objects}
\end{subfigure}
\begin{subfigure}{\linewidth}
  \begin{mathpar}
  \small
    \inferrule*
    {
      \strut
    }
    {
      !_{\cat{D}} : \cat{D} \to \One \in \Mor(\muPoly_{\cat{C}})
    }
    \and
    \inferrule*
    {
      \strut
    }
    {
      F : \One \to \cat{D} \in \Mor(\muPoly_{\cat{C}})
    }
    \and
    \inferrule*
    {
      \strut
    }
    {
      - \times - : \cat{C} \times \cat{C} \to \cat{C} \in \Mor(\muPoly_{\cat{C}})
    }
    \and
    \inferrule*
    {
      \strut
    }
    {
      - \textstyle\coprod - : \cat{C} \times \cat{C} \to \cat{C} \in \Mor(\muPoly_{\cat{C}})
    }
    \and
    \inferrule*
    {
      \cat{D} \in \Ob(\muPoly_{\cat{C}})
      \\
      \cat{D}' \in \Ob(\muPoly_{\cat{C}})
    }
    {
      \pi_1 : \cat{D} \times \cat{D}' \to \cat{D} \in \Mor(\muPoly_{\cat{C}})
    }
    \and
    \inferrule*
    {
      \cat{D} \in \Ob(\muPoly_{\cat{C}})
      \\
      \cat{D}' \in \Ob(\muPoly_{\cat{C}})
    }
    {
      \pi_2 : \cat{D} \times \cat{D}' \to \cat{D}' \in \Mor(\muPoly_{\cat{C}})
    }
    \and
    \inferrule*
    {
      \cat{E} \in \Ob(\muPoly_{\cat{C}})
      \\
      \cat{D} \in \Ob(\muPoly_{\cat{C}})
      \\
      \cat{D}' \in \Ob(\muPoly_{\cat{C}})
      \\
      F: \cat{E} \to \cat{D} \in \Mor(\muPoly_{\cat{C}})
      \\
      G: \cat{E} \to \cat{D}' \in \Mor(\muPoly_{\cat{C}})
    }
    {
      \prodM{F}{G} : \cat{E} \to \cat{C} \times \cat{D}' \in \Mor(\muPoly_{\cat{C}})
    }
    \and
    \inferrule*
    {
      \text{TODO}
    }
    {
      \mu F : \cat{D} \to \cat{C} \in \Mor(\muPoly_{\cat{C}})
    }
  \end{mathpar}
  \caption{Morphisms}
\end{subfigure}
\caption{Rules inductively defining $\muPoly_{\cat{C}}$ for any $\cat{C}$ with finite coproducts and finite
products}
\label{fig:mu-polynomial}
\end{figure}

\begin{figure}
\begin{subfigure}{\linewidth}
  \begin{align*}
  \small
  \sem{\vec\alpha \vdash \alpha_i}(\vec X) &=
  X_i
  \\
  \sem{\Delta \vdash \tyZero}(\vec X) &=
  \Zero
  \\
  \sem{\Delta \vdash \sigma \tySum \tau}(\vec X) &=
  \textstyle {\sem{\sigma}(\vec X)} \coprod {\sem{\tau}(\vec X)}
  \\
  \sem{\Delta \vdash \tyUnit}(\vec X) &=
  \One
  \\
  \sem{\Delta \vdash \sigma \tyProd \tau}(\vec X) &=
  \sem{\Delta \vdash \sigma}(\vec X) \times \sem{\Delta \vdash \tau}(\vec X)
  \\
  \sem{\Delta \vdash \sigma \tyFun \tau}(\vec X) &=
  \internalHom{\sem{\Delta \vdash \sigma}(\vec X)}{\sem{\Delta \vdash \tau}(\vec X)}
  \\
  \sem{\vec\alpha \vdash \mu\alpha_i.\tau}(\vec X) &=
  \mu(\lambda Y.\sem{\vec\alpha_{\neq \alpha_i}, \alpha \vdash \tau}(\vec X_{\neq X_i}, Y))
  \\
  \sem{\Delta \vdash \mu\alpha.\tau}(\vec X) &=
  \mu(\lambda Y.\sem{\Delta, \alpha \vdash \tau}(\vec X, Y))
  \textit{ if }\alpha \notin \Delta
  \\
  \sem{\Delta \vdash \tyLift\;\tau}(\vec X) &=
  \mathcal{L}(\sem{\Delta \vdash \tau})(\vec X)
  \end{align*}
  \caption{Parameterised types $\Delta \vdash \tau$ as multifunctors $\cat{C}^n \to \cat{C}$}
  \label{fig:default-semantics:types}
\end{subfigure}
\begin{subfigure}{\linewidth}
  \begin{align*}
  \small
  \sem{\emptyCxt} &= \One
  \\
  \sem{\Gamma, x: \tau} &= \sem{\Gamma} \times \sem{\tau}
  \end{align*}
  \caption{Contexts $\Delta \vdash \Gamma$}
\end{subfigure}
\begin{subfigure}{\linewidth}
  \begin{align*}
  \small
  \sem{x} &= \pi_i
  \textit{ where }\Gamma = x_1: \tau_1, \ldots, x_i: \tau_i, \ldots, x_n: \tau_n
  \\
  \sem{\tmInl{t}} &= \mathsf{inj}_1 \comp \sem{t}
  \\
  \sem{\tmInr{t}} &= \mathsf{inj}_2 \comp \sem{t}
  \\
  \sem{\tmCase{s}{x}{t_1}{y}{t_2}} &= \coprodM{t_1}{t_2} \comp \prodM{\id_{\sem{\Gamma}}}{\sem{s}}
  \\
  \sem{\tmUnit} &=\;!_{\sem{\Gamma}}
  \\
  \sem{\tmPair{s}{t}} &= \prodM{\sem{s}}{\sem{t}}
  \\
  \sem{\tmFst{t}} &= \pi_1 \comp \sem{t}
  \\
  \sem{\tmSnd{t}} &= \pi_2 \comp \sem{t}
  \\
  \sem{\tmFun{x}{t}} &= \lambda(\sem{t})
  \\
  \sem{\tmApp{s}{t}} &= \eval \comp \prodM{\sem{s}}{\sem{t}}
  \\
  \sem{\tmRoll{t}} &= \inMap_{\sem{\sigma}} \comp \sem{t}
  \textit{ where }\tau = \mu\alpha.\sigma
  \\
  \sem{\tmFold{s}{t}} &= \phi_{\sem{s}} \comp \sem{t}
  \\
  \sem{\tmReturn{t}} &= \eta_{\sem{\sigma}} \comp \sem{t}
  \textit{ where }\tau = \tyLift\;\sigma
  \\
  \sem{\tmBind{s}{t}} &= \mu_{\sem{\tau'}} \comp \mathcal{L}(\sem{t}) \comp \mathsf{st}_{\sem{\Gamma},\sem{\sigma}} \comp \prodM{\id_{\sem{\Gamma}}}{\sem{s}}
  \textit{ where }\Gamma \vdash s: \sigma\textit{ and }\tau = \tyLift\;\tau'
  \end{align*}
  \caption{Terms $\Gamma \vdash t: \tau$}
  \label{fig:default-semantics:terms}
\end{subfigure}
\caption{Conventional semantics in a suitable Cartesian closed category $\cat{C}$}
\end{figure}


\begin{definition}[$\mu$-polynomial]
Suppose $\cat{C}$ a category with finite coproducts and finite products. Define $\muPoly_{\cat{C}}$ to be the
smallest subcategory of $\Cat$ generated inductively by the rules in \figref{mu-polynomial}.
\end{definition}

For any $\mu$-polynomial endofunctor $F: \cat{C} \to \cat{C}$ with initial algebra $\mu F$, write $\inMap_F$
for the structure map $F\mu F \to \mu F$ and write $\cata_f$ for the unique morphism $\mu F \to X$ from the
initial $F$-algebra to any $F$-algebra $(X, f: FX \to X)$.

For the sake of a ``default'' (baseline) semantics, fix a bicartesian closed category $\cat{C}$ with finite
products $(\times, \One)$, finite coproducts $(\coprod, \Zero)$, exponentials $\internalHom{X}{Y}$, evaluation
morphisms $\eval_{X,Y}$ and currying isomorphisms $\lambda_{X,Y,Z}$, plus the following additional structure:
\begin{enumerate}
\item a strong monad $\mathcal{L}$ (lifting) with unit $\eta_X: X \to \mathcal{L}(X)$, multiplication $\mu_X:
\mathcal{L}^2(X) \to \mathcal{L}(X)$ and strength $\mathsf{st}_{X,Y}: X \times \mathcal{L}(Y) \to
\mathcal{L}(X \times Y)$;
\item $\mu$-polynomial endofunctors given by $\muPoly_{\cat{C}}$;
\item for each primitive type $\rho \in \PrimTy$, an object $\sem{\rho}$ of $\cat{C}$;
\item for each primitive operation $\phi \in \PrimOp^\rho_{\rho_1,\ldots,\rho_n}$, a morphism $\sem{\rho}:
\sem{\rho_1} \times \ldots \times \sem{\rho_n} \to \sem{\rho}$.
\end{enumerate}
\figref{default-semantics:types} gives the default interpretation of types $\Delta \vdash \tau$ with
$\length{\Delta} = n$ as multifunctors $\cat{C}^n \to \cat{C}$ (or as objects of $\cat{C}$ in the case $n =
0$). \figref{default-semantics:terms} gives the default interpretation of terms $\Gamma \vdash t: \tau$ as
morphisms $\sem{\Gamma} \to \sem{\tau}$.
