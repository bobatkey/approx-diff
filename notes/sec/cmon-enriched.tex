\subsection{$\CMon$-enriched categories}
\label{sec:cmon-enriched}

\begin{definition}[Commutative monoid]
A \emph{commutative monoid} $A = (A, \varepsilon, \bullet)$ is a set $A$ equipped with distinguished element
$\varepsilon \in A$ called the \emph{unit} and associative binary operation $\bullet$ satisfying $\varepsilon
\bullet x = x$ and $x \bullet \varepsilon = x$ for any $x \in A$.
\end{definition}

A commutative monoid homomorphism from $A$ to $B$ is any function $f: A \to B$ preserving $\varepsilon$ and
$\bullet$.

\begin{definition}[Category of commutative monoids]
Define $\CMon$ to be the category whose objects $A$ are the commutative monoids $A$ and morphisms $f: A \to B$
are the commutative monoid homomorphisms.
\end{definition}

$\CMon$ is closed monoidal, with the trivial one-element monoid $\One$ as terminal object and monoidal product
$A \tensor B$ given by the Cartesian product $A \times B$. $\CMon$ is complete and cocomplete, inheriting all
limits and colimits from $\Set$.

\subsubsection{$\CMon$-enriched category}

Recall that if a category $C$ is $\CMon$-enriched, then:
\begin{enumerate}
\item Every hom-object $\Hom{C}{A}{B}$ is a commutative monoid of morphisms; we write $\zero_{A,B}$ (zero
morphism) for the unit, and $+_{A,B}$ (addition of morphisms) for the binary operation.
\item Composition is \emph{bilinear}, i.e.~given by a family of morphisms $\Hom{C}{B}{C} \tensor
\Hom{C}{A}{B} \to \Hom{C}{A}{C}$ in $\CMon$ that preserve the additive structure in $\Hom{C}{B}{C}$ and
$\Hom{C}{A}{B}$ separately:

\begin{salign*}
f \comp \zero = f = \zero \comp f
\end{salign*}
\begin{salign*}
(f + g) \comp h &= (f \comp h) + (g \comp h) \\
h \comp (f + g) &= (h \comp f) + (h \comp g)
\end{salign*}
\end{enumerate}

Because $\CMon$ is closed monoidal, it is enriched over itself; every hom-object $\Hom{\CMon}{A}{B}$ is a
commutative monoid where:

\begin{enumerate}
\item the unit $0$ is the constant homomorphism sending every element of $A$ to $\varepsilon_B$;
\item the binary operation $+$ is pointwise addition of homomorphisms $(f + g)(a) = f(a) + g(a)$.
\end{enumerate}

\begin{proposition}
If $C$ is $\CMon$-enriched then any functor category $\Func{D}{C}$ is $\CMon$-enriched.
\end{proposition}

\subsubsection{$\CMon$-enriched presheaves}

\begin{definition}[$\mathscr{V}$-enriched functor]
For any monoidal category $(\mathscr{V}, \tensor, I)$ and $\mathscr{V}$-enriched categories $C$ and $D$, a
\emph{$\mathscr{V}$-enriched} functor $F: C \to D$ on morphisms is a family of morphisms $F_{X,Y}:
\Hom{C}{X}{Y} \to \Hom{D}{F(X)}{F(Y)}$ between hom-objects in $\mathscr{V}$ that makes the following diagrams
in $\mathscr{V}$ commute:

\begin{center}
\begin{tikzcd}[column sep=1.8cm]
I \arrow[d, equals] \arrow[r, "\id_X"] & \Hom{C}{X}{X} \arrow[d, "F_{X,X}"] \\
I \arrow[r, "\id_{F(X)}"'] & \Hom{D}{F(X)}{F(X)}
\end{tikzcd}
\hspace{5mm}
\begin{tikzcd}[column sep=2.6cm]
   \Hom{C}{Y}{Z} \tensor \Hom{C}{X}{Y} \arrow[r, "\comp_{X,Y,Z}"] \arrow[d, "F_{Y,Z} \tensor F_{X,Y}"'] & \Hom{C}{X}{Z} \arrow[d, "F_{X,Z}"] \\
   \Hom{D}{F(Y)}{F(Z)} \tensor \Hom{D}{F(X)}{F(Y)} \arrow[r, "\comp_{F(X),F(Y),F(Z)}"'] & \Hom{D}{F(X)}{F(Z)}
\end{tikzcd}
\end{center}
\end{definition}

Write $\Func{C}{D}_{\mathscr{V}}$ for the category of $\mathscr{V}$-enriched functors from $C$ to $D$.

In the specific setting of a $\CMon$-enriched presheaf $F: C^\op \to \CMon$, enrichment means that $F$
preserves the additive structure of morphisms:
\begin{itemize}
\item $F(0_{X,Y}) = 0_{F(Y),F(X)}$
\item $F(f + g) = F(f) + F(g)$
\end{itemize}
It then follows automatically that $F$ also preserves bilinear composition, i.e.:
\begin{itemize}
\item $F(f) \comp F(0) = F(f) = F(0) \comp F(f)$
\item $F(f + g) \comp F(h) = F(h \comp f) + F(h \comp g)$
\item $F(h) \comp F(f + g) = F(f \comp h) + F(g \comp h)$
\end{itemize}
in addition to the usual enriched functor properties.

\subsubsection{$\CMon$-enriched Yoneda embedding}

Suppose $C$ a small category. The usual \emph{Yoneda embedding} for $C$ is the functor $\Yoneda: C \to
\Func{C^\op}{\Set}$ sending:
\begin{itemize}
\item any object $X$ in $C$ to the contravariant hom-functor $\Yoneda(X) = \Hom{C}{-}{X}: C^\op \to \Set$
which sends:
   \begin{itemize}
   \item any object $Y$ in $C$ to the hom-set $\Hom{C}{Y}{X}$;
   \item any morphism $f: Y \to Z$ in $C$ to the function $\Hom{C}{f}{X} = (- \comp f): \Hom{C}{Z}{X} \to
   \Hom{C}{Y}{X}$.
   \end{itemize}
\item any morphism $f: X \to Y$ in $C$ to the natural transformation $\Yoneda(f)$ where $\Yoneda(f)_Z = (f
\comp -): \Hom{C}{Z}{X} \to \Hom{C}{Z}{Y}$.
\end{itemize}

\noindent Now suppose $C$ is $\mathscr{V}$-enriched with $\mathscr{V}$ closed monoidal; then the hom-functor
$\Hom{C}{-}{X}: C^\op \to \mathscr{V}$ is $\mathscr{V}$-enriched. The following diagram commutes because
precomposition with the identity is the identity on morphisms:

\begin{center}
\begin{tikzcd}[column sep=1.8cm]
I \arrow[d, equals] \arrow[r, "\id_X"] & \Hom{C}{X}{X} \arrow[d, "\Hom{C}{-}{W}_{X,X}"] \\
I \arrow[r, "\id_{\Hom{C}{X}{W}}"'] & \Hom{\mathscr{V}}{\Hom{C}{X}{W}}{\Hom{C}{X}{W}}
\end{tikzcd}
\end{center}

\noindent and the following by the associativity of composition:

\begin{center}
\begin{tikzcd}[column sep=2.7cm]
   \Hom{C}{Z}{Y} \tensor \Hom{C}{Y}{X} \arrow[r, "\comp_{X,Y,Z}"] \arrow[d, "\Hom{C}{-}{W}_{Y,Z}\,\tensor\,\Hom{C}{-}{W}_{X,Y}"'] & \Hom{C}{Z}{X} \arrow[d, "\Hom{C}{-}{W}_{X,Z}"] \\
   \Hom{\mathscr{V}}{\Hom{C}{Y}{W}}{\Hom{C}{Z}{W}} \tensor \Hom{\mathscr{V}}{\Hom{C}{X}{W}}{\Hom{C}{Y}{W}} \arrow[r, "\comp_{\Hom{C}{X}{W},\Hom{C}{Y}{W},\Hom{C}{Z}{W}}"'] & \Hom{\mathscr{V}}{\Hom{C}{X}{W}}{\Hom{C}{Z}{W}}
\end{tikzcd}
\end{center}

\noindent \todo{Then there is an enriched version of the Yoneda embedding, namely a functor
$\Yoneda_{\mathscr{V}}: C \to \Func{C^\op}{\mathscr{V}}_{\mathscr{V}}$ analogous to $\Yoneda$ which is itself
$\mathscr{V}$-enriched.}

% First we note that the hom-functor $\Hom{C}{-}{X}: C \to \CMon$ is $\CMon$-enriched because composition is
% bilinear and so $(- \comp f)$ preserves the additive structure of morphisms.
