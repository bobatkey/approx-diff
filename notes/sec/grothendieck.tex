
\subsection{Category of families}

\begin{definition}[Category of $I$-indexed families of objects]
For any set $I$ and any category $C$ write $\Fam(I,C)$ for the category where:
\begin{itemize}
\item objects are the $I$-indexed families of objects of $C$;
\item morphisms from $X$ to $Y$ are families of morphisms $f_i: X_i \to Y_i$ in $C$ for any $i \in I$.
\end{itemize}
\end{definition}

\noindent Equivalently $\Fam(I,C)$ is the functor category $\Func{I}{C}$ where we interpret $I$ as a discrete
category.

\begin{definition}[Reindexing]
For any $f: I \to J$ the \emph{reindexing} functor $\reindex{-}{f}: \Fam(J,C) \to \Fam(I,C)$ sends any
$J$-indexed family $X$ to the $I$-indexed family $X[f]$ where $X[f]_i = X_{f(i)}$ for any $i \in I$ and
similarly for morphisms.
\end{definition}

\begin{definition}[$\Fam(-,C)$ functor]
$\Fam(-,C): \Set^{\op} \to \Cat$ is then the functor which sends any set $I$ to $\Fam(I,C)$ and any function
$f: I \to J$ to the functor $\reindex{-}{f}: \Fam(J,C) \to \Fam(I,C)$.
\end{definition}

\begin{definition}[Grothendieck construction for $F$]
Suppose a functor $F: D \to \Cat$. The \emph{Grothendieck construction} $\Grothendieck{D}F$ for $F$ is the
category where:
\begin{itemize}
\item objects are pairs $(I, X)$ of an object $I$ of $D$ and an object $X$ of $F(I)$;
\item morphisms from $(I, X)$ to $(J, Y)$ are morphisms $f: I \to J$ in $D$ paired with a morphism $F(f)(X)
\to Y$ in $F(J)$.
\end{itemize}
\end{definition}

\noindent When $F$ is contravariant, a morphism from $(I, X)$ to $(J, Y)$ is a morphism $f: I \to J$ in $D$
paired with a morphism $X \to F(f)(Y)$ in $F(I)$.

\begin{definition}[Category of families]
For any category $C$ define $\Fam(C)$ to be the Grothendieck construction
$\Grothendieck{\Set^{\op}}\Fam(-,C)$.
\end{definition}

In $\Fam(C)$ we then have that:
\begin{itemize}
\item objects are pairs $(I, X)$ of a set $I$ and an indexed family $X$ in $\Fam(I,C)$;
\item morphisms from $(I, X)$ to $(J, Y)$ are functions $f: I \to J$ paired with a morphism $X \to Y[f]$ in
$\Fam(I,C)$.
\end{itemize}

\begin{proposition}
If $C$ has binary products then so does $\Fam(C)$.
\end{proposition}

\begin{proposition}
Suppose $C$ locally small.
\begin{enumerate}
\item If $C$ has binary coproducts and set-indexed products, then $\Fam(C)$ is symmetric monoidal, with the
coproduct as the monoidal product.
\item If the coproducts are in fact biproducts, then $\Fam(C)$ is Cartesian closed.
\end{enumerate}
\end{proposition}

\begin{proof}
Suppose $X = (I, X)$ and $Y = (I, Y)$ are objects in $\Fam(C)$. Define $[X, Y]$ to be the object $(K,Z)$ in
$\Fam(C)$ where $K$ is the set of morphisms from $\Fam(I,C)$ to $\Fam(J,C)$ and $Z$ is the $K$-indexed family
which maps every $f: \Fam(I,C) \to \Fam(J,C)$ in $K$ to $\prod_{i \in I}\reindex{Y}{f}_i$. Then the following
families of morphisms in $\Fam(C)$:
\begin{enumerate}
\item $\lambda_{X,Y,Z}: \Fam(C)(X \times Y, Z) \to \Fam(C)(X, [Y, Z])$
\item $\eval_{X,Y}: [X, Y] \times X \to Y$
\end{enumerate}
\end{proof}
