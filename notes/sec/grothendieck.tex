\subsection{Category of families}

\begin{definition}[Grothendieck construction for $F$]
Suppose a functor $F: D \to \Cat$. The \emph{Grothendieck construction} $\Grothendieck{D}F$ for $F$ is the
category where:
\begin{itemize}
\item objects $X$ are pairs of an object $\idx{X}$ of $D$ and an object $\faml{X}$ of $F(\idx{X})$;
\item morphisms $f: X \to Y$ are morphisms $\idx{f}: \idx{X} \to \idx{Y}$ in $D$ paired with morphisms
$\faml{f}: F(f)(\faml{X}) \to \faml{Y}$ in $F(\idx{Y})$.
\end{itemize}
\end{definition}

\noindent When $F$ is contravariant, a morphism $f: X \to Y$ is a morphism $\idx{f}: \idx{X} \to \idx{Y}$ in
$D$ paired with morphisms $\faml{f}: \faml{X} \to F(f)(\faml{Y})$ in $F(\idx{X})$.

\begin{definition}[Category of families]
For any category $C$ define $\Fam(C)$ to be the Grothendieck construction
$\Grothendieck{\Set^{\op}}\Fam(-,C)$.
\end{definition}

\noindent In $\Fam(C)$ the objects $X$ are thus sets $\idx{X}$ paired with indexed families $\faml{X}$ in
$\Fam(\idx{X},C)$ and morphisms $f: X \to Y$ are functions $\idx{f}: \idx{X} \to \idx{Y}$ paired with
morphisms $\faml{f}: \faml{X} \to \reindex{\faml{Y}}{\idx{f}}$ in $\Fam(\idx{X},C)$.

\begin{proposition}
\item
\begin{enumerate}
\item If $C$ is locally small then so is $\Fam(C)$.
\item If $C$ has binary products then so does $\Fam(C)$.
\end{enumerate}
\end{proposition}

\begin{proposition}
Suppose $C$ locally small. If $C$ has binary biproducts and set-indexed products, then $\Fam(C)$ is Cartesian
closed. \todo{Establish first that if $C$ has binary coproducts and set-indexed products, then $\Fam(C)$ is
symmetric monoidal, with the coproduct as monoidal product.}
\end{proposition}

\begin{proof}
Suppose $C$ is locally small, with binary biproducts given by $(\biprod,\inj,\proj)$ and set-indexed products
given by $(\prod,\eval_{\prod},\lambda_{\prod})$. For any objects $X, Y, Z$ in $\Fam(C)$ define:

\begin{enumerate}
\item $[X, Y]$ to be the object in $\Fam(C)$ with $\idx{[X,Y]} = \Fam(C)(X,Y)$ and $\faml{[X, Y]}(f) =
\prod_{\idx{X}}\reindex{\faml{Y}}{\idx{f}}$ for every $f: X \to Y$.
\item The family of isomorphisms $\lambda_{X,Y,Z}$ (natural in $X$) sending any $f: X \times Y \to Z$ to
$\lambda_{X,Y,Z}(f): X \to [Y, Z]$ with:
\begin{itemize}
\item $\idx{\lambda}_{X,Y,Z}(f)(x) = g$ where $\idx{g} = \idx{f} \comp (x, -): \idx{Y} \to \idx{Z}$ and
$\faml{g} = \faml{f} \comp \inj_{\faml{Y}}: \faml{Y} \to \reindex{\faml{Z}}{\idx{g}}$ where $\inj_{\faml{Y}}:
Y \to \const_{\idx{Y}}(\faml{X}_x) \biprod Y$
\item $\faml{\lambda}_{X,Y,Z}(f)(x) = \lambda_{\prod_{\idx{Y}}}(y \mapsto \faml{f}_{x, y} \comp
\inj_{\faml{X}_x}: \faml{X}_x \to \faml{Z}_{f(x,y)}$ where $\inj_{\faml{X}_x}: \faml{X}_x \to \faml{X}_x
\biprod \faml{Y}_y$)
\end{itemize}
\item The family of morphisms $\eval_{X,Y}: [X, Y] \times X \to Y$ in $\Fam(C)$ with
\begin{itemize}
\item $\idx{\eval}_{X,Y}(f, x) = \idx{f}(x)$
\item $\faml{\eval}_{X,Y}(f, x) = \coprodM{{\eval_{\prod}}_x}{\faml{f}_x}:
\prod_{\idx{X}}\reindex{\faml{Y}}{f} \biprod \faml{X}_x \to \faml{Y}_{\idx{f}(x)}$
\end{itemize}
using $\coprodM{-}{-}$ to denote coproduct of morphisms.
\end{enumerate}
Then for any $f: X \times Y \to Z$ in $\Fam(C)$ we have:
\begin{enumerate}
\item To show that the diagram on the left in $\Fam(C)$ commutes, it suffices to show that the diagram on the
right in $C$ commutes for every $(x, y) \in \idx{X} \times \idx{Y}$:

\begin{center}
\begin{tikzcd}
   {{[Y, Z]}} \tensor Y \arrow[dr, "\eval_{Y,Z}"] \\
   X \tensor Y \arrow[u, "\lambda_{X,Y,Z}(f) \tensor \id_Y"] \arrow[r, "f"'] & Z
\end{tikzcd}
\hspace{1cm}
\begin{tikzcd}
   \prod_{\idx{Y}}\reindex{\faml{Z}}{y' \mapsto \idx{f}(x,y')} \biprod \faml{Y}_y \arrow[dr,
   "\coprodM{\eval_{\prod_{\idx{Y}}}}{\faml{f}_x}"] \\
   \faml{X}_x \biprod \faml{Y}_y \arrow[u, "\lambda_{\prod_{\idx{Y}}}(y'\;\mapsto\;\faml{f}_{x, y'}\;\comp\;
\inj_{\faml{X}_x}) \biprod \id_{\faml{Y}_y}"] \arrow[r,
   "\faml{f}_{x,y}"'] & \faml{Z}_{\idx{f}(x,y)}
\end{tikzcd}
\end{center}

\item \todo{(exhibit $\lambda^{-1}$ and then show) $\lambda^{-1}_{X,Y,Z}(\lambda_{X,Y,Z}(f)) = f$}
\end{enumerate}
\end{proof}
