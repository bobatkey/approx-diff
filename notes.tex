\documentclass[acmsmall,screen]{acmart}

\usepackage{tabularx}
\usepackage{bbm}
\usepackage{tikz-cd}
\usepackage{xspace}

%\geometry{paperwidth=8.3in, paperheight=11.7in} % force to A4 for now
\settopmatter{printacmref=false}
\citestyle{acmauthoryear}
\raggedbottom

\newcommand*{\note}[1]{\textcolor{blue}{\textbf{note:} #1}}
\newcommand*{\todo}[1]{\textcolor{blue}{\textbf{todo:} #1}}

\newenvironment{salign*}
   {\par\nobreak\small\noindent\csname align*\endcsname}
   {\csname endalign*\endcsname}

\newcommand*{\secref}[1]{\S\ref{sec:#1}}
\newcommand*{\propref}[1]{Proposition~\ref{prop:#1}}

\newcommand*{\biinj}{\mathsf{i}}
\newcommand*{\biprod}{\oplus}
\newcommand*{\biproj}{\mathsf{p}}
\newcommand*{\cat}[1]{\mathbf{#1}}
\newcommand*{\comp}{\circ}
\newcommand*{\const}{\mathsf{const}}
\newcommand*{\coprodM}[2]{[#1, #2]}
\newcommand*{\eval}{\varepsilon}
\newcommand*{\faml}[1]{\overline{#1}} % \fam already defined
\newcommand*{\id}{\mathsf{id}}
\newcommand*{\idx}[1]{\widehat{#1}}
\newcommand*{\inj}{\mathsf{inj}}
\newcommand*{\iso}{\cong}
\newcommand*{\join}{\vee}
\newcommand*{\meet}{\wedge}
\newcommand*{\mult}{\cdot}
\newcommand*{\op}{\mathsf{op}}
\newcommand*{\prodM}[2]{\langle #1, #2\rangle}
\newcommand*{\proj}{\pi}
\newcommand*{\reindex}[2]{#1[#2]}
\newcommand*{\tensor}{\otimes}
\newcommand*{\zero}{0}

\newcommand*{\One}{\mathbbm{1}}
\newcommand*{\Hom}[3]{{#1}(#2,#3)}

% Specific categories
\newcommand*{\Cat}{\cat{Cat}}
\newcommand*{\CMon}{\cat{CMon}}
\newcommand*{\Fam}{\cat{Fam}}
\newcommand*{\FdVect}{\cat{FdVect}}
\newcommand*{\Func}[2]{[#1,#2]}
\newcommand*{\Grothendieck}[1]{\int_{#1}}
\newcommand*{\LatGal}{\cat{LatGal}}
\newcommand*{\Set}{\cat{Set}}
\newcommand*{\Setoid}{\cat{Setoid}}


\begin{document}

\title{Approximation as Differentiation: Notes}

\maketitle

\subsection{Overview}

Covered here:
\begin{itemize}
\item automatic differentiation (\secref{auto-diff})
\item stable functions (\secref{stable-functions})
\item $\CMon$, the category of commutative monoids (\secref{cmon})
\item $\CMon$-enrichment, including enriched functors and enriched Yoneda embedding (\secref{cmon-enriched})
\item biproducts and semi-additive categories (\secref{biproduct})
\item useful semi-additive categories (\secref{useful-semi-additive-categories}):
   \begin{itemize}
   \item category $\LatGal$ of bounded lattices and Galois connections (\secref{categories-with-biproducts:latgal})
   \item category $\FinVect_F$ of finite-dimensional vector spaces over a field $F$
   (\secref{categories-with-biproducts:fdvect})
   \end{itemize}
\item category $\Fam(I,\cat{C})$ of $I$-indexed families of objects of $\cat{C}$ (\secref{fam})
\item set-indexed products (\secref{set-indexed-product})
\item Grothendieck construction $\Grothendieck{\cat{C}}F$ and category of families $\Fam(\cat{C})$
(\secref{grothendieck})
\item useful categories of families for automatic approximation
(\secref{galois-slicing-auto-diff-via-fam}):
   \begin{itemize}
      \item Galois slicing via $\Fam(\LatGal)$ (\secref{galois-slicing-auto-diff-via-fam:galois-slicing})
      \item automatic differentiation via $\Fam(\FinVect)$ (\secref{galois-slicing-auto-diff-via-fam:auto-diff})
   \end{itemize}
\end{itemize}

\noindent $\Set$ will usually be $\Setoid$ in the Agda implementation but we will gloss that detail for now.

\subsection{Preliminaries}

\subsubsection{Notation}

\paragraph{Coproduct and product of morphisms.} In a category with coproducts $X + Y$ or products $X \times
Y$, we write:
\begin{itemize}
\item $\coprodM{f}{g}: X \coprod Y \to Z$ for the coproduct of morphisms $f: X \to Z$ and $g: Y \to Z$
\item $\prodM{f}{g}: X \to Y \times Z$ for the product of morphisms $f: X \to Y$ and $g: X \to Z$
\item $(f,g): X \times Y \to Z \times W$ for $\prodM{f \comp \proj_1}{g \comp \proj_2}$ where $f: X \to Z$ and $g: Y \to W$.
\end{itemize}

\paragraph{Constant function.} Write $\const(y): X \to Y$ for the function which sends any $x \in X$ to $y \in
Y$.

\paragraph{Functor category.} Write $\Func{\cat{C}}{\cat{D}}$ for the category of functors $\cat{C} \to
\cat{D}$.

\subsubsection{$\mathscr{V}$-categories (for concrete $\mathscr{V}$)}

Let $\mathscr{V}$ range over monoidal categories $(\mathscr{V}, \tensor, I)$ that are \emph{concrete},
i.e.~have a forgetful functor to $\Set$. In any $\mathscr{V}$-enriched category (or
\emph{$\mathscr{V}$-category}) $\cat{C}$, the hom-objects $\Hom{\cat{C}}{X}{Y}$ have underlying sets and the
functoriality and naturality conditions for $\mathscr{V}$-functors and $\mathscr{V}$-natural transformations
can be formulated elementwise (as per the $\Set$-enriched case) rather than more generally in terms of
morphisms from the tensor unit $I$. \todo{Be explicit about composition being a family of morphisms
$\Hom{\cat{C}}{Y}{Z} \tensor \Hom{\cat{C}}{X}{Y} \to \Hom{\cat{C}}{X}{Z}$ in $\mathscr{V}$.}

\begin{definition}[$\mathscr{V}$-functor]
\label{def:cmon-enriched:enriched-functor}
For any $\mathscr{V}$-enriched categories $\cat{C}$ and $\cat{D}$, a \emph{$\mathscr{V}$-enriched} functor $F:
\cat{C} \to \cat{D}$ on morphisms is a map sending every object $X$ of $\cat{C}$ to an object $F(X)$ of
$\cat{D}$, plus a family of morphisms $F_{X,Y}: \Hom{\cat{C}}{X}{Y} \to \Hom{\cat{D}}{F(X)}{F(Y)}$ between
hom-objects in $\mathscr{V}$, obeying the usual functor laws.
\end{definition}

\noindent The action of a $\mathscr{V}$-functor on morphisms thus necessarily preserves the
$\mathscr{V}$-structure on the hom-objects, since it is is given by a morphism in $\mathscr{V}$. The notion of
$\mathscr{V}$-natural transformation is defined similarly.

\begin{definition}[$\mathscr{V}$-natural transformation]
For any $\mathscr{V}$-enriched categories $\cat{C}$ and $\cat{D}$ and $\mathscr{V}$-functors $F, G: \cat{C}
\to \cat{D}$, a {$\mathscr{V}$-natural transformation} $\eta: F \naturalto G$ is a family of morphisms
$\eta_X: F(X) \to G(X)$ in $\cat{D}$ satisfying the usual naturality square for every $f: X \to Y$ in
$\cat{C}$. \todo{This is sufficient because $\comp_{X,Y,Z}$ as well as $F$ and $G$ respects the
$\mathscr{V}$-structure?}
\end{definition}

% making the following commute:
%
% \begin{center}
% \begin{tikzcd}[column sep=1.8cm]
% I \arrow[d, equals] \arrow[r, "\id_X"] & \Hom{\cat{C}}{X}{X} \arrow[d, "F_{X,X}"] \\
% I \arrow[r, "\id_{F(X)}"'] & \Hom{\cat{D}}{F(X)}{F(X)}
% \end{tikzcd}
% \hspace{5mm}
% \begin{tikzcd}[column sep=2.6cm]
%    \Hom{\cat{C}}{Y}{Z} \tensor \Hom{\cat{C}}{X}{Y} \arrow[r, "\comp_{X,Y,Z}"] \arrow[d, "F_{Y,Z} \tensor F_{X,Y}"']
%    & \Hom{\cat{C}}{X}{Z} \arrow[d, "F_{X,Z}"] \\
%    \Hom{\cat{D}}{F(Y)}{F(Z)} \tensor \Hom{\cat{D}}{F(X)}{F(Y)} \arrow[r, "\comp_{F(X),F(Y),F(Z)}"'] &
%    \Hom{\cat{D}}{F(X)}{F(Z)}
% \end{tikzcd}
% \end{center}
%
% \noindent These generalise the usual functor laws to a setting where the hom-objects are not necessarily sets
% and thus do not support an elementwise formulation, although here we are only interested in the situation
% where $\mathscr{V}$ is concrete.

\begin{definition}[Category of enriched functors]
Write $\Func{\cat{C}}{\cat{D}}_{\mathscr{V}}$ for the category of $\mathscr{V}$-enriched functors from
$\cat{C}$ to $\cat{D}$.
\end{definition}

\subsection{Automatic differentiation}
\label{sec:auto-diff}

For a differentiable manifold $M$, write $\tangents_x(M)$ for the tangent space at a point $x \in M$ and
$\cotangents_x(M)$ for the cotangent space at $x$. Now consider flat manifolds of the form $\RR^n$.

\begin{itemize}
\item The \emph{forward derivative} (tangent map or pushforward) $\pushf{f}_x$ of a differentiable function
$f: \RR^m \to \RR^n$ at $x \in \RR^m$ is the unique linear map $\tangents_x(\RR^m) \linearto
\tangents_{f(x)}(\RR^n)$ given by the Jacobean matrix of $f$ at $x$. Since $\RR^n$ is flat,
$\tangents_x(\RR^n)$ is canonically isomorphic to $\RR^n$, so in fact $\pushf{f}_x$ is a linear map $\RR^m
\linearto \RR^n$.
\item The \emph{backward derivative} (cotangent map or pullback) $\pullf{f}_x$ is the unique linear map
$\cotangents_{f(x)}(\RR^n) \linearto \cotangents_x(\RR^m)$ given by the transpose (adjoint) of the Jacobean
matrix of $f$ at $x$. Since $\RR^n$ is flat, $\cotangents_x(\RR^n)$ is also canonically isomorphic to $\RR^n$,
so in fact $\pullf{f}_x$ is a linear map $\RR^n \linearto \RR^m$.
\end{itemize}

\subsubsection{Chain rule}

Derivatives respect composition. Suppose $f: \RR^m \to \RR^n$ and $g: \RR^n \to \RR^k$. For any $x \in \RR^m$
we have:

\begin{itemize}
\item $\pushf{(g \comp f)}_x = \pushf{g}_{f(x)} \comp \pushf{f}_x: \RR^m \to \RR^k$
\item $\pullf{(g \comp f)}_x = \pullf{f}_{x} \comp \pullf{g}_{f(x)}: \RR^k \to \RR^m$
\end{itemize}

\subsubsection{Forward mode automatic differentiation}

\begin{definition}
Define the functor $\tangents$ which sends every vector space $\RR^n$ to itself and every differentiable map
$f: \RR^m \to \RR^n$ to the function $\tangents(f) = x \mapsto (f(x), \pushf{f}_x): \RR^m \to \RR^n \times
(\RR^m \linearto \RR^n)$, associating to every point its image in $f$ and the forward derivative of $f$ at
that point.
\end{definition}

For any map $f: \RR^m \to \RR^n \times (\RR^m \linearto \RR^n)$ define $f_1 = \pi_1 \comp f: \RR^m \to \RR^n$
and $f_2 = \pi_2 \comp f: \RR^m \to (\RR^m \linearto \RR^n)$.

\begin{definition}[$\Diff$]
\label{def:auto-diff:Diff}
Define $\Diff$ to be the category where the objects are all finite vector spaces $\RR^n$ and the morphisms
are all maps $f: \RR^m \to \RR^n \times (\RR^m \linearto \RR^n)$ with $f_1$ differentiable. Composition in
$\Diff$ is given by:
\begin{align*}
(g \comp f)_1(x) &= f_1(g_1(x)) \\
(g \comp f)_2(x) &= g_2(f_1(x)) \comp f_2(x)
\end{align*}
\end{definition}

We can verify that $\tangents$ is indeed a functor, regardless of whether the linear maps $\RR^m \linearto
\RR^n$ happen to be derivatives. But if $f_2(x)$ is the derivative of $f_1$ at $x$ for any $x \in \RR^m$ and
similarly for $g$ then the derivatives compose according to the forward chain rule and $(g \comp f)_2(x)$ is
the derivative of $(g \comp f)_1$ at $x$.

\subsubsection{Reverse mode automatic differentiation}

\begin{definition}
Define the functor $\cotangents$ which sends every vector space $\RR^n$ to itself and every differentiable map
$f: \RR^m \to \RR^n$ to the function $\cotangents(f) = x \mapsto (f(x), \pullf{f}_x): \RR^m \to \RR^n \times
(\RR^n \linearto \RR^m)$.
\end{definition}

\noindent and again define a corresponding category $\Diff^*$ of such maps where composition of the cotangent
maps respects the backward chain rule. \todo{In practice can't combine $\Diff$ and $\Diff^*$ into a single
category with morphisms $f: \RR^m \to \RR^n \times (\RR^m \linearto \RR^n) \times (\RR^n \linearto \RR^m)$
because..?}

\subsection{Stable Functions}
\label{sec:stable-functions}

Stable functions are an antecedent of Galois slicing that arose in
domain theory as an attempt to capture the idea that any function
describing a computational process must explore its input in a way
that allows identification of minimal parts of the input that are
needed for chosen approximations of the output.

Stability was invented by \citet{berry79} in an attempt to capture
sequentiality, and was also instrumental in the discovery of Linear
Logic via Coherence Spaces \cite{girard}. A textbook description of
stable functions in the context of domain theory is given by
\citet{amadio-curien} (Chapter 12). Here, we are not
concerned with completeness properties considered in domain theory, so
we start with only partially ordered sets. Stable functions can be
defined as a certain subset of the set of monotone functions between
two partially ordered sets:

\begin{definition}
  Let $f : X \to Y$ be a monotone function between posets $X$ and
  $Y$. The function $f$ is \emph{stable} if for all $x \in X$ and
  $y \leq f(x)$:
  \begin{enumerate}
  \item (\textsc{Existence}) there exists an $x_0 \leq x$ such that $y \leq f(x_0)$, and
  \item (\textsc{Minimality}) for any $x'_0 \leq x$ such that $y \leq f(x'_0)$ then
    $x_0 \leq x'_0$ .
  \end{enumerate}
\end{definition}

\begin{lemma}
  Stable functions are closed under identities and composition.
\end{lemma}

\begin{example}[Stability reveals intensional information]
  To see how stability works, consider the following examples of the
  OR operation on the lifted booleans $\mathbb{B}_\bot$. Two functions
  that are stable are the strict and left-short-circuiting
  ORs\footnote{The clauses in these examples are shorthand for the
    graph of the function. They are not to be understood as pattern
    matching clauses in a language like Haskell, where it is not
    possible to match on $\bot$.}:
  \begin{displaymath}
    \begin{array}[t]{l@{(}l@{,~}l@{)~}c@{~}l}
      \mathrm{strictOr}&\mathsf{tt}&\mathsf{tt}&=&\mathsf{tt} \\
      \mathrm{strictOr}&\mathsf{tt}&\mathsf{ff}&=&\mathsf{tt} \\
      \mathrm{strictOr}&\mathsf{ff}&\mathsf{tt}&=&\mathsf{tt} \\
      \mathrm{strictOr}&\mathsf{ff}&\mathsf{ff}&=&\mathsf{ff} \\
      \mathrm{strictOr}&\bot&\_&=&\bot \\
      \mathrm{strictOr}&\_&\bot&=&\bot \\
    \end{array}
    \qquad
    \begin{array}[t]{l@{(}l@{,~}l@{)~}c@{~}l}
      \mathrm{shortCircuitOR}&\mathsf{tt}&\_&=&\mathsf{tt} \\
      \mathsf{shortCircuitOR}&\mathsf{ff}&x&=&x \\
      \mathsf{shortCircuitOR}&\bot&\_& =& \bot
    \end{array}
  \end{displaymath}
  The function $\mathrm{strictOr}$ is stable. For example, for the
  input-output pair $(\mathsf{tt},\mathsf{ff}), \mathsf{tt}$, the
  minimal input that gives this output is exactly
  $(\mathsf{tt}, \mathsf{ff})$. If we take the approximation
  $\bot \leq \mathsf{tt}$ of the output, then the corresponding
  minimal input is $(\bot, \bot)$. The function
  $\mathrm{shortCircuitOR}$ is also stable. For the input-output pair
  $(\mathsf{tt},\mathsf{ff}),\mathsf{tt}$, the minimal input that
  gives this input is $(\mathsf{tt},\bot)$, indicating that the
  presence of $\mathsf{ff}$ in the second argument was not necessary
  to produce this output. As with $\mathrm{strictOr}$, the minimal
  input required to produce the output $\bot \leq \mathsf{tt}$ is
  again $(\bot,\bot)$.

  The fact that these two functions's stability witnesses reveal
  information about how they depend on their input is what we will
  exploit in order to use the idea of stability for slicing.
\end{example}

\begin{example}[Functions that are not stable]
  A function that is not stable is Plotkin's Parallel OR~\cite{lcf77},
  which short-circuits in both arguments, returning $\mathsf{tt}$ if
  either argment is $\mathsf{tt}$, even if the other argument is not
  defined:
  \begin{displaymath}
    \begin{array}{l@{(}l@{,~}l@{)~}c@{~}l}
      \mathrm{parallelOR}&\mathsf{tt}&\_&=&\mathsf{tt} \\
      \mathrm{parallelOR}&\_&\mathsf{tt}&=&\mathsf{tt} \\
      \mathsf{parallelOR}&\mathsf{ff}&\mathsf{ff}&=&\mathsf{ff} \\
      \mathsf{parallelOR}&\bot&\bot&=&\bot
    \end{array}
  \end{displaymath}
  For the input-output pair $(\mathsf{tt},\mathsf{tt}),\mathsf{tt}$,
  there is no one minimal input that produces this output. We have
  both $\mathrm{parallelOR}(\mathsf{tt},\bot) = \mathsf{tt}$ and
  $\mathrm{parallelOR}(\bot,\mathsf{tt}) = \mathsf{tt}$, which are
  incomparable and their greatest lower bound $(\bot,\bot)$ gives the
  output $\bot$.

  Parallel OR is famous because it is not \emph{sequential}, meaning
  intuitively that it cannot be implemented without running the two
  arguments in parallel to see if one of them returns
  $\mathsf{tt}$. That fact that it exists in the standard domain
  theoretic semantics of PCF means that this semantics is not fully
  abstract. Since Parallel OR is not stable, one might hope that
  stability is enough to capture sequentiality, and hence potentially
  give a fully abstract model of PCF. However, the following ternary
  function $\mathbb{B}_\bot^3 \to \{\top,\bot\}$ is stable but admits
  no sequential implementation that fixes an order that the arguments
  are examined in:
  \begin{displaymath}
    \begin{array}{l@{(}l@{,~}l@{,~}l@{)~}c@{~}l}
      \mathrm{gustave}&\mathsf{tt}&\mathsf{ff}&\_&=&\top \\
      \mathrm{gustave}&\mathsf{ff}&\_&\mathsf{tt}&=&\top \\
      \mathrm{gustave}&\_&\mathsf{tt}&\mathsf{ff}&=&\top \\
      \mathrm{gustave}&\_&\_&\_&=&\bot
    \end{array}
  \end{displaymath}
  Despite there being no one minimal input that achieves the output
  $\top$, each of the minimal inputs that can achieve this output are
  pairwise incomparable, so for each specific input that gets output
  $\top$ there is a unique minimal input that achieves it (listed in
  the first three lines of the definition).

  In terms of program slicing, however, the $\mathrm{gustave}$
  function does not present a problem. For any particular run (i.e.,
  input-output pair) of the program, there is an unambiguous minimal
  input that achieves the output, no matter that it was not achieved
  by a sequential processing of the input.
\end{example}

The definition of stability has a nice operational meaning, but seems
a little ad-hoc. We can rephrase stability in terms of Galois
connections / adjunctions between principal downsets / slices of the
partial orders. The alternative presentation also has the advantage
that it explicitly defines the stability witness as ``backwards''
function from approximations of the output to approximations of the
input.

\newcommand{\downset}[1]{\mathop{\downarrow}(#1)}

\begin{definition}
  For a poset $X$ and $x \in X$, the \emph{principal
    downset of $x$} is $\downset{x} = \{ x' \in X \mid x' \leq x \}$.
\end{definition}

\begin{lemma}
  If $f : X \to Y$ is monotonic, then for all $x \in X$, the
  restriction of $f$ to $\downset{x}$ is a monotone function
  $f_x : \downset{x} \to \downset{f(x)}$.
\end{lemma}

\begin{proof}
  $f_x$ is well defined: for any $x' \leq x$,
  $f_x(x') = f(x') \leq f(x)$ by monotonicity, so
  $f_x(x') \in \downset{f(x)}$. The function $f_x$ is monotone because
  $f$ is.
\end{proof}

\begin{lemma}
  A monotone function $f : X \to Y$ is stable if and only if for all
  $x \in X$, the restriction of $f_x : \downset{x} \to \downset{f(x)}$
  has a left adjoint.
\end{lemma}

\begin{proof}
  If $f$ is stable, then define a left adjoint
  $g_x : \downset{f(x)} \to \downset{x}$ by setting $g_x(y)$ to be the
  minimal $x_0$ required by stability. This is monotone: if
  $y \leq y'$, then we know that $y \leq y' \leq f(g_x(y'))$ by
  definition of $g_x$, so $g_x(y) \leq g_x(y')$ by minimality of
  $g_x(y)$. For the adjointness, let $x' \leq x$ and $y \leq
  f(x)$. Then if $g_x(y) \leq x'$, we have
  $y \leq f(g_x(y)) \leq f(x')$ by monotonicity of $f$ and the first
  part of stability. In the other direction, if we have
  $y \leq f(x')$, then by uniqueness we have $g_x(y) \leq x'$.

  If, for every $x$, $f_x$ has a left adjoint $g_x$, then for any
  $x', y$ we have $y \leq f_x(x') \Leftrightarrow g_x(y) \leq x'$. So
  $g_x(y)$ is the element that satisfies $y \leq f(g_x(y))$, and it is
  minimal since if $y \leq f_x(x'_0)$ then $g_x(y) \leq x'_0$.
\end{proof}

For what follows\footnote{Need a better way of motivating this.}, it
will be useful for the principal downsets to be bounded lattices, so
that we are guaranteed to have meets and joints of approximations of
points. Note that the principal downsets always have a top element
(the original element), but now we are also requiring a bottom
element, as well as finite meets and joins.

FIXME: is this a good name? Maybe just $\mathbf{Stable}$ or something
like that?

FIXME: if $x \leq x'$ then $\downset{x} \subseteq \downset{x'}$ and I
assume that the inclusion preserves lattice structure?

\begin{definition}
  An \emph{approximation set} is a partial order such that every
  principal downset $\downset{x}$ is a bounded lattice.

  A morphism of approximation sets is monotone function whose
  restriction to each slice has a left adjoint.
\end{definition}

Note that, by standard properties of Galois connections/adjoints, the
pairs $g_x \vdash f_x$ preserve the finite joins and meets respectively.

\begin{proposition}
  Approximation sets and their morphisms form a category.
\end{proposition}

FIXME: what properties does this category have? I think it has finite
products, and probably finite coproducts?

FIXME: there is a full and faithful functor from approximation sets to
$\Fam(\LatGal)$.

\subsection{Commutative monoids}
\label{sec:cmon}

\begin{definition}[Commutative monoid]
A \emph{commutative monoid} $X = (X, \bullet, \varepsilon)$ is a set $X$ equipped with distinguished element
$\varepsilon \in X$ called the \emph{unit} and associative binary operation $\bullet: X^2 \to X$ satisfying
$\varepsilon \bullet x = x$ and $x \bullet \varepsilon = x$ for any $x \in X$.
\end{definition}

A commutative monoid homomorphism from $X$ to $Y$ is any function $f: X \to Y$ preserving $\varepsilon$ and
$\bullet$.

\subsubsection{Category of commutative monoids}

\begin{definition}[Category $\CMon$]
\label{def:cmon:cmon}
Define $\CMon$ to be the category which has as objects $X$ all commutative monoids and as morphisms $f: X \to
Y$ all commutative monoid homomorphisms.
\end{definition}

$\CMon$ is complete and cocomplete, inheriting all limits and colimits from $\Set$. $\CMon$ is also monoidal
closed. \todo{But the monoidal product is not the Cartesian product?}

\subsubsection{Example of commutative monoid}

\begin{definition}[Bounded semilattice]
\label{def:cmon-enriched:bounded-semilattice}
A \emph{bounded semilattice} $X = (X, \bullet, \varepsilon)$ is a commutative monoid where $\bullet$ is
\emph{idempotent}, i.e.~satisfies $x \bullet x = x$.
\end{definition}

\noindent The idempotence of $\bullet$ (together with commutativity and associativity) induces a partial order
$\le_\bullet$ on $X$, with $x \le_{\bullet} y \iff x \bullet y = x$. With respect to this partial order, $x
\bullet y$ is the greatest lower bound (meet) of $x$ and $y$ and $\varepsilon$ is the top element; with
respect to the opposite order, $x \bullet y$ is the least upper bound (join) of $x$ and $y$ and $\varepsilon$
is the bottom element. This therefore provides an algebraic characterisation of the usual (dual)
order-theoretic notions of bounded meet semilattices $(X, \meet, \top)$ and bounded join semilattices $(X,
\join, \bot)$.

\begin{definition}[Category $\SemiLat$]
Define $\SemiLat$ to be the category which has as objects all bounded semilattices and as morphisms all
bounded semilattice homomorphisms.
\end{definition}

\subsection{$\CMon$-enriched categories}
\label{sec:cmon-enriched}

Recall that if a category $\cat{C}$ is $\CMon$-enriched, then:
\begin{enumerate}
\item Every hom-object $\Hom{\cat{C}}{X}{Y}$ is a commutative monoid of morphisms; we write $\zero_{X,Y}$
(zero morphism) for the unit, and $+_{X,Y}$ (addition of morphisms) for the binary operation (omitting the
indices where implied by the context).
\item Composition is \emph{bilinear}, i.e.~given by a family of morphisms $\Hom{\cat{C}}{Y}{Z} \tensor
\Hom{\cat{C}}{X}{Y} \to \Hom{\cat{C}}{X}{Z}$ in $\CMon$ that preserve the additive structure in
$\Hom{\cat{C}}{Y}{Z}$ and $\Hom{\cat{C}}{X}{Y}$ separately:

\begin{salign*}
f \comp \zero = f = \zero \comp f
\end{salign*}
\begin{salign*}
(f + g) \comp h &= (f \comp h) + (g \comp h) \\
h \comp (f + g) &= (h \comp f) + (h \comp g)
\end{salign*}
\end{enumerate}

Because $\CMon$ is monoidal closed, it is enriched over itself; every hom-object $\Hom{\CMon}{X}{Y}$ is a
commutative monoid with:

\begin{enumerate}
\item unit $0_{X,Y} = \const(\varepsilon_Y)$, the constant homomorphism sending every element of $X$ to
$\varepsilon_Y$;
\item binary operation $+_{X,Y}$ given by pointwise addition of homomorphisms $(f + g)(x) = f(x) + g(x)$.
\end{enumerate}

Suppose $F, G: \cat{D} \to \cat{C}$ are functors where $\cat{C}$ is $\CMon$-enriched. Then the hom-object
$\Hom{\Func{\cat{C}}{\cat{D}}}{F}{G}$ of natural transformations between $F$ and $G$ has a zero $0_{F,G}$ and
addition $\eta + \mu$ given component-wise as $(0_{F,G})_X = 0_{F(X),G(X)}$ and $(\eta + \mu)_X = \eta_X +
\mu_X$.

\begin{proposition}
If $C$ is $\CMon$-enriched then any functor category $\Func{\cat{D}}{\cat{C}}$ is $\CMon$-enriched.
\end{proposition}

\subsubsection{$\CMon$-enriched presheaves}

In the specific setting of a $\CMon$-enriched presheaf $F: \cat{C}^\op \to \CMon$, enrichment means that $F$
preserves the additive structure of morphisms:
\begin{itemize}
\item $F(0_{X,Y}) = 0_{F(Y),F(X)}$
\item $F(f + g) = F(f) + F(g)$
\end{itemize}
It then follows automatically that $F$ also preserves bilinear composition, i.e.:
\begin{itemize}
\item $F(f) \comp F(0) = F(f) = F(0) \comp F(f)$
\item $F(f + g) \comp F(h) = F(h \comp f) + F(h \comp g)$
\item $F(h) \comp F(f + g) = F(f \comp h) + F(g \comp h)$
\end{itemize}
in addition to the usual enriched functor properties from \defref{cmon-enriched:enriched-functor}.

\subsubsection{Enriched Yoneda embedding}

Suppose a monoidal category $\mathscr{V}$ and a $\mathscr{V}$-enriched category $\cat{C}$, again restricted to
the case where $\mathscr{V}$ is concrete.

\begin{definition}[Hom-functor]
Define the contravariant \emph{hom-functor} $\Hom{\cat{C}}{-}{X}: \cat{C}^\op \to \mathscr{V}$ sending
\begin{itemize}
\item any object $Y$ in $\cat{C}$ to the hom-object $\Hom{\cat{C}}{Y}{X}$ in $\mathscr{V}$;
\item any morphism $f: Y \to Z$ in $\cat{C}$ to the function $\Hom{\cat{C}}{f}{X} = (- \comp f):
\Hom{\cat{C}}{Z}{X} \to \Hom{\cat{C}}{Y}{X}$ on the underlying hom-sets.
\end{itemize}
\end{definition}

\begin{definition}[Yoneda embedding]
The \emph{Yoneda embedding} for $\cat{C}$ is the functor $\Yoneda: \cat{C} \to
\Func{\cat{C}^\op}{\mathscr{V}}$ sending
\begin{itemize}
\item any object $X$ to the hom-functor $\Hom{\cat{C}}{-}{X}: \cat{C}^\op \to \mathscr{V}$;
\item any morphism $f: X \to Y$ to the $\mathscr{V}$-natural transformation $\Hom{\cat{C}}{-}{X} \naturalto
\Hom{\cat{C}}{-}{Y}$ where $\Yoneda(f)_Z = (f \comp -): \Hom{\cat{C}}{Z}{X} \to \Hom{\cat{C}}{Z}{Y}$.
\end{itemize}
\end{definition}

\noindent Now suppose $\mathscr{V}$ is monoidal closed. Then the hom-functor $\Hom{\cat{C}}{-}{X}: \cat{C}^\op
\to \mathscr{V}$ is itself $\mathscr{V}$-enriched, providing a family of morphisms $\Hom{\cat{C}}{-}{X}_{Y,Z}:
\Hom{\cat{C}}{Y}{Z} \to \Hom{\mathscr{V}}{\Hom{\cat{C}}{Z}{X}}{\Hom{\cat{C}}{Y}{X}}$ in $\mathscr{V}$ for any
$Y, Z$ in $\cat{C}$. Moreover there exists an enriched Yoneda embedding, namely a functor
$\Yoneda_{\mathscr{V}}: \cat{C} \to \PSh_{\mathscr{V}}(\cat{C})$ which is itself $\mathscr{V}$-enriched,
sending objects $X$ to the $\mathscr{V}$-enriched hom-functor $\Hom{\cat{C}}{-}{X}$ and providing a family of
morphisms $(\Yoneda_{\mathscr{V}})_{X,Y}: \Hom{\cat{C}}{X}{Y} \to
\Hom{\PSh_{\mathscr{V}}(\cat{C})}{\Hom{\cat{C}}{-}{X}}{\Hom{\cat{C}}{-}{Y}}$ acting concretely at each
component as post-composition with $f: X \to Y$.

The $\CMon$-enriched Yoneda embedding \todo{think this works for any monoidal closed $\mathscr{V} with
biproducts?$} preserves biproducts, where (as in any functor category) biproducts in $\PSh_{\CMon}(\cat{C})$
are computed pointwise, and noting that $\CMon$ itself has biproducts.

\begin{proposition}
$\Yoneda_{\CMon}(X \biprod Y) \iso \Yoneda_{\CMon}(X) \biprod \Yoneda_{\CMon}(Y)$.
\end{proposition}

\begin{proof}
Suppose $Z$ an object in $C$. Then $\Yoneda_{\CMon}(X \biprod Y)(Z) = \Hom{\cat{C}}{Z}{X \biprod Y}$. But for
any $f: Z \to X \biprod Y$ in $\cat{C}$, there exist unique morphisms $f_1: Z \to X$ and $f_2: Z \to Y$ such
that $f = \biinj_X \comp f_1 + \biinj_Y \comp f_2$ by the universal property of the biproduct. Thus we have
$\Hom{\cat{C}}{Z}{X \biprod Y} \iso \Hom{\cat{C}}{Z}{X} \biprod \Hom{\cat{C}}{Z}{Y} = \Yoneda_{\CMon}(X)(Z)
\biprod \Yoneda_{\CMon}(Y)(Z) = (\Yoneda_{\CMon}(X) \biprod \Yoneda_{\CMon}(Y))(Z)$.
\end{proof}

The \emph{enriched Yoneda lemma} states that, for any $\mathscr{V}$-enriched presheaf $F$ and any object $X
\in \cat{C}$, there is an isomorphism between $F(X)$ and the hom-object (also living in $\mathscr{V}$) of
$\mathscr{V}$-natural transformations from the $\mathscr{V}$-enriched presheaf $\Hom{\cat{C}}{-}{X}$ to $F$:

\begin{lemma}
Suppose $\cat{C}$ is $\mathscr{V}$-enriched with $\mathscr{V}$ monoidal closed. For any $F: \cat{C}^\op \to
\mathscr{V}$ and any object $X$ in $\cat{C}$:
\[\Hom{\PSh_{\mathscr{V}}{(\cat{C})}}{\Hom{\cat{C}}{-}{X}}{F} \iso F(X)\]
\end{lemma}


% The following diagram commutes because precomposition with the identity is the identity on morphisms:
%
% \begin{center}
% \begin{tikzcd}[column sep=1.8cm]
% I \arrow[d, equals] \arrow[r, "\id_X"] & \Hom{\cat{C}}{X}{X} \arrow[d, "\Hom{\cat{C}}{-}{W}_{X,X}"] \\
% I \arrow[r, "\id_{\Hom{\cat{C}}{X}{W}}"'] & \Hom{\mathscr{V}}{\Hom{\cat{C}}{X}{W}}{\Hom{\cat{C}}{X}{W}}\
% \end{tikzcd}
% \end{center}
%
% \noindent and the following commutes by the associativity of composition:
%
% \begin{center}
% \begin{tikzcd}[column sep=2.7cm]
%    \Hom{\cat{C}}{Z}{Y} \tensor \Hom{\cat{C}}{Y}{X} \arrow[r, "\comp_{X,Y,Z}"] \arrow[d,
%    "\Hom{\cat{C}}{-}{W}_{Y,Z}\,\tensor\,\Hom{\cat{C}}{-}{W}_{X,Y}"'] & \Hom{\cat{C}}{Z}{X} \arrow[d,
%    "\Hom{\cat{C}}{-}{W}_{X,Z}"] \\
%    \Hom{\mathscr{V}}{\Hom{\cat{C}}{Y}{W}}{\Hom{\cat{C}}{Z}{W}} \tensor
%    \Hom{\mathscr{V}}{\Hom{\cat{C}}{X}{W}}{\Hom{\cat{C}}{Y}{W}} \arrow[r,
%    "\comp_{\Hom{\cat{C}}{X}{W},\Hom{\cat{C}}{Y}{W},\Hom{\cat{C}}{Z}{W}}"'] &
%    \Hom{\mathscr{V}}{\Hom{\cat{C}}{X}{W}}{\Hom{\cat{C}}{Z}{W}}
% \end{tikzcd}
% \end{center}

\subsection{Biproducts}
\label{sec:biproduct}

\begin{definition}[Zero object]
A \emph{zero} object is an object which is both terminal and initial.
\end{definition}

\begin{definition}[Biproduct]
Suppose $\cat{C}$ a $\CMon$-enriched category. For any objects $X, Y$, the \emph{biproduct} of $X$ and $Y$ is
an object $X \biprod Y$ of $C$ together with morphisms

\begin{center}
\begin{tikzcd}
   X \arrow[r, "\biinj_X", shift left] &
   X \biprod Y \arrow[l, "\biproj_X", shift left] \arrow[r, "\biproj_Y"', shift right] &
   Y \arrow[l, "\biinj_Y"', shift right]
\end{tikzcd}
\end{center}

\noindent satisfying

\begin{minipage}[t]{0.45\textwidth}
\begin{center}
\begin{salign*}
   \biproj_X \comp \biinj_X &= \id_X \\
   \biproj_Y \comp \biinj_X &= \zero_{X,Y}
\end{salign*}
\end{center}
\end{minipage}%
\begin{minipage}[t]{0.45\textwidth}
\begin{center}
\begin{salign*}
   \biproj_Y \comp \biinj_Y &= \id_Y \\
   \biproj_X \comp \biinj_Y &= \zero_{Y,X}
\end{salign*}
\end{center}
\end{minipage}

\noindent and

\begin{salign*}
(\biinj_X \comp \biproj_X) + (\biinj_Y \comp \biproj_Y) &= \id_{X \biprod Y}
\end{salign*}
\end{definition}

We call a $\CMon$-enriched $\cat{C}$ \emph{semi-additive} if it also has finite biproducts; having products or
coproducts is sufficient.

\begin{proposition}
\label{prop:biproduct:from-product-or-coproduct}
Suppose $\cat{C}$ is $\CMon$-enriched.
\begin{itemize}
\item If $\cat{C}$ has biproducts $(X \biprod Y, \biinj_X, \biinj_Y, \biproj_X, \biproj_Y)$ then $(X
\biprod Y, \biproj_X, \biproj_Y)$ is a product and $(X \biprod Y, \biinj_X, \biinj_Y)$ is a coproduct.
\item If $\cat{C}$ has binary products $(X \times Y, \proj_1, \proj_2)$ then:
\begin{enumerate}
\item $\prodM{0_{X,Y}}{0_{X,Z}} = 0_{X,Y \times Z}$
\item $\prodM{f_1 + g_1}{f_2 + g_2} = \prodM{f_1}{f_2} + \prodM{g_1}{g_2}$
\item $(X \times Y, \biinj_X, \biinj_Y, \proj_1, \proj_2)$ is a biproduct with $\biinj_X =
\prodM{\id_X}{0_{X,Y}}: X \to X \times Y$ and $\biinj_Y = \prodM{0_{X,Y}}{\id_Y}: Y \to X \times Y$.
\end{enumerate}
\item If $\cat{C}$ has binary coproducts $(X + Y, \inj_1, \inj_2)$ then
\begin{enumerate}
\item $\coprodM{0_{X,Z}}{0_{Y,Z}} = 0_{X + Y,Z}$
\item $\coprodM{f_1 + g_1}{f_2 + g_2} = \coprodM{f_1}{f_2} + \coprodM{g_1}{g_2}$
\item $(X + Y, \inj_1, \inj_2, \biproj_X, \biproj_Y)$ is a biproduct with $\biproj_X: X + Y \to X =
\coprodM{\id_X}{0_{Y,X}}$ and $\biproj_Y: X + Y \to Y = \coprodM{0_{X,Y}}{\id_Y}$.
\end{enumerate}
\item If $\cat{C}$ has a terminal (or initial) object then it is a zero object.
\end{itemize}
\end{proposition}

\begin{proposition}
\label{prop:biproduct:prod-coprod}
Suppose $\cat{C}$ a category with biproducts. Then the following diagram in $\cat{C}$ commutes:

\begin{center}
\begin{tikzcd}[row sep=3em, column sep=3em]
   & Y \arrow[d, "\biinj_Y", shift left] \arrow[dr, "f_2"] \\
   X \arrow[r, "\prodM{f_1}{g_1}"] \arrow[dr, "g_1"'] \arrow[ur, "f_1"] & Y \biprod Z \arrow[u, "\biproj_Y", shift left] \arrow[d, "\biproj_Z"', shift right] \arrow[r, pos=0.35, "\coprodM{f_2}{g_2}"] & W \\
   & B \arrow[u, "\biinj_Z"', shift right] \arrow[ur, "g_2"']
\end{tikzcd}
\end{center}
\end{proposition}

\input{notes/useful-semi-additive-categories}
\subsection{Category of $I$-indexed families}
\label{sec:fam}

\begin{definition}[Category of $I$-indexed families of objects]
\label{def:fam:Fam}
For any set $I$ and any category $\cat{C}$ write $\Fam(I,\cat{C})$ for the category which has:
\begin{itemize}
\item as objects all $I$-indexed families of objects of $\cat{C}$;
\item as morphisms from $X$ to $Y$ all families of morphisms $f_i: X_i \to Y_i$ in $\cat{C}$ for any $i \in
I$.
\end{itemize}
\end{definition}

\noindent Equivalently $\Fam(I,\cat{C})$ is the functor category $\Func{I}{\cat{C}}$, interpreting $I$ as a
discrete category.

\begin{definition}[Constant family]
For any object $X$ of $\cat{C}$ write $\const_I(X)$ for the constant family $\{X\}_{i \in I}$.
\end{definition}

\begin{definition}[Reindexing]
For any $f: I \to J$ define the \emph{reindexing} functor $\reindex{-}{f}: \Fam(J,\cat{C}) \to
\Fam(I,\cat{C})$ which:
\begin{itemize}
\item sends any $J$-indexed family $X$ to the $I$-indexed family $\reindex{X}{f}$ where $\reindex{X}{f}_i =
X_{f(i)}$ for any $i \in I$;
\item sends any morphism $g: X \to Y$ between $J$-indexed families to the morphism $\reindex{g}{f}:
\reindex{X}{f} \to \reindex{Y}{f}$ where $\reindex{g}{f}_i = g_{f(i)}$.
\end{itemize}
\end{definition}

By \emph{(strictly) indexed category over $\cat{C}$} we shall mean any (strict) functor $\cat{C}^\op \to
\Cat$.

\begin{definition}
Define $\Fam(-,\cat{C}): \Set^{\op} \to \Cat$ to be the indexed category which sends any set $I$ to
$\Fam(I,\cat{C})$ and any function $f: I \to J$ to the functor $\reindex{-}{f}: \Fam(J,\cat{C}) \to
\Fam(I,\cat{C})$.
\end{definition}

\begin{proposition}
If $\cat{C}$ has limits (resp.~colimits) then $\Fam(I,\cat{C})$ has limits (resp.~colimits) computed
pointwise.
\end{proposition}

\subsection{Set-indexed products}
\label{sec:set-indexed-product}

$\cat{C}$ \emph{has set-indexed products} if, for any set $I$ and any $I$-indexed family $X$ of objects of
$\cat{C}$, there exists an object $\prod_{i \in I}X_i$ (also written $\prod_{I} X$) and family of morphisms
$\eval_i$ in $\cat{C}$ such that for any object $Y$ and any family of morphisms $\{f_i: Y \to X_i\}_{i \in I}$
in $\cat{C}$, there exists a unique morphism $\lambda(f)$ making the following diagram commute for every $i
\in I$:

\begin{center}
\begin{tikzcd}
   \prod_{I} X \arrow[r, "\eval_i"] & X_i \\
   Y \arrow[ru, "f_i"'] \arrow[u, dotted, "\lambda(f)"]
\end{tikzcd}
\end{center}

Note that $f: \const_{I}(Y) \to X$ and $\eval: \const_{I}(\prod_{I} X) \to X$ are also morphisms in $\Fam(I,\cat{C})$.

\subsection{Category of families}
\label{sec:grothendieck}

\subsubsection{Grothendieck construction}

\begin{definition}[Grothendieck construction for $F$]
\label{def:grothendieck:Grothendieck}
Suppose an indexed category $T: \cat{C}^\op \to \Cat$. The \emph{Grothendieck construction} for $T$ is the
category $\Grothendieck{X}T(X)$ (also written $\Grothendieck{}T$) which has:
\begin{itemize}
\item as objects, all pairs $(X, \partial X)$ of an object $X$ of $\cat{C}$ and an object $\partial X$ of
$T(X)$;
\item as morphisms $(X, \partial X) \to (Y, \partial Y)$, all pairs $(f, \partial f)$ of morphisms $f: X \to
Y$ in $\cat{C}$ and morphisms $\partial f: \partial X \to T(f)(\partial Y)$ in $T(X)$.
\end{itemize}
\end{definition}

\noindent The $\partial X$ notation is justified by the fact that the Grothendieck construction formalises the
idea of tangent spaces and derivatives.

For any $(f, \partial f): (X, \partial X) \to (Y, \partial Y)$ and $(g, \partial g): (Y, \partial Y)
\to (Z, \partial Z)$ in $\Grothendieck{}T$, we have:

\begin{itemize}
\item $\partial f: \partial X \to T(f)(\partial Y)$;
\item $T(f)(\partial g): T(f)(\partial Y) \to T(f)(T(g)(\partial Z)) = T(g \comp f)(\partial Z)$.
\end{itemize}

\noindent The composition $(g, \partial g) \comp (f, \partial f): (X, \partial X) \to (Z, \partial Z)$ is
given by $(g \comp f, T(f)(\partial g) \comp \partial f)$.

\subsubsection{Category of families}

\begin{definition}[Category of families]
For any category $\cat{C}$ define $\Fam(\cat{C})$ to be the Grothendieck construction
$\Grothendieck{}\Fam(-,\cat{C})$.
\end{definition}

\noindent In $\Fam(\cat{C})$ the objects $(X, \partial X)$ are thus sets $X$ paired with indexed families
$\partial X$ in $\Fam(X,\cat{C})$ and morphisms $(f, \partial f): (X, \partial X) \to (Y, \partial Y)$ are
functions $f: X \to Y$ paired with morphisms $\partial f: \partial X \to \reindex{\partial Y}{f}$ in
$\Fam(X,\cat{C})$.

\begin{proposition}
\item
\begin{enumerate}
\item If $\cat{C}$ is locally small then so is $\Fam(\cat{C})$.
\item If $\cat{C}$ has binary products then so does $\Fam(\cat{C})$.
\end{enumerate}
\end{proposition}

The following is a special case of a construction due to \citet{nunes2023}:

\begin{proposition}
Suppose $\cat{C}$ locally small. If $\cat{C}$ has binary biproducts and set-indexed products, then
$\Fam(\cat{C})$ is Cartesian closed.
\end{proposition}

\begin{proof}
Suppose $\cat{C}$ is locally small, with binary biproducts given by $(\biprod,\biinj,\biproj)$ and set-indexed
products given by $(\prod,\eval_{\prod},\lambda_{\prod})$. Write $(\times,\pi_1,\pi_2)$ for the product in
$\Fam(\cat{C})$ inherited from the biproduct in $\cat{C}$. For any objects $X, Y, Z$ in $\Fam(\cat{C})$
define:

\begin{enumerate}
\item $[X, Y]$ to be the object $(X',Y')$ in $\Fam(\cat{C})$ with $X' = \Fam(\cat{C})(X,Y)$ and $Y'_f =
\prod_{\idx{X}}\reindex{\faml{Y}}{\idx{f}}$ for every $f: X \to Y$.
\item The family of isomorphisms $\lambda_{X,Y,Z}$ (natural in $X$) sending any $(f, g): X \times Y \to Z$ to
$(f', g'): X \to [Y, Z]$ with:
\begin{itemize}
\item $f'(x) = (f \comp (x, -), \reindex{g}{(x,-)} \comp \biinj_{\faml{Y}}):
Y \to Z$ where $\biinj_{\faml{Y}}: \faml{Y} \to \const_{\idx{Y}}(\faml{X}_x) \biprod \faml{Y}$
\item $g'_x = \lambda_{\prod_{\idx{Y}}}(\reindex{g}{(x,-)} \comp \biinj_{\faml{X}_x}:
\const_{\idx{Y}}(\faml{X}_x) \to \reindex{\faml{Z}}{f})$ where $\biinj_{\faml{X}_x}: \const_{\idx{Y}}(\faml{X}_x)
\to \const_{\idx{Y}}(\faml{X}_x) \biprod \faml{Y}$
\end{itemize}
Here $(x,-): \idx{Y} \to \idx{X} \times \idx{Y}$ denotes the function $\prodM{\const(x)}{\id_{\idx{Y}}}$ which
pairs any $y \in \idx{Y}$ with $x \in \idx{X}$.
\item The family of morphisms $\eval_{X,Y}: [X, Y] \times X \to Y$ in $\Fam(\cat{C})$ with
\begin{itemize}
\item $\idx{\eval_{X,Y}}(f, x) = \idx{f}(x)$
\item $\faml{\eval_{X,Y}}_{f, x} = \coprodM{{\eval_{\prod}}_x}{\faml{f}_x}:
\prod_{\idx{X}}\reindex{\faml{Y}}{\idx{f}} \biprod \faml{X}_x \to \faml{Y}_{\idx{f}(x)}$
\end{itemize}
\end{enumerate}

Then the following triangle identities hold.
\begin{enumerate}
\item $\lambda_{X,Y,Z}(\eval_{Y,Z} \comp (f, \id_Y)) = f$. Suppose $(f,g): X \to [Y,Z]$ and define
$(f^\dagger, g^\dagger) = \eval_{Y,Z} \comp ((f,g), \id_Y): X \times Y \to Z$. We then have:
\begin{itemize}
\item $f^\dagger(x,y) = (\idx{\eval_{Y,Z}} \comp (f, \id_{\idx{Y}}))(x,y) = \idx{f(x)}(y)$
\item $g^\dagger_{x,y} = (\faml{\eval_{Y,Z}} \comp (g, \id_{\faml{Y}}))_{x,y} = \faml{\eval_{Y,Z}}_{f(x), y}
\comp (g, \id_{\faml{Y}})_{x,y} = \coprodM{{\eval_{\prod}}_y}{\faml{f(x)}_y} \comp (g_x, {\id_{\faml{Y}}}_y) =
\coprodM{{\eval_{\prod}}_y \comp g_x}{\faml{f(x)}_y}$.
\end{itemize}

To show $\lambda_{X,Y,Z}(f^\dagger, g^\dagger) = (f,g)$, suppose $\lambda_{X,Y,Z}(f^\dagger, g^\dagger) = (f',
g')$. Then:
\begin{itemize}
\item $f'(x) = (f^\dagger \comp (x, -), \reindex{g^\dagger}{(x,-)} \comp \biinj_{\faml{Y}}): Y \to Z$ where
$\biinj_{\faml{Y}}: \faml{Y} \to \const_{\idx{Y}}(\faml{X}_x) \biprod \faml{Y}$
\item $g'_x = \lambda_{\prod_{\idx{Y}}}(\reindex{g^\dagger}{(x,-)} \comp \biinj_{\faml{X}_x}:
\const_{\idx{Y}}(\faml{X}_x) \to \reindex{\faml{Z}}{f^\dagger})$ where $\biinj_{\faml{X}_x}:
\const_{\idx{Y}}(\faml{X}_x) \to \const_{\idx{Y}}(\faml{X}_x) \biprod \faml{Y}$.
\end{itemize}
\noindent To show $f' = f : \idx{X} \to \Fam(\cat{C})(Y,Z)$, suppose $x \in \idx{X}$ and show $f'(x) = f(x): Y
\to Z$. Suppose $y \in \idx{Y}$.
\begin{enumerate}
\item $\idx{f'(x)}(y) = (f^\dagger \comp (x, -))(y) = f^\dagger(x,y) = \idx{f(x)}(y)$.
\item $\faml{f'(x)}_y = (\reindex{g^\dagger}{(x,-)} \comp \biinj_{\faml{Y}})_y = \reindex{g^\dagger}{(x,-)}_y \comp
{\biinj_{\faml{Y}}}_y = g^\dagger_{x,y} \comp {\biinj_{\faml{Y}}}_y = \coprodM{{\eval_{\prod}}_y \comp
g_x}{\faml{f(x)}_y} \comp {\biinj_{\faml{Y}}}_y = \faml{f(x)}_y$.
\end{enumerate}

To show $g' = g$, suppose $x \in \idx{X}$.
\begin{align*}
\small
g'_x
&=
\lambda_{\prod_{\idx{Y}}}(\reindex{g^\dagger}{(x,-)} \comp \biinj_{\faml{X}_x}) \\
&=
\lambda_{\prod_{\idx{Y}}}(\{g^\dagger_{(x,y)} \comp (\biinj_{\faml{X}_x})_y\}_{y \in \idx{Y}}) \\
&=
\lambda_{\prod_{\idx{Y}}}(\{\coprodM{{\eval_{\prod}}_y \comp g_x}{\faml{f(x)}_y} \comp (\biinj_{\faml{X}_x})_y\}_{y \in \idx{Y}}) \\
&=
\lambda_{\prod_{\idx{Y}}}(\{{\eval_{\prod}}_y \comp g_x\}_{y \in \idx{Y}}) \\
&=
g_x
\end{align*}
\qed

\item $\eval_{Y,Z} \comp (\lambda_{X,Y,Z}(f), \id_{Y}) = f$. Suppose $f: X \times Y \to Z$ in $\Fam(\cat{C})$
and $(x, y) \in \idx{X} \times \idx{Y}$. The diagram on the left below commutes by the universal property of
the set-indexed product; the commutativity of the diagram on the right is trivial.

\begin{center}
\begin{tikzcd}
   \prod_{\idx{Y}}\reindex{\faml{Z}}{\idx{f} \comp (x,-)} \arrow[dr, "\eval_{\prod_{y}}"] \\
   \faml{X}_x \arrow[u, "\lambda_{\prod_{\idx{Y}}}(\reindex{\faml{f}}{{(x,-)}}\,\comp\,
\biinj_{\faml{X}_x})"] \arrow[r, "\faml{f}_{x,y}\,\comp\,\biinj_{\faml{X}_x}"'] & \faml{Z}_{\idx{f}(x,y)}
\end{tikzcd}
\hspace{1cm}
\begin{tikzcd}[column sep=5em]
   \faml{Y}_y \arrow[dr,
   "\faml{f}_{(x,y)}\,\comp\,\biinj_{\faml{Y}_y}"] \\
   \faml{Y}_y \arrow[u, -, double, "\id_{\faml{Y}_y}"] \arrow[r, "\faml{f}_{x,y}\,\comp\,\biinj_{\faml{Y}_y}"'] & \faml{Z}_{\idx{f}(x,y)}
\end{tikzcd}
\end{center}

The diagram on the left below then commutes by Proposition~\ref{prop:biproduct:prod-coprod}, with
$\coprodM{\faml{f}_{x,y}\,\comp\,\biinj_{\faml{X}_x}}{\faml{f}_{x,y}\,\comp\,\biinj_{\faml{Y}_y}} =
\faml{f}_{(x,y)}$ by the uniqueness of the coproduct of morphisms. Then the diagram on the right in
$\Fam(\cat{C})$ commutes by the definitions of $\lambda_{X,Y,Z}, \eval_{Y,Z}$ and $[Y,Z]$.

\begin{center}
\begin{tikzcd}[column sep=5.5em]
   \prod_{\idx{Y}}\reindex{\faml{Z}}{\idx{f} \comp (x,-)} \biprod \faml{Y}_y \arrow[dr,
   "\coprodM{\eval_{\prod_{y}}}{\faml{f}_{(x,y)}\,\comp\,\biinj_{\faml{Y}_y}}"] \\
   \faml{X}_x \biprod \faml{Y}_y \arrow[u, "(\lambda_{\prod_{\idx{Y}}}(\reindex{\faml{f}}{{(x,-)}}\,\comp\,
\biinj_{\faml{X}_x}){,}\,\id_{\faml{Y}_y})"] \arrow[r,
"\coprodM{\faml{f}_{x,y}\,\comp\,\biinj_{\faml{X}_x}}{\faml{f}_{x,y}\,\comp\,\biinj_{\faml{Y}_y}}"'] &
\faml{Z}_{\idx{f}(x,y)}
\end{tikzcd}
\hspace{1cm}
\begin{tikzcd}
   {{[Y, Z]}} \times Y \arrow[dr, "\eval_{Y,Z}"] \\
   X \times Y \arrow[u, "(\lambda_{X,Y,Z}(f){,}\,\id_Y)"] \arrow[r, "f"'] & Z
\end{tikzcd}
\end{center}
\end{enumerate}
\end{proof}

\subsection{Galois slicing and automatic differentiation via categories of families}
\label{sec:galois-slicing-auto-diff-via-fam}

Recall that $\Fam(\cat{C})$ is the Grothendieck construction $\Grothendieck{X}\Fam(X, \cat{C})$.

\subsubsection{\GPS}
\label{sec:galois-slicing-auto-diff-via-fam:galois-slicing}

$\Fam(\LatGal)$ has:
\begin{itemize}
\item as objects $(X, \partial X)$ all pairs of a set $X$ and and for every $x \in X$, a bounded lattice
$\partial X_x$;
\item as morphisms $(X, \partial X) \to (Y, \partial Y)$, all pairs $(f, \partial f)$ of a function $f: X \to
Y$ and for every $x \in X$, a Galois connection $\partial f_x: \partial X_x \to \partial Y_{f(x)}$.
\end{itemize}

\noindent with the role of the pullback along $f$ in the composition $(g, \partial g) \comp (f, \partial f) =
(g \comp f, \reindex{\partial g}{f} \comp \partial f)$ being to select the appropriate lattice of
approximations and Galois connection at each point.

\subsubsection{Automatic differentiation}
\label{sec:galois-slicing-auto-diff-via-fam:auto-diff}

Consider the functor $\Diff \hookrightarrow \Fam(\FinVect)$ which send objects $\RR^n$ to constant families of
tangent spaces $(\RR^n, \const(\RR^n))$ and morphisms $f: \RR^m \to \RR^n \times (\RR^m \linearto \RR^n)$ to
pairs $(f_1, f_2)$ of differentiable functions $f_1: \RR^m \to \RR^n$ and families of linear maps $f_2:
\{{f_2}_x: \RR^m \linearto \RR^n\}_{x \in \RR^m}$. This exhibits $\Diff$ as equivalent to a full subcategory
of $\Fam(\FinVect)$.

In that subcategory, for any $(f, \partial f): (\RR^m, \const(\RR^m)) \to (\RR^n, \const(\RR^n))$ and $(g,
\partial g): (\RR^n, \const(\RR^n)) \to (\RR^k, \const(\RR^k))$, we have that for any $x \in \RR^m$:
\begin{itemize}
\item $\partial f_x: \RR^m \to \RR^n$;
\item $\reindex{\partial g}{f}_x = \partial g_{f(x)}: \RR^n \to \RR^k$.
\end{itemize}

\noindent The composition $(g, \partial g) \comp (f, \partial f) = (g \comp f, \reindex{\partial g}{f} \comp
\partial f): (\RR^m, \const(\RR^m)) \to (\RR^k, \const(\RR^k))$ corresponds exactly to the forward chain rule,
since $(\reindex{\partial g}{f} \comp \partial f)_x = {\partial g}_{f(x)} \comp {\partial f}_x$.

Similarly, $\Diff^*$ is equivalent to a full subcategory of $\Fam(\FinVect^\op)$ with composition
corresponding to the backward chain rule.


\subsection{Higher-order language}

We introduce a standard total functional language with a (reasonably) expressive type
system~\cite{crole94,pitts01,santocanale02}, over a set $\PrimTy$ of primitive types $\rho$ and sets
$\PrimOp^\rho_{\rho_1,\ldots,\rho_n}$ of primitive operations $\phi$.

\subsubsection{Syntax}
\label{sec:language:syntax}

% \begin{figure}
  \begin{mathpar}
  \small
  \inferrule*
  {
    \strut
  }
  {
    \Pol(+,\alpha,\alpha)
  }
  \and
  \inferrule*
  {
    \alpha \neq \beta
  }
  {
    \Pol(p,\alpha,\beta)
  }
  \and
  \inferrule*
  {
    \strut
  }
  {
    \Pol(p,\alpha,\tyZero)
  }
  \and
  \inferrule*
  {
    \Pol(p,\alpha,\sigma)
    \\
    \Pol(p,\alpha,\tau)
  }
  {
    \Pol(p,\alpha,\sigma \tySum \tau)
  }
  \and
  \inferrule*
  {
    \strut
  }
  {
    \Pol(p,\alpha,\tyUnit)
  }
  \and
  \inferrule*
  {
    \Pol(p,\alpha,\sigma)
    \\
    \Pol(p,\alpha,\tau)
  }
  {
    \Pol(p,\alpha,\sigma \tyProd \tau)
  }
  \and
  \inferrule*
  {
    \Pol(\neg p,\alpha,\sigma)
    \\
    \Pol(p,\alpha,\tau)
  }
  {
    \Pol(p,\alpha,\sigma \tyFun \tau)
  }
  \and
  \inferrule*
  {
    \strut
  }
  {
    \Pol(p,\alpha,\mu\alpha.\tau)
  }
  \and
  \inferrule*
  {
    \alpha \neq \beta
    \\
    \Pol(p,\alpha, \tau)
  }
  {
    \Pol(p,\alpha,\mu\beta.\tau)
  }
  \end{mathpar}
\caption{Polarity checking}
\end{figure}
 -- follow nunes2023 instead
\begin{figure}
  \begin{subfigure}[t]{0.48\linewidth}
  \small
  \[
  \begin{array}{lllll}
    & \textit{Types}
    \\
    &
    \sigma, \tau
    & ::= &
    \rho
    &
    \text{primitive type}
    \\
    && \mid &
    \sigma \tySum \tau
    &
    \text{sum}
    \\
    && \mid &
    \tyUnit
    &
    \text{unit}
    \\
    && \mid &
    \sigma \tyProd \tau
    &
    \text{product}
    \\
    && \mid &
    \sigma \tyFun \tau
    &
    \text{function}
    \\
    && \mid &
    \tyList\;\tau
    &
    \text{list}
    \\
    && \mid &
    \tyLift\;\tau
    &
    \text{lifting}
  \end{array}
  \]
  \end{subfigure}%
  \begin{subfigure}[t]{0.48\linewidth}
  \small
  \[
  \begin{array}{lllll}
    & \textit{Terms}
    \\
    &
    t, s
    & ::= &
    x
    &
    \text{variable}
    \\
    && \mid &
    \phi(\vec t)
    &
    \text{primitive op}
    \\
    && \mid &
    \tmInl{t} \mid \tmInr{t}
    &
    \text{injection}
    \\
    && \mid &
    \tmCase{s}{x}{t_1}{y}{t_2}
    &
    \text{case}
    \\
    && \mid &
    \tmUnit
    &
    \text{unit}
    \\
    && \mid &
    \tmPair{s}{t}
    &
    \text{pair}
    \\
    && \mid &
    \tmFst{t} \mid \tmSnd{t}
    &
    \text{projection}
    \\
    && \mid &
    \tmFun{x}{t}
    &
    \text{function}
    \\
    && \mid &
    \tmApp{s}{t}
    &
    \text{application}
    \\
    && \mid &
    \tmNil
    &
    \text{nil}
    \\
    && \mid &
    \tmCons{s}{t}
    &
    \text{cons}
    \\
    && \mid &
    \tmFoldList{s_1}{s_2}{t}
    &
    \text{fold}
    \\
    && \mid &
    \tmReturn{t}
    &
    \text{return}
    \\
    && \mid &
    \tmBind{s}{t}
    &
    \text{bind}
  \end{array}
  \]
  \end{subfigure}
  \caption{Syntax of types and terms}
  \label{fig:syntax}
\end{figure}

\begin{figure}
\begin{subfigure}{\linewidth}
  \begin{mathpar}
  \small
  \inferrule*
  {
    \alpha: \kType \in \Delta
  }
  {
    \Delta \vdash \alpha: \kType
  }
  \and
  \inferrule*
  {
    \strut
  }
  {
    \Delta \vdash \tyZero: \kType
  }
  \and
  \inferrule*
  {
    \Delta \vdash \sigma: \kType
    \\
    \Delta \vdash \tau: \kType
  }
  {
    \Delta \vdash \sigma \tySum \tau: \kType
  }
  \and
  \inferrule*
  {
    \strut
  }
  {
    \Delta \vdash \tyUnit: \kType
  }
  \and
  \inferrule*
  {
    \Delta \vdash \sigma: \kType
    \\
    \Delta \vdash \tau: \kType
  }
  {
    \Delta \vdash \sigma \tyProd \tau: \kType
  }
  \and
  \inferrule*
  {
    \Delta \vdash \sigma: \kType
    \\
    \Delta \vdash \tau: \kType
  }
  {
    \Delta \vdash \sigma \tyFun \tau: \kType
  }
  \and
  \inferrule*
  {
    \Delta, \alpha: \kType \vdash \tau: \kType
    \\
    \Pol(+,\alpha,\tau)
  }
  {
    \Delta \vdash \mu\alpha.\tau: \kType
  }
  \and
  \inferrule*
  {
    \Delta \vdash \tau: \kType
  }
  {
    \Delta \vdash \tyLift\;\tau: \kType
  }
  \end{mathpar}
  \caption{Well-kinded types}
\end{subfigure}
\begin{subfigure}{\linewidth}
  \begin{mathpar}
    \small
    \inferrule*
    {
      x : \tau \in \Gamma
    }
    {
      \Gamma \vdash x: \tau
    }
    \and
    \inferrule*
    {
      \Gamma \vdash t : \sigma
    }
    {
      \Gamma \vdash \tmInl{t}: \sigma \tySum \tau
    }
    \and
    \inferrule*
    {
      \Gamma \vdash t : \tau
    }
    {
      \Gamma \vdash \tmInr{t}: \sigma \tySum \tau
    }
    \and
    \inferrule*
    {
      \Gamma \vdash s : \sigma \tySum \tau
      \\
      \Gamma, x: \sigma \vdash t_1 : \tau'
      \\
      \Gamma, y : \tau \vdash t_2 : \tau'
    }
    {
      \Gamma \vdash \tmCase{s}{x}{t_1}{y}{t_2}: \tau'
    }
    \and
    \inferrule*
    {
      \strut
    }
    {
      \Gamma \vdash \tmUnit : \tyUnit
    }
    \and
    \inferrule*
    {
      \Gamma \vdash s : \sigma
      \\
      \Gamma \vdash t : \tau
    }
    {
      \Gamma \vdash \tmPair{s}{t}: \sigma \tyProd \tau
    }
    \and
    \inferrule*
    {
      \Gamma \vdash t : \sigma \tyProd \tau
    }
    {
      \Gamma \vdash \tmFst{t}: \sigma
    }
    \and
    \inferrule*
    {
      \Gamma \vdash t : \sigma \tyProd \tau
    }
    {
      \Gamma \vdash \tmSnd{t}: \tau
    }
    \and
    \inferrule*
    {
      \Gamma, x: \sigma \vdash t : \tau
    }
    {
      \Gamma \vdash \tmFun{x}{t}: \sigma \tyFun \tau
    }
    \and
    \inferrule*
    {
      \Gamma \vdash s: \sigma \tyFun \tau
      \\
      \Gamma \vdash t : \sigma
    }
    {
      \Gamma \vdash \tmApp{s}{t}: \tau
    }
    \and
    \inferrule*
    {
      \Gamma \vdash t : \subst{\tau}{\mu \alpha.\tau}{\alpha}
    }
    {
      \Gamma \vdash \tmRoll{t}: \mu\alpha.\tau
    }
    \and
    \inferrule*
    {
      \Gamma \vdash s : \subst{\sigma}{\tau}{\alpha} \tyFun \tau
      \\
      \Gamma \vdash t : \mu\alpha.\sigma
    }
    {
      \Gamma \vdash \tmFold{s}{t} : \tau
    }
    \and
    \inferrule*
    {
      \Gamma \vdash t : \tau
    }
    {
      \Gamma \vdash \tmReturn{t} : \tyLift\;\tau
    }
    \and
    \inferrule*
    {
      \Gamma \vdash s : \tyLift\;\sigma
      \\
      \Gamma \vdash t : \sigma \tyFun \tyLift\;\tau
    }
    {
      \Gamma \vdash \tmBind{s}{t} : \tyLift\;\tau
    }
  \end{mathpar}
  \caption{Well-typed terms (all types well-kinded)}
\end{subfigure}
\caption{Kinding and typing rules}
\label{fig:typing}
\end{figure}


\figrefTwo{syntax}{typing} give the syntax and typing rules of the higher-order language.

\subsubsection{Semantics}
\label{sec:language:semantics}

\begin{figure}
\begin{subfigure}{\linewidth}
  \begin{mathpar}
  \small
    \inferrule*
    {
      \strut
    }
    {
      \cat{C} \in \Ob(\muPoly_{\cat{C}})
    }
    \and
    \inferrule*
    {
      \strut
    }
    {
      \One \in \Ob(\muPoly_{\cat{C}})
    }
    \and
    \inferrule*
    {
      \cat{D} \in \Ob(\muPoly_{\cat{C}})
      \\
      \cat{D}' \in \Ob(\muPoly_{\cat{C}})
    }
    {
      \cat{D} \times \cat{D}' \in \Ob(\muPoly_{\cat{C}})
    }
  \end{mathpar}
  \caption{Objects}
\end{subfigure}
\begin{subfigure}{\linewidth}
  \begin{mathpar}
  \small
    \inferrule*
    {
      \strut
    }
    {
      !_{\cat{D}} : \cat{D} \to \One \in \Mor(\muPoly_{\cat{C}})
    }
    \and
    \inferrule*
    {
      \strut
    }
    {
      F : \One \to \cat{D} \in \Mor(\muPoly_{\cat{C}})
    }
    \and
    \inferrule*
    {
      \strut
    }
    {
      - \times - : \cat{C} \times \cat{C} \to \cat{C} \in \Mor(\muPoly_{\cat{C}})
    }
    \and
    \inferrule*
    {
      \strut
    }
    {
      - \textstyle\coprod - : \cat{C} \times \cat{C} \to \cat{C} \in \Mor(\muPoly_{\cat{C}})
    }
    \and
    \inferrule*
    {
      \cat{D} \in \Ob(\muPoly_{\cat{C}})
      \\
      \cat{D}' \in \Ob(\muPoly_{\cat{C}})
    }
    {
      \pi_1 : \cat{D} \times \cat{D}' \to \cat{D} \in \Mor(\muPoly_{\cat{C}})
    }
    \and
    \inferrule*
    {
      \cat{D} \in \Ob(\muPoly_{\cat{C}})
      \\
      \cat{D}' \in \Ob(\muPoly_{\cat{C}})
    }
    {
      \pi_2 : \cat{D} \times \cat{D}' \to \cat{D}' \in \Mor(\muPoly_{\cat{C}})
    }
    \and
    \inferrule*
    {
      \cat{E} \in \Ob(\muPoly_{\cat{C}})
      \\
      \cat{D} \in \Ob(\muPoly_{\cat{C}})
      \\
      \cat{D}' \in \Ob(\muPoly_{\cat{C}})
      \\
      F: \cat{E} \to \cat{D} \in \Mor(\muPoly_{\cat{C}})
      \\
      G: \cat{E} \to \cat{D}' \in \Mor(\muPoly_{\cat{C}})
    }
    {
      \prodM{F}{G} : \cat{E} \to \cat{C} \times \cat{D}' \in \Mor(\muPoly_{\cat{C}})
    }
    \and
    \inferrule*
    {
      \text{TODO}
    }
    {
      \mu F : \cat{D} \to \cat{C} \in \Mor(\muPoly_{\cat{C}})
    }
  \end{mathpar}
  \caption{Morphisms}
\end{subfigure}
\caption{Rules inductively defining $\muPoly_{\cat{C}}$ for any $\cat{C}$ with finite coproducts and finite
products}
\label{fig:mu-polynomial}
\end{figure}

\begin{figure}
  \begin{subfigure}[t]{0.47\linewidth}
    \small
    \begin{align*}
      \sem{\rho} &= \sem{\rho}_{\PrimTy}
      \\
      \sem{\sigma \tySum \tau} &= \textstyle {\sem{\sigma}} + {\sem{\tau}}
      \\
      \sem{\tyUnit} &= 1
      \\
      \sem{\sigma \tyProd \tau} &= \sem{\sigma} \times \sem{\tau}
      \\
      \sem{\sigma \tyFun \tau} &= \internalHom{\sem{\sigma}}{\sem{\tau}}
      \\
      \sem{\tyList\;\tau} &= \List(\sem{\tau})
      % \\
      % \sem{\tyLift\;\tau} &= \Lift(\sem{\tau})
    \end{align*}
    \caption{Interpretation of Types}
    \label{fig:semantics:types}
  \end{subfigure}
  \begin{subfigure}[t]{0.47\linewidth}
    \small
    \begin{align*}
      \sem{\emptyCxt} &= 1
      \\
      \sem{\Gamma, x: \tau} &= \sem{\Gamma} \times \sem{\tau}
    \end{align*}
    \caption{Interpretation of Contexts}
    \label{fig:semantics:contexts}
\end{subfigure}
\begin{subfigure}{0.8\linewidth}
  \small
  \begin{align*}
  \sem{x_i} &= \pi_i
  \\
  \sem{\phi(t_1, \ldots, t_n)}
  &=
  \sem{\phi}_{\Op} \comp \prodM{\sem{t_1}}{\ldots, \sem{t_n}}
  \\
  \sem{\tmInl{t}} &= \mathsf{inj}_1 \comp \sem{t}
  \\
  \sem{\tmInr{t}} &= \mathsf{inj}_2 \comp \sem{t}
  \\
  \sem{\tmCase{s}{x}{t_1}{y}{t_2}} &= \coprodM{\sem{t_1}}{\sem{t_2}} \comp \prodM{\id}{\sem{s}}
  \\
  \sem{\tmUnit} &=\;!_{\sem{\Gamma}}
  \\
  \sem{\tmPair{s}{t}} &= \prodM{\sem{s}}{\sem{t}}
  \\
  \sem{\tmFst{t}} &= \pi_1 \comp \sem{t}
  \\
  \sem{\tmSnd{t}} &= \pi_2 \comp \sem{t}
  \\
  \sem{\tmFun{x}{t}} &= \lambda(\sem{t})
  \\
  \sem{\tmApp{s}{t}} &= \eval \comp \prodM{\sem{s}}{\sem{t}}
%  \\
%  \sem{\tmRoll{t}} &= \inMap_{\sem{\sigma}} \comp \sem{t}
%  \textit{ where }\tau = \mu\alpha.\sigma
  \\
  \sem{\tmNil} &= \nil \comp {!_{\sem{\Gamma}}}
  \\
  \sem{\tmCons{s}{t}} &= \cons \comp \prodM{\sem{s}}{\sem{t}}
  \\
  \sem{\tmFoldList{t_1}{t_2}{s}} &= \fold(\sem{t_1},\sem{t_2}) \comp \prodM{\id}{\sem{s}}
%  \\
%  \sem{\tmFold{s}{t}} &= \eval \comp \prodM{\phi \comp \sem{s}}{\sem{t}}
  % \\
  % \sem{\tmReturn{t}} &= \eta_{\sem{\sigma}} \comp \sem{t}
  % \tag*{($\Gamma \vdash t: \sigma$)}
  % \\
  % \sem{\tmBind{s}{t}} &= \mu_{\sem{\tau}} \comp \Lift(\sem{t}) \comp \mathsf{st}_{\sem{\Gamma},\sem{\sigma}} \comp \prodM{\id}{\sem{s}}
  % \tag*{($\Gamma \vdash t: \sigma \to \tyLift\;\tau$)}
  \end{align*}
  \caption{Terms as morphisms}
  \label{fig:semantics:terms}
\end{subfigure}
\caption{Interpretation of types, contexts and terms in category $\Sem$}
\end{figure}


\begin{definition}[$\mu$-polynomial]
Suppose $\cat{C}$ a category with finite coproducts $(\coprod, \Zero)$ and finite products $(\times, \One)$.
Define $\muPoly_{\cat{C}}$ to be the smallest subcategory of $\Cat$ generated inductively by the rules in
\figref{mu-polynomial}.
\end{definition}

For any $\mu$-polynomial endofunctor $F: \cat{C} \to \cat{C}$ with initial algebra $\mu F$, write $\inMap_F$
for the structure map $F\mu F \to \mu F$ and write $\cata_f$ for the unique morphism $\mu F \to X$ from the
initial $F$-algebra to any $F$-algebra $(X, f: FX \to X)$.

To give the semantics for the language defined in \figrefTwo{syntax}{typing}, we fix a bicartesian closed
category $\Sem$ with finite products $(\times, \One)$, finite coproducts $(\coprod, \Zero)$, exponentials
$\internalHom{X}{Y}$, evaluation morphisms $\eval_{X,Y}$ and currying isomorphisms $\lambda_{X,Y,Z}$, plus the
following additional structure:
\begin{enumerate}
\item a strong monad $\Lift$ (lifting) with unit $\eta_X: X \to \Lift(X)$, multiplication $\mu_X: \Lift^2(X)
\to \Lift(X)$ and strength $\mathsf{st}_{X,Y}: X \times \Lift(Y) \to \Lift(X \times Y)$;
\item $\mu$-polynomial endofunctors given by $\muPoly_{\Sem}$;
\item for each primitive type $\rho \in \PrimTy$, an object $\sem{\rho}$;
\item for each primitive operation $\phi \in \PrimOp^\rho_{\rho_1,\ldots,\rho_n}$, a morphism $\sem{\phi}:
\sem{\rho_1} \times \ldots \times \sem{\rho_n} \to \sem{\rho}$.
\end{enumerate}

\figref{semantics:types} gives the interpretation of types $\Delta \vdash \tau$ with $\length{\Delta}
= n$ as functors $\Sem^n \to \Sem$, or as objects of $\Sem$ in the case $n = 0$.
\figref{semantics:terms} gives the interpretation of terms $\Gamma \vdash t: \tau$ as morphisms
$\sem{\Gamma} \to \sem{\tau}$.

\todo{Notation for partially applied functor used in \figref{semantics:types}}


\bibliographystyle{ACM-Reference-Format}
\bibliography{bib}

\end{document}
