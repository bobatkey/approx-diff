\documentclass[acmsmall,nonacm]{acmart}

\usepackage{bbm}
\usepackage{tikz-cd}

\geometry{paperwidth=8.3in, paperheight=11.7in} % force to A4 for now
\settopmatter{printacmref=false}

\newcommand*{\note}[1]{\textcolor{blue}{\textbf{note:} #1}}
\newcommand*{\todo}[1]{\textcolor{blue}{\textbf{todo:} #1}}

\newenvironment{salign*}
   {\par\nobreak\small\noindent\csname align*\endcsname}
   {\csname endalign*\endcsname}

\newcommand*{\secref}[1]{\S\ref{sec:#1}}
\newcommand*{\propref}[1]{Proposition~\ref{prop:#1}}

\newcommand*{\biinj}{\mathsf{i}}
\newcommand*{\biprod}{\oplus}
\newcommand*{\biproj}{\mathsf{p}}
\newcommand*{\cat}[1]{\mathbf{#1}}
\newcommand*{\comp}{\circ}
\newcommand*{\const}{\mathsf{const}}
\newcommand*{\coprodM}[2]{[#1, #2]}
\newcommand*{\eval}{\varepsilon}
\newcommand*{\faml}[1]{\overline{#1}} % \fam already defined
\newcommand*{\id}{\mathsf{id}}
\newcommand*{\idx}[1]{\widehat{#1}}
\newcommand*{\inj}{\mathsf{inj}}
\newcommand*{\iso}{\cong}
\newcommand*{\join}{\vee}
\newcommand*{\meet}{\wedge}
\newcommand*{\mult}{\cdot}
\newcommand*{\op}{\mathsf{op}}
\newcommand*{\prodM}[2]{\langle #1, #2\rangle}
\newcommand*{\proj}{\pi}
\newcommand*{\reindex}[2]{#1[#2]}
\newcommand*{\tensor}{\otimes}
\newcommand*{\zero}{0}

\newcommand*{\One}{\mathbbm{1}}
\newcommand*{\Hom}[3]{{#1}(#2,#3)}

% Specific categories
\newcommand*{\Cat}{\cat{Cat}}
\newcommand*{\CMon}{\cat{CMon}}
\newcommand*{\Fam}{\cat{Fam}}
\newcommand*{\FdVect}{\cat{FdVect}}
\newcommand*{\Func}[2]{[#1,#2]}
\newcommand*{\Grothendieck}[1]{\int_{#1}}
\newcommand*{\LatGal}{\cat{LatGal}}
\newcommand*{\Set}{\cat{Set}}
\newcommand*{\Setoid}{\cat{Setoid}}


\begin{document}

\title{Approximation as Differentiation}
\maketitle

\section{Overview}

Covered so far:
\begin{itemize}
\item $\CMon$-enriched categories (\secref{cmon-enriched})
\item biproducts (\secref{biproduct})
\item category $\Fam(I,C)$ of $I$-indexed families of objects of $C$ (\secref{fam})
\item set-indexed products (\secref{set-indexed-product})
\item Grothendieck construction $\Grothendieck{C}F$ and category of families $\Fam(C)$ (\secref{grothendieck})
\item Examples of useful categories with biproducts (\secref{categories-with-biproducts}):
   \begin{itemize}
   \item category $\LatGal$ of bounded lattices and Galois connections (\secref{categories-with-biproducts:latgal})
   \item category $\FinVect_F$ of finite-dimensional vector spaces over a field $F$
   (\secref{categories-with-biproducts:fdvect})
   \end{itemize}
\end{itemize}

\noindent $\Set$ will usually be $\Setoid$ in the Agda implementation but we will gloss that detail for now.

\section{Notation}

We write:
\begin{itemize}
\item $\coprodM{f}{g}: A + B \to C$ for the coproduct of morphisms $f: A \to C$ and $g: B \to C$
\item $\prodM{f}{g}: A \to B \times C$ for the product of morphisms $f: A \to B$ and $g: A \to C$
\item $(f,g): A \times B \to C \times D$ for $\prodM{f \comp \proj_1}{g \comp \proj_2}$ where $f: A \to B$ and $g: C \to D$
\end{itemize}


\section{Definitions}

\subsection{$\CMon$-enriched categories}
\label{sec:cmon-enriched}

\begin{definition}[Commutative monoid]
A \emph{commutative monoid} $A = (A, \varepsilon, \bullet)$ is a set $A$ equipped with distinguished element
$\varepsilon \in A$ called the \emph{unit} and associative binary operation $\bullet$ satisfying $\varepsilon
\bullet x = x$ and $x \bullet \varepsilon = x$ for any $x \in A$.
\end{definition}

A commutative monoid homomorphism from $A$ to $B$ is any function $f: A \to B$ preserving $\varepsilon$ and
$\bullet$.

\begin{definition}[Category of commutative monoids]
Define $\CMon$ to be the category whose objects are the commutative monoids $A$ and morphisms $A \to B$ are
the commutative monoid homomorphisms.
\end{definition}

$\CMon$ is symmetric closed monoidal, with the trivial one-element monoid $\One$ as terminal object, monoidal
product $A \tensor B$ given by the Cartesian product of $A$ and $B$, and braiding given by the isomorphism $A
\times B \iso B \times A$. $\CMon$ is complete and cocomplete, inheriting all limits and colimits from $\Set$.

\subsubsection{$\CMon$-enriched category}

We note that if a category $C$ is $\CMon$-enriched, then:
\begin{enumerate}
\item Every hom-object $\Hom{C}{A}{B}$ is a commutative monoid of morphisms; we write $\zero_{A,B}$ (zero
morphism) for the unit, and $+_{A,B}$ (addition of morphisms) for the binary operation.
\item Composition is \emph{bilinear}, i.e.~given by a family of morphisms $\Hom{C}{B}{C} \tensor
\Hom{C}{A}{B} \to \Hom{C}{A}{C}$ in $\CMon$ that preserve the additive structure in $\Hom{C}{B}{C}$ and
$\Hom{C}{A}{B}$ separately:

\begin{salign*}
f \comp \zero = f = \zero \comp f
\end{salign*}
\begin{salign*}
(f + g) \comp h &= (f \comp h) + (g \comp h) \\
h \comp (f + g) &= (h \comp f) + (h \comp g)
\end{salign*}
\end{enumerate}

$\CMon$ is closed monoidal, and so enriched over itself. More specifically every hom-object $\Hom{CMon}{A}{B}$
is a commutative monoid of commutative monoid homomorphisms where:

\begin{enumerate}
\item the unit $0$ is the constant homomorphism sending every element of $A$ to $\varepsilon_B$;
\item addition of morphisms $+$ is defined pointwise so that $(f + g)(a) = f(a) + g(a)$.
\end{enumerate}

\begin{proposition}
If $C$ is $\CMon$-enriched then for any category $D$ the functor category $\Func{D}{C}$ is $\CMon$-enriched.
\end{proposition}

\subsubsection{$\CMon$-enriched presheaves}

\begin{definition}[$\mathscr{V}$-enriched functor]
For any monoidal category $(\mathscr{V}, \tensor, I)$ and $V$-enriched categories $C$ and $D$, a
\emph{$\mathscr{V}$-enriched} functor $F: C \to D$ on morphisms is a family of morphisms $F_{X,Y}:
\Hom{C}{X}{Y} \to \Hom{D}{F(X)}{F(Y)}$ between hom-objects in $\mathscr{V}$ that makes the following diagrams
in $\mathscr{V}$ commute:

\begin{center}
\begin{tikzcd}[column sep=1.8cm]
I \arrow[d, equals] \arrow[r, "\id_X"] & \Hom{C}{X}{X} \arrow[d, "F_{X,X}"] \\
I \arrow[r, "\id_{F(X)}"'] & \Hom{D}{F(X)}{F(X)}
\end{tikzcd}
\hspace{5mm}
\begin{tikzcd}[column sep=2.6cm]
   \Hom{C}{Y}{Z} \tensor \Hom{C}{X}{Y} \arrow[r, "\comp_{X,Y,Z}"] \arrow[d, "F_{Y,Z} \tensor F_{X,Y}"'] & \Hom{C}{X}{Z} \arrow[d, "F_{X,Z}"] \\
   \Hom{D}{F(Y)}{F(Z)} \tensor \Hom{D}{F(X)}{F(Y)} \arrow[r, "\comp_{F(X),F(Y),F(Z)}"'] & \Hom{D}{F(X)}{F(Z)}
\end{tikzcd}
\end{center}
\end{definition}

Write $\Func{C}{D}_{\mathscr{V}}$ for the category of $\mathscr{V}$-enriched functors from $C$ to $D$.

In the specific setting of a $\CMon$-enriched presheaf $F: C^\op \to \CMon$, this means that $F$ preserves the
additive structure of morphisms:
\begin{itemize}
\item $F(0_{X,Y}) = 0_{F(Y),F(X)}$
\item $F(f + g) = F(f) + F(g)$
\end{itemize}
It then follows automatically that $F$ also preserves bilinear composition, i.e.:
\begin{itemize}
\item $F(f) \comp F(0) = F(f) = F(0) \comp F(f)$
\item $F(f + g) \comp F(h) = F(h \comp f) + F(h \comp g)$
\item $F(h) \comp F(f + g) = F(f \comp h) + F(g \comp h)$
\end{itemize}
in addition to the usual functorial properties.

\subsubsection{$\CMon$-enriched Yoneda embedding}

Suppose $C$ a small category. The usual \emph{Yoneda embedding} for $C$ is the functor $\Yoneda: C \to
\Func{C^\op}{\Set}$ sending:
\begin{itemize}
\item any object $X$ in $C$ to the contravariant hom-functor $\Yoneda(X) = \Hom{C}{-}{X}: C \to \Set$ which
sends:
   \begin{itemize}
   \item any object $Y$ in $C$ to the hom-set $\Hom{C}{Y}{X}$;
   \item any morphism $f: Y \to Z$ in $C$ to the function $\Hom{C}{f}{X} = (- \comp f): \Hom{C}{Z}{X} \to
   \Hom{C}{Y}{X}$.
   \end{itemize}
\item any morphism $f: X \to Y$ in $C$ to the natural transformation $\Yoneda(f)$ where $\Yoneda(f)_Z = (f
\comp -): \Hom{C}{Z}{X} \to \Hom{C}{Z}{Y}$.
\end{itemize}

\noindent For a $\mathscr{V}$-enriched category $C$, the hom-functor $\Hom{C}{-}{X}: C \to \mathscr{V}$ is
$\mathscr{V}$-enriched.

\noindent For $\CMon$-enriched $C$, there is an enriched version of the Yoneda embedding, namely the
$\CMon$-enriched functor $\Yoneda_{\CMon}: C \to \Func{C^\op}{\CMon}_{\CMon}$. First we note that the
hom-functor $\Hom{C}{-}{X}: C \to \CMon$ is $\CMon$-enriched because composition is bilinear and so $(- \comp
f)$ preserves the additive structure of morphisms.

\subsection{Biproducts}
\label{sec:biproduct}

\begin{definition}[Zero object]
A \emph{zero} object is an object which is both terminal and initial.
\end{definition}

Suppose $C$ a $\CMon$-enriched category.

\begin{definition}[Biproduct]
For any objects $A, B$, the \emph{biproduct} of $A$ and $B$ is an object $A \biprod B$ of $C$ together with
morphisms

\begin{center}
\begin{tikzcd}
   A \arrow[r, "\biinj_A", shift left] &
   A \biprod B \arrow[l, "\biproj_A", shift left] \arrow[r, "\biproj_B"', shift right] &
   B \arrow[l, "\biinj_B"', shift right]
\end{tikzcd}
\end{center}

\noindent satisfying

\begin{minipage}[t]{0.45\textwidth}
\begin{center}
\begin{salign*}
   \biproj_A \comp \biinj_A &= \id_A \\
   \biproj_B \comp \biinj_A &= \zero_{A,B}
\end{salign*}
\end{center}
\end{minipage}%
\begin{minipage}[t]{0.45\textwidth}
\begin{center}
\begin{salign*}
   \biproj_B \comp \biinj_B &= \id_B \\
   \biproj_A \comp \biinj_B &= \zero_{B,A}
\end{salign*}
\end{center}
\end{minipage}

\noindent and

\begin{salign*}
\biinj_A \comp \biproj_A + \biinj_B \comp \biproj_B &= \id_{A \biprod B}
\end{salign*}
\end{definition}

\begin{proposition}
\item
\begin{itemize}
\item If $C$ has biproducts then it has products given by $(A \biprod B, \biproj_A, \biproj_B)$ and coproducts
given by $(A \biprod B, \biinj_A, \biinj_B)$.
\item If $C$ has products (or coproducts) then they are biproducts.
\item If $C$ has a terminal (or initial) object then it is a zero object.
\end{itemize}
\end{proposition}

\noindent We say that $C$ is \emph{semi-additive} if it has finite biproducts. Having products or coproducts
is sufficient for semi-additivity:

\begin{proposition}
\label{prop:biproduct:from-product-or-coproduct}
Suppose $C$ is $\CMon$-enriched.
\begin{itemize}
\item If $C$ has products given by $(A \times B, \proj_1, \proj_2)$ then:
\begin{enumerate}
\item $\prodM{0_{A,B}}{0_{A,C}} = 0_{A,B \times C}$
\item $\prodM{f_1 + g_1}{f_2 + g_2} = \prodM{f_1 + f_2}{g_1 + g_2}$
\item $(A \times B, \biinj_A, \biinj_B, \proj_1, \proj_2)$ is a
biproduct with $\biinj_A: A \to A \times B = \prodM{\id_A}{0_{A,B}}$ and $\biinj_B: B \to A \times B =
\prodM{0_{A,B}}{\id_B}$.
\end{enumerate}
\item If $C$ has coproducts given by $(A + B, \inj_1, \inj_2)$ then
\begin{enumerate}
\item $\coprodM{0_{A,B}}{0_{A,C}} = 0_{A,B + C}$
\item $(A + B, \inj_1, \inj_2, \biproj_A, \biproj_B)$ is a
biproduct with $\biproj_A: A + B \to A = \coprodM{\id_A}{0_{B,A}}$ and $\biproj_B: A + B \to B =
\coprodM{0_{A,B}}{\id_B}$.
\end{enumerate}
\end{itemize}
\end{proposition}

\begin{proposition}
\label{prop:biproduct:prod-coprod}
Suppose $C$ a category with biproducts. Then the following commutes for any morphisms $f_1, g_1, f_2,
g_2$ in $C$:

\begin{center}
\begin{tikzcd}[row sep=3em, column sep=3em]
   & A \arrow[d, "\biinj_A", shift left] \arrow[dr, "f_2"] \\
   A' \arrow[r, "\prodM{f_1}{g_1}"] \arrow[dr, "g_1"'] \arrow[ur, "f_1"] & A \biprod B \arrow[u, "\biproj_A", shift left] \arrow[d, "\biproj_B"', shift right] \arrow[r, pos=0.35, "\coprodM{f_2}{g_2}"] & B' \\
   & B \arrow[u, "\biinj_B"', shift right] \arrow[ur, "g_2"']
\end{tikzcd}
\end{center}
\end{proposition}


\subsection{Category of families}

\begin{definition}[Category of $I$-indexed families of objects]
For any set $I$ and any category $C$ write $\Fam(I,C)$ for the category where:
\begin{itemize}
\item objects are the $I$-indexed families of objects of $C$;
\item morphisms from $X$ to $Y$ are families of morphisms $f_i: X_i \to Y_i$ in $C$ for any $i \in I$.
\end{itemize}
\end{definition}

\noindent Equivalently $\Fam(I,C)$ is the functor category $\Func{I}{C}$ where we interpret $I$ as a discrete
category.

\begin{definition}[Reindexing]
For any $f: I \to J$ the \emph{reindexing} functor $\reindex{-}{f}: \Fam(J,C) \to \Fam(I,C)$ sends any
$J$-indexed family $X$ to the $I$-indexed family $X[f]$ where $X[f]_i = X_{f(i)}$ for any $i \in I$ and
similarly for morphisms.
\end{definition}

\begin{definition}[$\Fam(-,C)$ functor]
$\Fam(-,C): \Set^{\op} \to \Cat$ is then the functor which sends any set $I$ to $\Fam(I,C)$ and any function
$f: I \to J$ to the functor $\reindex{-}{f}: \Fam(J,C) \to \Fam(I,C)$.
\end{definition}

\begin{definition}[Grothendieck construction for $F$]
Suppose a functor $F: D \to \Cat$. The \emph{Grothendieck construction} $\Grothendieck{D}F$ for $F$ is the
category where:
\begin{itemize}
\item objects are pairs $(I, X)$ of an object $I$ of $D$ and an object $X$ of $F(I)$;
\item morphisms from $(I, X)$ to $(J, Y)$ are morphisms $f: I \to J$ in $D$ paired with a morphism $F(f)(X)
\to Y$ in $F(J)$.
\end{itemize}
\end{definition}

\noindent When $F$ is contravariant, a morphism from $(I, X)$ to $(J, Y)$ is a morphism $f: I \to J$ in $D$
paired with a morphism $X \to F(f)(Y)$ in $F(I)$.

\begin{definition}[Category of families]
For any category $C$ define $\Fam(C)$ to be the Grothendieck construction
$\Grothendieck{\Set^{\op}}\Fam(-,C)$.
\end{definition}

In $\Fam(C)$ we then have that:
\begin{itemize}
\item objects are pairs $(I, X)$ of a set $I$ and an indexed family $X$ in $\Fam(I,C)$;
\item morphisms from $(I, X)$ to $(J, Y)$ are functions $f: I \to J$ paired with a morphism $X \to Y[f]$ in
$\Fam(I,C)$.
\end{itemize}

\begin{proposition}
Suppose $C$ locally small.
\begin{enumerate}
\item If $C$ has binary coproducts and set-indexed products, then $\Fam(C)$ is symmetric monoidal, with the
coproduct as the monoidal product.
\item If the coproducts are in fact biproducts, then $\Fam(C)$ is Cartesian closed.
\end{enumerate}
\end{proposition}

The internal hom $[(I, X), (J, Y)]$ is the object $(K,Z)$ in $\Fam(C)$ where $K$ is the set of morphisms from
$\Fam(I,C)$ to $\Fam(J,C)$ and $Z$ is the $K$-indexed family in $\Fam(K,C)$ which maps every $f: \Fam(I,C) \to
\Fam(J,C)$ in $K$ to $\prod_{i \in I}\reindex{Y}{f}_i$.

\subsection{Set-indexed products}

$C$ \emph{has set-indexed products} if, for any set $I$ and any $I$-indexed family $X$ of objects of $C$,
there exists an object $\prod_{i \in I}X_i$ (also written $\prod X$) and family of morphisms $\eval_i$ in $C$
such that for any object $Y$ and any family of morphisms $\{f_i: Y \to X_i\}_{i \in I}$ in $C$, there exists a
unique morphism $\lambda f$ making the following diagram commute:

\begin{center}
\begin{tikzcd}
   \prod X \arrow[r, "\eval_i"] &
   X_i
   \\
   Y \arrow[ru, "f_i"'] \arrow[u, dotted, "\lambda(f)"]
\end{tikzcd}
\end{center}

Note that $f$ is a morphism in $\Fam(I,C)$ from the constant family $\{Y\}_{i \in I}$ to $X$.

\subsection{Category of families}
\label{sec:grothendieck}

\begin{definition}[Grothendieck construction for $F$]
Suppose a functor $F: D \to \Cat$. The \emph{Grothendieck construction} $\Grothendieck{D}F$ for $F$ is the
category where:
\begin{itemize}
\item objects $X$ are pairs of an object $\idx{X}$ of $D$ and an object $\faml{X}$ of $F(\idx{X})$;
\item morphisms $f: X \to Y$ are morphisms $\idx{f}: \idx{X} \to \idx{Y}$ in $D$ paired with morphisms
$\faml{f}: F(f)(\faml{X}) \to \faml{Y}$ in $F(\idx{Y})$.
\end{itemize}
\end{definition}

\noindent When $F$ is contravariant, a morphism $f: X \to Y$ is a morphism $\idx{f}: \idx{X} \to \idx{Y}$ in
$D$ paired with a morphism $\faml{f}: \faml{X} \to F(f)(\faml{Y})$ in $F(\idx{X})$.

\begin{definition}[Category of families]
For any category $C$ define $\Fam(C)$ to be the Grothendieck construction
$\Grothendieck{\Set^{\op}}\Fam(-,C)$.
\end{definition}

\noindent In $\Fam(C)$ the objects $X$ are thus sets $\idx{X}$ paired with indexed families $\faml{X}$ in
$\Fam(\idx{X},C)$ and morphisms $f: X \to Y$ are functions $\idx{f}: \idx{X} \to \idx{Y}$ paired with
morphisms $\faml{f}: \faml{X} \to \reindex{\faml{Y}}{\idx{f}}$ in $\Fam(\idx{X},C)$.

\begin{proposition}
\item
\begin{enumerate}
\item If $C$ is locally small then so is $\Fam(C)$.
\item If $C$ has binary products then so does $\Fam(C)$.
\end{enumerate}
\end{proposition}

The following is a special case of a construction due to \citet{nunes2023}:

\begin{proposition}
Suppose $C$ locally small. If $C$ has binary biproducts and set-indexed products, then $\Fam(C)$ is Cartesian
closed. \todo{Establish first that if $C$ has binary coproducts and set-indexed products,
then $\Fam(C)$ is symmetric monoidal, with the coproduct as monoidal product.}
\end{proposition}

\begin{proof}
Suppose $C$ is locally small, with binary biproducts given by $(\biprod,\inj,\proj)$ and set-indexed products
given by $(\prod,\eval_{\prod},\lambda_{\prod})$. For any objects $X, Y, Z$ in $\Fam(C)$ define:

\begin{enumerate}
\item $[X, Y]$ to be the object $(X',Y')$ in $\Fam(C)$ with $X' = \Fam(C)(X,Y)$ and $Y'_{f,g} =
\prod_{\idx{X}}\reindex{\faml{Y}}{f}$ for every $(f,g): X \to Y$.
\item The family of isomorphisms $\lambda_{X,Y,Z}$ (natural in $X$) sending any $(f, g): X \times Y \to Z$ to
$(f', g'): X \to [Y, Z]$ with:
\begin{itemize}
\item $f'(x) = (f \comp (x, -), \reindex{g}{(x,-)} \comp \inj_{\faml{Y}}):
Y \to Z$ where $\inj_{\faml{Y}}: \faml{Y} \to \const_{\idx{Y}}(\faml{X}_x) \biprod \faml{Y}$
\item $g'_x = \lambda_{\prod_{\idx{Y}}}(\reindex{g}{(x,-)} \comp \inj_{\faml{X}_x}:
\const_{\idx{Y}}(\faml{X}_x) \to \reindex{\faml{Z}}{f})$ where $\inj_{\faml{X}_x}: \const_{\idx{Y}}(\faml{X}_x)
\to \const_{\idx{Y}}(\faml{X}_x) \biprod \faml{Y}$
\end{itemize}
\item The family of morphisms $\eval_{X,Y} = (f',g'): [X, Y] \times X \to Y$ in $\Fam(C)$ with
\begin{itemize}
\item $f'((f, g), x) = f(x)$
\item $g'_{(f,g), x} = \coprodM{{\eval_{\prod}}_x}{g_x}:
\prod_{\idx{X}}\reindex{\faml{Y}}{f} \biprod \faml{X}_x \to \faml{Y}_{f(x)}$
\end{itemize}
\end{enumerate}

Then:

\begin{enumerate}
\item Suppose $(f,g): X \to [Y,Z]$ and define $(f^\dagger, g^\dagger): X \times Y \to Z$ with:
\begin{itemize}
\item $f^\dagger(x,y) = \idx{f(x)}(y)$
\item $g^\dagger_{x,y} = \coprodM{{\eval_{\prod}}_y \comp g_x}{\faml{f}_y}$
\end{itemize}

To show $\lambda_{X,Y,Z}(f^\dagger, g^\dagger) = (f,g)$, suppose $\lambda_{X,Y,Z}(f^\dagger, g^\dagger) = (f',
g')$ with
\begin{itemize}
\item $f'(x) = (f^\dagger \comp (x, -), \reindex{g^\dagger}{(x,-)} \comp \inj_{\faml{Y}}): Y \to Z$ where
$\inj_{\faml{Y}}: \faml{Y} \to \const_{\idx{Y}}(\faml{X}_x) \biprod \faml{Y}$
\item $g'_x = \lambda_{\prod_{\idx{Y}}}(\reindex{g^\dagger}{(x,-)} \comp \inj_{\faml{X}_x}:
\const_{\idx{Y}}(\faml{X}_x) \to \reindex{\faml{Z}}{f^\dagger})$ where $\inj_{\faml{X}_x}:
\const_{\idx{Y}}(\faml{X}_x) \to \const_{\idx{Y}}(\faml{X}_x) \biprod \faml{Y}$
\end{itemize}
\noindent To show $f' = f : X \to \Fam(C)(Y,Z)$, suppose $x \in \idx{X}$. To show $f'(x) = f(x): Y \to Z$,
suppose $y \in \idx{Y}$.
\begin{enumerate}
\item $(f^\dagger \comp (x, -))(y) = f^\dagger(x,y) = \idx{f(x)}(y)$.
\item To show $\reindex{g^\dagger}{(x,-)} \comp \inj_{\faml{Y}} = g_x$, suppose $y \in \idx{Y}$.
\end{enumerate}

To show $g' = g$, suppose $x \in \idx{X}$.

\item Suppose $f: X \times Y \to Z$ in $\Fam(C)$ and $(x, y) \in \idx{X} \times \idx{Y}$. The diagram on the
left below commutes by the universal property of the set-indexed product; the commutativity of the diagram on
the right is trivial.

\begin{center}
\begin{tikzcd}
   \prod_{\idx{Y}}\reindex{\faml{Z}}{\idx{f} \comp (x,-)} \arrow[dr, "\eval_{\prod_{y}}"] \\
   \faml{X}_x \arrow[u, "\lambda_{\prod_{\idx{Y}}}(\reindex{\faml{f}}{{(x,-)}}\,\comp\,
\inj_{\faml{X}_x})"] \arrow[r, "\faml{f}_{x,y}\,\comp\,\inj_{\faml{X}_x}"'] & \faml{Z}_{\idx{f}(x,y)}
\end{tikzcd}
\hspace{1cm}
\begin{tikzcd}[column sep=5em]
   \faml{Y}_y \arrow[dr,
   "\faml{f}_{(x,y)}\,\comp\,\inj_{\faml{Y}_y}"] \\
   \faml{Y}_y \arrow[u, -, double, "\id_{\faml{Y}_y}"] \arrow[r, "\faml{f}_{x,y}\,\comp\,\inj_{\faml{Y}_y}"'] & \faml{Z}_{\idx{f}(x,y)}
\end{tikzcd}
\end{center}

The diagram on the left below then commutes by Proposition~\ref{prop:biproduct:prod-coprod}, with
$\coprodM{\faml{f}_{x,y}\,\comp\,\inj_{\faml{X}_x}}{\faml{f}_{x,y}\,\comp\,\inj_{\faml{Y}_y}} =
\faml{f}_{(x,y)}$ by the uniqueness of the coproduct of morphisms. Then the diagram on the right in $\Fam(C)$
commutes by the definitions of $\lambda_{X,Y,Z}, \eval_{Y,Z}$ and $[Y,Z]$.

\begin{center}
\begin{tikzcd}[column sep=5.5em]
   \prod_{\idx{Y}}\reindex{\faml{Z}}{\idx{f} \comp (x,-)} \biprod \faml{Y}_y \arrow[dr,
   "\coprodM{\eval_{\prod_{y}}}{\faml{f}_{(x,y)}\,\comp\,\inj_{\faml{Y}_y}}"] \\
   \faml{X}_x \biprod \faml{Y}_y \arrow[u, "(\lambda_{\prod_{\idx{Y}}}(\reindex{\faml{f}}{{(x,-)}}\,\comp\,
\inj_{\faml{X}_x}){,}\,\id_{\faml{Y}_y})"] \arrow[r,
"\coprodM{\faml{f}_{x,y}\,\comp\,\inj_{\faml{X}_x}}{\faml{f}_{x,y}\,\comp\,\inj_{\faml{Y}_y}}"'] &
\faml{Z}_{\idx{f}(x,y)}
\end{tikzcd}
\hspace{1cm}
\begin{tikzcd}
   {{[Y, Z]}} \tensor Y \arrow[dr, "\eval_{Y,Z}"] \\
   X \tensor Y \arrow[u, "\lambda_{X,Y,Z}(f) \tensor \id_Y"] \arrow[r, "f"'] & Z
\end{tikzcd}
\end{center}
\end{enumerate}
\end{proof}

\subsection{Useful categories with biproducts}
\label{sec:categories-with-biproducts}

\subsubsection{Category of bounded lattices and Galois connections}
\label{sec:categories-with-biproducts:latgal}

\begin{definition}[Bounded lattice]
A \emph{bounded lattice} $X = (X, \meet, \top, \join, \bot)$ is a partial order $X$ equipped with binary
operations $\meet$ and $\join$ and distinguished elements $\top$ and $\bot$ such that $(X, \meet, \top)$ is a
bounded meet-semilattice and $(X, \join, \bot)$ is a bounded join-semilattice.
\end{definition}

\noindent Since $(X, \meet, \top)$ is a commutative monoid, and thus a $\CMon$-enriched poset category, for
any $X, Y$ there is a zero meet-semilattice homomorphism $0_{X,Y}: X \to \top_Y$ and addition of homomorphisms
$f + g: X \to Y$ given by pointwise application of $\meet$. The dual construction applies to $(X, \join,
\bot)$. Now define the category $\LatGal$ where:
\begin{itemize}
\item objects $X, Y$ are the bounded lattices;
\item morphisms $f \dashv g: X \to Y$ are meet-semilattice homomorphisms $f: X \to Y$ paired with
join-semilattice homomorphisms $g: Y \to X$ satisfying $y \leq f(x) \iff g(y) \leq x$ for any $x \in X$ and any
$y \in Y$.
\end{itemize}

\noindent and composition is component-wise.

$\LatGal$ is itself $\CMon$-enriched. For any $X, Y$ we have that $\LatGal(X,Y)$ is a commutative monoid,
where:
\begin{itemize}
\item $0_{X,Y}$ is the Galois connection $0_{X,Y} \dashv 0_{Y,X}: X \to Y$
\item $(f \dashv g) + (f' \dashv g') = (f + f') \dashv (g + g'): X \to Y$.
\end{itemize}

\noindent and biadditivity of composition is component-wise. Moreover $\LatGal$ has finite products, with the
trivial 1-point lattice as terminal object, product lattice $X \times Y$ as binary product, and product of
Galois connections defined component-wise; it follows that $\LatGal$ has finite biproducts
(\propref{biproduct:from-product-or-coproduct}).

\subsubsection{Category of finite-dimensional vector spaces}
\label{sec:categories-with-biproducts:fdvect}

\begin{definition}[Vector space over a field $F$]
Suppose $F$ a field with addition of elements written $a + b$ and multiplication written $a \mult b$.
Overloading $+$ and $\mult$, define a \emph{vector space $V = (V, +, \mult)$ over $F$} to be a set $V$
equipped with addition of vectors $+: V^2 \to V$ () and scalar multiplication $\mult: F \times V \to V$ where
$(V,+)$ is an Abelian group and the vector operations are compatible with the field operations in that the
following equations hold:
\begin{itemize}
\item $0 + a = a$
\item $1 \mult a = a$
\item $(a \mult b) \mult v = a \mult (b \mult v)$
\item $(a + b) \mult v = (a \mult v) + (b \mult v)$
\item $a \mult (u + v) = (a \mult u) + (b \mult v)$
\end{itemize}
\end{definition}

% \subsubsection{$\CMon$-valued presheaves over $\LatGal$}
%
% A presheaf $F: \LatGal^\op \to \CMon$ assigns to every bounded lattice $L$ a commutative monoid $F(L)$ and to
% every Galois connection $f: L \to L'$ a commutative monoid homomorphism $F(f): F(L) \to F(L')$.
% $\Func{\LatGal^\op}{\CMon}$ is the category of presheaves of commutative monoids on $\LatGal$.


% \subsection{Higher-order language}

We introduce a standard total functional language with a (reasonably) expressive type
system~\cite{crole94,pitts01,santocanale02}, over a set $\PrimTy$ of primitive types $\rho$ and sets
$\PrimOp^\rho_{\rho_1,\ldots,\rho_n}$ of primitive operations $\phi$.

\subsubsection{Syntax}
\label{sec:language:syntax}

% \begin{figure}
  \begin{mathpar}
  \small
  \inferrule*
  {
    \strut
  }
  {
    \Pol(+,\alpha,\alpha)
  }
  \and
  \inferrule*
  {
    \alpha \neq \beta
  }
  {
    \Pol(p,\alpha,\beta)
  }
  \and
  \inferrule*
  {
    \strut
  }
  {
    \Pol(p,\alpha,\tyZero)
  }
  \and
  \inferrule*
  {
    \Pol(p,\alpha,\sigma)
    \\
    \Pol(p,\alpha,\tau)
  }
  {
    \Pol(p,\alpha,\sigma \tySum \tau)
  }
  \and
  \inferrule*
  {
    \strut
  }
  {
    \Pol(p,\alpha,\tyUnit)
  }
  \and
  \inferrule*
  {
    \Pol(p,\alpha,\sigma)
    \\
    \Pol(p,\alpha,\tau)
  }
  {
    \Pol(p,\alpha,\sigma \tyProd \tau)
  }
  \and
  \inferrule*
  {
    \Pol(\neg p,\alpha,\sigma)
    \\
    \Pol(p,\alpha,\tau)
  }
  {
    \Pol(p,\alpha,\sigma \tyFun \tau)
  }
  \and
  \inferrule*
  {
    \strut
  }
  {
    \Pol(p,\alpha,\mu\alpha.\tau)
  }
  \and
  \inferrule*
  {
    \alpha \neq \beta
    \\
    \Pol(p,\alpha, \tau)
  }
  {
    \Pol(p,\alpha,\mu\beta.\tau)
  }
  \end{mathpar}
\caption{Polarity checking}
\end{figure}
 -- follow nunes2023 instead
\begin{figure}
  \begin{subfigure}[t]{0.48\linewidth}
  \small
  \[
  \begin{array}{lllll}
    & \textit{Types}
    \\
    &
    \sigma, \tau
    & ::= &
    \rho
    &
    \text{primitive type}
    \\
    && \mid &
    \sigma \tySum \tau
    &
    \text{sum}
    \\
    && \mid &
    \tyUnit
    &
    \text{unit}
    \\
    && \mid &
    \sigma \tyProd \tau
    &
    \text{product}
    \\
    && \mid &
    \sigma \tyFun \tau
    &
    \text{function}
    \\
    && \mid &
    \tyList\;\tau
    &
    \text{list}
    \\
    && \mid &
    \tyLift\;\tau
    &
    \text{lifting}
  \end{array}
  \]
  \end{subfigure}%
  \begin{subfigure}[t]{0.48\linewidth}
  \small
  \[
  \begin{array}{lllll}
    & \textit{Terms}
    \\
    &
    t, s
    & ::= &
    x
    &
    \text{variable}
    \\
    && \mid &
    \phi(\vec t)
    &
    \text{primitive op}
    \\
    && \mid &
    \tmInl{t} \mid \tmInr{t}
    &
    \text{injection}
    \\
    && \mid &
    \tmCase{s}{x}{t_1}{y}{t_2}
    &
    \text{case}
    \\
    && \mid &
    \tmUnit
    &
    \text{unit}
    \\
    && \mid &
    \tmPair{s}{t}
    &
    \text{pair}
    \\
    && \mid &
    \tmFst{t} \mid \tmSnd{t}
    &
    \text{projection}
    \\
    && \mid &
    \tmFun{x}{t}
    &
    \text{function}
    \\
    && \mid &
    \tmApp{s}{t}
    &
    \text{application}
    \\
    && \mid &
    \tmNil
    &
    \text{nil}
    \\
    && \mid &
    \tmCons{s}{t}
    &
    \text{cons}
    \\
    && \mid &
    \tmFoldList{s_1}{s_2}{t}
    &
    \text{fold}
    \\
    && \mid &
    \tmReturn{t}
    &
    \text{return}
    \\
    && \mid &
    \tmBind{s}{t}
    &
    \text{bind}
  \end{array}
  \]
  \end{subfigure}
  \caption{Syntax of types and terms}
  \label{fig:syntax}
\end{figure}

\begin{figure}
\begin{subfigure}{\linewidth}
  \begin{mathpar}
  \small
  \inferrule*
  {
    \alpha: \kType \in \Delta
  }
  {
    \Delta \vdash \alpha: \kType
  }
  \and
  \inferrule*
  {
    \strut
  }
  {
    \Delta \vdash \tyZero: \kType
  }
  \and
  \inferrule*
  {
    \Delta \vdash \sigma: \kType
    \\
    \Delta \vdash \tau: \kType
  }
  {
    \Delta \vdash \sigma \tySum \tau: \kType
  }
  \and
  \inferrule*
  {
    \strut
  }
  {
    \Delta \vdash \tyUnit: \kType
  }
  \and
  \inferrule*
  {
    \Delta \vdash \sigma: \kType
    \\
    \Delta \vdash \tau: \kType
  }
  {
    \Delta \vdash \sigma \tyProd \tau: \kType
  }
  \and
  \inferrule*
  {
    \Delta \vdash \sigma: \kType
    \\
    \Delta \vdash \tau: \kType
  }
  {
    \Delta \vdash \sigma \tyFun \tau: \kType
  }
  \and
  \inferrule*
  {
    \Delta, \alpha: \kType \vdash \tau: \kType
    \\
    \Pol(+,\alpha,\tau)
  }
  {
    \Delta \vdash \mu\alpha.\tau: \kType
  }
  \and
  \inferrule*
  {
    \Delta \vdash \tau: \kType
  }
  {
    \Delta \vdash \tyLift\;\tau: \kType
  }
  \end{mathpar}
  \caption{Well-kinded types}
\end{subfigure}
\begin{subfigure}{\linewidth}
  \begin{mathpar}
    \small
    \inferrule*
    {
      x : \tau \in \Gamma
    }
    {
      \Gamma \vdash x: \tau
    }
    \and
    \inferrule*
    {
      \Gamma \vdash t : \sigma
    }
    {
      \Gamma \vdash \tmInl{t}: \sigma \tySum \tau
    }
    \and
    \inferrule*
    {
      \Gamma \vdash t : \tau
    }
    {
      \Gamma \vdash \tmInr{t}: \sigma \tySum \tau
    }
    \and
    \inferrule*
    {
      \Gamma \vdash s : \sigma \tySum \tau
      \\
      \Gamma, x: \sigma \vdash t_1 : \tau'
      \\
      \Gamma, y : \tau \vdash t_2 : \tau'
    }
    {
      \Gamma \vdash \tmCase{s}{x}{t_1}{y}{t_2}: \tau'
    }
    \and
    \inferrule*
    {
      \strut
    }
    {
      \Gamma \vdash \tmUnit : \tyUnit
    }
    \and
    \inferrule*
    {
      \Gamma \vdash s : \sigma
      \\
      \Gamma \vdash t : \tau
    }
    {
      \Gamma \vdash \tmPair{s}{t}: \sigma \tyProd \tau
    }
    \and
    \inferrule*
    {
      \Gamma \vdash t : \sigma \tyProd \tau
    }
    {
      \Gamma \vdash \tmFst{t}: \sigma
    }
    \and
    \inferrule*
    {
      \Gamma \vdash t : \sigma \tyProd \tau
    }
    {
      \Gamma \vdash \tmSnd{t}: \tau
    }
    \and
    \inferrule*
    {
      \Gamma, x: \sigma \vdash t : \tau
    }
    {
      \Gamma \vdash \tmFun{x}{t}: \sigma \tyFun \tau
    }
    \and
    \inferrule*
    {
      \Gamma \vdash s: \sigma \tyFun \tau
      \\
      \Gamma \vdash t : \sigma
    }
    {
      \Gamma \vdash \tmApp{s}{t}: \tau
    }
    \and
    \inferrule*
    {
      \Gamma \vdash t : \subst{\tau}{\mu \alpha.\tau}{\alpha}
    }
    {
      \Gamma \vdash \tmRoll{t}: \mu\alpha.\tau
    }
    \and
    \inferrule*
    {
      \Gamma \vdash s : \subst{\sigma}{\tau}{\alpha} \tyFun \tau
      \\
      \Gamma \vdash t : \mu\alpha.\sigma
    }
    {
      \Gamma \vdash \tmFold{s}{t} : \tau
    }
    \and
    \inferrule*
    {
      \Gamma \vdash t : \tau
    }
    {
      \Gamma \vdash \tmReturn{t} : \tyLift\;\tau
    }
    \and
    \inferrule*
    {
      \Gamma \vdash s : \tyLift\;\sigma
      \\
      \Gamma \vdash t : \sigma \tyFun \tyLift\;\tau
    }
    {
      \Gamma \vdash \tmBind{s}{t} : \tyLift\;\tau
    }
  \end{mathpar}
  \caption{Well-typed terms (all types well-kinded)}
\end{subfigure}
\caption{Kinding and typing rules}
\label{fig:typing}
\end{figure}


\figrefTwo{syntax}{typing} give the syntax and typing rules of the higher-order language.

\subsubsection{Semantics}
\label{sec:language:semantics}

\begin{figure}
\begin{subfigure}{\linewidth}
  \begin{mathpar}
  \small
    \inferrule*
    {
      \strut
    }
    {
      \cat{C} \in \Ob(\muPoly_{\cat{C}})
    }
    \and
    \inferrule*
    {
      \strut
    }
    {
      \One \in \Ob(\muPoly_{\cat{C}})
    }
    \and
    \inferrule*
    {
      \cat{D} \in \Ob(\muPoly_{\cat{C}})
      \\
      \cat{D}' \in \Ob(\muPoly_{\cat{C}})
    }
    {
      \cat{D} \times \cat{D}' \in \Ob(\muPoly_{\cat{C}})
    }
  \end{mathpar}
  \caption{Objects}
\end{subfigure}
\begin{subfigure}{\linewidth}
  \begin{mathpar}
  \small
    \inferrule*
    {
      \strut
    }
    {
      !_{\cat{D}} : \cat{D} \to \One \in \Mor(\muPoly_{\cat{C}})
    }
    \and
    \inferrule*
    {
      \strut
    }
    {
      F : \One \to \cat{D} \in \Mor(\muPoly_{\cat{C}})
    }
    \and
    \inferrule*
    {
      \strut
    }
    {
      - \times - : \cat{C} \times \cat{C} \to \cat{C} \in \Mor(\muPoly_{\cat{C}})
    }
    \and
    \inferrule*
    {
      \strut
    }
    {
      - \textstyle\coprod - : \cat{C} \times \cat{C} \to \cat{C} \in \Mor(\muPoly_{\cat{C}})
    }
    \and
    \inferrule*
    {
      \cat{D} \in \Ob(\muPoly_{\cat{C}})
      \\
      \cat{D}' \in \Ob(\muPoly_{\cat{C}})
    }
    {
      \pi_1 : \cat{D} \times \cat{D}' \to \cat{D} \in \Mor(\muPoly_{\cat{C}})
    }
    \and
    \inferrule*
    {
      \cat{D} \in \Ob(\muPoly_{\cat{C}})
      \\
      \cat{D}' \in \Ob(\muPoly_{\cat{C}})
    }
    {
      \pi_2 : \cat{D} \times \cat{D}' \to \cat{D}' \in \Mor(\muPoly_{\cat{C}})
    }
    \and
    \inferrule*
    {
      \cat{E} \in \Ob(\muPoly_{\cat{C}})
      \\
      \cat{D} \in \Ob(\muPoly_{\cat{C}})
      \\
      \cat{D}' \in \Ob(\muPoly_{\cat{C}})
      \\
      F: \cat{E} \to \cat{D} \in \Mor(\muPoly_{\cat{C}})
      \\
      G: \cat{E} \to \cat{D}' \in \Mor(\muPoly_{\cat{C}})
    }
    {
      \prodM{F}{G} : \cat{E} \to \cat{C} \times \cat{D}' \in \Mor(\muPoly_{\cat{C}})
    }
    \and
    \inferrule*
    {
      \text{TODO}
    }
    {
      \mu F : \cat{D} \to \cat{C} \in \Mor(\muPoly_{\cat{C}})
    }
  \end{mathpar}
  \caption{Morphisms}
\end{subfigure}
\caption{Rules inductively defining $\muPoly_{\cat{C}}$ for any $\cat{C}$ with finite coproducts and finite
products}
\label{fig:mu-polynomial}
\end{figure}

\begin{figure}
  \begin{subfigure}[t]{0.47\linewidth}
    \small
    \begin{align*}
      \sem{\rho} &= \sem{\rho}_{\PrimTy}
      \\
      \sem{\sigma \tySum \tau} &= \textstyle {\sem{\sigma}} + {\sem{\tau}}
      \\
      \sem{\tyUnit} &= 1
      \\
      \sem{\sigma \tyProd \tau} &= \sem{\sigma} \times \sem{\tau}
      \\
      \sem{\sigma \tyFun \tau} &= \internalHom{\sem{\sigma}}{\sem{\tau}}
      \\
      \sem{\tyList\;\tau} &= \List(\sem{\tau})
      % \\
      % \sem{\tyLift\;\tau} &= \Lift(\sem{\tau})
    \end{align*}
    \caption{Interpretation of Types}
    \label{fig:semantics:types}
  \end{subfigure}
  \begin{subfigure}[t]{0.47\linewidth}
    \small
    \begin{align*}
      \sem{\emptyCxt} &= 1
      \\
      \sem{\Gamma, x: \tau} &= \sem{\Gamma} \times \sem{\tau}
    \end{align*}
    \caption{Interpretation of Contexts}
    \label{fig:semantics:contexts}
\end{subfigure}
\begin{subfigure}{0.8\linewidth}
  \small
  \begin{align*}
  \sem{x_i} &= \pi_i
  \\
  \sem{\phi(t_1, \ldots, t_n)}
  &=
  \sem{\phi}_{\Op} \comp \prodM{\sem{t_1}}{\ldots, \sem{t_n}}
  \\
  \sem{\tmInl{t}} &= \mathsf{inj}_1 \comp \sem{t}
  \\
  \sem{\tmInr{t}} &= \mathsf{inj}_2 \comp \sem{t}
  \\
  \sem{\tmCase{s}{x}{t_1}{y}{t_2}} &= \coprodM{\sem{t_1}}{\sem{t_2}} \comp \prodM{\id}{\sem{s}}
  \\
  \sem{\tmUnit} &=\;!_{\sem{\Gamma}}
  \\
  \sem{\tmPair{s}{t}} &= \prodM{\sem{s}}{\sem{t}}
  \\
  \sem{\tmFst{t}} &= \pi_1 \comp \sem{t}
  \\
  \sem{\tmSnd{t}} &= \pi_2 \comp \sem{t}
  \\
  \sem{\tmFun{x}{t}} &= \lambda(\sem{t})
  \\
  \sem{\tmApp{s}{t}} &= \eval \comp \prodM{\sem{s}}{\sem{t}}
%  \\
%  \sem{\tmRoll{t}} &= \inMap_{\sem{\sigma}} \comp \sem{t}
%  \textit{ where }\tau = \mu\alpha.\sigma
  \\
  \sem{\tmNil} &= \nil \comp {!_{\sem{\Gamma}}}
  \\
  \sem{\tmCons{s}{t}} &= \cons \comp \prodM{\sem{s}}{\sem{t}}
  \\
  \sem{\tmFoldList{t_1}{t_2}{s}} &= \fold(\sem{t_1},\sem{t_2}) \comp \prodM{\id}{\sem{s}}
%  \\
%  \sem{\tmFold{s}{t}} &= \eval \comp \prodM{\phi \comp \sem{s}}{\sem{t}}
  % \\
  % \sem{\tmReturn{t}} &= \eta_{\sem{\sigma}} \comp \sem{t}
  % \tag*{($\Gamma \vdash t: \sigma$)}
  % \\
  % \sem{\tmBind{s}{t}} &= \mu_{\sem{\tau}} \comp \Lift(\sem{t}) \comp \mathsf{st}_{\sem{\Gamma},\sem{\sigma}} \comp \prodM{\id}{\sem{s}}
  % \tag*{($\Gamma \vdash t: \sigma \to \tyLift\;\tau$)}
  \end{align*}
  \caption{Terms as morphisms}
  \label{fig:semantics:terms}
\end{subfigure}
\caption{Interpretation of types, contexts and terms in category $\Sem$}
\end{figure}


\begin{definition}[$\mu$-polynomial]
Suppose $\cat{C}$ a category with finite coproducts $(\coprod, \Zero)$ and finite products $(\times, \One)$.
Define $\muPoly_{\cat{C}}$ to be the smallest subcategory of $\Cat$ generated inductively by the rules in
\figref{mu-polynomial}.
\end{definition}

For any $\mu$-polynomial endofunctor $F: \cat{C} \to \cat{C}$ with initial algebra $\mu F$, write $\inMap_F$
for the structure map $F\mu F \to \mu F$ and write $\cata_f$ for the unique morphism $\mu F \to X$ from the
initial $F$-algebra to any $F$-algebra $(X, f: FX \to X)$.

To give the semantics for the language defined in \figrefTwo{syntax}{typing}, we fix a bicartesian closed
category $\Sem$ with finite products $(\times, \One)$, finite coproducts $(\coprod, \Zero)$, exponentials
$\internalHom{X}{Y}$, evaluation morphisms $\eval_{X,Y}$ and currying isomorphisms $\lambda_{X,Y,Z}$, plus the
following additional structure:
\begin{enumerate}
\item a strong monad $\Lift$ (lifting) with unit $\eta_X: X \to \Lift(X)$, multiplication $\mu_X: \Lift^2(X)
\to \Lift(X)$ and strength $\mathsf{st}_{X,Y}: X \times \Lift(Y) \to \Lift(X \times Y)$;
\item $\mu$-polynomial endofunctors given by $\muPoly_{\Sem}$;
\item for each primitive type $\rho \in \PrimTy$, an object $\sem{\rho}$;
\item for each primitive operation $\phi \in \PrimOp^\rho_{\rho_1,\ldots,\rho_n}$, a morphism $\sem{\phi}:
\sem{\rho_1} \times \ldots \times \sem{\rho_n} \to \sem{\rho}$.
\end{enumerate}

\figref{semantics:types} gives the interpretation of types $\Delta \vdash \tau$ with $\length{\Delta}
= n$ as functors $\Sem^n \to \Sem$, or as objects of $\Sem$ in the case $n = 0$.
\figref{semantics:terms} gives the interpretation of terms $\Gamma \vdash t: \tau$ as morphisms
$\sem{\Gamma} \to \sem{\tau}$.

\todo{Notation for partially applied functor used in \figref{semantics:types}}


\section{Some references}

\cite{karvonen2020}

\bibliographystyle{plainnat}
\bibliography{bib}

\end{document}
