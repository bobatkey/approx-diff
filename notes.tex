\documentclass[acmsmall,nonacm]{acmart}

\geometry{paperwidth=8.3in, paperheight=11.7in} % force to A4 for now
\settopmatter{printacmref=false}

\newcommand*{\note}[1]{\textcolor{blue}{\textbf{note:} #1}}
\newcommand*{\todo}[1]{\textcolor{blue}{\textbf{todo:} #1}}

\newenvironment{salign*}
   {\par\nobreak\small\noindent\csname align*\endcsname}
   {\csname endalign*\endcsname}

\newcommand*{\secref}[1]{\S\ref{sec:#1}}
\newcommand*{\propref}[1]{Proposition~\ref{prop:#1}}

\newcommand*{\biinj}{\mathsf{i}}
\newcommand*{\biprod}{\oplus}
\newcommand*{\biproj}{\mathsf{p}}
\newcommand*{\cat}[1]{\mathbf{#1}}
\newcommand*{\comp}{\circ}
\newcommand*{\const}{\mathsf{const}}
\newcommand*{\coprodM}[2]{[#1, #2]}
\newcommand*{\eval}{\varepsilon}
\newcommand*{\faml}[1]{\overline{#1}} % \fam already defined
\newcommand*{\id}{\mathsf{id}}
\newcommand*{\idx}[1]{\widehat{#1}}
\newcommand*{\inj}{\mathsf{inj}}
\newcommand*{\iso}{\cong}
\newcommand*{\join}{\vee}
\newcommand*{\meet}{\wedge}
\newcommand*{\mult}{\cdot}
\newcommand*{\op}{\mathsf{op}}
\newcommand*{\prodM}[2]{\langle #1, #2\rangle}
\newcommand*{\proj}{\pi}
\newcommand*{\reindex}[2]{#1[#2]}
\newcommand*{\tensor}{\otimes}
\newcommand*{\zero}{0}

\newcommand*{\One}{\mathbbm{1}}
\newcommand*{\Hom}[3]{{#1}(#2,#3)}

% Specific categories
\newcommand*{\Cat}{\cat{Cat}}
\newcommand*{\CMon}{\cat{CMon}}
\newcommand*{\Fam}{\cat{Fam}}
\newcommand*{\FdVect}{\cat{FdVect}}
\newcommand*{\Func}[2]{[#1,#2]}
\newcommand*{\Grothendieck}[1]{\int_{#1}}
\newcommand*{\LatGal}{\cat{LatGal}}
\newcommand*{\Set}{\cat{Set}}
\newcommand*{\Setoid}{\cat{Setoid}}


\begin{document}

\title{Approximation as Differentiation notes}
\maketitle

\section{Overview}

Some of the key constructions:
\begin{itemize}
\item Category of families $\Fam$
\item Category of $\LatGal$ of bounded lattices and Galois connections
\end{itemize}

\noindent $\Set$ will often be $\Setoid$ in the Agda implementation but will gloss that here.

\section{Definitions}

\subsection{Indexed families of objects}

For any set $I$ and category $C$ write $\Fam(I,C)$ for the category where:
\begin{itemize}
\item the objects are the $I$-indexed families of objects of $C$;
\item a morphism from $X$ to $Y$ is a family of morphisms $f_i: X_i \to Y_i$ in $C$ for any $i \in I$.
\end{itemize}

\noindent Equivalently $\Fam(I,C)$ is the functor category $\Func{I}{C}$ where we interpret $I$ as a discrete
category.

\begin{definition}[Reindexing functor]
For any $X \in \Fam(J,C)$ and $f: I \to J$ define $X[f] \in \Fam(I,C)$ where $X[f]_i = X_{f(i)}$.
\end{definition}

\subsection{Category of families}

For a functor $F: C \to \Set$, we have the category where:
\begin{itemize}
\item objects are pairs $(I, X)$ of an object $I$ of $C$ and an $I$-indexed family $X \in FI$;
\item morphisms from $(I, X)$ to $(J, Y)$ are morphisms $f: I \to J$ in $C$ where $(Ff)(X) = Y$.
\end{itemize}

\noindent This is the Grothendieck construction for a functor $F: C \to \Cat$ for the special case where $F: C
\to \Set$ (where we read a $\Set$-valued functor as a $\Cat$-valued functor restricted to discrete
categories).

Define the fibred category $\Fam(C)$ over $\Set$ where:
\begin{itemize}
\item objects are pairs $(I, X)$ of a set $I$ and indexed family $X$ in $\Fam(I,C)$;
\item morphisms from $(I, X)$ to $(J, Y)$ are functions $f: I \to J$ paired with morphisms $X \to Y[f]$ in
$\Fam(I,C)$.
\end{itemize}

\noindent This is the Grothendieck construction for the functor $\Set^{\op} \to \Cat$ which sends any set $I$
to the category $\Fam(I,C)$ and any function $f: I \to J$ to the reindexing functor $-[f]: \Fam(J,C) \to
\Fam(I,C)$.

\noindent \todo{Some properties of $\Fam(C)$.}

\subsection{Category of bounded lattices and Galois connections}

Define the category $\LatGal$ where:
\begin{itemize}
\item objects $L$ are the bounded lattices;
\item morphisms $f \vdash g: L \to L'$ are meet-semilattice homomorphisms $f: L \to L'$ paired with
join-semilattice homomorphisms $g: L' \to L$ satisfying $y \leq f(x) \iff g(y) \leq x$ for any $x \in L$ and any
$y \in L'$.
\end{itemize}

\noindent $\LatGal$ has finite biproducts. \todo{Tie this to pre-additive structure.}

\section{Language}

\subsection{Syntax}

We parameterise our language over a signature $\Sigma = (S, \phi, R)$ consisting of:
\begin{itemize}
\item a set $S$ of sorts
\item a set of operations $\phi$ with signature $\Sigma(\phi) \in S^* \times S$
\item a set of relations $R$ with signature $\Sigma(R) \in S^*$
\end{itemize}

\subsection{Semantic domain}

We interpret our language into any category with:
\begin{itemize}
\item finite products (binary products and a terminal object)
\item Booleans
\end{itemize}
\note{Ignoring lists for now. Booleans can be derived from coproducts.}

\end{document}
