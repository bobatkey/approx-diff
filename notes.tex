\documentclass[acmsmall,nonacm]{acmart}

\usepackage{bbm}
\usepackage{tikz-cd}

\geometry{paperwidth=8.3in, paperheight=11.7in} % force to A4 for now
\settopmatter{printacmref=false}

\newcommand*{\note}[1]{\textcolor{blue}{\textbf{note:} #1}}
\newcommand*{\todo}[1]{\textcolor{blue}{\textbf{todo:} #1}}

\newenvironment{salign*}
   {\par\nobreak\small\noindent\csname align*\endcsname}
   {\csname endalign*\endcsname}

\newcommand*{\cat}[1]{\mathbf{#1}}
\newcommand*{\comp}{\circ}
\newcommand*{\eval}{\mathsf{ev}}
\newcommand*{\id}{\mathsf{id}}
\newcommand*{\iso}{\cong}
\newcommand*{\op}{\mathsf{op}}
\newcommand*{\biprod}{\oplus}
\newcommand*{\reindex}[2]{#1[#2]}
\newcommand*{\tensor}{\otimes}
\newcommand*{\zero}{0}

\newcommand*{\One}{\mathbbm{1}}
\newcommand*{\Hom}[3]{{#1}(#2,#3)}

% Specific categories
\newcommand*{\Cat}{\cat{Cat}}
\newcommand*{\CMon}{\cat{CMon}}
\newcommand*{\Fam}{\cat{Fam}}
\newcommand*{\Func}[2]{[#1,#2]}
\newcommand*{\Grothendieck}[1]{\int_{#1}}
\newcommand*{\LatGal}{\cat{LatGal}}
\newcommand*{\Set}{\cat{Set}}
\newcommand*{\Setoid}{\cat{Setoid}}


\begin{document}

\title{Approximation as Differentiation}
\maketitle

\section{Overview}

Covered so far:
\begin{itemize}
\item $\CMon$-enriched categories (\secref{cmon-enriched})
\item biproducts (\secref{biproduct})
\item category $\Fam(I,C)$ of $I$-indexed families of objects of $C$ (\secref{fam})
\item set-indexed products (\secref{set-indexed-product})
\item Grothendieck construction $\Grothendieck{C}F$ and category of families $\Fam(C)$ (\secref{grothendieck})
\item Examples of useful categories with biproducts:
   \begin{itemize}
   \item category $\LatGal$ of bounded lattices and Galois connections (\secref{categories-with-biproducts:latgal})
   \item category $\FdVect_k$ of finite-dimensional vector spaces over a field $k$ (\secref{categories-with-biproducts:fdvect})
   \end{itemize}
\end{itemize}

\noindent $\Set$ will usually be $\Setoid$ in the Agda implementation but we will gloss that detail for now.

\section{Notation}

We write:
\begin{itemize}
\item $\coprodM{f}{g}: A + B \to C$ for the coproduct of morphisms $f: A \to C$ and $g: B \to C$
\item $\prodM{f}{g}: A \to B \times C$ for the product of morphisms $f: A \to B$ and $g: A \to C$
\item $(f,g): A \times B \to C \times D$ for $\prodM{f \comp \pi_1}{g \comp \pi_2}$ where $f: A \to B$ and $g: C \to D$
\end{itemize}


\section{Definitions}

\subsection{$\CMon$-enriched categories}
\label{sec:cmon-enriched}

\begin{definition}[Commutative monoid]
A \emph{commutative monoid} $A = (A, \varepsilon, \bullet)$ is a set $A$ equipped with distinguished element
$\varepsilon \in A$ called the \emph{unit} and associative binary operation $\bullet$ satisfying $\varepsilon
\bullet x = x$ and $x \bullet \varepsilon = x$ for any $x \in A$.
\end{definition}

A commutative monoid homomorphism from $A$ to $B$ is any function $f: A \to B$ preserving $\varepsilon$ and
$\bullet$.

\begin{definition}[Category of commutative monoids]
Define $\CMon$ to be the category whose objects $A$ are the commutative monoids $A$ and morphisms $f: A \to B$
are the commutative monoid homomorphisms.
\end{definition}

$\CMon$ is closed monoidal, with the trivial one-element monoid $\One$ as terminal object and monoidal product
$A \tensor B$ given by the Cartesian product $A \times B$. $\CMon$ is complete and cocomplete, inheriting all
limits and colimits from $\Set$.

\subsubsection{$\CMon$-enriched category}

Recall that if a category $C$ is $\CMon$-enriched, then:
\begin{enumerate}
\item Every hom-object $\Hom{C}{A}{B}$ is a commutative monoid of morphisms; we write $\zero_{A,B}$ (zero
morphism) for the unit, and $+_{A,B}$ (addition of morphisms) for the binary operation.
\item Composition is \emph{bilinear}, i.e.~given by a family of morphisms $\Hom{C}{B}{C} \tensor
\Hom{C}{A}{B} \to \Hom{C}{A}{C}$ in $\CMon$ that preserve the additive structure in $\Hom{C}{B}{C}$ and
$\Hom{C}{A}{B}$ separately:

\begin{salign*}
f \comp \zero = f = \zero \comp f
\end{salign*}
\begin{salign*}
(f + g) \comp h &= (f \comp h) + (g \comp h) \\
h \comp (f + g) &= (h \comp f) + (h \comp g)
\end{salign*}
\end{enumerate}

Because $\CMon$ is closed monoidal, it is enriched over itself; every hom-object $\Hom{\CMon}{A}{B}$ is a
commutative monoid where:

\begin{enumerate}
\item the unit $0$ is the constant homomorphism sending every element of $A$ to $\varepsilon_B$;
\item the binary operation $+$ is pointwise addition of homomorphisms $(f + g)(a) = f(a) + g(a)$.
\end{enumerate}

\begin{proposition}
If $C$ is $\CMon$-enriched then any functor category $\Func{D}{C}$ is $\CMon$-enriched.
\end{proposition}

\subsubsection{$\CMon$-enriched presheaves}

\begin{definition}[$\mathscr{V}$-enriched functor]
For any monoidal category $(\mathscr{V}, \tensor, I)$ and $\mathscr{V}$-enriched categories $C$ and $D$, a
\emph{$\mathscr{V}$-enriched} functor $F: C \to D$ on morphisms is a family of morphisms $F_{X,Y}:
\Hom{C}{X}{Y} \to \Hom{D}{F(X)}{F(Y)}$ between hom-objects in $\mathscr{V}$ that makes the following diagrams
in $\mathscr{V}$ commute:

\begin{center}
\begin{tikzcd}[column sep=1.8cm]
I \arrow[d, equals] \arrow[r, "\id_X"] & \Hom{C}{X}{X} \arrow[d, "F_{X,X}"] \\
I \arrow[r, "\id_{F(X)}"'] & \Hom{D}{F(X)}{F(X)}
\end{tikzcd}
\hspace{5mm}
\begin{tikzcd}[column sep=2.6cm]
   \Hom{C}{Y}{Z} \tensor \Hom{C}{X}{Y} \arrow[r, "\comp_{X,Y,Z}"] \arrow[d, "F_{Y,Z} \tensor F_{X,Y}"'] & \Hom{C}{X}{Z} \arrow[d, "F_{X,Z}"] \\
   \Hom{D}{F(Y)}{F(Z)} \tensor \Hom{D}{F(X)}{F(Y)} \arrow[r, "\comp_{F(X),F(Y),F(Z)}"'] & \Hom{D}{F(X)}{F(Z)}
\end{tikzcd}
\end{center}
\end{definition}

Write $\Func{C}{D}_{\mathscr{V}}$ for the category of $\mathscr{V}$-enriched functors from $C$ to $D$.

In the specific setting of a $\CMon$-enriched presheaf $F: C^\op \to \CMon$, enrichment means that $F$
preserves the additive structure of morphisms:
\begin{itemize}
\item $F(0_{X,Y}) = 0_{F(Y),F(X)}$
\item $F(f + g) = F(f) + F(g)$
\end{itemize}
It then follows automatically that $F$ also preserves bilinear composition, i.e.:
\begin{itemize}
\item $F(f) \comp F(0) = F(f) = F(0) \comp F(f)$
\item $F(f + g) \comp F(h) = F(h \comp f) + F(h \comp g)$
\item $F(h) \comp F(f + g) = F(f \comp h) + F(g \comp h)$
\end{itemize}
in addition to the usual enriched functor properties.

\subsubsection{$\CMon$-enriched Yoneda embedding}

Suppose $C$ a small category. The usual \emph{Yoneda embedding} for $C$ is the functor $\Yoneda: C \to
\Func{C^\op}{\Set}$ sending:
\begin{itemize}
\item any object $X$ in $C$ to the contravariant hom-functor $\Yoneda(X) = \Hom{C}{-}{X}: C^\op \to \Set$
which sends:
   \begin{itemize}
   \item any object $Y$ in $C$ to the hom-set $\Hom{C}{Y}{X}$;
   \item any morphism $f: Y \to Z$ in $C$ to the function $\Hom{C}{f}{X} = (- \comp f): \Hom{C}{Z}{X} \to
   \Hom{C}{Y}{X}$.
   \end{itemize}
\item any morphism $f: X \to Y$ in $C$ to the natural transformation $\Yoneda(f)$ where $\Yoneda(f)_Z = (f
\comp -): \Hom{C}{Z}{X} \to \Hom{C}{Z}{Y}$.
\end{itemize}

\noindent Now suppose $C$ is $\mathscr{V}$-enriched with $\mathscr{V}$ closed monoidal; then the hom-functor
$\Hom{C}{-}{X}: C^\op \to \mathscr{V}$ is $\mathscr{V}$-enriched. The following diagram commutes because
precomposition with the identity is the identity on morphisms:

\begin{center}
\begin{tikzcd}[column sep=1.8cm]
I \arrow[d, equals] \arrow[r, "\id_X"] & \Hom{C}{X}{X} \arrow[d, "\Hom{C}{-}{W}_{X,X}"] \\
I \arrow[r, "\id_{\Hom{C}{X}{W}}"'] & \Hom{\mathscr{V}}{\Hom{C}{X}{W}}{\Hom{C}{X}{W}}
\end{tikzcd}
\end{center}

\noindent and the following by the associativity of composition:

\begin{center}
\begin{tikzcd}[column sep=2.7cm]
   \Hom{C}{Z}{Y} \tensor \Hom{C}{Y}{X} \arrow[r, "\comp_{X,Y,Z}"] \arrow[d, "\Hom{C}{-}{W}_{Y,Z}\,\tensor\,\Hom{C}{-}{W}_{X,Y}"'] & \Hom{C}{Z}{X} \arrow[d, "\Hom{C}{-}{W}_{X,Z}"] \\
   \Hom{\mathscr{V}}{\Hom{C}{Y}{W}}{\Hom{C}{Z}{W}} \tensor \Hom{\mathscr{V}}{\Hom{C}{X}{W}}{\Hom{C}{Y}{W}} \arrow[r, "\comp_{\Hom{C}{X}{W},\Hom{C}{Y}{W},\Hom{C}{Z}{W}}"'] & \Hom{\mathscr{V}}{\Hom{C}{X}{W}}{\Hom{C}{Z}{W}}
\end{tikzcd}
\end{center}

\noindent \todo{Then there is an enriched version of the Yoneda embedding, namely a functor
$\Yoneda_{\mathscr{V}}: C \to \Func{C^\op}{\mathscr{V}}_{\mathscr{V}}$ analogous to $\Yoneda$ which is itself
$\mathscr{V}$-enriched.}

% First we note that the hom-functor $\Hom{C}{-}{X}: C \to \CMon$ is $\CMon$-enriched because composition is
% bilinear and so $(- \comp f)$ preserves the additive structure of morphisms.

\subsection{Biproducts}

\begin{definition}[Zero object]
A \emph{zero} object is an object which is both terminal and initial.
\end{definition}

Suppose $C$ a $\CMon$-enriched category.

\begin{definition}[Biproduct]
For any objects $A, B$, the \emph{biproduct} of $A$ and $B$ is an object $A \biprod B$ of $C$ together with
morphisms

\begin{center}
\begin{tikzcd}
   A \arrow[r, "\inj_A", shift left] &
   A \biprod B \arrow[l, "\proj_A", shift left] \arrow[r, "\proj_B"', shift right] &
   B \arrow[l, "\inj_B"', shift right]
\end{tikzcd}
\end{center}

\noindent satisfying

\begin{minipage}[t]{0.45\textwidth}
\begin{center}
\begin{salign*}
   \proj_A \comp \inj_A &= \id_A \\
   \proj_B \comp \inj_A &= \zero_{A,B}
\end{salign*}
\end{center}
\end{minipage}%
\begin{minipage}[t]{0.45\textwidth}
\begin{center}
\begin{salign*}
   \proj_B \comp \inj_B &= \id_B \\
   \proj_A \comp \inj_B &= \zero_{B,A}
\end{salign*}
\end{center}
\end{minipage}

\noindent and

\begin{salign*}
\inj_A \comp \proj_A + \inj_B \comp \proj_B &= \id_{A \biprod B}
\end{salign*}
\end{definition}

\begin{proposition}
\item
\begin{itemize}
\item If $C$ has biproducts then it has products given by $(A \biprod B, \proj_A, \proj_B)$ and coproducts
given by $(A \biprod B, \inj_A, \inj_B)$.
\item If $C$ has products (or coproducts) then they are biproducts.
\item If $C$ has a terminal (or initial) object then it is a zero object.
\end{itemize}
\end{proposition}

\noindent We say that $C$ is \emph{semi-additive} if it has finite biproducts.

\begin{proposition}
\label{prop:biproduct:prod-coprod}
Suppose $C$ a category with biproducts. Then the following diagram in $C$ commutes for any $f_1, g_1, f_2,
g_2$:

\begin{center}
\begin{tikzcd}[row sep=3em]
   & A' \\
   A \arrow[ur, "f_1"] \arrow[r, "\inj_A", shift left] &
   A \biprod B \arrow[l, "\proj_A", shift left] \arrow[r, "\proj_B"', shift right] \arrow[u, pos=0.3, "\coprodM{f_1}{g_1}"'] &
   B \arrow[l, "\inj_B"', shift right] \arrow[ul, "g_1"'] \\
   & B' \arrow[ul, "f_2"] \arrow[u, pos=0.6, "\prodM{f_2}{g_2}"'] \arrow[ur, "g_2"']
\end{tikzcd}
\end{center}
\end{proposition}


\subsection{Category of families}

\begin{definition}[$\Fam(I,C)$]
For any set $I$ and any category $C$ write $\Fam(I,C)$ for the category where:
\begin{itemize}
\item objects are the $I$-indexed families of objects of $C$;
\item morphisms from $X$ to $Y$ are families of morphisms $f_i: X_i \to Y_i$ in $C$ for any $i \in I$.
\end{itemize}
\end{definition}

\noindent Equivalently $\Fam(I,C)$ is the functor category $\Func{I}{C}$ where we interpret $I$ as a discrete
category.

\begin{definition}[Reindexing]
For any $f: I \to J$ the \emph{reindexing} functor $\reindex{-}{f}: \Fam(J,C) \to \Fam(I,C)$ sends any
$J$-indexed family $X$ to the $I$-indexed family $X[f]$ where $X[f]_i = X_{f(i)}$ for any $i \in I$ and
similarly for morphisms.
\end{definition}

\begin{definition}[$\Fam(-,C)$ functor]
$\Fam(-,C): \Set^{\op} \to \Cat$ is then the functor which sends any set $I$ to $\Fam(I,C)$ and any function
$f: I \to J$ to the functor $\reindex{-}{f}: \Fam(J,C) \to \Fam(I,C)$.
\end{definition}

\begin{definition}[Grothendieck construction for $F$]
Suppose a functor $F: D \to \Cat$. The \emph{Grothendieck construction} $\Grothendieck{D}F$ for $F$ is the
category where:
\begin{itemize}
\item objects are pairs $(I, X)$ of an object $I$ of $D$ and an object $X$ of $F(I)$;
\item morphisms from $(I, X)$ to $(J, Y)$ are morphisms $f: I \to J$ in $D$ paired with a morphism $F(f)(X)
\to Y$ in $F(J)$.
\end{itemize}
\end{definition}

\noindent When $F$ is contravariant, a morphism from $(I, X)$ to $(J, Y)$ is a morphism $f: I \to J$ in $D$
paired with a morphism $X \to F(f)(Y)$ in $F(I)$.

\begin{definition}[Category of families]
For any category $C$ define $\Fam(C)$ to be the Grothendieck construction
$\Grothendieck{\Set^{\op}}\Fam(-,C)$.
\end{definition}

In $\Fam(C)$ we then have that:
\begin{itemize}
\item objects are pairs $(I, X)$ of a set $I$ and an indexed family $X$ in $\Fam(I,C)$;
\item morphisms from $(I, X)$ to $(J, Y)$ are functions $f: I \to J$ paired with a morphism $X \to Y[f]$ in
$\Fam(I,C)$.
\end{itemize}

\subsection{Set-indexed products}

$C$ \emph{has set-indexed products} if, for any set $I$ and any $I$-indexed family $X$ of objects of $C$,
there exists an object $\prod_{i \in I}X_i$ and family of morphisms $\eval_i$ in $C$ such that for any object
$Y$ of $C$ and any family of morphisms $\{f_i: Y \to X_i\}_{i \in I}$ in $C$, there exists a unique morphism
$\lambda f$ making the following diagram commute:

\begin{center}
\begin{tikzcd}
   \prod_{i \in I}X_i \arrow[r, "\eval_i"] &
   X_i
   \\
   Y \arrow[ru, "f_i"'] \arrow[u, "\lambda f"]
\end{tikzcd}
\end{center}

Note that $f$ is a morphism in $\Fam(I,C)$ from the constant family $\{Y\}_{i \in I}$ to $X$.

\subsection{Category of families}

\begin{definition}[Category of $I$-indexed families of objects]
For any set $I$ and any category $C$ write $\Fam(I,C)$ for the category where:
\begin{itemize}
\item objects are the $I$-indexed families of objects of $C$;
\item morphisms from $X$ to $Y$ are families of morphisms $f_i: X_i \to Y_i$ in $C$ for any $i \in I$.
\end{itemize}
\end{definition}

\noindent Equivalently $\Fam(I,C)$ is the functor category $\Func{I}{C}$ where we interpret $I$ as a discrete
category.

\begin{definition}[Reindexing]
For any $f: I \to J$ the \emph{reindexing} functor $\reindex{-}{f}: \Fam(J,C) \to \Fam(I,C)$ sends any
$J$-indexed family $X$ to the $I$-indexed family $X[f]$ where $X[f]_i = X_{f(i)}$ for any $i \in I$ and
similarly for morphisms.
\end{definition}

\begin{definition}[$\Fam(-,C)$ functor]
$\Fam(-,C): \Set^{\op} \to \Cat$ is then the functor which sends any set $I$ to $\Fam(I,C)$ and any function
$f: I \to J$ to the functor $\reindex{-}{f}: \Fam(J,C) \to \Fam(I,C)$.
\end{definition}

\begin{definition}[Grothendieck construction for $F$]
Suppose a functor $F: D \to \Cat$. The \emph{Grothendieck construction} $\Grothendieck{D}F$ for $F$ is the
category where:
\begin{itemize}
\item objects are pairs $(I, X)$ of an object $I$ of $D$ and an object $X$ of $F(I)$;
\item morphisms from $(I, X)$ to $(J, Y)$ are morphisms $f: I \to J$ in $D$ paired with a morphism $F(f)(X)
\to Y$ in $F(J)$.
\end{itemize}
\end{definition}

\noindent When $F$ is contravariant, a morphism from $(I, X)$ to $(J, Y)$ is a morphism $f: I \to J$ in $D$
paired with a morphism $X \to F(f)(Y)$ in $F(I)$.

\begin{definition}[Category of families]
For any category $C$ define $\Fam(C)$ to be the Grothendieck construction
$\Grothendieck{\Set^{\op}}\Fam(-,C)$.
\end{definition}

In $\Fam(C)$ we then have that:
\begin{itemize}
\item objects are pairs $(I, X)$ of a set $I$ and an indexed family $X$ in $\Fam(I,C)$;
\item morphisms from $(I, X)$ to $(J, Y)$ are functions $f: I \to J$ paired with a morphism $X \to Y[f]$ in
$\Fam(I,C)$.
\end{itemize}

\begin{proposition}
\item
\begin{enumerate}
\item If $C$ is locally small then so is $\Fam(C)$.
\item If $C$ has binary products then so does $\Fam(C)$.
\end{enumerate}
\end{proposition}

\begin{proposition}
Suppose $C$ locally small. If $C$ has binary biproducts and set-indexed products, then $\Fam(C)$ is Cartesian
closed. \todo{Establish first that if $C$ has binary coproducts and set-indexed products, then $\Fam(C)$ is
symmetric monoidal, with the coproduct as monoidal product.}
\end{proposition}

\begin{proof}
Suppose $C$ is locally small, with binary biproducts given by $(\biprod,\inj,\proj)$ and set-indexed products
given by $(\prod,\eval_{\prod},\lambda_{\prod})$. For any objects $X = (I, X), Y = (J, Y)$ and $Z = (K, Z)$ in
$\Fam(C)$ define:

\begin{enumerate}
\item $[X, Y]$ to be the object $(\Fam(C)(X,Y),[X, Y])$ in $\Fam(C)$ where we also write $[X,Y]$ for the
indexed family which maps every $(f: I \to J, g: X \to \reindex{Y}{f})$ in $\Fam(C)(X,Y)$ to
$\prod\reindex{Y}{f}$.
\item The family of isomorphisms $\lambda_{X,Y,Z}$ (natural in $X$) which sends any morphism $(f: I \times J
\to K, g: X \times Y \to \reindex{Z}{f}): X \times Y \to Z$ to the morphism $(f': I \to \Fam(C)(Y,Z), g': X
\to \reindex{[Y, Z]}{f'}): X \to [Y, Z]$ with:
\begin{itemize}
\item $f'(i) = (j \mapsto f(i, j): J \to K, j \mapsto g_{i,j} \comp \inj_{Y_j}: Y \to \reindex{Z}{j \mapsto
f(i, j)})$ where $\inj_{Y_j}: Y_j \to X_i \biprod Y_j$
\item $g' = \lambda_{\prod}(i \mapsto g_{i, j} \comp \inj_{X_i})$ where $\inj_{X_i}: X_i \to X_i \biprod
Y_j$
\end{itemize}
\item The family of morphisms $\eval_{X,Y} = (f', g'): [X, Y] \times X \to Y$ in $\Fam(C)$ with
\begin{itemize}
\item $f': \Fam(C)(X,Y) \times I \to J = ((f, g), i) \mapsto f(i)$
\item $g': [X, Y] \times X \to \reindex{Y}{f} = ((f, g), i) \mapsto [{\eval_{\prod}}_i, g_i]$ where
${\eval_{\prod}}_i: \prod_I\reindex{Y}{f} \to Y_{f(i)}$
\end{itemize}
using $[-,-]$ to denote coproduct of morphisms.
\end{enumerate}
Then for any $f: X \times Y \to Z$ in $\Fam(C)$ the following diagrams commute:
\begin{enumerate}
\item $\eval_{Y,Z} \comp (\lambda_{X,Y,Z}(f) \times \id_{Y}) = f$

\begin{center}
\begin{tikzcd}
   {{[Y, Z]}} \tensor Y \arrow[dr, "\eval_{Y,Z}"] \\
   X \tensor Y \arrow[u, "\lambda_{X,Y,Z}(f) \tensor \id_Y"] \arrow[r, "f"'] & Z
\end{tikzcd}
\end{center}

\item $\lambda^{-1}_{X,Y,Z}(\lambda_{X,Y,Z}(f)) = f$
\end{enumerate}
\todo{Prove this bit}
\end{proof}

\input{notes/sec/galois}

% \subsection{Language Syntax and Interpretation}

\subsubsection{Syntax}

We assume a language parameterised by a signature $\Sigma = (S, \Op, \Rel)$ consisting of:
\begin{itemize}
\item a set $S$ of sorts;
\item a set $\Op(s_1, \ldots, s_n; s)$ of $n$-ary operations with input sorts $s_1, \ldots, s_n$ and output
sort $s$;
\item a set $\Rel(s_1, \ldots, s_n)$ of $n$-ary relations over sorts $s_1, \ldots, s_n$.
\end{itemize}

\noindent \todo{Are relations just for examples -- perhaps not needed in the formalism?}

\begin{figure}
  \begin{subfigure}[t]{0.48\linewidth}
  \small
  \[
  \begin{array}{lllll}
    & \textit{Types}
    \\
    &
    \sigma, \tau
    & ::= &
%    \alpha
%    &
%    \text{type variable}
%    \\
%    && \mid &
    \rho
    &
    \text{primitive type}
    \\
    && \mid &
    \tyZero
    &
    \text{zero}
    \\
    && \mid &
    \sigma \tySum \tau
    &
    \text{sum}
    \\
    && \mid &
    \tyUnit
    &
    \text{unit}
    \\
    && \mid &
    \sigma \tyProd \tau
    &
    \text{product}
    \\
    && \mid &
    \sigma \tyFun \tau
    &
    \text{function}
    \\
%    && \mid &
%    \mu\alpha.\tau
%    &
%    \text{inductive type}
%    \\
    && \mid &
    \tyList\;\tau
    &
    \text{list}
    \\
    && \mid &
    \tyLift\;\tau
    &
    \text{lifting}
  \end{array}
  \]
  \end{subfigure}%
  \begin{subfigure}[t]{0.48\linewidth}
  \small
  \[
  \begin{array}{lllll}
    & \textit{Terms}
    \\
    &
    t, s
    & ::= &
    x
    &
    \text{variable}
    \\
    && \mid &
    \phi(\vec t)
    &
    \text{primitive op}
    \\
    && \mid &
    \tmInl{t} \mid \tmInr{t}
    &
    \text{injection}
    \\
    && \mid &
    \tmCase{s}{x}{t_1}{y}{t_2}
    &
    \text{case}
    \\
    && \mid &
    \tmUnit
    &
    \text{unit}
    \\
    && \mid &
    \tmPair{s}{t}
    &
    \text{pair}
    \\
    && \mid &
    \tmFst{t} \mid \tmSnd{t}
    &
    \text{projection}
    \\
    && \mid &
    \tmFun{x}{t}
    &
    \text{function}
    \\
    && \mid &
    \tmApp{s}{t}
    &
    \text{application}
%    \\
%    && \mid &
%    \tmRoll{t}
%    &
%    \text{roll}
    \\
    && \mid &
    \tmNil
    &
    \text{nil}
    \\
    && \mid &
    \tmCons{s}{t}
    &
    \text{cons}
    \\
%    && \mid &
%    \tmFold{s}{t}
%    &
%    \text{fold}
%    \\
    && \mid &
    \tmFoldList{s}{t_1}{t_2}
    &
    \text{fold}
    \\
    && \mid &
    \tmReturn{t}
    &
    \text{return}
    \\
    && \mid &
    \tmBind{s}{t}
    &
    \text{bind}
  \end{array}
  \]
  \end{subfigure}
  \caption{Syntax of types and terms}
  \label{fig:syntax}
\end{figure}


% \subsubsection{Semantic domain}
%
% We interpret our language into any category with:
% \begin{itemize}
% \item finite products (binary products and a terminal object)
% \item Booleans
% \end{itemize}
% \note{Ignoring lists for now. Booleans can be derived from coproducts.}


\section{Some references}

\cite{karvonen2020}

\bibliographystyle{plainnat}
\bibliography{bib}

\end{document}
