\section{Interpreting Higher-Order Programs}
\label{sec:higher-order}

$\Fam(\FinVect)$ and $\Fam(\LatGal)$ provide appropriate notions of composition for automatic differentiation
and Galois slicing respectively, but neither is Cartesian closed and thereby able to interpret higher-order
programs directly. However the following is a special case of a construction due to \citet{nunes2023}:

\begin{proposition}
Suppose $\cat{C}$ locally small. If $\cat{C}$ has binary biproducts and all small products, then
$\Fam(\cat{C})$ is Cartesian closed.
\end{proposition}

A category $\cat{C}$ has all small (set-indexed) products if, for any set $X$ and any $X$-indexed family $A$
of objects of $\cat{C}$, there exists an object $\prod_{x \in X}A_x$ (also written $\prod_{X} A$) and family
of projection morphisms $\{\pi_x: \prod_{X} A \to A_x\}_{x \in X}$ in $\cat{C}$ such that for any object $B$
and family of morphisms $\{f_x: B \to A_x\}_{x \in X}$ in $\cat{C}$, there exists a unique product morphism
$\prodFam{f}$ making the following diagram commute for every $x \in X$:

\begin{center}
\begin{tikzcd}
   \prod_{X} A \arrow[r, "\pi_x"] & A_x \\
   B \arrow[ru, "f_x"'] \arrow[u, dotted, "\prodFam{f}"]
\end{tikzcd}
\end{center}

Abusing notation somewhat, we shall write $X$ for an object $(X, \partial X)$ in $\Fam(C)$, and similarly for
morphisms $f = (f, \partial f)$. If $\cat{C}$ has all small products and also biproducts, then the internal
hom $\internalHom{X}{Y}$ in $\Fam(\cat{C})$ is the object $Z$ where:
\begin{itemize}
\item the indexing set $Z$ is the $\Fam(\cat{C})(X, Y)$, which is indeed a set by
\propref{Grothendieck:fam-inherits-local-smallness};
\item the indexed family $\partial Z$ is the family of products $\prod_{X}\reindex{\partial Y}{f}$ for every
${f: X \to Y}$.
\end{itemize}

The products in $\Fam(C)$ arise from the products in $\cat{C}$ (\propref{Grothendieck:fam-inherits-products}).
The evaluation morphism $\eval_{X,Y}: (\internalHom{X}{Y}) \times X \to Y$ is a family of morphisms in
$\cat{C}$ given by:
\begin{align*}
\eval_{X,Y}(f,x) &= f(x) \\
\partial\eval_{X,Y}(f,x) &= \coprodM{\pi_x}{\partial f(x)}
  : \textstyle\prod_{X}\reindex{\partial Y}{f} \biprod \partial X(x) \to \partial Y(f(x))
\end{align*}
\noindent using the coproduct of morphisms from the biproduct in $\cat{C}$.

The currying isomorphism $\lambda_{X,Y,Z}: \Hom{\Fam(C)}{X \times Y}{Z} \to
\Hom{\Fam(C)}{X}{\internalHom{Y}{Z}}$ natural in $X, Y, Z$ is given by:

\vspace{-4mm}
\begin{align*}
\lambda_{X,Y,Z}(f)(x) &= g \quad\textit{with $g = f \comp (x,-)$ and $\partial g(x) = \reindex{\partial f}{(x,-)}$} \\
\partial\lambda_{X,Y,Z}(f)(x) &= \prodFam{\{\partial f(x,y) \comp \biinj_{\partial X(x)}\}_{y \in Y}}
\end{align*}
\noindent Here $(x,-): Y \to X \times Y$ is the ``reindexing'' function that pairs any $y$ with a fixed $x$,
and for any $y$ $\biinj_{\partial X(x)}$ denotes the canonical injection $\partial X(x) \to \partial X(x)
\biprod \partial Y(y)$.

\todo{
\begin{enumerate}
  \item $\FinVect$ and $\LatGal$ both have biproducts but neither has all small products
  \item $\PSh(\cat{C})$ has all small products (inherited from $\Set$), but doesn't automatically inherit
  biproducts from $\cat{C}$
  \item however, $\PSh_{\CMon}(\cat{C})$ does indeed have biproducts if $\cat{C}$ does
  \item thus $\Fam(\PSh_{\CMon}(\FinVect))$ and $\Fam(\PSh_{\CMon}(\LatGal))$ are both Cartesian closed
  \item also need that $\CMon$-enriched Yoneda embedding preserves biproducts \todo{check: because this is how
  we will interpret base types? E.g.~$\sem{\mathsf{Bool}} = \Hom{\cat{C}}{-}{2}$ assuming $\cat{C}$ provides a
  suitable object 2?}
  \end{enumerate}
}

\subsection{Biproducts and Semi-Additive Categories}
\label{sec:biproducts}

We noted that $\LatGal$ is enriched in $\SemiLat$, the category of semilattices, and moreover that the product
of lattices is also a coproduct, making $\LatGal$ into a category with \emph{biproducts}. In fact the presence
of biproducts is sufficient for enrichment in $\CMon$, the category of commutative monoids (including
semilattices); such a category is called \emph{semi-additive}.

\begin{definition}[Semi-additive category]
\label{def:biproducts:semi-additive}
A category with finite products and coproducts is \emph{semi-additive} if the canonical morphisms (projections
and injections) give an isomorphism
\[\textstyle X \coprod Y \iso X \times Y\] that is natural in both variables.
\end{definition}

The product/coproduct is called a \emph{biproduct}, with the biproduct structure denoted by $(\biprod, 0)$ and
projections $\biproj_X, \biproj_Y$ and injections $\biinj_X$ and $\biinj_Y$. The unit $0$ is both terminal and
initial and is called a \emph{zero} object. A semi-additive category $\cat{C}$ is enriched in $\CMon$: for any
two morphisms $f, g: X \to Y$ in $\cat{C}$, the biproduct structure provides a way to ``add'' them together,
forming a morphism $f + g: X \to Y$. Diagrammatically:

\begin{center}
\begin{tikzcd}
   X \arrow[r, "\diag"] & X \biprod X \arrow[r, "f \biprod g"] & Y \biprod Y \arrow[r, "\codiag"] & Y
\end{tikzcd}
\end{center}

Here $\diag$ denotes the diagonal $\prodM{\id_X}{\id_X}$ given by the universal property of the product and
$\codiag$ denotes the codiagonal $\coprodM{\id_X}{\id_Y}$ given by the universal property of the coproduct.
The $f \oplus g$ morphism is the component-wise map $\prodM{f \comp \biproj_X}{g \comp \biproj_Y} =
\coprodM{\biinj_X \comp f}{\biinj_Y \comp g}$. Similarly we can exhibit a \emph{zero} morphism $0_{X,Y}$ by
composing the unique maps in and out of the zero object:

\begin{center}
\begin{tikzcd}
   X \arrow[r, "!_X"] & 0 \arrow[r, "!^Y"] & Y
\end{tikzcd}
\end{center}

It is easy to verify that $+$ is associative and commutative and that $f + 0_{X,Y} = f$, and thus that every
hom-object $\cat{C}(X,Y)$ is an object in $\CMon$. Moreover composition is \emph{bilinear}, i.e.~given by a
family of morphisms $\Hom{\cat{C}}{Y}{Z} \tensor \Hom{\cat{C}}{X}{Y} \to \Hom{\cat{C}}{X}{Z}$ in $\CMon$ that
preserve the additive structure in $\Hom{\cat{C}}{Y}{Z}$ and $\Hom{\cat{C}}{X}{Y}$ separately:

\begin{salign*}
f \comp \zero_{X,Y} = 0_{X,Z} = \zero_{Y,Z} \comp f
\end{salign*}
\begin{salign*}
(f + g) \comp h &= (f \comp h) + (g \comp h) \\
h \comp (f + g) &= (h \comp f) + (h \comp g)
\end{salign*}

\subsubsection{Biproduct laws}
It is also easy to show that in a semi-additive category the following equations hold:
%\vspace{-4mm}
\begin{minipage}[t]{0.45\textwidth}
\begin{center}
\begin{salign*}
   \biproj_X \comp \biinj_X &= \id_X \\
   \biproj_Y \comp \biinj_X &= \zero_{X,Y}
\end{salign*}
\end{center}
\end{minipage}%
\begin{minipage}[t]{0.45\textwidth}
\begin{center}
\begin{salign*}
   \biproj_Y \comp \biinj_Y &= \id_Y \\
   \biproj_X \comp \biinj_Y &= \zero_{Y,X}
\end{salign*}
\end{center}
\end{minipage}

\begin{salign*}
(\biinj_X \comp \biproj_X) + (\biinj_Y \comp \biproj_Y) &= \id_{X \biprod Y}
\end{salign*}

\vspace{3mm}
\noindent It is also possible to use these laws to define biproducts, in the case where $\cat{C}$ is enriched
in $\CMon$ and so equipped with addition of morphisms and zero morphisms. In that case any products or
coproducts in $\cat{C}$ are necessarily biproducts.

\begin{proposition}
\label{prop:biproducts:from-product-or-coproduct}
In a category $\cat{C}$ enriched in $\CMon$:
\begin{enumerate}
\item If $(X \times Y, \pi_1, \pi_2)$ is a product then $(X \times Y, \inj_1, \inj_2)$ is a coproduct, where
$\inj_X = \prodM{\id_X}{0_{X,Y}}: X \to X \times Y$ and $\inj_Y = \prodM{0_{X,Y}}{\id_Y}: Y \to X \times Y$.
\item If $(X \coprod Y, \inj_1, \inj_2)$ is a coproduct then $(X \coprod Y, \pi_1, \pi_2)$ is a product, where
$\pi_1 = \coprodM{\id_X}{0_{Y,X}}: X \coprod Y \to X$ and $\pi_2 = \coprodM{0_{X,Y}}{\id_Y}: X \coprod Y \to
Y$.
\item If $X$ is terminal (resp.~initial) then $X$ is initial (resp.~terminal).
\end{enumerate}
\end{proposition}

\subsection{Higher-order language}
\todo{Introduce earlier?}

We introduce a standard total functional languages with a (reasonably) expressive type
system~\cite{crole94,pitts01,santocanale02}.

\subsubsection{Syntax}

\begin{figure}
  \begin{mathpar}
  \small
  \inferrule*
  {
    \strut
  }
  {
    \Pol(+,\alpha,\alpha)
  }
  \and
  \inferrule*
  {
    \alpha \neq \beta
  }
  {
    \Pol(p,\alpha,\beta)
  }
  \and
  \inferrule*
  {
    \strut
  }
  {
    \Pol(p,\alpha,\tyZero)
  }
  \and
  \inferrule*
  {
    \Pol(p,\alpha,\sigma)
    \\
    \Pol(p,\alpha,\tau)
  }
  {
    \Pol(p,\alpha,\sigma \tySum \tau)
  }
  \and
  \inferrule*
  {
    \strut
  }
  {
    \Pol(p,\alpha,\tyUnit)
  }
  \and
  \inferrule*
  {
    \Pol(p,\alpha,\sigma)
    \\
    \Pol(p,\alpha,\tau)
  }
  {
    \Pol(p,\alpha,\sigma \tyProd \tau)
  }
  \and
  \inferrule*
  {
    \Pol(\neg p,\alpha,\sigma)
    \\
    \Pol(p,\alpha,\tau)
  }
  {
    \Pol(p,\alpha,\sigma \tyFun \tau)
  }
  \and
  \inferrule*
  {
    \strut
  }
  {
    \Pol(p,\alpha,\mu\alpha.\tau)
  }
  \and
  \inferrule*
  {
    \alpha \neq \beta
    \\
    \Pol(p,\alpha, \tau)
  }
  {
    \Pol(p,\alpha,\mu\beta.\tau)
  }
  \end{mathpar}
\caption{Polarity checking}
\end{figure}

\begin{figure}
  \begin{subfigure}[t]{0.48\linewidth}
  \small
  \[
  \begin{array}{lllll}
    & \textit{Types}
    \\
    &
    \sigma, \tau
    & ::= &
    \rho
    &
    \text{primitive type}
    \\
    && \mid &
    \sigma \tySum \tau
    &
    \text{sum}
    \\
    && \mid &
    \tyUnit
    &
    \text{unit}
    \\
    && \mid &
    \sigma \tyProd \tau
    &
    \text{product}
    \\
    && \mid &
    \sigma \tyFun \tau
    &
    \text{function}
    \\
    && \mid &
    \tyList\;\tau
    &
    \text{list}
    \\
    && \mid &
    \tyLift\;\tau
    &
    \text{lifting}
  \end{array}
  \]
  \end{subfigure}%
  \begin{subfigure}[t]{0.48\linewidth}
  \small
  \[
  \begin{array}{lllll}
    & \textit{Terms}
    \\
    &
    t, s
    & ::= &
    x
    &
    \text{variable}
    \\
    && \mid &
    \phi(\vec t)
    &
    \text{primitive op}
    \\
    && \mid &
    \tmInl{t} \mid \tmInr{t}
    &
    \text{injection}
    \\
    && \mid &
    \tmCase{s}{x}{t_1}{y}{t_2}
    &
    \text{case}
    \\
    && \mid &
    \tmUnit
    &
    \text{unit}
    \\
    && \mid &
    \tmPair{s}{t}
    &
    \text{pair}
    \\
    && \mid &
    \tmFst{t} \mid \tmSnd{t}
    &
    \text{projection}
    \\
    && \mid &
    \tmFun{x}{t}
    &
    \text{function}
    \\
    && \mid &
    \tmApp{s}{t}
    &
    \text{application}
    \\
    && \mid &
    \tmNil
    &
    \text{nil}
    \\
    && \mid &
    \tmCons{s}{t}
    &
    \text{cons}
    \\
    && \mid &
    \tmFoldList{s_1}{s_2}{t}
    &
    \text{fold}
    \\
    && \mid &
    \tmReturn{t}
    &
    \text{return}
    \\
    && \mid &
    \tmBind{s}{t}
    &
    \text{bind}
  \end{array}
  \]
  \end{subfigure}
  \caption{Syntax of types and terms}
  \label{fig:syntax}
\end{figure}

\begin{figure}
\begin{subfigure}{\linewidth}
  \begin{mathpar}
  \small
  \inferrule*
  {
    \alpha: \kType \in \Delta
  }
  {
    \Delta \vdash \alpha: \kType
  }
  \and
  \inferrule*
  {
    \strut
  }
  {
    \Delta \vdash \tyZero: \kType
  }
  \and
  \inferrule*
  {
    \Delta \vdash \sigma: \kType
    \\
    \Delta \vdash \tau: \kType
  }
  {
    \Delta \vdash \sigma \tySum \tau: \kType
  }
  \and
  \inferrule*
  {
    \strut
  }
  {
    \Delta \vdash \tyUnit: \kType
  }
  \and
  \inferrule*
  {
    \Delta \vdash \sigma: \kType
    \\
    \Delta \vdash \tau: \kType
  }
  {
    \Delta \vdash \sigma \tyProd \tau: \kType
  }
  \and
  \inferrule*
  {
    \Delta \vdash \sigma: \kType
    \\
    \Delta \vdash \tau: \kType
  }
  {
    \Delta \vdash \sigma \tyFun \tau: \kType
  }
  \and
  \inferrule*
  {
    \Delta, \alpha: \kType \vdash \tau: \kType
    \\
    \Pol(+,\alpha,\tau)
  }
  {
    \Delta \vdash \mu\alpha.\tau: \kType
  }
  \and
  \inferrule*
  {
    \Delta \vdash \tau: \kType
  }
  {
    \Delta \vdash \tyLift\;\tau: \kType
  }
  \end{mathpar}
  \caption{Well-kinded types}
\end{subfigure}
\begin{subfigure}{\linewidth}
  \begin{mathpar}
    \small
    \inferrule*
    {
      x : \tau \in \Gamma
    }
    {
      \Gamma \vdash x: \tau
    }
    \and
    \inferrule*
    {
      \Gamma \vdash t : \sigma
    }
    {
      \Gamma \vdash \tmInl{t}: \sigma \tySum \tau
    }
    \and
    \inferrule*
    {
      \Gamma \vdash t : \tau
    }
    {
      \Gamma \vdash \tmInr{t}: \sigma \tySum \tau
    }
    \and
    \inferrule*
    {
      \Gamma \vdash s : \sigma \tySum \tau
      \\
      \Gamma, x: \sigma \vdash t_1 : \tau'
      \\
      \Gamma, y : \tau \vdash t_2 : \tau'
    }
    {
      \Gamma \vdash \tmCase{s}{x}{t_1}{y}{t_2}: \tau'
    }
    \and
    \inferrule*
    {
      \strut
    }
    {
      \Gamma \vdash \tmUnit : \tyUnit
    }
    \and
    \inferrule*
    {
      \Gamma \vdash s : \sigma
      \\
      \Gamma \vdash t : \tau
    }
    {
      \Gamma \vdash \tmPair{s}{t}: \sigma \tyProd \tau
    }
    \and
    \inferrule*
    {
      \Gamma \vdash t : \sigma \tyProd \tau
    }
    {
      \Gamma \vdash \tmFst{t}: \sigma
    }
    \and
    \inferrule*
    {
      \Gamma \vdash t : \sigma \tyProd \tau
    }
    {
      \Gamma \vdash \tmSnd{t}: \tau
    }
    \and
    \inferrule*
    {
      \Gamma, x: \sigma \vdash t : \tau
    }
    {
      \Gamma \vdash \tmFun{x}{t}: \sigma \tyFun \tau
    }
    \and
    \inferrule*
    {
      \Gamma \vdash s: \sigma \tyFun \tau
      \\
      \Gamma \vdash t : \sigma
    }
    {
      \Gamma \vdash \tmApp{s}{t}: \tau
    }
    \and
    \inferrule*
    {
      \Gamma \vdash t : \subst{\tau}{\mu \alpha.\tau}{\alpha}
    }
    {
      \Gamma \vdash \tmRoll{t}: \mu\alpha.\tau
    }
    \and
    \inferrule*
    {
      \Gamma \vdash s : \subst{\sigma}{\tau}{\alpha} \tyFun \tau
      \\
      \Gamma \vdash t : \mu\alpha.\sigma
    }
    {
      \Gamma \vdash \tmFold{s}{t} : \tau
    }
    \and
    \inferrule*
    {
      \Gamma \vdash t : \tau
    }
    {
      \Gamma \vdash \tmReturn{t} : \tyLift\;\tau
    }
    \and
    \inferrule*
    {
      \Gamma \vdash s : \tyLift\;\sigma
      \\
      \Gamma \vdash t : \sigma \tyFun \tyLift\;\tau
    }
    {
      \Gamma \vdash \tmBind{s}{t} : \tyLift\;\tau
    }
  \end{mathpar}
  \caption{Well-typed terms (all types well-kinded)}
\end{subfigure}
\caption{Kinding and typing rules}
\label{fig:typing}
\end{figure}


\figrefTwo{syntax}{typing} give the syntax and typing rules of the higher-order language.

\subsubsection{Semantics}

\begin{figure}
\begin{subfigure}{\linewidth}
  \begin{mathpar}
  \small
    \inferrule*
    {
      \strut
    }
    {
      \cat{C} \in \Ob(\muPoly_{\cat{C}})
    }
    \and
    \inferrule*
    {
      \strut
    }
    {
      \One \in \Ob(\muPoly_{\cat{C}})
    }
    \and
    \inferrule*
    {
      \cat{D} \in \Ob(\muPoly_{\cat{C}})
      \\
      \cat{D}' \in \Ob(\muPoly_{\cat{C}})
    }
    {
      \cat{D} \times \cat{D}' \in \Ob(\muPoly_{\cat{C}})
    }
  \end{mathpar}
  \caption{Objects}
\end{subfigure}
\begin{subfigure}{\linewidth}
  \begin{mathpar}
  \small
    \inferrule*
    {
      \strut
    }
    {
      !_{\cat{D}} : \cat{D} \to \One \in \Mor(\muPoly_{\cat{C}})
    }
    \and
    \inferrule*
    {
      \strut
    }
    {
      F : \One \to \cat{D} \in \Mor(\muPoly_{\cat{C}})
    }
    \and
    \inferrule*
    {
      \strut
    }
    {
      - \times - : \cat{C} \times \cat{C} \to \cat{C} \in \Mor(\muPoly_{\cat{C}})
    }
    \and
    \inferrule*
    {
      \strut
    }
    {
      - \textstyle\coprod - : \cat{C} \times \cat{C} \to \cat{C} \in \Mor(\muPoly_{\cat{C}})
    }
    \and
    \inferrule*
    {
      \cat{D} \in \Ob(\muPoly_{\cat{C}})
      \\
      \cat{D}' \in \Ob(\muPoly_{\cat{C}})
    }
    {
      \pi_1 : \cat{D} \times \cat{D}' \to \cat{D} \in \Mor(\muPoly_{\cat{C}})
    }
    \and
    \inferrule*
    {
      \cat{D} \in \Ob(\muPoly_{\cat{C}})
      \\
      \cat{D}' \in \Ob(\muPoly_{\cat{C}})
    }
    {
      \pi_2 : \cat{D} \times \cat{D}' \to \cat{D}' \in \Mor(\muPoly_{\cat{C}})
    }
    \and
    \inferrule*
    {
      \cat{E} \in \Ob(\muPoly_{\cat{C}})
      \\
      \cat{D} \in \Ob(\muPoly_{\cat{C}})
      \\
      \cat{D}' \in \Ob(\muPoly_{\cat{C}})
      \\
      F: \cat{E} \to \cat{D} \in \Mor(\muPoly_{\cat{C}})
      \\
      G: \cat{E} \to \cat{D}' \in \Mor(\muPoly_{\cat{C}})
    }
    {
      \prodM{F}{G} : \cat{E} \to \cat{C} \times \cat{D}' \in \Mor(\muPoly_{\cat{C}})
    }
    \and
    \inferrule*
    {
      \text{TODO}
    }
    {
      \mu F : \cat{D} \to \cat{C} \in \Mor(\muPoly_{\cat{C}})
    }
  \end{mathpar}
  \caption{Morphisms}
\end{subfigure}
\caption{Rules inductively defining $\muPoly_{\cat{C}}$ for any $\cat{C}$ with finite coproducts and finite
products}
\label{fig:mu-polynomial}
\end{figure}

\begin{figure}
\begin{subfigure}{\linewidth}
  \begin{align*}
  \small
  \sem{\vec\alpha \vdash \alpha_i}(\vec X) &=
  X_i
  \\
  \sem{\Delta \vdash \tyZero}(\vec X) &=
  \Zero
  \\
  \sem{\Delta \vdash \sigma \tySum \tau}(\vec X) &=
  \textstyle {\sem{\sigma}(\vec X)} \coprod {\sem{\tau}(\vec X)}
  \\
  \sem{\Delta \vdash \tyUnit}(\vec X) &=
  \One
  \\
  \sem{\Delta \vdash \sigma \tyProd \tau}(\vec X) &=
  \sem{\Delta \vdash \sigma}(\vec X) \times \sem{\Delta \vdash \tau}(\vec X)
  \\
  \sem{\Delta \vdash \sigma \tyFun \tau}(\vec X) &=
  \internalHom{\sem{\Delta \vdash \sigma}(\vec X)}{\sem{\Delta \vdash \tau}(\vec X)}
  \\
  \sem{\vec\alpha \vdash \mu\alpha_i.\tau}(\vec X) &=
  \mu(\lambda Y.\sem{\vec\alpha_{\neq \alpha_i}, \alpha \vdash \tau}(\vec X_{\neq X_i}, Y))
  \\
  \sem{\Delta \vdash \mu\alpha.\tau}(\vec X) &=
  \mu(\lambda Y.\sem{\Delta, \alpha \vdash \tau}(\vec X, Y))
  \textit{ if }\alpha \notin \Delta
  \\
  \sem{\Delta \vdash \tyLift\;\tau}(\vec X) &=
  \mathcal{L}(\sem{\Delta \vdash \tau})(\vec X)
  \end{align*}
  \caption{Parameterised types $\Delta \vdash \tau$ as multifunctors $\cat{C}^n \to \cat{C}$}
  \label{fig:default-semantics:types}
\end{subfigure}
\begin{subfigure}{\linewidth}
  \begin{align*}
  \small
  \sem{\emptyCxt} &= \One
  \\
  \sem{\Gamma, x: \tau} &= \sem{\Gamma} \times \sem{\tau}
  \end{align*}
  \caption{Contexts $\Delta \vdash \Gamma$}
\end{subfigure}
\begin{subfigure}{\linewidth}
  \begin{align*}
  \small
  \sem{x} &= \pi_i
  \textit{ where }\Gamma = x_1: \tau_1, \ldots, x_i: \tau_i, \ldots, x_n: \tau_n
  \\
  \sem{\tmInl{t}} &= \mathsf{inj}_1 \comp \sem{t}
  \\
  \sem{\tmInr{t}} &= \mathsf{inj}_2 \comp \sem{t}
  \\
  \sem{\tmCase{s}{x}{t_1}{y}{t_2}} &= \coprodM{t_1}{t_2} \comp \prodM{\id_{\sem{\Gamma}}}{\sem{s}}
  \\
  \sem{\tmUnit} &=\;!_{\sem{\Gamma}}
  \\
  \sem{\tmPair{s}{t}} &= \prodM{\sem{s}}{\sem{t}}
  \\
  \sem{\tmFst{t}} &= \pi_1 \comp \sem{t}
  \\
  \sem{\tmSnd{t}} &= \pi_2 \comp \sem{t}
  \\
  \sem{\tmFun{x}{t}} &= \lambda(\sem{t})
  \\
  \sem{\tmApp{s}{t}} &= \eval \comp \prodM{\sem{s}}{\sem{t}}
  \\
  \sem{\tmRoll{t}} &= \inMap_{\sem{\sigma}} \comp \sem{t}
  \textit{ where }\tau = \mu\alpha.\sigma
  \\
  \sem{\tmFold{s}{t}} &= \phi_{\sem{s}} \comp \sem{t}
  \\
  \sem{\tmReturn{t}} &= \eta_{\sem{\sigma}} \comp \sem{t}
  \textit{ where }\tau = \tyLift\;\sigma
  \\
  \sem{\tmBind{s}{t}} &= \mu_{\sem{\tau'}} \comp \mathcal{L}(\sem{t}) \comp \mathsf{st}_{\sem{\Gamma},\sem{\sigma}} \comp \prodM{\id_{\sem{\Gamma}}}{\sem{s}}
  \textit{ where }\Gamma \vdash s: \sigma\textit{ and }\tau = \tyLift\;\tau'
  \end{align*}
  \caption{Terms $\Gamma \vdash t: \tau$}
  \label{fig:default-semantics:terms}
\end{subfigure}
\caption{Conventional semantics in a suitable Cartesian closed category $\cat{C}$}
\end{figure}


\begin{definition}[$\mu$-polynomial]
Suppose $\cat{C}$ a category with finite coproducts and finite products. Define $\muPoly_{\cat{C}}$ to be the
smallest subcategory of $\Cat$ generated inductively by the rules in \figref{mu-polynomial}.
\end{definition}

Suppose $\cat{C}$ a Cartesian closed category with finite coproducts, a strong monad $\mathcal{L}$ and
$\mu$-polynomial endofunctors given by $\muPoly_{\cat{C}}$. \figref{default-semantics:types} gives the
interpretation of a type $\tau$ with a single free type variable as a $\mu$-polynomial endofunctor on
$\cat{C}$.
