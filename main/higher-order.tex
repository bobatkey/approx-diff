\section{CHAD at Higher-Order}
\label{sec:higher-order}

$\Fam(\FinVect)$ and $\Fam(\LatGal)$, introduced in \secref{first-order}, provide appropriate notions of
composition for automatic differentiation and Galois slicing respectively, but neither is Cartesian closed and
thereby able to interpret higher-order programs directly. However the following is a special case of a
construction due to \citet{nunes2023}:

\begin{proposition}
Suppose $\cat{C}$ locally small. If $\cat{C}$ has binary biproducts and all small products, then
$\Fam(\cat{C})$ is Cartesian closed.
\end{proposition}

A category $\cat{C}$ has all small (set-indexed) products if, for any set $X$ and any $X$-indexed family $A$
of objects of $\cat{C}$, there exists an object $\prod_{x \in X}A_x$ (also written $\prod_{X} A$) and family
of projection morphisms $\{\pi_x: \prod_{X} A \to A_x\}_{x \in X}$ in $\cat{C}$ such that for any object $B$
and family of morphisms $\{f_x: B \to A_x\}_{x \in X}$ in $\cat{C}$, there exists a unique product morphism
$\prodFam{f}$ making the following diagram commute for every $x \in X$:

\begin{center}
\begin{tikzcd}
   \prod_{X} A \arrow[r, "\pi_x"] & A_x \\
   B \arrow[ru, "f_x"'] \arrow[u, dotted, "\prodFam{f}"]
\end{tikzcd}
\end{center}

Abusing notation somewhat, we shall write $X$ for an object $(X, \partial X)$ in $\Fam(C)$, and similarly for
morphisms $f = (f, \partial f)$. If $\cat{C}$ has all small products and also biproducts, then the internal
hom $\internalHom{X}{Y}$ in $\Fam(\cat{C})$ is the object $Z$ where:
\begin{itemize}
\item the indexing set $Z$ is $\Fam(\cat{C})(X, Y)$, which is indeed a set by
\propref{Grothendieck:fam-inherits-local-smallness};
\item the indexed family $\partial Z$ is the family of products $\prod_{X}\reindex{\partial Y}{f}$ for every
${f: X \to Y}$.
\end{itemize}

The products in $\Fam(C)$ arise from the products in $\cat{C}$ (\propref{Grothendieck:fam-inherits-products}).
The evaluation morphism $\eval_{X,Y}: (\internalHom{X}{Y}) \times X \to Y$ is a family of morphisms in
$\cat{C}$ given by:
\begin{align*}
\eval_{X,Y}(f,x) &= f(x) \\
\partial\eval_{X,Y}(f,x) &= \coprodM{\pi_x}{\partial f(x)}
  : \textstyle\prod_{X}\reindex{\partial Y}{f} \biprod \partial X(x) \to \partial Y(f(x))
\end{align*}
\noindent using the coproduct of morphisms from the biproduct in $\cat{C}$.

The currying isomorphism $\lambda_{X,Y,Z}: \Hom{\Fam(C)}{X \times Y}{Z} \to
\Hom{\Fam(C)}{X}{\internalHom{Y}{Z}}$ natural in $X, Y, Z$ is given by:

\vspace{-4mm}
\begin{align*}
\lambda_{X,Y,Z}(f)(x) &= g \quad\textit{with $g = f \comp (x,-)$ and $\partial g(x) = \reindex{\partial f}{(x,-)}$} \\
\partial\lambda_{X,Y,Z}(f)(x) &= \prodFam{\{\partial f(x,y) \comp \biinj_{\partial X(x)}\}_{y \in Y}}
\end{align*}
\noindent Here $(x,-): Y \to X \times Y$ is the ``reindexing'' function that pairs any $y$ with a fixed $x$,
and for any $y$ $\biinj_{\partial X(x)}$ denotes the canonical injection $\partial X(x) \to \partial X(x)
\biprod \partial Y(y)$.

\todo{
\begin{enumerate}
  \item $\FinVect$ and $\LatGal$ both have biproducts but neither has all small products
  \item $\PSh(\cat{C})$ has all small products (inherited from $\Set$), but doesn't automatically inherit
  biproducts from $\cat{C}$
  \item however, $\PSh_{\CMon}(\cat{C})$ does indeed have biproducts if $\cat{C}$ does
  \item thus $\Fam(\PSh_{\CMon}(\FinVect))$ and $\Fam(\PSh_{\CMon}(\LatGal))$ are both Cartesian closed
  \item also need that $\CMon$-enriched Yoneda embedding preserves biproducts \todo{check: because this is how
  we will interpret base types? E.g.~$\sem{\mathsf{Bool}} = \Hom{\cat{C}}{-}{2}$ assuming $\cat{C}$ provides a
  suitable object 2?}
  \end{enumerate}
}
