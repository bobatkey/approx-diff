\section{Related work}
\label{sec:related-work}

\subsection{Stable Domain Theory}

Stable Domain Theory was originally proposed by \citet{berry79} as a refinement of domain theory aimed at
capturing the intensional behaviour of sequential programs, and elaborated on subsequently by \citet{berry82}
and \citet{amadio-curien}. Standard domain-theoretic models interpret programs as continuous functions,
preserving directed joins; Berry observed that this continuity condition alone is too permissive to model
sequentiality. Stability imposes additional constraints to reflect how functions preserve bounded meets of
approximants, effectively requiring that the evaluation of a function respect a specific computational order.
Though stable functions do not fully characterise sequentiality, because they admit $\mathrm{gustave}$-style
counterexamples (\exref{parallel-or}), they remain an appropriate notion for studying the sensitivity of a
program to partial data at a specific point.

Our use of Stable Domain Theory diverges from the traditional aim of modelling infinite or partial data,
however. Instead, we follow a line of work that uses partiality as a qualitative notion of approximation
suitable for provenance and program slicing (discussed in more detail in \secref{related-work:galois-slicing}
below). Paul Taylor’s characterisation of stable functions via local Galois connections on principle downsets
provides the semantic underpinning for the reverse maps used in Galois slicing~\cite{taylor99}. Our work
builds on these ideas by interpreting Galois slicing as a form of differentiable programming, using the
machinery of stable functions and Galois connections to present Galois slicing in a denotational style.

\subsection{Automatic Differentiation}

\subsection{\GPS}
\label{sec:related-work:galois-slicing}

\cite{berry79,berry82}

\subsection{Tangent Categories}

\cite{cockett14,cockett18}

\subsection{Lens Categories}

\cite{spivak19}
