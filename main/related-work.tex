\section{Related work}

\subsection{Stable Domain Theory}

Stable Domain Theory was originally proposed by \citet{berry79} as a refinement of domain theory aimed at
capturing the intensional behaviour of sequential programs. Standard domain-theoretic models interpret
programs as continuous functions (preserving directed joins) Berry observed that this continuity condition
alone is too permissive to model sequentiality. His introduction of stability imposed additional constraints
to reflect how functions preserve bounded meets of approximants, ensuring that the evaluation of a function
respects a notion of computational order. Subsequent work by \citet{berry82} and Amadio and Curien [1998, Ch.
12], elaborated these ideas, showing that stable functions admit local Galois adjoints on principal downsets,
and offering connections to linear logic and sequential algorithms. Though stable functions do not fully
characterise sequentiality (e.g., they admit Gustave-style counterexamples), they remain a foundational
concept for refining program denotations by their sensitivity to approximation.

Our use of stable domain theory diverges from the traditional aim of modelling infinite or partial data.
Instead, we follow a growing line of work that reinterprets domains as qualitative approximation structures
suitable for provenance and slicing applications. In particular, Paul Taylor’s characterisation of stable
functions via local Galois connections provides the semantic underpinning for the reverse maps used in Galois
slicing, allowing us to reconstruct slices from output approximants [Taylor, 1991]. This perspective has been
fruitfully adopted in settings such as data provenance [Cheney et al. 2007], program differentiation [Elliott
2018], and recently by Vákár and collaborators [Lucatelli Nunes and Vákár 2023] in their CHAD framework. Our
work builds on these ideas by interpreting Galois slicing as a form of differentiable programming, using the
machinery of stable functions and Galois adjoints to unify forward and backward provenance in a denotational
setting.

\subsection{Automatic Differentiation}

\subsection{Galois Program slicing}

\cite{berry79,berry82}

\subsection{Tangent Categories}

\cite{cockett14,cockett18}

\subsection{Lens Categories}

\cite{spivak19}
