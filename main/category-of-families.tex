\subsection{Grothendieck Construction for Indexed Categories}
\label{sec:Grothendieck}

The goal now is to equip the interpretation of a (first-order) program with a Galois connection relating
approximations of its inputs to approximations of its output. Since the specific lattices of approximations
depend on the value being approximated, the semantics of every type $\tau$ must have both a conventional
component as well as an approximating component, i.e.~a set $X$ which serves as the usual interpretation of
$\tau$, plus an associated lattice of approximations $\partial X(x)$ for every point $x \in X$. The semantics
of an expression $e$ of type $\sigma$ with a single free variable of type $\tau$ (with $\tau$ and $\sigma$
denoting $X$ and $Y$) must then also have two components: a function $f: X \to Y$ giving the conventional
interpretation of $e$, plus a family of Galois connections $\partial f(x): \partial X(x) \to \partial Y(f(x))$
for every $x \in X$ which can be used for slicing over that expression.

This pattern of objects and morphisms is captured by the construction used in CHAD~\cite{vákár22,nunes2023} to
model automatic differentiation, namely the \emph{Grothendieck construction} for a particular set-indexed
category $T: \Set^\op \to \Cat$. For automatic differentiation, $T(X)$ will be the functor category
$\Func{X}{\FinVect}$, regarded as the category of $X$-indexed families of finite vector spaces, with indexed
linear maps as morphisms. For Galois slicing, we will take $T(X) = \Func{X}{\LatGal}$, the category of
$X$-indexed families of bounded lattices, with indexed Galois connections as morphisms.

The Grothendieck construction $\Grothendieck{}T$~(\defref{Grothendieck} below) for an indexed category $T$,
sometimes called the \emph{total category} for $T$, is the category of all set-indexed families of objects of
$T$, together with morphisms that account for both structure within each family and how families transform
under reindexing functions $f: X \to Y$. First we formalise $X$-indexed families and their categories.

\begin{definition}[Category of $X$-indexed families]
For any set $X$ (regarded as a discrete category) and any category $\cat{C}$, recall that the functor
category $\Func{X}{\cat{C}}$ has as objects all functors $F: X \to \cat{C}$, namely $X$-indexed families of
objects of $\cat{C}$, and as morphisms from $F \to G$, all families of morphisms $\eta(x): F(x) \to G(x)$ in
$\cat{C}$ for every $x \in X$, with naturality trivial because $X$ has only identity morphisms.
\end{definition}

Like any functor category $\Func{X}{\cat{C}}$ inherits limits and colimits pointwise from its codomain.

\begin{proposition}
If $\cat{C}$ has limits (resp.~colimits) then $\Func{X}{\cat{C}}$ has limits (resp.~colimits) computed
pointwise.
\end{proposition}

\begin{definition}[Reindexing]
For any function $f: X \to Y$ define the \emph{reindexing} functor $\reindex{-}{f}: \Func{Y}{\cat{C}} \to
\Func{X}{\cat{C}}$ which sends any $Y$-indexed family $F$ over $\cat{C}$ to the $X$-indexed family
$\reindex{F}{f} = F \comp f$ (regarding $f$ as a functor between $X$ and $Y$ as discrete categories) and any
family of morphisms $\eta: F \to G$ to $\reindex{\eta}{f}: \reindex{F}{f} \to \reindex{G}{f}$ with
$\reindex{\eta}{f}(x) = \eta(f(x))$.
\end{definition}

Although reindexing is just precomposition with $f$, writing it as $\reindex{-}{f}$ avoids notational
confusion when combining composition of morphisms in $\Func{X}{C}$ with reindexing.

\begin{definition}
\label{def:Grothendieck}
\todo{Move earlier because doesn't assume $\cat{C} = \Set$.} Suppose an indexed category $T: \cat{C}^\op \to
\Cat$. The \emph{Grothendieck construction} for $T$ is the category $\Grothendieck{X}T(X)$ (also written
$\Grothendieck{}T$) which has as objects, all pairs $(X, \partial X)$ of an object $X$ of $\cat{C}$ and an
object $\partial X$ of $T(X)$, and as morphisms $(X, \partial X) \to (Y, \partial Y)$, all pairs $(f, \partial
f)$ of morphisms $f: X \to Y$ in $\cat{C}$ and morphisms $\partial f: \partial X \to T(f)(\partial Y)$ in
$T(X)$.
\end{definition}

\noindent The $\partial X$ notation is intended to suggest the connection between the Grothendieck
construction and the idea of tangent spaces and derivatives. To understand how the $\partial f$ components
compose, consider any $(f, \partial f): (X, \partial X) \to (Y, \partial Y)$ and $(g, \partial g): (Y,
\partial Y) \to (Z, \partial Z)$ in $\Grothendieck{}T$. We have:

\begin{itemize}
\item $\partial f: \partial X \to T(f)(\partial Y)$, and
\item $T(f)(\partial g): T(f)(\partial Y) \to T(f)(T(g)(\partial Z)) = T(g \comp f)(\partial Z)$
\end{itemize}

\noindent where $T(f)(\partial g)$ is the image of $\partial g$ in the ``reindexing'' functor $T(f)$. The
composition $(g, \partial g) \comp (f, \partial f): (X, \partial X) \to (Z, \partial Z)$ is then given by $(g
\comp f, T(f)(\partial g) \comp \partial f)$.

\subsection{Category of Families}
\label{sec:Fam}

Of interest to us is the Grothendieck construction for $\Func{-}{\cat{C}}: \Set^{\op} \to \Cat$, the indexed
category which sends any set $X$ to the category of $X$-indexed families $\Func{X}{\cat{C}}$ and any function
$f: X \to Y$ to the reindexing functor $\reindex{-}{f}: \Func{Y}{\cat{C}} \to \Func{X}{\cat{C}}$ .

\begin{definition}[Categories of families]
\label{def:Fam}
For any category $\cat{C}$ define $\Fam(\cat{C})$ to be the Grothendieck construction
$\Grothendieck{X}\Func{X}{\cat{C}}$.
\end{definition}

\noindent In $\Fam(\cat{C})$ the objects $(X, \partial X)$ are thus sets $X$ paired with indexed families
$\partial X: X \to \cat{C}$ and morphisms $(f, \partial f): (X, \partial X) \to (Y, \partial Y)$ are functions
$f: X \to Y$ paired with morphisms $\partial f: \partial X \to \partial \reindex{Y}{f}$ in
$\Func{X}{\cat{C}}$.

\subsection{Galois Slicing via a Category of Families}
\label{sec:fam:galois-slicing}

The total category $\Fam(\LatGal)$ is then basic setting for interpreting first-order programs for \GPS. This
has as objects $(X, \partial X)$, all pairs of a set $X$ and and for every $x \in X$, a bounded lattice
$\partial X(x)$, and as morphisms $(X, \partial X) \to (Y, \partial Y)$, all pairs $(f, \partial f)$ of a
function $f: X \to Y$ and for every $x \in X$, a Galois connection $\partial f(x): \partial X(x) \to \partial
Y(f(x))$. This is precisely the setup we identified informally at the beginning of this section. The
reindexing along $f$ in the composition $(g, \partial g) \comp (f, \partial f) = (g \comp f, \reindex{\partial
g}{f} \comp \partial f)$ selects the appropriate lattice of approximations and Galois connection at each
point, using the baseline (unapproximated) function $f$.

\subsection{Automatic Differentiation via a Category of Families}

Following the CHAD approach, automatic differentiation for first-order programs can also be recovered in this
setup, via the total category $\Fam(\FinVect)$. We recapitulate the basic idea here. For flat manifolds like
$\RR^n$, the tangent space and cotangent space at a point $x$ are both canonically isomorphic to $\RR^n$, so
the forward and backward derivatives of a differentiable function $f: \RR^m \to \RR^n$ at a point $x$ have the
following form:

\begin{itemize}
\item The \emph{forward derivative} (tangent map or pushforward) $\pushf{f}_x$ of $f$ at $x \in \RR^m$ is the
unique linear map $\RR^m \linearto \RR^n$ given by the Jacobean matrix of $f$ at $x$.
\item The \emph{backward derivative} (cotangent map or pullback) $\pullf{f}_x$ is the unique linear map
$\RR^n \linearto \RR^m$ given by the transpose (adjoint) of the Jacobean matrix of $f$ at $x$.
\end{itemize}

\subsubsection{Chain Rule}

Derivatives respect composition~\cite{spivak65}. Suppose $f: \RR^m \to \RR^n$ and $g: \RR^n \to \RR^k$. For
any $x \in \RR^m$ we have:

\begin{itemize}
\item $\pushf{(g \comp f)}_x = \pushf{g}_{f(x)} \comp \pushf{f}_x: \RR^m \to \RR^k$
\item $\pullf{(g \comp f)}_x = \pullf{f}_{x} \comp \pullf{g}_{f(x)}: \RR^k \to \RR^m$
\end{itemize}

\subsubsection{Forward-Mode Automatic Differentiation}

The basic idea of (forward-mode) automatic differentiation, as first described by \citet{linnainmaa76}, was to
recognise that one could efficiently compute the pushforward map $\pushf{f}: \RR^m \to (\RR^m \linearto
\RR^n)$ of a differentiable function $f: \RR^m \to \RR^n$ alongside the computation of $f$ itself by defining
a single map $\tangents(f) = x \mapsto (f(x), \pushf{f}_x): \RR^m \to \RR^n \times (\RR^m \linearto \RR^n)$.
Defining $f_1 = \pi_1 \comp f: \RR^m \to \RR^n$ and $f_2 = \pi_2 \comp f: \RR^m \to (\RR^m \linearto \RR^n)$,
composition of such maps can be expressed as:
\begin{align*}
(g \comp f)_1(x) &= f_1(g_1(x)) \\
(g \comp f)_2(x) &= g_2(f_1(x)) \comp f_2(x)
\end{align*}

We can verify that $\tangents$ is functorial, regardless of whether the linear maps $\RR^m \linearto \RR^n$
happen to be derivatives. But if $f_2(x)$ is the derivative of $f_1$ at $x$ for any $x \in \RR^m$ and
similarly for $g$ then the derivatives compose according to the forward chain rule and $(g \comp f)_2(x)$ is
the derivative of $(g \comp f)_1$ at $x$. Thus this setup provides a way of computing the forward derivative
of a composite function, without have to recompute the function itself, as long as all atomic operations
provide derivatives. An efficient implementation of \GPS based on $\Fam(\LatGal)$ (\secref{fam:galois-slicing}
above) would most likely apply a similar optimisation to make it possible to compute forwards slices without
having to reexecute the program at the same time.

We can see these ``optimised'' forward-mode maps as living in $\Fam(\FinVect)$ if we regard every finite
vector space $\RR^n$ as paired with the constant family of tangent spaces at $\RR^n$ (also written $\RR^n$,
for convenience), and each map $f: \RR^m \to \RR^n \times (\RR^m \linearto \RR^n)$ as the pair $(f_1, f_2)$.
Then the composition $(g, \partial g) \comp (f, \partial f) = (g \comp f, \reindex{\partial g}{f} \comp
\partial f)$ corresponds exactly to the forward chain rule, since $(\reindex{\partial g}{f} \comp \partial
f)_x = {\partial g}_{f(x)} \comp {\partial f}_x$, with the reindexing along $f$ selecting the appropriate
pushforward by recomputing $f(x)$.

\subsubsection{Reverse-Mode Automatic Differentiation}

\todo{brief summary}
