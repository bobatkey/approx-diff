\section{Conclusion and Future Work}
\label{sec:conclusion}

We have presented a semantic version of \GPS, inspired by connections to differentiable programming and
automatic differentation, and shown that it can be used to interpret an expressive higher-order language
suitable for writing simple queries and data manipulation. Our model elucidates some of the decisions
implicitly taken in previous works on \GPS (\secref{cbn-translation}), and reveals new applications such as
approximation by intervals \secref{interval-approx}. Our categorical approach admits a modular construction of
our model, and the use of general theorems, such as Fiore and Simpson's \thmref{glr-definability}, to prove
properties of the interpretation. We have focused on constructions that enable an executable implementation in
Agda in this work, but have conjectured that there are connections to established notions of categorical
differentiation in \conref{tangent-stable-fns} and \conref{tangent-fam}.

\paragraph{Quantitative slicing and XAI}  Explainable AI (XAI) techniques like Gradient-weighted Class
Activation Mapping (Grad-CAM)~\cite{selvaraju20} use reverse-mode AD selectively to calculate \emph{heat maps}
(or \emph{saliency maps}) that highlight input regions contributing to a given classification or other
outcome. We would like to investigate quantitative approximation structures where $\top$ represents the
original input image and lower elements represent ``slices'' of the image where individual pixels have been
ablated to some degree (partly removed or blurred). This might allow for composing some of these techniques
with Galois slicing, for use in hybrid systems such as physical simulations with ML-based parameterisations.

\paragraph{Refinement of the model} As we discussed in \remref{further-structure}, there are possible ways
that the model we have proposed here could be refined to both remove nonsensical elements of the model, and to
augment the model with enough power to prove additional properties such as ``functions are insenstive to
unused inputs''. We are also planning to explore more examples of approximation along the lines of the
intervals example in \secref{interval-approx}. One route might be to follow \citet{edalat-heckmann98}'s
embedding of metric spaces in Scott domains and explore whether metric spaces (which already provide a native
notion of approximation) can be embedded in $\Fam(\LatGal)$.

\paragraph{Categorical Models of Differentation} We have already identified \conref{tangent-stable-fns} and
\conref{tangent-fam} as potential connections to categorical models of differentation. It would also be
interesting to discover any possible links to Differential Linear Logic
\cite{ehrhard_differential_2006}. Differential Linear Logic and the Dialectica translation have been used to
give an account of reverse differentiation by \citet{kerjean-pedrot2024} that is similar to the CHAD approach
of \citet{vakar22}.

\paragraph{General Inductive and Coinductive Types}

Lists are the only recursive data type we provided in our source language, so important future work is
supporting general inductive and coinductive types. \citet{nunes2023} support automatic differentiation for
datatypes defined as the least or greatest fixed points of $\mu\nu$-polynomial functors; we could potentially
adopt a similar approach. Full inductive types would allow us to embed an interpreter for a small language;
combined with the CBN monadic translation described in \secref{cbn-translation} which uniformly inserts
approximation points, we should be able to obtain the program slicing behaviour of earlier \GPS work ``for
free''. Coinductive types (e.g.~streams) present additional challenges, especially for defining
join-preserving backward maps, but also open the door to slicing (finite prefixes of) infinite data sources,
with some likely relationship to the problem of dealing with partial or non-terminating computations.

\paragraph{Recursion and Partiality} This work has only examined the case for total programs. Even though we
took stable domain theory as our starting point, we did not make any use of directed completeness or similar
properties of domains. We expect that an account of recursion in an extension of the framework discussed so
far would likely involve families of bounded lattices indexed by DCPOs, where the order of the DCPO would be
reflected in embedding-projection relationships between the lattices. There are also other refinements of
stable domain theory such as \citet{berry79}'s bidomains and \citet{laird07} bistable biorders that separate
the extensional and stable orders on the same set, in a way that might be similar to
\exref{intervals-and-maxima-elements}.

\paragraph{Source-To-Source Translation Techniques}

% To formalise this, we'd need to... See also $\Fam(\PSh_{\CMon}(\namedcat{Syn}))$?

An interesting alternative to the denotational approach presented here, and to the trace-based approaches used
in earlier \GPS implementations, would be to develop a source-to-source transformation, in direct analogy with
the CHAD approach to automatic differentiation \cite{vakar22,nunes2023}. In their approach, forward and
reverse-mode AD are implemented as compositional transformations on source code, guided by a universal
property: they arise as the unique structure-preserving functors from the source language to a suitably
structured target language formalised as a Grothendieck construction. Adapting this to Galois slicing would
allow slicing to ``compiled in'', avoiding the need for a custom interpreter and potentially exposing
opportunities for optimisation.
