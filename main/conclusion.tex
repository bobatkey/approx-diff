\section{Conclusion and Future Work}
\label{sec:conclusion}

\paragraph{Integrating XAI techniques with Galois slicing.} \roly{Sketch what this might look like.}
Explainable AI (XAI) techniques like Gradient-weighted Class Activation Mapping (Grad-CAM)~\cite{selvaraju20}
use reverse-mode AD selectively on specific activations, calculating \emph{heat maps} that highlight input
regions contributing to a given classification decision. It might be possible to compose these techniques with
\GPS whilst still preserving useful round-tripping properties; if so this could be a basis for end-to-end
explainability in systems that combine neural networks with traditional symbolic computation.

\paragraph{More Approximation Domains.} The intervals example is
interesting, but is it useful, and are there more examples like this?
One route might be to follow \cite{edalat-heckmann98} and explore
whether metric spaces (which already provide a native notion of
approximation) can be embedded in $\Fam(\LatGal)$.

\paragraph{Alternative Higher-order Models}

1. $\Fam(\PSh_{\CMon}(\LatGal))$

2. Concrete sheaves over $\Fam(\LatGal)$ analogous to Diffeological spaces?

\paragraph{Categorical Models of Differentation}



\paragraph{Recursion and Partiality.}
1. Despite using stable domain theory as our starting point, we have ignored

2. Conjecture that the category with $(X : Cpo, \partial X : X \to \CLatGal)$, where $\CLatGal$ is the category of complete lattices and galois connections is the right setting. Would need to investigate the connections with dI-domains, bidomains, and bistable bidomains (Laird).

\paragraph{Implementation}

Would be better to have an implementation that worked by translation,
instead of interpreting directly as Agda functions. See work by
Smeding and Vákár on translations. To formalise this, we'd need
to... See also $\Fam(\PSh_{\CMon}(Syn))$?
