\section{Higher-order language}

We introduce a total higher-order functional language, over a set $\PrimTy$ of primitive types $\rho$ and sets
$\PrimOp^\rho_{\rho_1,\ldots,\rho_n}$ of primitive operations $\phi$, equipped with sums, products, functions
and a data type of lists. We omit a treatment of general inductive or coinductive types.

\subsubsection{Syntax}
\label{sec:language:syntax}

\begin{figure}
  \begin{subfigure}[t]{0.48\linewidth}
  \small
  \[
  \begin{array}{lllll}
    & \textit{Types}
    \\
    &
    \sigma, \tau
    & ::= &
    \rho
    &
    \text{primitive type}
    \\
    && \mid &
    \sigma \tySum \tau
    &
    \text{sum}
    \\
    && \mid &
    \tyUnit
    &
    \text{unit}
    \\
    && \mid &
    \sigma \tyProd \tau
    &
    \text{product}
    \\
    && \mid &
    \sigma \tyFun \tau
    &
    \text{function}
    \\
    && \mid &
    \tyList\;\tau
    &
    \text{list}
    \\
    && \mid &
    \tyLift\;\tau
    &
    \text{lifting}
  \end{array}
  \]
  \end{subfigure}%
  \begin{subfigure}[t]{0.48\linewidth}
  \small
  \[
  \begin{array}{lllll}
    & \textit{Terms}
    \\
    &
    t, s
    & ::= &
    x
    &
    \text{variable}
    \\
    && \mid &
    \phi(\vec t)
    &
    \text{primitive op}
    \\
    && \mid &
    \tmInl{t} \mid \tmInr{t}
    &
    \text{injection}
    \\
    && \mid &
    \tmCase{s}{x}{t_1}{y}{t_2}
    &
    \text{case}
    \\
    && \mid &
    \tmUnit
    &
    \text{unit}
    \\
    && \mid &
    \tmPair{s}{t}
    &
    \text{pair}
    \\
    && \mid &
    \tmFst{t} \mid \tmSnd{t}
    &
    \text{projection}
    \\
    && \mid &
    \tmFun{x}{t}
    &
    \text{function}
    \\
    && \mid &
    \tmApp{s}{t}
    &
    \text{application}
    \\
    && \mid &
    \tmNil
    &
    \text{nil}
    \\
    && \mid &
    \tmCons{s}{t}
    &
    \text{cons}
    \\
    && \mid &
    \tmFoldList{s_1}{s_2}{t}
    &
    \text{fold}
    \\
    && \mid &
    \tmReturn{t}
    &
    \text{return}
    \\
    && \mid &
    \tmBind{s}{t}
    &
    \text{bind}
  \end{array}
  \]
  \end{subfigure}
  \caption{Syntax of types and terms}
  \label{fig:syntax}
\end{figure}

\begin{figure}
\begin{subfigure}{\linewidth}
  \begin{mathpar}
  \small
  \inferrule*
  {
    \alpha: \kType \in \Delta
  }
  {
    \Delta \vdash \alpha: \kType
  }
  \and
  \inferrule*
  {
    \strut
  }
  {
    \Delta \vdash \tyZero: \kType
  }
  \and
  \inferrule*
  {
    \Delta \vdash \sigma: \kType
    \\
    \Delta \vdash \tau: \kType
  }
  {
    \Delta \vdash \sigma \tySum \tau: \kType
  }
  \and
  \inferrule*
  {
    \strut
  }
  {
    \Delta \vdash \tyUnit: \kType
  }
  \and
  \inferrule*
  {
    \Delta \vdash \sigma: \kType
    \\
    \Delta \vdash \tau: \kType
  }
  {
    \Delta \vdash \sigma \tyProd \tau: \kType
  }
  \and
  \inferrule*
  {
    \Delta \vdash \sigma: \kType
    \\
    \Delta \vdash \tau: \kType
  }
  {
    \Delta \vdash \sigma \tyFun \tau: \kType
  }
  \and
  \inferrule*
  {
    \Delta, \alpha: \kType \vdash \tau: \kType
    \\
    \Pol(+,\alpha,\tau)
  }
  {
    \Delta \vdash \mu\alpha.\tau: \kType
  }
  \and
  \inferrule*
  {
    \Delta \vdash \tau: \kType
  }
  {
    \Delta \vdash \tyLift\;\tau: \kType
  }
  \end{mathpar}
  \caption{Well-kinded types}
\end{subfigure}
\begin{subfigure}{\linewidth}
  \begin{mathpar}
    \small
    \inferrule*
    {
      x : \tau \in \Gamma
    }
    {
      \Gamma \vdash x: \tau
    }
    \and
    \inferrule*
    {
      \Gamma \vdash t : \sigma
    }
    {
      \Gamma \vdash \tmInl{t}: \sigma \tySum \tau
    }
    \and
    \inferrule*
    {
      \Gamma \vdash t : \tau
    }
    {
      \Gamma \vdash \tmInr{t}: \sigma \tySum \tau
    }
    \and
    \inferrule*
    {
      \Gamma \vdash s : \sigma \tySum \tau
      \\
      \Gamma, x: \sigma \vdash t_1 : \tau'
      \\
      \Gamma, y : \tau \vdash t_2 : \tau'
    }
    {
      \Gamma \vdash \tmCase{s}{x}{t_1}{y}{t_2}: \tau'
    }
    \and
    \inferrule*
    {
      \strut
    }
    {
      \Gamma \vdash \tmUnit : \tyUnit
    }
    \and
    \inferrule*
    {
      \Gamma \vdash s : \sigma
      \\
      \Gamma \vdash t : \tau
    }
    {
      \Gamma \vdash \tmPair{s}{t}: \sigma \tyProd \tau
    }
    \and
    \inferrule*
    {
      \Gamma \vdash t : \sigma \tyProd \tau
    }
    {
      \Gamma \vdash \tmFst{t}: \sigma
    }
    \and
    \inferrule*
    {
      \Gamma \vdash t : \sigma \tyProd \tau
    }
    {
      \Gamma \vdash \tmSnd{t}: \tau
    }
    \and
    \inferrule*
    {
      \Gamma, x: \sigma \vdash t : \tau
    }
    {
      \Gamma \vdash \tmFun{x}{t}: \sigma \tyFun \tau
    }
    \and
    \inferrule*
    {
      \Gamma \vdash s: \sigma \tyFun \tau
      \\
      \Gamma \vdash t : \sigma
    }
    {
      \Gamma \vdash \tmApp{s}{t}: \tau
    }
    \and
    \inferrule*
    {
      \Gamma \vdash t : \subst{\tau}{\mu \alpha.\tau}{\alpha}
    }
    {
      \Gamma \vdash \tmRoll{t}: \mu\alpha.\tau
    }
    \and
    \inferrule*
    {
      \Gamma \vdash s : \subst{\sigma}{\tau}{\alpha} \tyFun \tau
      \\
      \Gamma \vdash t : \mu\alpha.\sigma
    }
    {
      \Gamma \vdash \tmFold{s}{t} : \tau
    }
    \and
    \inferrule*
    {
      \Gamma \vdash t : \tau
    }
    {
      \Gamma \vdash \tmReturn{t} : \tyLift\;\tau
    }
    \and
    \inferrule*
    {
      \Gamma \vdash s : \tyLift\;\sigma
      \\
      \Gamma \vdash t : \sigma \tyFun \tyLift\;\tau
    }
    {
      \Gamma \vdash \tmBind{s}{t} : \tyLift\;\tau
    }
  \end{mathpar}
  \caption{Well-typed terms (all types well-kinded)}
\end{subfigure}
\caption{Kinding and typing rules}
\label{fig:typing}
\end{figure}


\subsubsection{Semantics}
\label{sec:language:semantics}

% \begin{figure}
\begin{subfigure}{\linewidth}
  \begin{mathpar}
  \small
    \inferrule*
    {
      \strut
    }
    {
      \cat{C} \in \Ob(\muPoly_{\cat{C}})
    }
    \and
    \inferrule*
    {
      \strut
    }
    {
      \One \in \Ob(\muPoly_{\cat{C}})
    }
    \and
    \inferrule*
    {
      \cat{D} \in \Ob(\muPoly_{\cat{C}})
      \\
      \cat{D}' \in \Ob(\muPoly_{\cat{C}})
    }
    {
      \cat{D} \times \cat{D}' \in \Ob(\muPoly_{\cat{C}})
    }
  \end{mathpar}
  \caption{Objects}
\end{subfigure}
\begin{subfigure}{\linewidth}
  \begin{mathpar}
  \small
    \inferrule*
    {
      \strut
    }
    {
      !_{\cat{D}} : \cat{D} \to \One \in \Mor(\muPoly_{\cat{C}})
    }
    \and
    \inferrule*
    {
      \strut
    }
    {
      F : \One \to \cat{D} \in \Mor(\muPoly_{\cat{C}})
    }
    \and
    \inferrule*
    {
      \strut
    }
    {
      - \times - : \cat{C} \times \cat{C} \to \cat{C} \in \Mor(\muPoly_{\cat{C}})
    }
    \and
    \inferrule*
    {
      \strut
    }
    {
      - \textstyle\coprod - : \cat{C} \times \cat{C} \to \cat{C} \in \Mor(\muPoly_{\cat{C}})
    }
    \and
    \inferrule*
    {
      \cat{D} \in \Ob(\muPoly_{\cat{C}})
      \\
      \cat{D}' \in \Ob(\muPoly_{\cat{C}})
    }
    {
      \pi_1 : \cat{D} \times \cat{D}' \to \cat{D} \in \Mor(\muPoly_{\cat{C}})
    }
    \and
    \inferrule*
    {
      \cat{D} \in \Ob(\muPoly_{\cat{C}})
      \\
      \cat{D}' \in \Ob(\muPoly_{\cat{C}})
    }
    {
      \pi_2 : \cat{D} \times \cat{D}' \to \cat{D}' \in \Mor(\muPoly_{\cat{C}})
    }
    \and
    \inferrule*
    {
      \cat{E} \in \Ob(\muPoly_{\cat{C}})
      \\
      \cat{D} \in \Ob(\muPoly_{\cat{C}})
      \\
      \cat{D}' \in \Ob(\muPoly_{\cat{C}})
      \\
      F: \cat{E} \to \cat{D} \in \Mor(\muPoly_{\cat{C}})
      \\
      G: \cat{E} \to \cat{D}' \in \Mor(\muPoly_{\cat{C}})
    }
    {
      \prodM{F}{G} : \cat{E} \to \cat{C} \times \cat{D}' \in \Mor(\muPoly_{\cat{C}})
    }
    \and
    \inferrule*
    {
      \text{TODO}
    }
    {
      \mu F : \cat{D} \to \cat{C} \in \Mor(\muPoly_{\cat{C}})
    }
  \end{mathpar}
  \caption{Morphisms}
\end{subfigure}
\caption{Rules inductively defining $\muPoly_{\cat{C}}$ for any $\cat{C}$ with finite coproducts and finite
products}
\label{fig:mu-polynomial}
\end{figure}

% \begin{figure}
  \begin{align*}
  \small
  \sem{\vec\alpha \vdash \alpha_i}(\vec X) &=
  X_i
  \\
  \sem{\Delta \vdash \rho}(\vec X) &=
  \sem{\rho}
  \\
  \sem{\Delta \vdash \tyZero}(\vec X) &=
  0
  \\
  \sem{\Delta \vdash \sigma \tySum \tau}(\vec X) &=
  \textstyle {\sem{\sigma}(\vec X)} \coprod {\sem{\tau}(\vec X)}
  \\
  \sem{\Delta \vdash \tyUnit}(\vec X) &=
  1
  \\
  \sem{\Delta \vdash \sigma \tyProd \tau}(\vec X) &=
  \sem{\Delta \vdash \sigma}(\vec X) \times \sem{\Delta \vdash \tau}(\vec X)
  \\
  \sem{\Delta \vdash \sigma \tyFun \tau}(\vec X) &=
  \internalHom{\sem{\Delta \vdash \sigma}(\vec X)}{\sem{\Delta \vdash \tau}(\vec X)}
  \\
  \sem{\vec\alpha \vdash \mu\alpha_i.\tau}(\vec X) &=
  \mu(\sem{\vec\alpha_{\neq \alpha_i}, \alpha \vdash \tau}^{\vec X_{\neq X_i}})
  \\
  \sem{\Delta \vdash \mu\alpha.\tau}(\vec X) &=
  \mu(\sem{\Delta, \alpha \vdash \tau}^{\vec X})
  \textit{ if }\alpha \notin \Delta
  \\
  \sem{\Delta \vdash \tyLift\;\tau}(\vec X) &=
  \mathcal{L}(\sem{\Delta \vdash \tau}(\vec X))
  \end{align*}
  \caption{Parameterised types $\Delta \vdash \tau$ as functors $\Sem^n \to \Sem$ \todo{Revisit; also action
  on morphisms important in $\mu F$ case}}
  \label{fig:semantics:types}
\end{figure}

% \figref{semantics:types} gives the interpretation of types $\Delta \vdash \tau$ with $\length{\Delta}
% = n$ as functors $\Sem^n \to \Sem$, or as objects of $\Sem$ in the case $n = 0$.
% \todo{Notation for partially applied functor used in \figref{semantics:types}}
%
% \begin{definition}[$\mu$-polynomial]
% Suppose $\cat{C}$ a category with finite coproducts $(\coprod, 0)$ and finite products $(\times, 1)$. Define
% $\muPoly_{\cat{C}}$ to be the smallest subcategory of $\Cat$ generated inductively by the rules in
% \figref{mu-polynomial}.
% \end{definition}
%
% For any $\mu$-polynomial endofunctor $F: \cat{C} \to \cat{C}$ with initial algebra $\mu F$, write $\inMap_F$
% for the structure map $F\mu F \to \mu F$ and write $\cata_f$ for the unique morphism $\mu F \to X$ from the
% initial $F$-algebra to any $F$-algebra $(X, f: FX \to X)$.
