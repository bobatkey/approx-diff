\section{Higher-Order Language}
\label{sec:language}

To model Galois slicing semantically for higher-order programs, we define a simple total functional
programming language, extending the simply-typed lambda calculus. The language is parameterised by a signature
$\Sigma = (\PrimTy, \PrimOp)$ consisting of a set $\PrimTy$ of base types $\rho$ and a family of sets
$\PrimOp^\rho_{\rho_1,\ldots,\rho_n}$ of primitive operations $\phi$ of arity $n$ over those base types.

\subsection{Syntax}
\label{sec:language:syntax}

The syntax is defined in \figref{syntax}. Types includes base types $\rho$ drawn from $\PrimTy$, along with
standard type formers for sums, products, functions and lists% , plus a type constructor $\tyLift$ to allow
% approximation points to be added explicitly, as discussed in \secref{models-of-total-gps}
. Terms include
variables, the usual introduction and elimination forms% , a monadic return and bind for lifted terms
, and
primitive operations $\phi$.

The language is intentionally minimal: it excludes general recursion, and general inductive or coinductive
types, which we will consider in future work (\secref{conclusion}). Typing judgments for terms are standard
and shown in \figref{typing}, with the usual rules for products, sums, functions, and lists.

\begin{figure}
  \begin{subfigure}[t]{0.48\linewidth}
  \small
  \[
  \begin{array}{lllll}
    & \textit{Types}
    \\
    &
    \sigma, \tau
    & ::= &
    \rho
    &
    \text{primitive type}
    \\
    && \mid &
    \sigma \tySum \tau
    &
    \text{sum}
    \\
    && \mid &
    \tyUnit
    &
    \text{unit}
    \\
    && \mid &
    \sigma \tyProd \tau
    &
    \text{product}
    \\
    && \mid &
    \sigma \tyFun \tau
    &
    \text{function}
    \\
    && \mid &
    \tyList\;\tau
    &
    \text{list}
    \\
    && \mid &
    \tyLift\;\tau
    &
    \text{lifting}
  \end{array}
  \]
  \end{subfigure}%
  \begin{subfigure}[t]{0.48\linewidth}
  \small
  \[
  \begin{array}{lllll}
    & \textit{Terms}
    \\
    &
    t, s
    & ::= &
    x
    &
    \text{variable}
    \\
    && \mid &
    \phi(\vec t)
    &
    \text{primitive op}
    \\
    && \mid &
    \tmInl{t} \mid \tmInr{t}
    &
    \text{injection}
    \\
    && \mid &
    \tmCase{s}{x}{t_1}{y}{t_2}
    &
    \text{case}
    \\
    && \mid &
    \tmUnit
    &
    \text{unit}
    \\
    && \mid &
    \tmPair{s}{t}
    &
    \text{pair}
    \\
    && \mid &
    \tmFst{t} \mid \tmSnd{t}
    &
    \text{projection}
    \\
    && \mid &
    \tmFun{x}{t}
    &
    \text{function}
    \\
    && \mid &
    \tmApp{s}{t}
    &
    \text{application}
    \\
    && \mid &
    \tmNil
    &
    \text{nil}
    \\
    && \mid &
    \tmCons{s}{t}
    &
    \text{cons}
    \\
    && \mid &
    \tmFoldList{s_1}{s_2}{t}
    &
    \text{fold}
    \\
    && \mid &
    \tmReturn{t}
    &
    \text{return}
    \\
    && \mid &
    \tmBind{s}{t}
    &
    \text{bind}
  \end{array}
  \]
  \end{subfigure}
  \caption{Syntax of types and terms}
  \label{fig:syntax}
\end{figure}

\begin{figure}
\begin{subfigure}{\linewidth}
  \begin{mathpar}
  \small
  \inferrule*
  {
    \alpha: \kType \in \Delta
  }
  {
    \Delta \vdash \alpha: \kType
  }
  \and
  \inferrule*
  {
    \strut
  }
  {
    \Delta \vdash \tyZero: \kType
  }
  \and
  \inferrule*
  {
    \Delta \vdash \sigma: \kType
    \\
    \Delta \vdash \tau: \kType
  }
  {
    \Delta \vdash \sigma \tySum \tau: \kType
  }
  \and
  \inferrule*
  {
    \strut
  }
  {
    \Delta \vdash \tyUnit: \kType
  }
  \and
  \inferrule*
  {
    \Delta \vdash \sigma: \kType
    \\
    \Delta \vdash \tau: \kType
  }
  {
    \Delta \vdash \sigma \tyProd \tau: \kType
  }
  \and
  \inferrule*
  {
    \Delta \vdash \sigma: \kType
    \\
    \Delta \vdash \tau: \kType
  }
  {
    \Delta \vdash \sigma \tyFun \tau: \kType
  }
  \and
  \inferrule*
  {
    \Delta, \alpha: \kType \vdash \tau: \kType
    \\
    \Pol(+,\alpha,\tau)
  }
  {
    \Delta \vdash \mu\alpha.\tau: \kType
  }
  \and
  \inferrule*
  {
    \Delta \vdash \tau: \kType
  }
  {
    \Delta \vdash \tyLift\;\tau: \kType
  }
  \end{mathpar}
  \caption{Well-kinded types}
\end{subfigure}
\begin{subfigure}{\linewidth}
  \begin{mathpar}
    \small
    \inferrule*
    {
      x : \tau \in \Gamma
    }
    {
      \Gamma \vdash x: \tau
    }
    \and
    \inferrule*
    {
      \Gamma \vdash t : \sigma
    }
    {
      \Gamma \vdash \tmInl{t}: \sigma \tySum \tau
    }
    \and
    \inferrule*
    {
      \Gamma \vdash t : \tau
    }
    {
      \Gamma \vdash \tmInr{t}: \sigma \tySum \tau
    }
    \and
    \inferrule*
    {
      \Gamma \vdash s : \sigma \tySum \tau
      \\
      \Gamma, x: \sigma \vdash t_1 : \tau'
      \\
      \Gamma, y : \tau \vdash t_2 : \tau'
    }
    {
      \Gamma \vdash \tmCase{s}{x}{t_1}{y}{t_2}: \tau'
    }
    \and
    \inferrule*
    {
      \strut
    }
    {
      \Gamma \vdash \tmUnit : \tyUnit
    }
    \and
    \inferrule*
    {
      \Gamma \vdash s : \sigma
      \\
      \Gamma \vdash t : \tau
    }
    {
      \Gamma \vdash \tmPair{s}{t}: \sigma \tyProd \tau
    }
    \and
    \inferrule*
    {
      \Gamma \vdash t : \sigma \tyProd \tau
    }
    {
      \Gamma \vdash \tmFst{t}: \sigma
    }
    \and
    \inferrule*
    {
      \Gamma \vdash t : \sigma \tyProd \tau
    }
    {
      \Gamma \vdash \tmSnd{t}: \tau
    }
    \and
    \inferrule*
    {
      \Gamma, x: \sigma \vdash t : \tau
    }
    {
      \Gamma \vdash \tmFun{x}{t}: \sigma \tyFun \tau
    }
    \and
    \inferrule*
    {
      \Gamma \vdash s: \sigma \tyFun \tau
      \\
      \Gamma \vdash t : \sigma
    }
    {
      \Gamma \vdash \tmApp{s}{t}: \tau
    }
    \and
    \inferrule*
    {
      \Gamma \vdash t : \subst{\tau}{\mu \alpha.\tau}{\alpha}
    }
    {
      \Gamma \vdash \tmRoll{t}: \mu\alpha.\tau
    }
    \and
    \inferrule*
    {
      \Gamma \vdash s : \subst{\sigma}{\tau}{\alpha} \tyFun \tau
      \\
      \Gamma \vdash t : \mu\alpha.\sigma
    }
    {
      \Gamma \vdash \tmFold{s}{t} : \tau
    }
    \and
    \inferrule*
    {
      \Gamma \vdash t : \tau
    }
    {
      \Gamma \vdash \tmReturn{t} : \tyLift\;\tau
    }
    \and
    \inferrule*
    {
      \Gamma \vdash s : \tyLift\;\sigma
      \\
      \Gamma \vdash t : \sigma \tyFun \tyLift\;\tau
    }
    {
      \Gamma \vdash \tmBind{s}{t} : \tyLift\;\tau
    }
  \end{mathpar}
  \caption{Well-typed terms (all types well-kinded)}
\end{subfigure}
\caption{Kinding and typing rules}
\label{fig:typing}
\end{figure}


\subsection{Semantics}
\label{sec:language:semantics}

\begin{figure}
  \begin{subfigure}[t]{0.47\linewidth}
    \small
    \begin{align*}
      \sem{\rho} &= \sem{\rho}_{\PrimTy}
      \\
      \sem{\sigma \tySum \tau} &= \textstyle {\sem{\sigma}} + {\sem{\tau}}
      \\
      \sem{\tyUnit} &= 1
      \\
      \sem{\sigma \tyProd \tau} &= \sem{\sigma} \times \sem{\tau}
      \\
      \sem{\sigma \tyFun \tau} &= \internalHom{\sem{\sigma}}{\sem{\tau}}
      \\
      \sem{\tyList\;\tau} &= \List(\sem{\tau})
      % \\
      % \sem{\tyLift\;\tau} &= \Lift(\sem{\tau})
    \end{align*}
    \caption{Interpretation of Types}
    \label{fig:semantics:types}
  \end{subfigure}
  \begin{subfigure}[t]{0.47\linewidth}
    \small
    \begin{align*}
      \sem{\emptyCxt} &= 1
      \\
      \sem{\Gamma, x: \tau} &= \sem{\Gamma} \times \sem{\tau}
    \end{align*}
    \caption{Interpretation of Contexts}
    \label{fig:semantics:contexts}
\end{subfigure}
\begin{subfigure}{0.8\linewidth}
  \small
  \begin{align*}
  \sem{x_i} &= \pi_i
  \\
  \sem{\phi(t_1, \ldots, t_n)}
  &=
  \sem{\phi}_{\Op} \comp \prodM{\sem{t_1}}{\ldots, \sem{t_n}}
  \\
  \sem{\tmInl{t}} &= \mathsf{inj}_1 \comp \sem{t}
  \\
  \sem{\tmInr{t}} &= \mathsf{inj}_2 \comp \sem{t}
  \\
  \sem{\tmCase{s}{x}{t_1}{y}{t_2}} &= \coprodM{\sem{t_1}}{\sem{t_2}} \comp \prodM{\id}{\sem{s}}
  \\
  \sem{\tmUnit} &=\;!_{\sem{\Gamma}}
  \\
  \sem{\tmPair{s}{t}} &= \prodM{\sem{s}}{\sem{t}}
  \\
  \sem{\tmFst{t}} &= \pi_1 \comp \sem{t}
  \\
  \sem{\tmSnd{t}} &= \pi_2 \comp \sem{t}
  \\
  \sem{\tmFun{x}{t}} &= \lambda(\sem{t})
  \\
  \sem{\tmApp{s}{t}} &= \eval \comp \prodM{\sem{s}}{\sem{t}}
%  \\
%  \sem{\tmRoll{t}} &= \inMap_{\sem{\sigma}} \comp \sem{t}
%  \textit{ where }\tau = \mu\alpha.\sigma
  \\
  \sem{\tmNil} &= \nil \comp {!_{\sem{\Gamma}}}
  \\
  \sem{\tmCons{s}{t}} &= \cons \comp \prodM{\sem{s}}{\sem{t}}
  \\
  \sem{\tmFoldList{t_1}{t_2}{s}} &= \fold(\sem{t_1},\sem{t_2}) \comp \prodM{\id}{\sem{s}}
%  \\
%  \sem{\tmFold{s}{t}} &= \eval \comp \prodM{\phi \comp \sem{s}}{\sem{t}}
  % \\
  % \sem{\tmReturn{t}} &= \eta_{\sem{\sigma}} \comp \sem{t}
  % \tag*{($\Gamma \vdash t: \sigma$)}
  % \\
  % \sem{\tmBind{s}{t}} &= \mu_{\sem{\tau}} \comp \Lift(\sem{t}) \comp \mathsf{st}_{\sem{\Gamma},\sem{\sigma}} \comp \prodM{\id}{\sem{s}}
  % \tag*{($\Gamma \vdash t: \sigma \to \tyLift\;\tau$)}
  \end{align*}
  \caption{Terms as morphisms}
  \label{fig:semantics:terms}
\end{subfigure}
\caption{Interpretation of types, contexts and terms in category $\Sem$}
\end{figure}


An interpretation of a signature $\Sigma = (\PrimTy, \PrimOp)$ can be given in any category $\cat{C}$ with
finite products, and assigns to each base type $\rho \in \PrimTy$ an object $\sem{\rho}_{\PrimTy}$ in
$\cat{C}$, and to each primitive operation $\phi \in \PrimOp^\rho_{\rho_1,\ldots,\rho_n}$, a morphism
$\sem{\phi}_{\Op}: \sem{\rho_1}_{\PrimTy} \times \ldots \times \sem{\rho_n}_{\PrimTy} \to
\sem{\rho}_{\PrimTy}$.

Assuming that $\cat{C}$ is bicartesian closed and has a list object (\eqnref{lists}), then we can extend an
interpretation of a signature $\Sigma$ to an interpretation of the whole language over
$\Sigma$. \figref{semantics:types} and \figref{semantics:contexts} define the interpretation of types and
contexts as objects of $\cat{C}$ respectively. Terms are interpreted as morphisms between the interpretations
of the context and type, as defined in \figref{semantics:terms}. We have used the notations $\pi_i$ for
projections, $\prodM{f}{g}$ for pairing, $\coprodM{f}{g}$ for (parameterised) copairing, $!_x$ for morphisms
to the terminal object, and $\lambda$ and $\eval$ for currying and evaluation for exponentials.

For the first-order definability result in \secref{definability}, we will need another interpretation
$\sem{-}_{\mathit{fo}}$ of the first-order types (those constructed from primitive types, sums, unit and
products) in any bicartesian category. Such interpretations are preserved by finite coproduct and coproduct
preserving functors:

\begin{lemma}\label{lem:first-order-agreement-types}
  If $\cat{C}$ and $\cat{D}$ are bicartesian and bicartesian closed categories with interpretations of the
  signature $\Sigma$, $F : \cat{C} \to \cat{D}$ is a bicartesian functor, and
  $F(\sem{\rho}_{\PrimTy}) \cong \sem{\rho}_{\PrimTy}$ for all $\rho$, then for all first-order types $\tau$,
  $F(\sem{\tau}_{\mathit{fo}}) \cong \sem{\tau}$, and similar for contexts only containing first-order types.
\end{lemma}

\subsubsection{Interpretation for Higher-Order \GPS}

Given the above, we can now interpret the language in any of the bicartesian closed categories with list
objects we constructed in \secref{models-of-total-gps}. Specifically, we assume that we have an interpretation
of our chosen signature in $\Fam(\LatGal)$. Signatures are first-order, so it does not matter that
$\Fam(\LatGal)$ is not closed. Any such interpretation can be transported to
$\Fam(\MeetSLat \times \JoinSLat^\op)$ along the functor $H$ from \propref{ho-embedding} because it preserves
finite products. We can then interpret the whole language in $\Fam(\MeetSLat \times \JoinSLat^\op)$.

Interpreting a whole program $\Gamma \vdash t : \tau$ yield morphisms in
$\Fam(\MeetSLat \times \JoinSLat^\op)$ which, as in \secref{semantic-gps}, are triples
$(f, \partial f_f, \partial f_r)$ of the underlying function and the forward and backward approximation
maps. However, unlike in $\Fam(\LatGal)$ are not guaranteed that the forward and backward maps even operate on
the same lattices, let alone form a Galois connection. \lemref{first-order-agreement-types} guarantees that
the lattices agree, but the fact that the pair form a Galois connection is less trivial. We will prove this
property in \secref{definability}.

% Given a model of $\Sigma$ in $\Fam(\LatGal)$, the category $\Fam(\MeetSLat \times \JoinSLat^\op)$ can
% implement $\Sem$. It is bicartesian closed by \corref{mslat-jslat-bcc}, and can model $\tyList$ using infinite
% coproducts (\secref{models-of-total-gps:coproducts-and-products}). The embedding $\HoEmbed: \Fam(\LatGal) \to
% \Fam(\MeetSLat \times \JoinSLat^\op)$ from \propref{ho-embedding} preserves finite products, and so can
% transport each $\sem{\rho}_{\PrimTy}$ and $\sem{\phi}_{\Op}$ into the higher-order setting to provide a model
% of $\Sigma$. Finally the lifting monad in $\Fam(\LatGal)$ (\secref{first-order:lifting}) is also preserved by
% $H$. \roly{don't think we actually show this}

% The correctness of this interpretation is considered in the next section.

\subsection{Examples}

Let the signature
$\Sigma_{\mathsf{num}} = (\{\mathsf{num}\}, \{\mathrm{zero} : \tyUnit \to \mathsf{num}, \mathrm{add} :
\mathsf{num} \times \mathsf{num} \to \mathsf{num}\})$. This signature suffices to write the simple query
function in \exref{introduction-example}, where we interpret the $\mathrm{Label}$ type as the sum
$\tyUnit \tySum \tyUnit$ and the labels $\mathsf{a}$ and $\mathsf{b}$ as $\tmInl{\tmUnit}$ and
$\tmInr{\tmUnit}$. We consider several interpretations of $\Sigma_{\mathsf{num}}$ in $\Fam(\LatGal)$ and their
behaviour on the query from \exref{introduction-example}.
\begin{enumerate}
\item If we take $\sem{\mathsf{num}}_{\PrimTy} = \Disc(\mathbb{R})$, with $\sem{\mathrm{zero}}_{\PrimOp}$ and
  $\sem{\mathrm{add}}_{\PrimOp}$ the embeddings of the usual zero and addition functions via $\Disc$, then the
  resulting interpretation contains no approximation information. The approximation lattices for all types are
  always the trivial one-point lattice, and the forward and reverse approximation maps are necessarily the
  identity.
\item To get interesting approximation information, we take
  $\sem{\mathsf{num}}_{\PrimTy} = \Lift(\Disc(\mathbb{R}))$, using the lifting monad from
  \secref{models-of-total-gps:lifting}, with $\sem{\mathrm{zero}}_{\PrimOp}$ and
  $\sem{\mathrm{add}}_{\PrimOp}$ defined from the unlifted interpretations above and the monad structure. This
  interpretation recovers the behaviour given in \exref{introduction-example}: running the reverse
  approximation map at the input $\mathsf{a}$ with the top approximation maps indeed reveals that only the
  numbers in the rows tagged with that label in the input are used, and likewise for $\mathsf{b}$.
\item More quantitative approximation information can be obtained by using the interval approximation
  interpretation from \exref{intervals-and-maxima-elements} and \secref{interval-approx}. We let
  $\sem{\mathsf{num}}_{\PrimTy} = \mathbb{R}_{\mathit{intv}}$ and interpret addition using the morphism given
  in \secref{interval-approx}.

  Recall that $\mathrm{query}\,\mathsf{a}\,\mathit{db} = 1$, so in the reverse direction we can pick an
  interval containing $1$ to discover the largest (i.e. least in the order) intervals that will give rise to
  this output as \emph{independent} changes to the input. For example, if we pick
  $[\frac{9}{10},\frac{11}{10}]$ as the interval, then:
  \begin{displaymath}
    (\partial\mathrm{query}\,\mathsf{a}\,\mathsf{db})_r([\frac{9}{10},\frac{11}{10}]) = [(\cdot,[-\frac{1}{10},\frac{1}{10}]),(\cdot,\bot),(\cdot,[\frac{9}{10},\frac{11}{10}])]
  \end{displaymath}
  Indicating that, to achieve an output within $[\frac{9}{10},\frac{11}{10}]$ for this query, either the first
  $\mathsf{a}$ row could be in $[-\frac{1}{10},\frac{1}{10}]$ or the second one could be in
  $[\frac{9}{10},\frac{11}{10}]$.
\end{enumerate}

\subsection{Systematic Insertion of Approximation via Moggi's CBN translation}
\label{sec:cbn-translation}

We can now carry out the systematic insertion of approximation points that we foreshadowed in
\secref{models-of-total-gps:lifting}, using \citet[\S 3.1]{notions-of-computation}'s monadic CBN
translation. We use the $T$ monad from \secref{tagging-monad} because it can be defined in terms of the
language constructed we already have. This requires that we have a signature $\Sigma$ that includes a
primitive type to be interpreted as the approximation object $\mathbb{A}$ and primitive operations to be
interpreted as the monoid operations on this object.

TODO: describe the syntactic CBN translation; show an example on the query example.
