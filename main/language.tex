\section{Higher-Order Language}
\label{sec:language}

To model Galois slicing semantically for higher-order programs, we define a simple total functional
programming language, extending the simply-typed lambda calculus.

\subsection{Syntax}
\label{sec:language:syntax}

The language is parameterised by a signature $\Sigma = (\PrimTy, \PrimOp)$ consisting of a set $\PrimTy$ of
base types $\rho$ and a family of sets $\PrimOp^\rho_{\rho_1,\ldots,\rho_n}$ of primitive operations $\phi$ of
arity $n$ over those base types. The syntax is defined in \figref{syntax}. Types includes base types $\rho$
drawn from $\PrimTy$, along with standard type formers for sums, products, functions and lists, plus a type
constructor $\tyLift$ to allow approximation points to be added explicitly, as discussed in
\secref{models-of-total-gps}. Terms include variables, the usual introduction and elimination forms, a monadic
return and bind for lifted terms, and primitive operations $\phi$.

The language is intentionally minimal: it excludes general recursion, and general inductive or coinductive
types, which we will consider in future work (\secref{conclusion}). Typing judgments for terms are standard
and shown in \figref{typing}, with the usual rules for products, sums, functions, lists and lifting.

\begin{figure}
  \begin{subfigure}[t]{0.48\linewidth}
  \small
  \[
  \begin{array}{lllll}
    & \textit{Types}
    \\
    &
    \sigma, \tau
    & ::= &
    \rho
    &
    \text{primitive type}
    \\
    && \mid &
    \sigma \tySum \tau
    &
    \text{sum}
    \\
    && \mid &
    \tyUnit
    &
    \text{unit}
    \\
    && \mid &
    \sigma \tyProd \tau
    &
    \text{product}
    \\
    && \mid &
    \sigma \tyFun \tau
    &
    \text{function}
    \\
    && \mid &
    \tyList\;\tau
    &
    \text{list}
    \\
    && \mid &
    \tyLift\;\tau
    &
    \text{lifting}
  \end{array}
  \]
  \end{subfigure}%
  \begin{subfigure}[t]{0.48\linewidth}
  \small
  \[
  \begin{array}{lllll}
    & \textit{Terms}
    \\
    &
    t, s
    & ::= &
    x
    &
    \text{variable}
    \\
    && \mid &
    \phi(\vec t)
    &
    \text{primitive op}
    \\
    && \mid &
    \tmInl{t} \mid \tmInr{t}
    &
    \text{injection}
    \\
    && \mid &
    \tmCase{s}{x}{t_1}{y}{t_2}
    &
    \text{case}
    \\
    && \mid &
    \tmUnit
    &
    \text{unit}
    \\
    && \mid &
    \tmPair{s}{t}
    &
    \text{pair}
    \\
    && \mid &
    \tmFst{t} \mid \tmSnd{t}
    &
    \text{projection}
    \\
    && \mid &
    \tmFun{x}{t}
    &
    \text{function}
    \\
    && \mid &
    \tmApp{s}{t}
    &
    \text{application}
    \\
    && \mid &
    \tmNil
    &
    \text{nil}
    \\
    && \mid &
    \tmCons{s}{t}
    &
    \text{cons}
    \\
    && \mid &
    \tmFoldList{s_1}{s_2}{t}
    &
    \text{fold}
    \\
    && \mid &
    \tmReturn{t}
    &
    \text{return}
    \\
    && \mid &
    \tmBind{s}{t}
    &
    \text{bind}
  \end{array}
  \]
  \end{subfigure}
  \caption{Syntax of types and terms}
  \label{fig:syntax}
\end{figure}

\begin{figure}
\begin{subfigure}{\linewidth}
  \begin{mathpar}
  \small
  \inferrule*
  {
    \alpha: \kType \in \Delta
  }
  {
    \Delta \vdash \alpha: \kType
  }
  \and
  \inferrule*
  {
    \strut
  }
  {
    \Delta \vdash \tyZero: \kType
  }
  \and
  \inferrule*
  {
    \Delta \vdash \sigma: \kType
    \\
    \Delta \vdash \tau: \kType
  }
  {
    \Delta \vdash \sigma \tySum \tau: \kType
  }
  \and
  \inferrule*
  {
    \strut
  }
  {
    \Delta \vdash \tyUnit: \kType
  }
  \and
  \inferrule*
  {
    \Delta \vdash \sigma: \kType
    \\
    \Delta \vdash \tau: \kType
  }
  {
    \Delta \vdash \sigma \tyProd \tau: \kType
  }
  \and
  \inferrule*
  {
    \Delta \vdash \sigma: \kType
    \\
    \Delta \vdash \tau: \kType
  }
  {
    \Delta \vdash \sigma \tyFun \tau: \kType
  }
  \and
  \inferrule*
  {
    \Delta, \alpha: \kType \vdash \tau: \kType
    \\
    \Pol(+,\alpha,\tau)
  }
  {
    \Delta \vdash \mu\alpha.\tau: \kType
  }
  \and
  \inferrule*
  {
    \Delta \vdash \tau: \kType
  }
  {
    \Delta \vdash \tyLift\;\tau: \kType
  }
  \end{mathpar}
  \caption{Well-kinded types}
\end{subfigure}
\begin{subfigure}{\linewidth}
  \begin{mathpar}
    \small
    \inferrule*
    {
      x : \tau \in \Gamma
    }
    {
      \Gamma \vdash x: \tau
    }
    \and
    \inferrule*
    {
      \Gamma \vdash t : \sigma
    }
    {
      \Gamma \vdash \tmInl{t}: \sigma \tySum \tau
    }
    \and
    \inferrule*
    {
      \Gamma \vdash t : \tau
    }
    {
      \Gamma \vdash \tmInr{t}: \sigma \tySum \tau
    }
    \and
    \inferrule*
    {
      \Gamma \vdash s : \sigma \tySum \tau
      \\
      \Gamma, x: \sigma \vdash t_1 : \tau'
      \\
      \Gamma, y : \tau \vdash t_2 : \tau'
    }
    {
      \Gamma \vdash \tmCase{s}{x}{t_1}{y}{t_2}: \tau'
    }
    \and
    \inferrule*
    {
      \strut
    }
    {
      \Gamma \vdash \tmUnit : \tyUnit
    }
    \and
    \inferrule*
    {
      \Gamma \vdash s : \sigma
      \\
      \Gamma \vdash t : \tau
    }
    {
      \Gamma \vdash \tmPair{s}{t}: \sigma \tyProd \tau
    }
    \and
    \inferrule*
    {
      \Gamma \vdash t : \sigma \tyProd \tau
    }
    {
      \Gamma \vdash \tmFst{t}: \sigma
    }
    \and
    \inferrule*
    {
      \Gamma \vdash t : \sigma \tyProd \tau
    }
    {
      \Gamma \vdash \tmSnd{t}: \tau
    }
    \and
    \inferrule*
    {
      \Gamma, x: \sigma \vdash t : \tau
    }
    {
      \Gamma \vdash \tmFun{x}{t}: \sigma \tyFun \tau
    }
    \and
    \inferrule*
    {
      \Gamma \vdash s: \sigma \tyFun \tau
      \\
      \Gamma \vdash t : \sigma
    }
    {
      \Gamma \vdash \tmApp{s}{t}: \tau
    }
    \and
    \inferrule*
    {
      \Gamma \vdash t : \subst{\tau}{\mu \alpha.\tau}{\alpha}
    }
    {
      \Gamma \vdash \tmRoll{t}: \mu\alpha.\tau
    }
    \and
    \inferrule*
    {
      \Gamma \vdash s : \subst{\sigma}{\tau}{\alpha} \tyFun \tau
      \\
      \Gamma \vdash t : \mu\alpha.\sigma
    }
    {
      \Gamma \vdash \tmFold{s}{t} : \tau
    }
    \and
    \inferrule*
    {
      \Gamma \vdash t : \tau
    }
    {
      \Gamma \vdash \tmReturn{t} : \tyLift\;\tau
    }
    \and
    \inferrule*
    {
      \Gamma \vdash s : \tyLift\;\sigma
      \\
      \Gamma \vdash t : \sigma \tyFun \tyLift\;\tau
    }
    {
      \Gamma \vdash \tmBind{s}{t} : \tyLift\;\tau
    }
  \end{mathpar}
  \caption{Well-typed terms (all types well-kinded)}
\end{subfigure}
\caption{Kinding and typing rules}
\label{fig:typing}
\end{figure}


\subsection{Semantics}
\label{sec:language:semantics}

\begin{figure}
  \begin{subfigure}[t]{0.47\linewidth}
    \small
    \begin{align*}
      \sem{\rho} &= \sem{\rho}_{\PrimTy}
      \\
      \sem{\sigma \tySum \tau} &= \textstyle {\sem{\sigma}} + {\sem{\tau}}
      \\
      \sem{\tyUnit} &= 1
      \\
      \sem{\sigma \tyProd \tau} &= \sem{\sigma} \times \sem{\tau}
      \\
      \sem{\sigma \tyFun \tau} &= \internalHom{\sem{\sigma}}{\sem{\tau}}
      \\
      \sem{\tyList\;\tau} &= \List(\sem{\tau})
      % \\
      % \sem{\tyLift\;\tau} &= \Lift(\sem{\tau})
    \end{align*}
    \caption{Interpretation of Types}
    \label{fig:semantics:types}
  \end{subfigure}
  \begin{subfigure}[t]{0.47\linewidth}
    \small
    \begin{align*}
      \sem{\emptyCxt} &= 1
      \\
      \sem{\Gamma, x: \tau} &= \sem{\Gamma} \times \sem{\tau}
    \end{align*}
    \caption{Interpretation of Contexts}
    \label{fig:semantics:contexts}
\end{subfigure}
\begin{subfigure}{0.8\linewidth}
  \small
  \begin{align*}
  \sem{x_i} &= \pi_i
  \\
  \sem{\phi(t_1, \ldots, t_n)}
  &=
  \sem{\phi}_{\Op} \comp \prodM{\sem{t_1}}{\ldots, \sem{t_n}}
  \\
  \sem{\tmInl{t}} &= \mathsf{inj}_1 \comp \sem{t}
  \\
  \sem{\tmInr{t}} &= \mathsf{inj}_2 \comp \sem{t}
  \\
  \sem{\tmCase{s}{x}{t_1}{y}{t_2}} &= \coprodM{\sem{t_1}}{\sem{t_2}} \comp \prodM{\id}{\sem{s}}
  \\
  \sem{\tmUnit} &=\;!_{\sem{\Gamma}}
  \\
  \sem{\tmPair{s}{t}} &= \prodM{\sem{s}}{\sem{t}}
  \\
  \sem{\tmFst{t}} &= \pi_1 \comp \sem{t}
  \\
  \sem{\tmSnd{t}} &= \pi_2 \comp \sem{t}
  \\
  \sem{\tmFun{x}{t}} &= \lambda(\sem{t})
  \\
  \sem{\tmApp{s}{t}} &= \eval \comp \prodM{\sem{s}}{\sem{t}}
%  \\
%  \sem{\tmRoll{t}} &= \inMap_{\sem{\sigma}} \comp \sem{t}
%  \textit{ where }\tau = \mu\alpha.\sigma
  \\
  \sem{\tmNil} &= \nil \comp {!_{\sem{\Gamma}}}
  \\
  \sem{\tmCons{s}{t}} &= \cons \comp \prodM{\sem{s}}{\sem{t}}
  \\
  \sem{\tmFoldList{t_1}{t_2}{s}} &= \fold(\sem{t_1},\sem{t_2}) \comp \prodM{\id}{\sem{s}}
%  \\
%  \sem{\tmFold{s}{t}} &= \eval \comp \prodM{\phi \comp \sem{s}}{\sem{t}}
  % \\
  % \sem{\tmReturn{t}} &= \eta_{\sem{\sigma}} \comp \sem{t}
  % \tag*{($\Gamma \vdash t: \sigma$)}
  % \\
  % \sem{\tmBind{s}{t}} &= \mu_{\sem{\tau}} \comp \Lift(\sem{t}) \comp \mathsf{st}_{\sem{\Gamma},\sem{\sigma}} \comp \prodM{\id}{\sem{s}}
  % \tag*{($\Gamma \vdash t: \sigma \to \tyLift\;\tau$)}
  \end{align*}
  \caption{Terms as morphisms}
  \label{fig:semantics:terms}
\end{subfigure}
\caption{Interpretation of types, contexts and terms in category $\Sem$}
\end{figure}


To give the semantics for the language defined in \figrefTwo{syntax}{typing}, we fix a bicartesian closed
category $\Sem$ with finite products $(\times, 1)$, finite coproducts $(+, 0)$ and exponentials
$\internalHom{X}{Y}$, with evaluation morphisms $\eval_{X,Y}$ and currying isomorphisms $\lambda_{X,Y,Z}$,
plus the following additional structure:
\begin{enumerate}
\item a strong monad $(\Lift, \eta, \mu)$ with strength $\mathsf{st}_{X,Y}: X \times \Lift(Y) \to \Lift(X
\times Y)$
\item for each base type $\rho \in \PrimTy$, an object $\sem{\rho}_{\PrimTy}$
\item for each primitive operation $\phi \in \PrimOp^\rho_{\rho_1,\ldots,\rho_n}$, a morphism
$\sem{\phi}_{\Op}: \sem{\rho_1} \times \ldots \times \sem{\rho_n} \to \sem{\rho}$
\item an endofunctor $\List: \Sem \to \Sem$, plus for any object $X$, morphisms $\nil: 1 \to \List(X)$ and
$\cons: X \times \List(X) \to \List(X)$, and for any objects $Y, Z$ and morphisms $f_\nil: Z \to Y$ and
$f_\cons: Z \times X \times Y \to Y$, a morphism $\fold(f_\nil,f_\cons): Z \times \List(X) \to Y$.
\end{enumerate}

\figref{semantics:types} gives the interpretations of types $\tau$ and contexts $\Gamma$ as objects of $\Sem$.
\figref{semantics:terms} gives the interpretation of terms $\Gamma \vdash t: \tau$ as morphisms $\sem{\Gamma}
\to \sem{\tau}$.

\subsubsection{Interpretation for higher-order \GPS}

\paragraph{Base types and operations}
A \emph{model} of a signature in a category $\cat{C}$ with finite products and a terminal object assigns to
each base type $\rho \in \PrimTy$ an object $\sem{\rho}_{\PrimTy}$ in $\cat{C}$ and to each primitive
operation $\phi \in \PrimOp^\rho_{\rho_1,\ldots,\rho_n}$ a morphism $\sem{\phi}_{\Op}: \sem{\rho_1}_{\PrimTy}
\times \ldots \times \sem{\rho_n}_{\PrimTy} \to \sem{\rho}_{\PrimTy}$.

Given an interpretation of base types and operations in $\Fam(\LatGal)$, the category $\Fam(\MeetSLat \times
\JoinSLat^\op)$ can implement $\Sem$. It is bicartesian closed (\corref{mslat-jslat-bcc}) and can model
$\List(X)$ as an infinite coproduct (\secref{models-of-total-gps:coproducts-and-products}). An interpretation
of base types and operations can be obtained from their interpretation in $\Fam(\LatGal)$ via the embedding
...
\roly{$\Lift$ monad.}
