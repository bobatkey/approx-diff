\section{Approximations as Tangents}

We motivate our approach to \GPS by showing how to combine ideas from differential geometry and stable domain theory to reconstruct \GPS in a dentational setting. From this basis, we apply the CHAD framework of V{\'a}k{\'a}r \etal to obtain a denotational model of automatic differentiation that gives us an effective way to interpret programs as functions along with their forward and backwards approximation maps\bob{and we did it in Agda, so we can run it}.

\subsection{Manifolds, Smooth Functions, and Automatic Differentiation}

For what follows we do not need to know in detail what a manifold is, or exactly how smooth functions are defined. We only state some definitions in order to fix terminology and to justify our claim of a connection to standard differentiablity. \bob{Need a good reference here.}

The general study of differentiable functions usually takes place on \emph{manifolds}, which are topological spaces that are ``locally'' like a Euclidean space $\RR^n$. The spaces $\RR^n$ are manifolds, but so are ``non-flat'' examples such as $n$-spheres and more exotic spaces. Every point $x$ in a manifold $M$ has an associated \emph{tangent vector space} $\tangents_x(M)$ and \emph{cotangent vector space} $\cotangents_x(M)$, the latter consisting of linear functions $\tangents_x(M) \linearto \RR$. The tangent and cotangent spaces are finite dimensional, so in the presence of a chosen basis they are canonically isomorphic. In the case when the manifold is $\RR^n$, then every tangent space is isomorphic to $\RR^n$ as well.

Smooth functions $f$ between manifolds $M$ and $N$ are functions on their points that are locally differentiable on $\RR^n$. Each smooth function induces maps of the (co)tangent spaces:
\begin{itemize}
\item The \emph{forward derivative} (tangent map or pushforward) $\pushf{f}_x$ is a linear map $\tangents_x(M) \linearto \tangents_{f(x)}(N)$. In the Eucdlidean case when $M = \RR^m$ and $N = \RR^n$, the tangent map can be represented by the Jacobian matrix of partial derivatives of $f$ at $x$.
\item The \emph{backward derivative} (cotangent map, or pullback) $\pullf{f}_x$ is a linear map $\cotangents_{f(x)}(N) \linearto \cotangents_x(M)$. In the Euclidean case, the backward derivative is represented by the transpose of the Jacobian of $f$ at $x$.
\end{itemize}

Computing the forward and backward derivatives of smooth functions $f$ has many applications of practical interest. For example, computation of the reverse derivative is of central interest in Machine Learning by Gradient Descent, the main technique used to train Deep Neural Networks. \bob{need a citation here}.

Derivatives can be computed numerically by computing $f$ on small perturbations of its input, or symbolically by examining a closed-form representation of $f$. However, a more common and practical technique is to use \emph{automatic differentiation}, where a program computing $f$ is instrumented to produce (a representation of) the forward and/or backward derivative as a side-effect of producing the output. \bob{cite something on autodiff}. This has led to the area of differentiable programming, where programming languages and their implementations are specifically designed to admit efficient automatic differentation algorithms. \bob{cite: JAX, TensorFlow, Dex, Jesse Sigal, etc.}

\subsection{Stable Domain Theory as Differentiability}

Our thesis is that \GPS can profitably be seen as a form of differentiable programming, where tangents are not linear approximations of functions but instead are qualitiative approximations of elements in the sense of domain theory. Smooth functions in this setting are Berry's \emph{stable functions} in domain theory. We now introduce these concepts and how they relate to \GPS.

\subsubsection{Domains as a Qualitative Theory of Approximation}

Domain theory is a method for defining the semantics of programs that
handle infinite data (e.g., functions or infinite
streams). \emph{Domains} are special partially ordered sets where the
ordering denotes a relationship of qualitative information content: if
$x \sqsubseteq y$, then $y$ may contain more information than $x$. For
example, if $x$ and $y$ are functions, then $y$ may be defined at more
places in its domain than $x$. Infinite objects are understood in
terms of their approximations in this sense, and domains are assumed
to be closed under least upper bounds (lubs) of directed sets, meaning
that any internally consistent collection of elements has a
``completion'' that contains all the information covered by the
set. Programs are interpreted as monotone functions that preserve
directed lubs. The latter property is called continuity. Monoticity
captures the idea that if the input gets more defined, then the output
can get more defined. Continuity intuitively expresses the idea that a
function interpreting a program cannot ``look at all the
approximations'' when determining the output, which corresponds to the
intuitive idea that a computable function cannot look at a non-finite
amount of input to produce an output.

For the purposes of \GPS, we are not interested in using
approximations to model computation on infinite objects, but rather to
use them for the related task of revealing how programs explore their
inputs for provenance tracking. Therefore, in what follows we dispense with completeness properties of the partially ordered sets we consider.

\subsubsection{Stable Functions as ``Smooth Functions''}

Continuous functions between domains capture the intuitive idea that
computable functions are restricted to finite observations of their
input for finite output. However, there are examples, such as
Plotkin's Parallel OR (\autoref{ex:parallel-or}, below) that are
continuous but do not treat their input in a way that is consistent
with a sequential implementation. Stability is an additional property
of monotone functions that was invented by Berry \cite{berry86} in an
attempt to capture sequentiality. This was not successful (see the
$\mathrm{gustave}$ function in \autoref{ex:parallel-or}), but we will
see now how it is closely related to problem of computing forward and
backwards maps of approximations needed in \GPS. A textbook
description of stable functions in the context of domain theory is
given by \citet{amadio-curien} (Chapter 12). We will
refer to this where necessary.

In a partially ordered set $X$, for any $x \in X$ the set of elements
below $x$, $\downset{x} = \{ x' \mid x' \sqsubseteq x \}$, is itself a
partially ordered set. These approximations of $x$ we will think of as
``derivatives'' of $x$, and the whole set $\downset{x}$ as the
``tangent space''. Tangent spaces are vector spaces, since they are
linear approximations to curves at that point. In the partially
ordered setting, we take elements of $\downset{x}$ to be
approximations of processes defined at $x$. As we can add tangents, we
assume we can take meets of approximations in $\downset{x}$:

\begin{definition}
  A \emph{bounded meet poset} is a partially ordered set $X$ where
  every for every $x \in X$, $\downset{x}$ is a meet semilattice, with
  $x$ as the top element.
\end{definition}

Smooth functions have a ``pushforward'' derivative that takes tangents
at a point $x$ to tangents at the point $f(x)$. In the partially
ordered setting, we require that the restriction of a monotone
function $f$ to each $\downset{x}$ preserves meets:

\begin{definition}
  A \emph{conditionally multiplicative} (cm) function $f : X \to Y$ is
  a monotone function such that for all $x \in X$, the restriction
  $f_x : \downset{x} \to \downset{f(x)}$ preserves meets.
\end{definition}

\begin{lemma}
  Morphisms of meet approximation spaces are closed under identities
  and composition, forming a category. FIXME: with products?
  functions? sums?
\end{lemma}

\begin{example}[Conditionally Multiplicative Functions]
  To see the effect of requiring preservation of bounded meets,
  consider several ways of defining disjunction on the lifted booleans
  $\mathbb{B}_\bot$. Two functions that are cm are the strict and
  left-short-circuiting ORs\footnote{The clauses in these examples are
    shorthand for the graph of the function. They are not to be
    understood as pattern matching clauses in a language like Haskell,
    where it is not possible to match on $\bot$.}:
  \begin{displaymath}
    \begin{array}[t]{l@{(}l@{,~}l@{)~}c@{~}l}
      \mathrm{strictOr}&\mathsf{tt}&\mathsf{tt}&=&\mathsf{tt} \\
      \mathrm{strictOr}&\mathsf{tt}&\mathsf{ff}&=&\mathsf{tt} \\
      \mathrm{strictOr}&\mathsf{ff}&\mathsf{tt}&=&\mathsf{tt} \\
      \mathrm{strictOr}&\mathsf{ff}&\mathsf{ff}&=&\mathsf{ff} \\
      \mathrm{strictOr}&\bot&\_&=&\bot \\
      \mathrm{strictOr}&\_&\bot&=&\bot \\
    \end{array}
    \qquad
    \begin{array}[t]{l@{(}l@{,~}l@{)~}c@{~}l}
      \mathrm{shortCircuitOR}&\mathsf{tt}&\_&=&\mathsf{tt} \\
      \mathsf{shortCircuitOR}&\mathsf{ff}&x&=&x \\
      \mathsf{shortCircuitOR}&\bot&\_& =& \bot
    \end{array}
  \end{displaymath}
  \bob{Demonstrate that they are conditionally multiplicative, by
    giving evidence of stability and stating that this implies cm}

  The function $\mathrm{strictOr}$ is stable. For example, for the
  input-output pair $(\mathsf{tt},\mathsf{ff}), \mathsf{tt}$, the
  minimal input that gives this output is exactly
  $(\mathsf{tt}, \mathsf{ff})$. If we take the approximation
  $\bot \leq \mathsf{tt}$ of the output, then the corresponding
  minimal input is $(\bot, \bot)$. The function
  $\mathrm{shortCircuitOR}$ is also stable. For the input-output pair
  $(\mathsf{tt},\mathsf{ff}),\mathsf{tt}$, the minimal input that
  gives this input is $(\mathsf{tt},\bot)$, indicating that the
  presence of $\mathsf{ff}$ in the second argument was not necessary
  to produce this output. As with $\mathrm{strictOr}$, the minimal
  input required to produce the output $\bot \leq \mathsf{tt}$ is
  again $(\bot,\bot)$.

  The fact that these two functions's stability witnesses reveal
  information about how they depend on their input is what we will
  exploit in order to use the idea of stability for slicing.
\end{example}

\begin{example}[A non-Conditionally Multiplicative Function]
  \label{ex:parallel-or}
  A function that is not cm is Plotkin's Parallel OR \cite{lcf77},
  which short-circuits in both arguments, returning $\mathsf{tt}$ if
  either argment is $\mathsf{tt}$, even if the other argument is not
  defined:
  \begin{displaymath}
    \begin{array}{l@{(}l@{,~}l@{)~}c@{~}l}
      \mathrm{parallelOR}&\mathsf{tt}&\_&=&\mathsf{tt} \\
      \mathrm{parallelOR}&\_&\mathsf{tt}&=&\mathsf{tt} \\
      \mathsf{parallelOR}&\mathsf{ff}&\mathsf{ff}&=&\mathsf{ff} \\
      \mathsf{parallelOR}&\bot&\bot&=&\bot
    \end{array}
  \end{displaymath}
  For the input-output pair $(\mathsf{tt},\mathsf{tt}),\mathsf{tt}$,
  there is no one minimal input that produces this output. We have
  both $\mathrm{parallelOR}(\mathsf{tt},\bot) = \mathsf{tt}$ and
  $\mathrm{parallelOR}(\bot,\mathsf{tt}) = \mathsf{tt}$, which are
  incomparable and their greatest lower bound $(\bot,\bot)$ gives the
  output $\bot$.

  Parallel OR is famous because it is not \emph{sequential}, meaning
  intuitively that it cannot be implemented without running the two
  arguments in parallel to see if one of them returns
  $\mathsf{tt}$. That fact that it exists in the standard domain
  theoretic semantics of PCF means that this semantics is complete for
  reasoning about observational equivalence in PCF. Since Parallel OR
  is not stable, one might hope that stability is enough to capture
  sequentiality, and hence potentially give a fully abstract model of
  PCF. However, the following ternary function
  $\mathbb{B}_\bot^3 \to \{\top,\bot\}$ is stable but admits no
  sequential implementation that fixes an order that the arguments are
  examined in:
  \begin{displaymath}
    \begin{array}{l@{(}l@{,~}l@{,~}l@{)~}c@{~}l}
      \mathrm{gustave}&\mathsf{tt}&\mathsf{ff}&\_&=&\top \\
      \mathrm{gustave}&\mathsf{ff}&\_&\mathsf{tt}&=&\top \\
      \mathrm{gustave}&\_&\mathsf{tt}&\mathsf{ff}&=&\top \\
      \mathrm{gustave}&\_&\_&\_&=&\bot
    \end{array}
  \end{displaymath}
  Despite there being no one minimal input that achieves the output
  $\top$, each of the minimal inputs that can achieve this output are
  pairwise incomparable, so for each specific input that gets output
  $\top$ there is a unique minimal input that achieves it (listed in
  the first three lines of the definition).

  In terms of \GPS the $\mathrm{gustave}$ function does not present a
  problem. For any particular run (i.e., input-output pair) of the
  program, there is an unambiguous minimal input that achieves the
  output, no matter that it was not achieved by a sequential
  processing of the input.
\end{example}

\begin{example}[Is Conditional Multiplicativity Enough?]
  \label{ex:non-stable-function}
  In the preceeding examples, we demonstrated conditional
  multiplicativity by giving a backwards mapping of approximations
  that forces preservation of meets. A natural question is whether or
  not we always have such a backwards map which we could use for
  backwards propagation of approximations. An example that is
  conditionally multiplicative, but does not admit such a backards map
  (from Amadio and Curien just before Lemma 12.2.3, originally due to
  Berry) is given by defining
  $\mathrm{unstable} : D \to \{\bot, \top\}$, where:
  \begin{displaymath}
    D = \bot \sqsubseteq \cdots \sqsubseteq n \sqsubseteq \cdots \sqsubseteq 1 \sqsubseteq 0
  \end{displaymath}
  as $\mathrm{unstable}(\bot) = \bot$ and
  $\mathrm{unstable}(n) = \top$. This is monotone, and preserves meets
  in every $\downset{x}$. However, there is no ``best'' (i.e., least)
  input that gives us any finite output.
\end{example}

In light of the preceeding example, we can give an alternative
definition of a function that requires the existence of a reverse
mapping directly, without assuming that any meets exist:

\begin{definition}[Stable function]
  Let $f : X \to Y$ be a monotone function between posets $X$ and
  $Y$. The function $f$ is \emph{stable} if for all $x \in X$ and
  $y \leq f(x)$:
  \begin{enumerate}
  \item (\textsc{Existence}) there exists an $x_0 \leq x$ such that $y \leq f(x_0)$, and
  \item (\textsc{Minimality}) for any $x'_0 \leq x$ such that $y \leq f(x'_0)$ then
    $x_0 \leq x'_0$ .
  \end{enumerate}
\end{definition}

\begin{lemma}
  Stable functions are closed under identities and composition.
\end{lemma}

\begin{example}
  \bob{Elaborate on the examples}
  \begin{enumerate}
  \item $\mathrm{strictOr}$ and $\mathrm{shortCircuitOR}$ are stable, as demonstrated above.
  \item $\mathrm{parallelOR}$ is not stable.
  \item $\mathrm{gustave}$ is stable.
  \item $\mathrm{unstable}$ is not stable, but is conditionally multiplicative.
  \end{enumerate}
\end{example}

Stability has an alternative definition in terms of Galois connections, which will be more useful for what follows. This characterisation is due to Paul Taylor\bob{cite}.

\begin{lemma}
  A monotone function $f : X \to Y$ is stable if and only if for all
  $x \in X$, the restriction of $f_x : \downset{x} \to \downset{f(x)}$
  has a left Galois adjoint.
\end{lemma}

\begin{proof}
  If $f$ is stable, then define a left adjoint
  $f^*_x : \downset{f(x)} \to \downset{x}$ by setting $f^*_x(y)$ to be
  the minimal $x_0$ required by stability. This is monotone: if
  $y \leq y'$, then we know that $y \leq y' \leq f(f^*_x(y'))$ by
  definition of $f^*_x$, so $f^*_x(y) \leq f^*_x(y')$ by minimality of
  $f^*_x(y)$. For the adjointness, let $x' \leq x$ and $y \leq
  f(x)$. Then if $f^*_x(y) \leq x'$, we have
  $y \leq f(f^*_x(y)) \leq f(x')$ by monotonicity of $f$ and the first
  part of stability. In the other direction, if we have
  $y \leq f(x')$, then by uniqueness we have $f^*_x(y) \leq x'$.

  If, for every $x$, $f_x$ has a left adjoint $f^*_x$, then for any
  $x', y$ we have $y \leq f_x(x') \Leftrightarrow f^*_x(y) \leq
  x'$. So $f^*_x(y)$ is the element that satisfies
  $y \leq f(f^*_x(y))$, and it is minimal since if $y \leq f_x(x'_0)$
  then $f^*_x(y) \leq x'_0$.
\end{proof}

Even though stable functions can be defined on any partially ordered set, in light of the analogy with tangent spaces, it makes sense to require that meets (for forward approximation maps) and joins (for backwards approximation maps) exist:

\begin{definition}
  An \emph{L-poset} is a partially ordered set $X$ such that for every
  $x \in X$, the principal downset $\downset{x}$ is a bounded lattice
  (i.e., have all finite meets and joins).
\end{definition}

The following lemma is an instance of standard facts about Galois connections preserving meets and joins:

\begin{lemma}
  For L-posets $X$ and $Y$, a stable function $f : X \to Y$ preserves
  meets in its forward part $f_x$ and joins in its reverse part
  $f^*_x$.
\end{lemma}

The converse to this lemma (that functions that preserve meets in
their forward part have a left Galois adjoint) is not true, as was
demonstrated by the non-stable function in
\autoref{ex:non-stable-function}. In the case when the posets
$\downset{x}$ are \emph{complete}, and $f_x$ preserves infinitary
meets, then we are guaranteed a left Galois adjoint. In that example,
the infinite set $\{0, 1, 2, \dots, n, n-1, \dots\}$ not including
$\bot$ of approximations of $0$ does not have a greatest lower bound,
so the order is not complete.

We could have required that L-posets' principal downsets were complete
lattices, which would mean that we would only need to require that
forward approximation maps preserve all meets and we could compute the
backwards map automatically. However, our goal is a computable theory
for \GPS, analogous to automatic differentiation. Requiring
completeness to compute backwards maps of approximations would require
that complete joins are always computable, which will not be the case for approximations of functions.

\subsection{Summary}
\label{sec:diff-stab-summary}

We now summarise the analogy between differentiable functions between manifolds and stable functions between L-posets:

\begin{itemize}
\item For \GPS, we assume that every value has an associated lattice of {\em approximations}. For differentiable programs, every point has an associated vector space of {\em tangents}.
\item For \GPS, every program has an associated forward approximation map that takes approximations forward from the input to the output. This map {\em preserves meets}. For differentiable programs, every program has a forward derivative that takes tangents of the input to tangents of the output. The forward derivative map is {\em linear}, so it preserves addition of tangents and the zero tangent.
\item For \GPS, every program has an associated backward approximation map that takes approximations of the output back to least approximations of the input. This map {\em preserves joins}. For differentiable programs, every program has a reverse derivative that takes tangents of the output to tangents of the input. This map is again {\em linear}.
\item For \GPS, the forward and backward approximation maps are related by being a Galois connection. For differentiable programming, the forward and reverse derivatives are related by being each others' transpose.
\end{itemize}

Given this close connection between \GPS and differentiable programming, we can take structures intended for modelling automatic differentiation and use them to model \GPS. This will enable us to generalise and expand the scope of \GPS to act as a foundation for data provenance in a wide range of computational settings.
