\section{CHAD at First-Order}
\label{sec:first-order}

Our denotational approach to \GPS is based on the CHAD (Combinatory Homomorphic Automatic Differentiation) work of Mattias Vákár and others~\cite{vákár22,nunes2023}. Before we go into details, we present a high-level overview of their approach and how we have applied it to \GPS.

The fundamental approach in CHAD is to consider Grothendieck constructions on indexed categories $T : \cat{C}^\op \to \Cat$. An object of the Grothendieck construction $\int T$, described in more detail in \secref{Grothendieck}, is a pair $(X \in \cat{C}, \partial X \in T(X))$, which we read as pairing a space $X$ of ``points'' from $\cat{C}$ with an associated bundle of tangent spaces. The maps from the ``base'' category $\cat{C}$ are used to interpret the programs we are modelling. The maps in the indexed category are the maps of tangents or approximations, being derivatives and Galois connections respectively.

In the case of differentiable programs and automatic differentiation, a basic model can be constructed by taking $\cat{C}$ to be $\Set$ and $T(X) = X \to \FinVect$: indexed collections of finite dimensional vector spaces, whose elements are interpreted as tangents at the given point, with linear maps. The Grothendieck category $\int T$ has objects that are pairs of a set $X$ and for each $x \in X$ a tangent vector space $\partial X(x)$. Morphisms are pairs of maps of points to points, accompanied by linear maps of tangents at those points. If we carefully choose an initial set of maps, we can read these maps as derivatives (note that nothing in the Grothendieck construction says that they must be derivatives!). Composition in the Grothendieck category is exactly the {\em chain rule} for composing derivatives, so we at least do know that if we start with derivatives on basic operations, then composing them will retain this property.

For \GPS, we again take $\cat{C}$ to be $\Set$ and now take $T(X) = X \to \LatGal$: indexed collections of lattices, whose elements are interpreted as approximations at the given point, with indexed Galois connections as the maps between them.

In these special cases when $\cat{C} = \Set$, the Grothendieck construction is the {\em families} constructions $\Fam(\FinVect)$ and $\Fam(\LatGal)$~(\secref{Fam}). These settings are adequate for interpreting first-order programs, but neither is Cartesian closed. The picture for higher-order programs is more complicated. As is well known, $\Fam(\cat{X})$ for any category $\cat{X}$ is the free coproduct completion of $\cat{X}$~\cite{lawvere63}. In the case that $\cat{X}$ has a (symmetric) monoidal product, then we always get a symmetric monoidal product on $\Fam(\cat{X})$, using the Cartesian products from $\Set$. When $\cat{X}$ has all (small) products, and the monoidal product is actually a coproduct, then $\Fam(\cat{X})$ is symmetric monoidal closed. If the coproducts are actually {\em biproducts}~(\secref{biproducts}), then $\Fam(\cat{X})$ is Cartesian closed, and we can interpret higher-order programs.

Happily, $\FinVect$ and $\LatGal$ both have biproducts, as a consequence of their having products and being enriched in commutative monoids~(\propref{biproducts:from-product-or-coproduct} below). However, neither of them has all small products, without moving to infinite dimensional vector spaces (in the case of $\FinVect$) or complete lattices (in the case of $\LatGal$). Neither of these is ideal. In infinite dimensional vector spaces, dualisation is not involutive and the connection between the forward and reverse derivatives is lost. In complete lattices, we can no longer easily implement the infinite meets and joins required, and we lose the computability of the model. Moreover, in Agda, there are issues with predicativity when considering complete lattices. \todo{be more specific}

Vákár and collaborators overcome this by using different interpretations for forward and backward derivatives. We overcome this by using $\CMon$-enriched presheaves, as described in \secref{higher-order}; first we present a suitable first-order setting.

\subsection{Biproducts and Semi-Additive Categories}
\label{sec:biproducts}

\todo{Introduce $\CMon$-category first.} We start with the idea of a biproduct and semi-additive categories,
which are always enriched in $\CMon$, the category of commutative monoids.

\begin{definition}[Semi-additive category]
\label{def:biproducts:semi-additive}
A category with finite products and coproducts is \emph{semi-additive} if the canonical morphisms (projections
and injections) give an isomorphism
\[\textstyle X \coprod Y \iso X \times Y\] that is natural in both variables.
\end{definition}

The product/coproduct is called a \emph{biproduct}, with the biproduct structure denoted by $(\biprod, 0)$ and
projections $\biproj_X, \biproj_Y$ and injections $\biinj_X$ and $\biinj_Y$. The unit $0$ is both terminal and
initial and is called a \emph{zero} object. A semi-additive category $\cat{C}$ is enriched in $\CMon$: for any
two morphisms $f, g: X \to Y$ in $\cat{C}$, the biproduct structure provides a way to ``add'' them together,
forming a morphism $f + g: X \to Y$. Diagrammatically:

\begin{center}
\begin{tikzcd}
   X \arrow[r, "\diag"] & X \biprod X \arrow[r, "f \biprod g"] & Y \biprod Y \arrow[r, "\codiag"] & Y
\end{tikzcd}
\end{center}

Here $\diag$ denotes the diagonal $\prodM{\id_X}{\id_X}$ given by the universal property of the product and
$\codiag$ denotes the codiagonal $\coprodM{\id_X}{\id_Y}$ given by the universal property of the coproduct.
The $f \oplus g$ morphism is the component-wise map $\prodM{f \comp \biproj_X}{g \comp \biproj_Y} =
\coprodM{\biinj_X \comp f}{\biinj_Y \comp g}$. Similarly we can exhibit a \emph{zero} morphism $0_{X,Y}$ by
composing the unique maps in and out of the zero object:

\begin{center}
\begin{tikzcd}
   X \arrow[r, "!_X"] & 0 \arrow[r, "!^Y"] & Y
\end{tikzcd}
\end{center}

It is easy to verify that $+$ is associative and commutative and that $f + 0_{X,Y} = f$, and thus that every
hom-object $\cat{C}(X,Y)$ is an object in $\CMon$. Moreover composition is \emph{bilinear}, i.e.~given by a
family of morphisms $\Hom{\cat{C}}{Y}{Z} \tensor \Hom{\cat{C}}{X}{Y} \to \Hom{\cat{C}}{X}{Z}$ in $\CMon$ that
preserve the additive structure in $\Hom{\cat{C}}{Y}{Z}$ and $\Hom{\cat{C}}{X}{Y}$ separately:

\begin{salign*}
f \comp \zero_{X,Y} = 0_{X,Z} = \zero_{Y,Z} \comp f
\end{salign*}
\begin{salign*}
(f + g) \comp h &= (f \comp h) + (g \comp h) \\
h \comp (f + g) &= (h \comp f) + (h \comp g)
\end{salign*}

\subsubsection{Biproduct laws}
It is also easy to show that in a semi-additive category the following equations hold:
%\vspace{-4mm}
\begin{minipage}[t]{0.45\textwidth}
\begin{center}
\begin{salign*}
   \biproj_X \comp \biinj_X &= \id_X \\
   \biproj_Y \comp \biinj_X &= \zero_{X,Y}
\end{salign*}
\end{center}
\end{minipage}%
\begin{minipage}[t]{0.45\textwidth}
\begin{center}
\begin{salign*}
   \biproj_Y \comp \biinj_Y &= \id_Y \\
   \biproj_X \comp \biinj_Y &= \zero_{Y,X}
\end{salign*}
\end{center}
\end{minipage}

\begin{salign*}
(\biinj_X \comp \biproj_X) + (\biinj_Y \comp \biproj_Y) &= \id_{X \biprod Y}
\end{salign*}

\vspace{3mm}
\noindent It is also possible to use these laws to define biproducts, in the case where $\cat{C}$ is enriched
in $\CMon$ and so equipped with addition of morphisms and zero morphisms. In that case any products or
coproducts in $\cat{C}$ are necessarily biproducts.

\begin{proposition}
\label{prop:biproducts:from-product-or-coproduct}
In a category $\cat{C}$ enriched in $\CMon$:
\begin{enumerate}
\item If $(X \times Y, \pi_1, \pi_2)$ is a product then $(X \times Y, \inj_1, \inj_2)$ is a coproduct, where
$\inj_X = \prodM{\id_X}{0_{X,Y}}: X \to X \times Y$ and $\inj_Y = \prodM{0_{X,Y}}{\id_Y}: Y \to X \times Y$.
\item If $(X \coprod Y, \inj_1, \inj_2)$ is a coproduct then $(X \coprod Y, \pi_1, \pi_2)$ is a product, where
$\pi_1 = \coprodM{\id_X}{0_{Y,X}}: X \coprod Y \to X$ and $\pi_2 = \coprodM{0_{X,Y}}{\id_Y}: X \coprod Y \to
Y$.
\item If $X$ is terminal (resp.~initial) then $X$ is initial (resp.~terminal).
\end{enumerate}
\end{proposition}

A biproduct category $\cat{C}$ cannot be Cartesian closed without being trivial. If $0 \times X \iso X$ then
$\Hom{\cat{C}}{X}{Y} \iso \Hom{\cat{C}}{0}{X \Rightarrow Y} \iso 1$, and so $\cat{C}$ is terminal, with only a
single object up to isomorphism.

\subsubsection{Example: Category of Finite Vector Spaces}
\label{sec:categories-with-biproducts:fdvect}

The category $\FinVect$ of finite vector spaces over $\RR$ (with addition of real numbers written $a + b$ and
multiplication written $a \mult b$) is semi-additive.

\begin{definition}[Finite dimensional vector space over $\RR$]
Up to isomorphism, an \emph{$n$-dimensional vector space over $\RR$} is the set $\RR^n$ equipped with
component-wise addition of vectors and scalar multiplication (also written $+$ and $\mult$), where $(\RR^n,+)$
is an abelian group (with identity $0$ and additive inverse $-v$ again defined component-wise), and where the
vector operations are compatible with the field operations in that the following equations hold \todo{probably
more detail than we need to go into}:

\vspace{-4mm}
\begin{minipage}[t]{0.45\textwidth}
\begin{center}
\begin{align*}
   1 \mult v &= v \\
   (a \mult b) \mult v &= a \mult (b \mult v)
\end{align*}
\end{center}
\end{minipage}%
\begin{minipage}[t]{0.45\textwidth}
\begin{center}
\begin{align*}
   (a + b) \mult v &= (a \mult v) + (b \mult v) \\
   a \mult (u + v) &= (a \mult u) + (b \mult v)
\end{align*}
\end{center}
\end{minipage}
\end{definition}

\vspace{1mm}
\noindent Composition is with bilinear. The direct sum $V \biprod W \iso \RR^{m + n}$ given by the Cartesian
product is a biproduct $(V \biprod W, \biinj_{V}, \biinj_{W}, \pi_1, \pi_2)$ with $\biinj_{V} =
\prodM{\id_{V}}{0_{V,W}}$ and $\biinj_{W} = \prodM{0_{V,W}}{\id_{W}}$.

\begin{definition}
Define $\FinVect$ to be the category which has as objects $\RR^n$ all finite-dimensional vector spaces, and as
morphisms $f: \RR^m \to \RR^n$ all functions $f$ satisfying $f(u + v) = f(u) + f(v)$ and $f(a \mult v) = a
\mult f(v)$, with $f$ preserving the abelian group structure automatically.
\end{definition}

$\FinVect$ is enriched in $\Ab$, the category of abelian groups \todo{Introduce $\CMon$-enrichment first, then
perhaps $\Ab$-enrichment parenthetically}. For any objects $V = \RR^m$ and $W = \RR^n$ the hom-set
$\FinVect(V,W)$ is an abelian group, with identity $0_{V,W}$ given by the constant map $v \mapsto 0_W$ and
addition and additive inverse given pointwise:

\vspace{-4mm}
\begin{center}
\begin{minipage}[t]{0.35\textwidth}
\begin{align*}
(f + g)(v) &= f(v) + g(v)
\end{align*}
\end{minipage}%
\begin{minipage}[t]{0.35\textwidth}
\begin{align*}
(-f)(v) &= -f(v)
\end{align*}
\end{minipage}
\end{center}

\subsubsection{Example: Category of Lattices and Galois Connections}

\todo{Rework to account for new ordering.} Galois connections are pairs of monotone functions $f: Y \to X$ and
$g: X \to Y$ between posets, where $f$ is the (pointwise) best approximation from below to an inverse of $g$,
and $g$ the best approximation from above to an inverse of $f$. The category $\LatGal$ of bounded lattices and
Galois connections is semi-additive.

\begin{definition}[Galois connection]
Suppose $X$ and $Y$ are posets. A \emph{Galois connection} $f \adj g: X \to Y$ is a pair of monotone functions
$f: Y \to X$ and $g: X \to Y$ satisfying $y \leq g(x) \iff f(y) \leq x$ for any $x \in X$ and $y \in Y$.
\end{definition}

\noindent The $\adj$ notation is justified because a Galois connection $f \adj g: X \to Y$ can also be seen an
adjunction between poset categories, with monotone $f$ and $g$ regarded as functors; $f$ is usually referred
to as the \emph{upper} (right) adjoint and $g$ as the \emph{lower} (left) adjoint. Galois connections compose
component-wise, with $\id_X \adj \id_X: X \to X$ as the unit for composition.

\begin{definition}
Define $\LatGal$ to be the category which has as objects $X = (X, \meet, \join, \top, \bot)$ all bounded
lattices, and as morphisms all Galois connections $f \adj g: X \to Y$.
\end{definition}

\noindent Right adjoints preserves limits and left adjoints preserves colimits, so for any $f \adj g: X \to Y$
in $\LatGal$, $g$ is a (bounded) \emph{meet-semilattice homomorphism}, i.e.~preserves the meet-semilattice
structure $(X, \meet, \top)$. Similarly, $f$ is a join-semilattice homomorphism with respect to $(X, \join,
\bot)$.

\todo{Again, introduce $\CMon$-enrichment first and relegate $\SemiLat$-enrichment to parenthetical remark.}
The meet-preserving maps from $X$ to $Y$ themselves form a meet-semilattice. The operation $\meet$ on such
maps is given pointwise, so that $(f \meet g)(x) = f(x) \meet g(x): X \to Y$, and preserves meets; the
constant map $\top_{X,Y}: X \to Y$ sending any $x \in X$ to $\top_Y$, which also preserves meets, provides the
unit. Dually the join-preserving maps from $Y$ to $X$ have a join-semilattice structure given pointwise by
$\join$ and the constant map $\bot_{Y,X}: Y \to X$ sending any $y \in Y$ to $\bot_X$. Since these two
constructions come in adjoint pairs, the hom-set $\LatGal(X,Y)$ forms a bounded meet semilattice with unit
$\top_{X,Y}$ given by the Galois connection $\bot_{Y,X} \adj \top_{X,Y}: X \to Y$ and meet of Galois
connections $(f \adj g) \meet (f' \adj g') = (f \join f') \adj (g \meet g'): X \to Y$. Thus $\LatGal$ is
enriched in $\SemiLat$, the category of semilattices and semilattice homomorphisms.

The projections $\pi_1: X \times Y \to X$ and $\pi_2: X \times Y \to Y$, where $X \times Y$ denotes the
product of lattices, have both upper and lower adjoints; this means that $X \times Y$, which we shall
hereafter write as $X \biprod Y$, is both a product and a coproduct, with projections $\biproj_X$ and
$\biproj_Y$ and injections $\biinj_X$ and $\biinj_Y$ given by:

\vspace{-4mm}
\begin{minipage}[t]{0.45\textwidth}
\begin{center}
\begin{align*}
   \biproj_X = \prodM{\id_X}{\bot_Y} \adj \proj_1: X \biprod Y \to X \\
   \biproj_Y = \prodM{\bot_X}{\id_Y} \adj \proj_2: X \biprod Y \to Y
\end{align*}
\end{center}
\end{minipage}%
\begin{minipage}[t]{0.45\textwidth}
\begin{center}
\begin{align*}
   \biinj_X = \pi_1 \adj \prodM{\id_X}{\top_Y} : X \to X \biprod Y \\
   \biinj_Y = \pi_2 \adj \prodM{\top_X}{\id_Y}: Y \to X \biprod Y
\end{align*}
\end{center}
\end{minipage}
\vspace{2mm}

\noindent The trivial 1-point lattice, which is both terminal and initial and we write as $0$, is the unit for
$\biprod$.

\subsection{Grothendieck Construction for Indexed Categories}
\label{sec:Grothendieck}

\todo{Rework.} The goal now is to equip the interpretation of a (first-order) program with a Galois connection
relating approximations of its inputs to approximations of its output. Since the specific lattices of
approximations depend on the value being approximated, the semantics of every type $\tau$ must have both a
conventional component as well as an approximating component, i.e.~a set $X$ which serves as the usual
interpretation of $\tau$, plus an associated lattice of approximations $\partial X(x)$ for every point $x \in
X$. The semantics of an expression $e$ of type $\sigma$ with a single free variable of type $\tau$ (with
$\tau$ and $\sigma$ denoting $X$ and $Y$) must then also have two components: a function $f: X \to Y$ giving
the conventional interpretation of $e$, plus a family of Galois connections $\partial f(x): \partial X(x) \to
\partial Y(f(x))$ for every $x \in X$ which can be used for slicing over that expression.

This pattern of objects and morphisms is captured by the construction used in CHAD~\cite{vákár22,nunes2023} to
model automatic differentiation, namely the \emph{Grothendieck construction} for a particular set-indexed
category $T: \Set^\op \to \Cat$. For automatic differentiation, $T(X)$ will be the functor category
$\Func{X}{\FinVect}$, regarded as the category of $X$-indexed families of finite vector spaces, with indexed
linear maps as morphisms. For Galois slicing, we will take $T(X) = \Func{X}{\LatGal}$, the category of
$X$-indexed families of bounded lattices, with indexed Galois connections as morphisms.

In its general form, the Grothendieck construction $\Grothendieck{}T$ for an arbitrary indexed category $T:
\cat{C}^\op \to \Cat$, sometimes called the \emph{total category} for $T$, assembles all the categories $T(X)$
(for objects $X$ in $\cat{C}$) into a single category, together with morphisms that account for both internal
structure and reindexing along morphisms $f: X \to Y$ in $\cat{C}$.

\begin{definition}
\label{def:Grothendieck}
Suppose an indexed category $T: \cat{C}^\op \to \Cat$. The \emph{Grothendieck construction} for $T$ is the
category $\Grothendieck{X}T(X)$ (also written $\Grothendieck{}T$) which has as objects, all pairs $(X,
\partial X)$ of an object $X$ of $\cat{C}$ and an object $\partial X$ of $T(X)$, and as morphisms $(X,
\partial X) \to (Y, \partial Y)$, all pairs $(f, \partial f)$ of morphisms $f: X \to Y$ in $\cat{C}$ and
morphisms $\partial f: \partial X \to T(f)(\partial Y)$ in $T(X)$.
\end{definition}

\noindent The $\partial X$ notation used here is intended to suggest the connection between the Grothendieck
construction and the idea of tangent spaces and derivatives. To understand how the $\partial f$ components
compose, consider any $(f, \partial f): (X, \partial X) \to (Y, \partial Y)$ and $(g, \partial g): (Y,
\partial Y) \to (Z, \partial Z)$ in $\Grothendieck{}T$. We have:

\begin{itemize}
\item $\partial f: \partial X \to T(f)(\partial Y)$, and
\item $T(f)(\partial g): T(f)(\partial Y) \to T(f)(T(g)(\partial Z)) = T(g \comp f)(\partial Z)$
\end{itemize}

\noindent where $T(f)(\partial g)$ is the image of $\partial g$ in the ``reindexing'' functor $T(f)$. The
composition $(g, \partial g) \comp (f, \partial f): (X, \partial X) \to (Z, \partial Z)$ is then given by $(g
\comp f, T(f)(\partial g) \comp \partial f)$.

\subsection{Category of Families}
\label{sec:Fam}

Now we formalise $X$-indexed families as functor categories $\Func{X}{\cat{C}}$ and consider the specific case
of the Grothendieck construction for set-indexed categories $\Func{-}{\cat{C}}: \Set^\op \to \Cat$.

\begin{definition}[Category of $X$-indexed families]
For any set $X$ (regarded as a discrete category) and any category $\cat{C}$, recall that the functor
category $\Func{X}{\cat{C}}$ has as objects all functors $F: X \to \cat{C}$, namely $X$-indexed families of
objects of $\cat{C}$, and as morphisms from $F \to G$, all families of morphisms $\eta(x): F(x) \to G(x)$ in
$\cat{C}$ for every $x \in X$, with naturality trivial because $X$ has only identity morphisms.
\end{definition}

Like any functor category $\Func{X}{\cat{C}}$ inherits limits and colimits pointwise from its codomain.

\begin{proposition}
If $\cat{C}$ has limits (resp.~colimits) then $\Func{X}{\cat{C}}$ has limits (resp.~colimits) computed
pointwise.
\end{proposition}

\begin{definition}[Reindexing]
For any function $f: X \to Y$ define the \emph{reindexing} functor $\reindex{-}{f}: \Func{Y}{\cat{C}} \to
\Func{X}{\cat{C}}$ which sends any $Y$-indexed family $F$ over $\cat{C}$ to the $X$-indexed family
$\reindex{F}{f} = F \comp f$ (regarding $f$ as a functor between $X$ and $Y$ as discrete categories) and any
family of morphisms $\eta: F \to G$ to $\reindex{\eta}{f}: \reindex{F}{f} \to \reindex{G}{f}$ with
$\reindex{\eta}{f}(x) = \eta(f(x))$. \todo{Should be clearer that $\reindex{-}{f}$ is actually
$\Func{f}{\cat{C}}$.}
\end{definition}

\noindent Although reindexing is just precomposition with $f$, writing it as $\reindex{-}{f}$ avoids
notational confusion when combining composition of morphisms in $\Func{X}{C}$ with reindexing.

The Grothendieck construction for $\Func{-}{\cat{C}}: \Set^{\op} \to \Cat$, the indexed category which sends
any set $X$ to the category of indexed families $\Func{X}{\cat{C}}$ and any function $f: X \to Y$ to the
reindexing functor $\reindex{-}{f}: \Func{Y}{\cat{C}} \to \Func{X}{\cat{C}}$, is called the \emph{families}
construction for $\cat{C}$.

\begin{definition}[Categories of families]
\label{def:Fam}
For any category $\cat{C}$ define $\Fam(\cat{C})$ to be the Grothendieck construction
$\Grothendieck{X}\Func{X}{\cat{C}}$.
\end{definition}

\noindent In $\Fam(\cat{C})$ the objects $(X, \partial X)$ are thus all sets $X$ paired with indexed families
$\partial X: X \to \cat{C}$ and the morphisms $(f, \partial f): (X, \partial X) \to (Y, \partial Y)$ are all
functions $f: X \to Y$ paired with morphisms $\partial f: \partial X \to \partial \reindex{Y}{f}$ in
$\Func{X}{\cat{C}}$.

\begin{proposition}
\label{prop:Grothendieck:fam-inherits-local-smallness}
If $\cat{C}$ is locally small then so is $\Fam(\cat{C})$.
\end{proposition}

\begin{proposition}
\label{prop:Grothendieck:fam-inherits-products}
If $\cat{C}$ is (symmetric) monoidal then $\Fam(\cat{C})$ is also (symmetric) monoidal, with tensor product
$(X, \partial X) \tensor (Y, \partial Y) = (X \times Y,(x, y) \mapsto \partial X(x) \tensor \partial Y(y))$.
\end{proposition}

The action of the families construction, regarded as a covariant 2-functor $\Cat \to \Cat$, sends functors $F:
\cat{C} \to \cat{D}$ to functors $\Fam(F): \Fam(\cat{C}) \to \Fam(\cat{D})$ that use $F$ to reassign the
target category, sending objects $(X, \partial X)$ to $(X, F \comp \partial X)$ and morphisms $(f, \partial
f)$ to $(f, F \comp \partial f)$, where $F \comp \partial f$ denotes the natural transformation with $(F \comp
\partial f)(x) = F(\partial f(x))$. If $F$ happens to be a (strong) monad on $\cat{C}$, then $\Fam(F)$ is a
(strong) monad on $\Fam(\cat{C})$, defined by lifting the original monad $F: \cat{C} \to \cat{C}$ pointwise
over families.

\subsection{First-Order Galois Slicing via CHAD}
\label{sec:fam:galois-slicing}

The total category $\Fam(\LatGal)$ is then a suitable setting for interpreting first-order programs for \GPS.
It has as objects $(X, \partial X)$, all pairs of a set $X$ and and for every $x \in X$, a bounded lattice
$\partial X(x)$, and as morphisms $(X, \partial X) \to (Y, \partial Y)$, all pairs $(f, \partial f)$ of a
function $f: X \to Y$ and for every $x \in X$, a Galois connection $\partial f(x): \partial X(x) \to \partial
Y(f(x))$. This is precisely the setup we identified informally at the beginning of this section. The
reindexing along $f$ in the composition $(g, \partial g) \comp (f, \partial f) = (g \comp f, \reindex{\partial
g}{f} \comp \partial f)$ selects the appropriate lattice of approximations and Galois connection at each
point, using the baseline (unapproximated) function $f$.

\todo{Show how $\LatGal$ has a lifting monad?}

\subsection{First-Order Automatic Differentiation via CHAD}

Following the CHAD approach, automatic differentiation for first-order programs can also be recovered in this
setup, via the total category $\Fam(\FinVect)$. We recapitulate the basic ideas here.

\subsubsection{Derivatives of Differentiable Functions}

For flat manifolds like $\RR^n$, the tangent space and cotangent space at a point $x$ are both canonically
isomorphic to $\RR^n$, so the forward and backward derivatives of a differentiable function $f: \RR^m \to
\RR^n$ at a point $x$ have the following form:

\begin{itemize}
\item The \emph{forward derivative} (tangent map or pushforward) $\pushf{f}_x$ of $f$ at $x \in \RR^m$ is the
unique linear map $\RR^m \linearto \RR^n$ given by the Jacobean matrix of $f$ at $x$.
\item The \emph{backward derivative} (cotangent map or pullback) $\pullf{f}_x$ is the unique linear map
$\RR^n \linearto \RR^m$ given by the transpose (adjoint) of the Jacobean matrix of $f$ at $x$.
\end{itemize}

\subsubsection{Chain Rule}

Derivatives respect composition~\cite{spivak65}. Suppose $f: \RR^m \to \RR^n$ and $g: \RR^n \to \RR^k$. Then
for any $x \in \RR^m$ we have:

\begin{itemize}
\item $\pushf{(g \comp f)}_x = \pushf{g}_{f(x)} \comp \pushf{f}_x: \RR^m \to \RR^k$
\item $\pullf{(g \comp f)}_x = \pullf{f}_{x} \comp \pullf{g}_{f(x)}: \RR^k \to \RR^m$
\end{itemize}

\subsubsection{Forward-Mode Automatic Differentiation}

The basic idea of (forward-mode) automatic differentiation, as first described by \citet{linnainmaa76}, was to
recognise that one could efficiently compute the pushforward map $\pushf{f}: \RR^m \to (\RR^m \linearto
\RR^n)$ of a differentiable function $f: \RR^m \to \RR^n$ alongside the computation of $f$ itself, by defining
a single map $\tangents(f) = x \mapsto (f(x), \pushf{f}_x): \RR^m \to \RR^n \times (\RR^m \linearto \RR^n)$.
Defining $f_1 = \pi_1 \comp f: \RR^m \to \RR^n$ and $f_2 = \pi_2 \comp f: \RR^m \to (\RR^m \linearto \RR^n)$,
composition of such maps can be expressed as:
\begin{align*}
(g \comp f)_1(x) &= f_1(g_1(x)) \\
(g \comp f)_2(x) &= g_2(f_1(x)) \comp f_2(x)
\end{align*}

We can verify that $\tangents$ is functorial, regardless of whether the linear maps $\RR^m \linearto \RR^n$
happen to be derivatives. But if $f_2(x)$ is the derivative of $f_1$ at $x$ for any $x \in \RR^m$ and
similarly for $g$ then the derivatives compose according to the forward chain rule and $(g \comp f)_2(x)$ is
the derivative of $(g \comp f)_1$ at $x$. Thus this setup provides an algorithmic method for computing the
forward derivative of a composite function, without have to recompute the function itself, as long as all
atomic operations provide derivatives. An efficient implementation of \GPS based on $\Fam(\LatGal)$
(\secref{fam:galois-slicing} above) would most likely apply a similar optimisation to make it possible to
compute forwards slices without having to reexecute the program at the same time.

We can see these ``optimised'' forward-mode maps as living in $\Fam(\FinVect)$ if we regard every finite
vector space $\RR^n$ as paired with the constant family of tangent spaces at $\RR^n$ (also written $\RR^n$,
for convenience), and each map $f: \RR^m \to \RR^n \times (\RR^m \linearto \RR^n)$ as the pair $(f_1, f_2)$.
Then the composition $(g, \partial g) \comp (f, \partial f) = (g \comp f, \reindex{\partial g}{f} \comp
\partial f)$ again corresponds exactly to the forward chain rule, since $(\reindex{\partial g}{f} \comp
\partial f)_x = {\partial g}_{f(x)} \comp {\partial f}_x$, with the reindexing along $f$ recomputing $f(x)$ in
order to select the appropriate pushforward.

\subsubsection{Reverse-Mode Automatic Differentiation}

\todo{brief summary}
