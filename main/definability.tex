\section{Correctness of the Higher-Order Interpretation}
\label{sec:definability}

As we noted at the end of \secref{language:gps-interpretation}, our
interpretation of the higher-order language is in the category
$\Fam(\MeetSLat \times \JoinSLat^\op)$, so it is not a priori evident
that we get a Galois connection from the interpretation of a program
with first-order type (that may use higher-order functions
internally). \citet{vakar22} and \citet{nunes2023} construct custom
instances of categorical sconing arguments to prove correctness of
their higher-order interpretation with respect to normal
differentiation. Instead of doing this, we make use of a general
\emph{syntax free} theorem due to \citet{fiore-simpson99}. The proof
of this depends on the construction of a Grothendieck Logical Relation
over the extensive topology on the category $\cat{C}$, but the
statement of the theorem does not rely on this. We have formalised
this proof in Agda (see \texttt{conservativity.agda} in the
supplementary material\footnote{Our Agda development is complete
  except for a proof that $\Fam(C)$ has extensive coproducts. We plan
  to complete this part of the proof before any final
  version. Moreover, this result does not yet apply to infinitary
  coproducts, though we believe it is a relatively minor extension to
  the proof to do so.}).

\begin{theorem}[\citet{fiore-simpson99}]
  \label{thm:glr-definability}
  Let $\cat{C}$ be an extensive bicartesian category, $\cat{D}$ be a
  bicartesian closed category, and $F : \cat{C} \to \cat{D}$ a functor
  preserving finite products and coproducts. Then there is a category
  $\GLR(F)$ and functors $p : \GLR(F) \to \cat{D}$ and
  $\hat{F} : \cat{C} \to \GLR(F)$, such that:
  \begin{enumerate}
  \item $\GLR(\cat{D},F)$ is bicartesian closed;
  \item $F = p \circ \hat{F} : \cat{C} \to \cat{D}$;
  \item The functor $p$ strictly preserves the bicartesian closed structure; and
  \item The functor $\hat{F}$ is full and preserves the bicartesian structure.
  \end{enumerate}
\end{theorem}

\begin{remark}
  Compared to the exact result stated at the end of
  \citet{fiore-simpson99}'s paper, we have made two modifications,
  justified by our Agda proof. First, we generalise to the case where
  $\cat{C}$ is not cartesian closed, and the functor $F$ does not
  preserve exponentials. Examination of the proof reveals that if this
  is the case, then $\hat{F}$ also preserves exponentials, but it is
  not needed for the result stated. Second, Fiore and Simpson restrict
  to the case when $\cat{C}$ is small to be able to construct
  Grothendieck sheaves on this category. We use Agda's universe
  hierarchy to simply construct ``large'' sheaves at the the
  appropriate universe level.
\end{remark}

\begin{theorem}
  \label{thm:language-definability}
  For all $\Gamma \vdash M : \tau$, where $\Gamma$ and $\tau$ are
  first-order, there exists a morphism
  $g \in \Fam(\LatGal)(\sem{\Gamma}_{\mathit{fo}},
  \sem{\tau}_{\mathit{fo}})$ such that
  $H(g) = (\cong) \circ \sem{\Gamma \vdash M : \tau} \circ (\cong)$,
  where the isomorphisms arise from
  \lemref{first-order-agreement-types}.
\end{theorem}

\begin{proof}
  Instantiate \thmref{glr-definability} with
  $\HoEmbed : \Fam(\LatGal) \to \Fam(\MeetSLat \times
  \JoinSLat^\op)$. By \propref{ho-embedding} we know that $F$
  preserves finite products and coproducts. The fullness of
  $\hat{\HoEmbed}$ means that for any morphism
  $f : \hat{\HoEmbed}(\sem{\Gamma}_{\mathit{fo}}) \to
  \hat{H}(\sem{\tau}_{\mathit{fo}})$ in $\GLR(\HoEmbed)$ there exists
  a $g : \sem{\Gamma}_{\mathit{fo}} \to \sem{\tau}_{\mathit{fo}}$ in
  $\Fam(\LatGal)$ such that $H(g) = f$. Since $\GLR(\HoEmbed)$ has
  enough structure, we can interpret the term $M$ in it to get a
  morphism
  $\sem{\Gamma \vdash M : \tau}_{\GLR(\HoEmbed)} : \sem{\Gamma} \to
  \sem{\tau}$ in $\GLR(\HoEmbed)$. Applying
  \lemref{first-order-agreement-types} and the fact that the
  strictness of $p$ means that
  $p(\sem{\Gamma \vdash M : \tau}_{\GLR(\HoEmbed)}) = \sem{\Gamma
    \vdash M : \tau}_{\Fam(\MeetSLat \times \JoinSLat^\op)}$ yields
  the result.
\end{proof}

\begin{remark}
  Note that even if we modified our base interpretation of semantic
  \GPS along the lines of \remref{further-structure} to give a refined
  version $\cat{G}$ of $\Fam(\LatGal)$, as long as there is a finite
  product and coproduct preserving functor
  $\cat{G} \to \Fam(\MeetSLat \times \JoinSLat^\op)$, then the
  analogous result to \thmref{language-definability} still holds.
\end{remark}

We can also use \thmref{glr-definability} to show that the
interpretation of the language in the category $\Set$ agrees with the
higher-order interpretation in $\Fam(\MeetSLat \times \JoinSLat^\op)$
on the underlying function at first order. This shows that the
higher-order interpretation does what we expect in the underlying
interpretation of terms, and that the approximation information does
not interfere.

\begin{theorem}
  \label{thm:underlying-interp-equal} For all
  $\Gamma \vdash M : \tau$, where $\Gamma$ and $\tau$ are first-order,
  the interpretation $\sem{\Gamma \vdash M : \tau}_{\Set}$ in $\Set$
  and the interpretation
  $\sem{\Gamma \vdash M : \tau}_{\Fam(\MeetSLat \times
    \JoinSLat^\op)}$ agree on their first components.
\end{theorem}

\begin{proof}
  Instantiate \thmref{glr-definability} with the functor
  $\langle \mathrm{Id} , \pi_1 \rangle : \Fam(\MeetSLat \times
  \JoinSLat^\op) \to \Fam(\MeetSLat \times \JoinSLat^\op) \times \Set$
  that is the identity in the first component and projects out the
  underlying function in the second. We get, for each
  $\Gamma \vdash M : \tau$, a $g$ such that
  $g = \sem{\Gamma \vdash M : \tau}_{\Fam(\MeetSLat \times
    \JoinSLat^\op)}$ and
  $\pi_1(g) = \sem{\Gamma \vdash M : \tau}_\Set$. Therefore,
  $\pi_1(\sem{\Gamma \vdash M : \tau}_{\Fam(\MeetSLat \times
    \JoinSLat^\op)}) = \sem{\Gamma \vdash M : \tau}_\Set$.
\end{proof}
