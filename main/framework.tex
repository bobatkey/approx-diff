\section{A categorical framework for \GPS}

We will consider a pure, total language, where types can be interpreted as sets. Our goal is to equip the
interpretations of programs with Galois connections relating approximations of their inputs to approximations
of their outputs. Since these lattices of approximations depends on the value being approximated, the
semantics of every type $\tau$ must have both a ``conventional'' component, providing the set $X$ which
interprets $\tau$, plus an associated lattice of approximations $\partial X(x)$ for every point $x \in X$. The
semantics of an expression $e$ of type $\sigma$ with a single free variable of type $\tau$ (with $\tau$ and
$\sigma$ conventionally interpreted by $X$ and $Y$) must then also have two components: a function $f: X \to
Y$ giving the conventional interpretation of $e$, and a family of Galois connections $\partial f(x): \partial
X(x) \to \partial Y(f(x))$ for every $x \in X$ which can be used for slicing over that expression.

\subsection{Grothendieck Construction for Indexed Categories}
\label{sec:Grothendieck}

This pattern of objects and morphisms is exactly the \emph{Grothendieck construction} for a particular indexed
category $\Set^\op \to \Cat$, namely the indexed category $\Func{-}{\LatGal}$ which assigns to every set $X$
the functor category $\Func{X}{\LatGal}$. Here the (discrete) category $X$ is a plain set and so
$\Func{X}{\LatGal}$ is exactly the category of $X$-indexed families of bounded lattices, with $X$-indexed
families of Galois connections between them as morphisms. The Grothendieck construction
$\Grothendieck{}T$~(\defref{Grothendieck} below) for an indexed category $T$, sometimes called the \emph{total
category} for $T$, is the category of all $\Set$-indexed families of objects of $T$, together with morphisms
that account for both structure within each family and how families transform under reindexing.

First we formalise $X$-indexed families and their categories.

\begin{definition}[Category of $X$-indexed families]
For any set $X$ (regarded as a discrete category) and any category $\cat{C}$, recall that the functor
category $\Func{X}{\cat{C}}$ has as objects all functors $F: X \to \cat{C}$, namely $X$-indexed families of
objects of $\cat{C}$, and as morphisms from $F \to G$, all families of morphisms $\eta(x): F(x) \to G(x)$ in
$\cat{C}$ for every $x \in X$, with naturality trivial because $X$ has only identity morphisms.
\end{definition}

Like any functor category $\Func{X}{\cat{C}}$ inherits limits and colimits pointwise from its codomain.

\begin{proposition}
If $\cat{C}$ has limits (resp.~colimits) then $\Func{X}{\cat{C}}$ has limits (resp.~colimits) computed
pointwise.
\end{proposition}

\begin{definition}[Reindexing]
For any function $f: X \to Y$ define the \emph{reindexing} functor $\reindex{-}{f}: \Func{Y}{\cat{C}} \to
\Func{X}{\cat{C}}$ which sends any $Y$-indexed family $F$ over $\cat{C}$ to the $X$-indexed family
$\reindex{F}{f} = F \comp f$ (regarding $f$ as a functor between $X$ and $Y$ as discrete categories) and any
family of morphisms $\eta: F \to G$ to $\reindex{\eta}{f}: \reindex{F}{f} \to \reindex{G}{f}$ with
$\reindex{\eta}{f}(x) = \eta(f(x))$.
\end{definition}

Although reindexing is just precomposition with $f$, writing it as $\reindex{-}{f}$ avoids notational
confusion when combining composition of morphisms in $\Func{X}{C}$ with reindexing.

\begin{definition}
\label{def:Grothendieck}
Suppose an indexed category $T: \cat{C}^\op \to \Cat$. The \emph{Grothendieck construction} for $T$ is the
category $\Grothendieck{X}T(X)$ (also written $\Grothendieck{}T$) which has as objects, all pairs $(X,
\partial X)$ of an object $X$ of $\cat{C}$ and an object $\partial X$ of $T(X)$, and as morphisms $(X,
\partial X) \to (Y, \partial Y)$, all pairs $(f, \partial f)$ of morphisms $f: X \to Y$ in $\cat{C}$ and
morphisms $\partial f: \partial X \to T(f)(\partial Y)$ in $T(X)$.
\end{definition}

\noindent The $\partial X$ notation is intended to suggest the connection between the Grothendieck
construction and the idea of tangent spaces and derivatives. To understand how the $\partial f$ components
compose, consider any $(f, \partial f): (X, \partial X) \to (Y, \partial Y)$ and $(g, \partial g): (Y,
\partial Y) \to (Z, \partial Z)$ in $\Grothendieck{}T$. We have:

\begin{itemize}
\item $\partial f: \partial X \to T(f)(\partial Y)$, and
\item $T(f)(\partial g): T(f)(\partial Y) \to T(f)(T(g)(\partial Z)) = T(g \comp f)(\partial Z)$
\end{itemize}

\noindent where $T(f)(\partial g)$ is the image of $\partial g$ in the ``reindexing'' functor $T(f)$. The
composition $(g, \partial g) \comp (f, \partial f): (X, \partial X) \to (Z, \partial Z)$ is then given by $(g
\comp f, T(f)(\partial g) \comp \partial f)$.

\subsection{Category of Families}
\label{sec:Fam}

Of interest to us is the Grothendieck construction for $\Func{-}{\cat{C}}: \Set^{\op} \to \Cat$, the indexed
category which sends any set $X$ to the category of $X$-indexed families $\Func{X}{\cat{C}}$ and any function
$f: X \to Y$ to the reindexing functor $\reindex{-}{f}: \Func{Y}{\cat{C}} \to \Func{X}{\cat{C}}$ .

\begin{definition}[Category of families]
\label{def:Fam}
For any category $\cat{C}$ define $\Fam(\cat{C})$ to be the Grothendieck construction
$\Grothendieck{X}\Func{X}{\cat{C}}$.
\end{definition}

\noindent In $\Fam(\cat{C})$ the objects $(X, \partial X)$ are thus sets $X$ paired with indexed families
$\partial X: X \to \cat{C}$ and morphisms $(f, \partial f): (X, \partial X) \to (Y, \partial Y)$ are functions
$f: X \to Y$ paired with morphisms $\partial f: \partial X \to \partial \reindex{Y}{f}$ in
$\Func{X}{\cat{C}}$.

\subsection{Galois Slicing via the Grothendieck Construction}

Now consider the total category $\Fam(\LatGal)$, which is our basic setting for interpreting first-order
programs for \GPS. This has as objects $(X, \partial X)$, all pairs of a set $X$ and and for every $x \in X$,
a bounded lattice $\partial X(x)$, and as morphisms $(X, \partial X) \to (Y, \partial Y)$, all pairs $(f,
\partial f)$ of a function $f: X \to Y$ and for every $x \in X$, a Galois connection $\partial f(x): \partial
X(x) \to \partial Y(f(x))$, as we sketched informally at the beginning of this section. The pullback
(reindexing) along $f$ in the composition $(g, \partial g) \comp (f, \partial f) = (g \comp f,
\reindex{\partial g}{f} \comp \partial f)$ selects the appropriate lattice of approximations and Galois
connection at each point, using the baseline (unapproximated) function $f$.
