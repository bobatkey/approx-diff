\section{Constructing Models of Total \GPS}
\label{sec:models-of-total-gps}

TODO:
\begin{enumerate}
\item Link to the discussion in the previous section
\item Category of Families construction over a category $\cat{C}$:
  \begin{enumerate}
  \item Has (all) coproducts
  \item Has products when $\cat{C}$ does, and is distributive and
    stable.
  \end{enumerate}
\item Categories of interest: $\LatGal$, $\MeetSLat$, $\JoinSLat$ and
  (for normal differentiation) $\FinVect$.
\item Some functors between these models and from L-poset. Some
  interesting objects in these categories.
\item For higher-order models: we use a theorem of Vákár and Lucatelli
  Nunes: if $\cat{C}$ has is cmon-enriched with biproducts and has all
  (small) products, then $\Fam(\cat{C})$ is cartesian closed.
\item We can construct a list object in $\Fam(\cat{C})$ when it has
  products.
\end{enumerate}




We have seen that the idea of stability from domain theory can be seen
as a bridging concept between smooth functions and their derivatives
on one hand, and \GPS on the other. For the next part of our
development, we want to transport a mathematical framework for
automatic differentation from normal differentation to \GPS. We take
the \emph{Combinatory Homomorphic Automatic Differentation} (CHAD)
framework of Vákár \etal{} and apply it to \GPS. In this section we
develop this framework for first-order programs only. In
\autoref{sec:chad-higher-order}, we extend the framework to deal with
the whole higher-order language we present in \autoref{sec:language}.

\subsection{CMon-enriched and Biproduct Categories}

As we noted above\bob{where?}, one commonality between tangent spaces
and approximations is the ability to add tangents and to take the meet
or join of two approximations. For our purposes, the suitable common
abstraction is given by categories enriched in commutative monoids.

\begin{definition}
  \bob{CMon-categories}
\end{definition}

\begin{definition}
  \bob{Biproducts}
\end{definition}

\begin{example}
  FinVect
\end{example}

\begin{example}
  $\LatGal$, the category of bounded lattices and Galois connections.
\end{example}

\subsection{Families}

The basic idea behind the CHAD framework is to use Cmon-categories
with biproducts as the representation of tangent objects. Tangent
objects are associated with points formally by constructing a category
where objects consist of a pair of a set (of points) and for each
point a corresponding tangent object. This is exactly the
\emph{families} construction that freely completes any category with
all coproducts:

\begin{definition}
  \label{def:category-of-families}
  For any category $\cat{C}$, the \emph{category of families over
    $\cat{C}$}, $\Fam(\cat{C})$ has:
  \begin{itemize}
  \item Objects $(A, X)$, where $A$ is a set and
    $X : A \to \Ob\cat{C}$ is an $A$-indexed familiy of objects of
    $\cat{C}$;
  \item Morphisms $(A, X) \to (B, Y)$ are pairs $(f, \partial f)$
    where $f : A \to B$ is a function and
    $\partial f : \Pi_{x : A}.~\cat{C}(X(x), Y(f\,x))$ is an
    $A$-indexed family of morphisms of $\cat{C}$.
  \end{itemize}
\end{definition}

The key observation is that the composition of morphisms in Fam(C)
matches the \emph{chain rule} for composition of
derivatives. Therefore, in Fam(FinVect), if we have two morphisms
$(f, \partial f)$ and $(g, \partial g)$ where the second components
are the actual derivatives of the first components, then their
composition's second component is the derivative of the
composition. More formally:\bob{is this in Vákár's work?}

\begin{proposition}
  There is a faithful functor $F : \Man \to \Fam(\FinVect)$ that sends
  a manifold $M$ to $(M, \lambda x. T_x(M))$, and each smooth function
  $f$ to $(f, f_*)$, the pair of $f$ and its forward derivative.
\end{proposition}

As similar result is given in \cite{cruttwell2022}, where Euclidean
spaces $\RR^n$ and smooth functions are embedded into a category of
lenses (the ``simply typed'' version of the $\Fam$ construction).

For \GPS, we have an analogous result:

\begin{lemma}
  Faithful functor from L-poset to LatGal.
\end{lemma}



\begin{lemma}
  Fam(C) has all coproducts.
\end{lemma}

\begin{lemma}
  Fam(C) has products when C does.
\end{lemma}

More generally, get monoidal products.

\subsection{\GPS in Fam(LatGal)}

\begin{enumerate}
\item The lifting monad
\item Show some examples
\end{enumerate}
