\section{Models of Galois Slicing for a Total Language} % \GPS doesn't use title caps
\label{sec:models-of-total-gps}

The previous section concluded that L-posets and stable functions give
a model of \GPS analogous to manifolds and smooth functions. However,
we noted a conceptual shortcoming of this model, for the purposes of
modelling total computations, that proper values and their
approximations live in the same category. In this section, we propose
a model for total \GPS based on the \emph{Category of Families}
construction. This construction, and the more general Grothendieck
construction, has been previously used by V{\'a}k{\'a}r and
collaborators \cite{vakar22} to model automatic differentiation for
higher-order programs on the reals. We reuse some of their results,
and discuss the commonalities as we go.

\subsection{The Category of Families Construction}
\label{sec:models-of-total-gps:fam}

L-posets are partially ordered sets where every principal downset
$\downset{x}$ is a bounded lattice of approximations/tangents. As we
explained in \secref{stab}, the shortcoming of this setup is that
proper values and their approximations live in the same set. We fix
this by changing our model to one where we have sets $X$ of values,
and for each $x \in X$, a bounded lattice $\partial X(x)$ of
approximations of $x$. This construction is an instance of the general
\emph{Category of Families} construction:

\begin{definition}
  Let $\cat{C}$ be a category. The \emph{Category of Families} over
  $\cat{C}$, $\Fam(\cat{C})$, has as objects pairs $(X, \partial X)$,
  where $X$ is a set and $\partial X : X \to \cat{C}$ is an
  $X$-indexed family of objects in $\cat{C}$. A morphism
  $f : (X, \partial X) \to (Y, \partial Y)$ consists of a pair of a
  function $f : X \to Y$ and a family of morphisms of $\cat{C}$,
  $\partial f : \Pi_{x : X}.\,\cat{C}(\partial X(x), \partial Y(f\,x))$.
\end{definition}

The reason for choosing the $\Fam$ construction is that composition in
this category is an abstract version of the chain rule that we have
seen in \remref{chain-rule}, \remref{chain-rule-cm}, and
\remref{chain-rule-stable}. Composition $f \circ g $ of morphisms
$f : (Y, \partial Y) \to (Z, \partial Z)$ and
$g : (X, \partial X) \to (Y, \partial Y)$ in this category is given by
normal function composition on the set components, and
$\partial (f \circ g)(x) = \partial f(f, x) \circ \partial g(x)$,
where the latter composition is in $\cat{C}$.

The fact that morphisms in $\Fam(C)$ compose according to a chain rule
means that the categories we considered in \secref{approx-as-tangents}
embed into $\Fam(C)$ for appropriate $C$. If we let $\FinVect$ be the
category of finite dimensional real vector spaces and linear maps,
then:

\begin{proposition}
  \label{prop:embed-manifolds}
  There is a faithful functor $\Man \to \Fam(\FinVect)$ that sends a
  manifold $M$ to $(M, \lambda x. T_x(M))$, and each smooth function
  $f$ to $(f, f_*)$, the pair of $f$ and its forward derivative.
\end{proposition}

As similar result is given by \citet{cruttwell2022}, where Euclidean
spaces $\RR^n$ and smooth functions are embedded into a category of
lenses (the ``simply typed'' version of the $\Fam$ construction). As
in \citet{vakar22}, the idea is to formally separate functions on
points and their forward/reverse tangent maps for the purposes of
implementation of automatic differentiation. In the case of smooth
maps, this process throws away information on higher derivatives by
turning smooth maps into pairs of plain functions and linear
functions. We conjecture at the end of this section that the analogous
construction in our partially ordered setting does not.

For the categories $\CM$ and $\Lposet$, we pick the appropriate
categories of partial orders and monotone maps:

\begin{enumerate}
\item The category $\LatGal$ has bounded lattices as objects and
  Galois connections as morphisms, with the right adjunct going in
  the ``forward'' direction. The category $\Fam(\LatGal)$ is our
  preferred model for total functions with Galois sliicing. We
  explore some specific examples in this category in
  \secref{semantic-gps}.
\item The category $\MeetSLat$ has meet-semilattices with top as
  objects and monotone finite meet preserving functions as
  morphisms. The category $\Fam(\MeetSLat)$ provides a model of
  total functions with ``forward derivatives'' only.
\item The category $\JoinSLat$ has join-semilattices with bottom as
  objects and monotone finite join preserving functions as
  morphisms. The category $\Fam(\JoinSLat^\op)$ provides a model of
  total functions with ``backwards derivatives'' only.
\end{enumerate}

We can get an analogous result to \propref{embed-manifolds} for
L-posets and stable maps:

\begin{proposition}
  \label{prop:embed-stable}
  There is a faithful functor $\Lposet \to \Fam(\LatGal)$ that maps an
  L-poset $X$ to $(X, \lambda x. \downset{x})$ and stable functions
  $f$ to $(f, \lambda x. (f_x, f^*_x))$. Likewise, there is a faithful
  functor $\CM \to \Fam(\MeetSLat)$.
\end{proposition}

Despite its similarity, this proposition has a lesser status than
\propref{embed-manifolds} because it is not clear that the category
$\Lposet$ (or $\CM$) is a canonical definition of approximable sets
and functions with approximation derivatives, as we discussed at the
end of \secref{stab}.
% We could restrict $\Lposet$ to morphisms that preserve maximal
% elements and take L-posets $X$ to
% $(\mathrm{Max}(M), \lambda x. \downset{x})$.
Our current hypothesis is that $\Fam(\LatGal)$, where values and their
approximations are separated by construction, is the natural model of
semantic \GPS in a total setting. We now investigate some categorical
properties of this category, with a view to modelling a higher-order
total programming language in \secref{language}.

\subsection{Categorical Properties of $\Fam(\cat{C})$}

\subsubsection{Coproducts and Products}
\label{sec:models-of-total-gps:coproducts-and-products}

The categories $\Fam(\cat{C})$ are the free coproduct completions of
categories $\cat{C}$, so they have all coproducts:

\begin{proposition}
  For any $\cat{C}$, $\Fam(\cat{C})$ has all coproducts, which can be
  given on objects by:
  \begin{displaymath}
    \coprod_i (X_i, \partial X_i) = (\coprod_i X_i, \lambda (i, x_i).\, \partial X_i(x))
  \end{displaymath}
  Coproducts in $\Fam(\cat{C})$ are \emph{extensive}
  \cite[Proposition 2.4]{carboni-lack-walters93}.
\end{proposition}

For $\Fam(\cat{C})$ to have finite products, we need $\cat{C}$ to have
finite products:

\begin{proposition}
  If $\cat{C}$ has finite products, then so does $\Fam(\cat{C})$. On
  objects, binary products can be defined by:
  \begin{displaymath}
    (X, \partial X) \times (Y, \partial Y) = (X \times Y, \lambda (x, y). \partial X(x) \times \partial Y(y))
  \end{displaymath}
  Since $\Fam(\cat{C})$ is extensive, products and coproducts
  distribute.
\end{proposition}

Using the infinitary coproducts and finite products, we can construct
a wide range of other useful semantic models of datatypes in
$\Fam(\cat{C})$. For example, lists can be constructed as a coproduct
\begin{displaymath}
  \List(X) = \coprod_{n \in \mathbb{N}} X^n
\end{displaymath}
where $X^0 = 1$ (the terminal object) and $X^{n+1} = X \times X^n$.

Our category of interest, $\Fam(\LatGal)$ has coproducts and finite
products, because $\LatGal$ has products. Similarly for
$\Fam(\MeetSLat)$. As we shall see below, the products in $\LatGal$
(and $\MeetSLat$ and $\JoinSLat)$ are also coproducts, which is
essential to obtaining cartesian closure.

\subsubsection{Cartesian Closure}

For cartesian closure of the categories $\Fam(C)$ that we are
interested in, we rely on the following theorem of \citet{nunes2023},
specialised from their setting with the general Grothendieck
construction to $\Fam(\cat{C})$. This relies on the definition of
\emph{biproducts}, which we discuss below in \secref{biproducts}.

\begin{theorem}[\cite{nunes2023}]
  \label{thm:fam-closed}
  \AGDA.  If $\cat{C}$ has biproducts (\defref{biproducts}) and all
  products, then $\Fam(\cat{C})$ is cartesian closed\footnote{More
    precisely, if $\cat{C}$ has coproducts then we have a monoidal
    product on $\Fam(\cat{C})$ which is closed by this
    construction. When these coproducts are in fact biproducts, we get
    cartesian closure.}. On objects, the internal hom can be given by:
  \begin{displaymath}
    (X, \partial X) \to (Y, \partial Y) = (\Pi_{x : X}. \Sigma_{y : Y}. \cat{C}(\partial X(x), \partial Y(y)), \lambda f. \Pi_{x : X}. \partial Y(\pi_1(f\, x)))
  \end{displaymath}
\end{theorem}

The $\Set$-component of $(X, \partial X) \to (Y, \partial Y)$ consists
of exactly the morphisms of $\Fam(\cat{C})$, rephrased into a single
object. When $\cat{C} = \FinVect$, these are functions with an
associated linear map at every point, and when $\cat{C} = \LatGal$,
these are functions with an associated Galois connection at every
point. A tangent to a function is then defined to be a mapping from
points in the domain to tangents in the codomain along the function.

The category $\MeetSLat$ satisfies the hypoetheses of
\thmref{fam-closed}, so:
\begin{corollary}
  $\Fam(\MeetSLat)$ is cartesian closed and has all coproducts.
\end{corollary}
Unfortunately, neither of the $\LatGal$ or $\FinVect$ satisfy the
hypotheses of this theorem, because neither of them have infinite
products. We will consider ways to rectify this below in
\secref{fixing-completeness}.

\begin{remark}
  \label{rem:hermida-exponentials}
  There is another construction of internal homs on $\Fam(\cat{C})$
  arising from the use of fibrations for categorical logical
  relations, due to \citet[Corollary 4.12]{hermida99}. If we assume
  that $\cat{C}$ is itself cartesian closed and has all products, then
  we could construct an internal hom as:
  \begin{displaymath}
    (X, \partial X) \to (Y, \partial Y) = (X \to Y, \lambda f. \Pi_{x : X}.\,\partial X(x) \to \partial Y(f\,x))
  \end{displaymath}
  However, for the purposes of modelling differentiable programs, this
  is fatally flawed in that neither of $\LatGal$ nor $\FinVect$ are
  cartesian closed, and there is no way of making them so without
  losing the property of being able to conjunct or add tangents, as we
  shall see below. We will implicitly use Hermida's construction in
  our definability proof in \secref{definability}, where we use a
  logical relations argument to show that every morphism definable in
  the higher order language is also first-order definable.
\end{remark}

\subsubsection{CMon-Categories and Biproducts}
\label{sec:biproducts}

Loosely stated, biproducts are objects that are both products and
coproducts. The concept can be defined in any category, as shown by
\citet{karvonen20}, but for our purposes it will be more convenient to
use the shorter definition in categories enriched in commutative
monoids:
\begin{definition}
  A category $\cat{C}$ is enriched in $\CMon$, the category of
  commutative monoids, if every homset $\cat{C}(X,Y)$ is a commutative
  monoid with $(+,0)$ and composition is bilinear:
  \begin{displaymath}
    f \comp \zero = 0 = \zero \comp f
  \end{displaymath}
  \begin{displaymath}
    (f + g) \comp h = (f \comp h) + (g \comp h) \qquad
    h \comp (f + g) = (h \comp f) + (h \comp g)
  \end{displaymath}
\end{definition}
In any $\CMon$-category we can define what it means to be the
biproduct of two objects:
\begin{definition}
  \label{def:biproducts}
  In a $\CMon$-category a biproduct is an object $X \biprod Y$
  together with morphisms

  \begin{center}
    \begin{tikzcd}
      X \arrow[r, "\biinj_X", shift left] &
      X \biprod Y \arrow[l, "\biproj_X", shift left] \arrow[r, "\biproj_Y"', shift right] &
      Y \arrow[l, "\biinj_Y"', shift right]
    \end{tikzcd}
  \end{center}

  \noindent satisfying

  \vspace{-3mm}
  \begin{minipage}[t]{0.45\textwidth}
    \begin{center}
      \begin{salign*}
        \biproj_X \comp \biinj_X &= \id_X \\
        \biproj_Y \comp \biinj_X &= \zero_{X,Y}
      \end{salign*}
    \end{center}
  \end{minipage}%
  \begin{minipage}[t]{0.45\textwidth}
    \begin{center}
      \begin{salign*}
        \biproj_Y \comp \biinj_Y &= \id_Y \\
        \biproj_X \comp \biinj_Y &= \zero_{Y,X}
      \end{salign*}
    \end{center}
  \end{minipage}

  \begin{salign*}
    (\biinj_X \comp \biproj_X) + (\biinj_Y \comp \biproj_Y) &= \id_{X \biprod Y}
  \end{salign*}
  A zero object is an object that is both initial and terminal.
\end{definition}
As the name suggests, biproducts in a category are both products and
coproducts:
\begin{proposition}
  \item
  \begin{enumerate}
  \item A $\CMon$-category that has biproducts $X \oplus Y$ for all
    $X$ and $Y$ also has products and coproducts with
    $X \times Y = X + Y = X \oplus Y$.
  \item A $\CMon$-category with (co)products also has biproducts, and
    any initial or terminal object is a zero object.
  \end{enumerate}
\end{proposition}

\begin{example}
  The following are $\CMon$-enriched and have finite products, and
  hence biproducts:
  \begin{enumerate}
  \item In $\FinVect$, morphisms are linear maps and so can be added
    and have a zero map. Finite products are given by cartesian
    products of the underlying sets, with the vector operations
    defined pointwise.
  \item In $\LatGal$, right adjoints are summed using meets and left
    adjoints are summed using joins. The zero maps are given by the
    constantly $\top$ and constantly $\bot$ functions
    respectively. Products are given by the cartesian product of the
    underlying set and the one-element lattice for the
    terminal/initial/zero object.
  \item $\MeetSLat$ and $\JoinSLat$ are both $\CMon$-enriched and have
    finite products for similar reasons to $\LatGal$.
  \end{enumerate}
\end{example}

\begin{remark}
  Categories with zero objects cannot be cartesian closed without
  being trivial in the sense of having exactly one morphism between
  every pair of objects because
  $\cat{C}(X, Y) \cong \cat{C}(1 \times X,Y) \cong \cat{C}(0 \times
  X,Y) \cong \cat{C}(0,X \to Y) \cong 1$. Consequently, we cannot
  apply the alternative construction of exponentials described in
  \remref{hermida-exponentials}.
\end{remark}

\subsubsection{Discrete Completeness}
\label{sec:fixing-completeness}

The second hypothesis of \thmref{fam-closed} is that the category
$\cat{C}$ has all (i.e., infinite) products. This is required to
gather together tangents for all of the points in the domain of the
function. Unfortunately, neither $\FinVect$ nor $\LatGal$ is complete
in this sense.

In the case of $\FinVect$, the solution is to expand to the category
of all vector spaces $\Vect$, where infinite direct products
exist. Note that these infinite products are not biproducts because
the vector space operations themselves are finitary. This is the
solution that \cite{vakar22} uses for the semantics of forward
($\Fam(\Vect)$) and reverse ($\Fam(\Vect^\op)$) automatic
differentiation for higher order programs. Since the forward and
reverse derivatives of a smooth map are intrinsically defined,
\citet{vakar22}'s correctness theorem shows that, for programs with
first-order type, the interpretation in $\Fam(\Vect)$ correctly yields
the forward derivative of the defined function on the reals (and
reverse derivative for $\Fam(\Vect^\op)$).

For $\LatGal$ we could expand to the category of complete lattices and
Galois connections between them. From a classical mathematical point
of view, this would give a model of \GPS that would be suitable for
reasoning about programs' behaviour and their forward and backward
approximations. However, in terms of building an executable model
inside the Agda proof assistant, and with an eye toward implementation
strategies, we seek a finitary solution. (Note that the solution of
moving to complete lattices is very different to moving to arbitrary
dimension vector spaces: in the former we have infinitary operations,
while the latter still has only finitary operations.)

We will avoid the need for infinitary operations by separating the
forward and backward parts of the Galois connections to act
independently by moving to the product category
$\MeetSLat \times \JoinSLat^\op$. Objects in this category consist of
\emph{separate} meet- and join-semilattices and potentially unrelated
forward meet-preserving and backward join-preserving maps. We first
check that this category satisfies the hypotheses of
\thmref{fam-closed}:

\begin{proposition}
  $\MeetSLat \times \JoinSLat^\op$ has biproducts and all products.
\end{proposition}

\begin{proof}
  $\MeetSLat$ and $\JoinSLat$ are both $\CMon$-enriched and have
  finite products, as noted above. The opposite of a category with
  biproducts also has biproducts (by swapping the injections $i$ and
  projections $p$), and products of categories with biproducts also
  have biproducts pointwise. Hence $\MeetSLat \times \JoinSLat^\op$
  has biproducts.

  $\MeetSLat$ has all products, indeed all limits, because it is the
  category of algebras for a Lawvere theory. Similarly, $\JoinSLat$
  has all coproducts, indeed all colimits, for the same reason. Note
  that these are very different constructions: elements of a product
  of meet-semilattices consist of (possibly infinite) tuples of
  elements, while elements of a coproduct of join-semilattices consist
  of \emph{finite} formal joins of elements quotiented by the
  join-semilattice equations. Since $\JoinSLat$ has all coproducts,
  $\JoinSLat^\op$ has all products, and so
  $\MeetSLat \times \JoinSLat^\op$ has all products, as required.
\end{proof}

\begin{corollary}
  \label{cor:mslat-jslat-bcc}
  $\Fam(\MeetSLat \times \JoinSLat^\op)$ is cartesian closed and has
  all coproducts.
\end{corollary}

This corollary means that, assuming a sensible intepretation of
primitive types and operations, we can use
$\Fam(\MeetSLat \times \JoinSLat^\op)$ to interpret the higher-order
language we describe in the next section. We still regard the category
$\Fam(\LatGal)$ as the reference model of approximable sets with
forward and backward approximation maps; the category
$\Fam(\MeetSLat \times \JoinSLat^\op)$ is a technical device to carry
out the interpretation of higher-order programs. To get
interpretations of first-order types and primitive operations, we can
embed $\Fam(\LatGal)$ into $\Fam(\MeetSLat \times \JoinSLat^\op)$:

\begin{proposition}
  \label{prop:ho-embedding}
  The functor
  $\HoEmbed : \Fam(\LatGal) \to \Fam(\MeetSLat \times \JoinSLat^\op)$ is defined
  on objects as
  $\HoEmbed(X, \partial X) = (X, \lambda x. (\partial X(x), \partial
  X(x)))$. This functor is faithful and preserves coproducts and
  finite products.
\end{proposition}

With this embedding functor, we see in
\propref{first-order-type-interp} that the interpretation of
first-order types will be the same up to isomorphism in
$\Fam(\LatGal)$ and $\Fam(\MeetSLat \times \JoinSLat^\op)$, as long as
we interpret the base types as objects in $\Fam(\LatGal)$. At
higher-order, however, the meet-semilattice and join-semilattice sides
of the interpretation will diverge, and it is no longer clear that the
interpretation of programs using higher-order functions internally
will result in Galois connections. In \secref{definability} we will
see that every program with first-order type (even if it uses
higher-order functions internally) does in fact have an interpretation
definable in $\Fam(\LatGal)$.

% \bob{In the conclusion, we have to admit to the possibility that
%   $\Fam(CLatGal)$ might be sensible. We could have also used
%   PShCMon(LatGal) and there is PShCMon(GraphLang) for
%   normalisation. We would have different problems with proving
%   correctness however.}

\subsection{Semantic Galois Slicing in $\Fam(\LatGal)$}
\label{sec:semantic-gps}

Our thesis is that $\Fam(\LatGal)$ is a suitable setting for
interpreting first-order programs for \GPS. The above discussion has
been somewhat abstract, so we now consider some examples in the
category $\Fam(\LatGal)$ and how they relate to \GPS.

Spelt out in full, $\Fam(\LatGal)$ has as objects $(X, \partial X)$,
all pairs of a set $X$ and and for every $x \in X$, a bounded lattice
$\partial X(x)$. Morphisms $(X, \partial X) \to (Y, \partial Y)$, are
triples $(f, \partial f_f, \partial f_r)$ of functions $f : X \to Y$
and families of monotone maps
$\partial f_f : \Pi_{x : X}.\partial X(x) \multimap \partial Y(f\,x)$
(``forward derivative'') and
$\partial f_r : \Pi_{x : X}. \partial Y(f\,x) \multimap \partial X(x)$
(``reverse derivative''), such that for all $x$,
$\partial f_r(x) \dashv \partial f_f(x)$.

\subsubsection{Unapproximated Functions}

$\LatGal$ has a terminal (also zero) object $\mathbb{1}$, so there is
a functor $\Disc : \Set \to \Fam(\LatGal)$ that maps a set $X$ to
$(X, \lambda x. \mathbb{1})$ and functions $f$ to morphisms
$(f, \lambda\_.\,\mathrm{id}_{\mathbb{1}})$. This functor preserves
products and coproducts. Therefore, we can take any sets and functions
of interest for modelling primitive types and operations of a
programming language and embed, albeit without any interesting
approximation information.

\subsubsection{Lifting Monad}

The operation of adding a new bottom element to a bounded lattice
forms part of a monad $L$ on $\LatGal$. This monad extends to a
(strong) monad $\Lift$ on $\Fam(\LatGal)$ with
$\Lift(X, \partial X) = (X, L \circ \partial X)$. The monad $\Lift$
does not affect the points of the original object, but adds a new
minimum approximation. This monad allows us to construct morphisms in
$\Fam(\LatGal)$ that track information about how values are used
similar to the examples of cm and stable functions we saw in
\secref{approx-as-tangents}.

\newcommand{\Bool}{\mathrm{Bool}}

Let $\Bool = \Disc(\{\mathsf{tt},\mathsf{ff}\})$ be the
(unapproximated) embedding of the booleans and
$\mathrm{or} : \Bool \times \Bool \to \Bool$ be the (unapproximated)
boolean or function. The, using a Moggi-style let notation
\cite{notions-of-computation} for morphisms constructed using the
Monad structure of $\Lift$, we can reproduce the functions
$\mathrm{strictOr}$ and $\mathrm{shortCircuitOr}$ functions from
\exref{strict-short-circuit} (we also assume an if-then-else operation
on booleans, definable from the fact that $\Fam(\LatGal)$ has
coproducts and $\Disc$ preserves them). Both of these expressions
define morphisms $\Lift(\Bool) \times \Lift(\Bool) \to \Lift(\Bool)$
in $\Fam(\LatGal)$:
\begin{displaymath}
  \begin{array}{lcl}
    \mathrm{strictOr}(x,y)&=&\mathrm{let}\,b_1 \Leftarrow x\,\mathrm{in}\,\mathrm{let}\,b_2 \Leftarrow y\,\mathrm{in}\,\eta(\mathrm{or}(b_1,b_2)) \\
    \mathrm{shortCircuitOr}(x,y)&=&\mathrm{let}\,b_1 \Leftarrow x\,\mathrm{in}\,\mathrm{if}\,b_1\,\mathrm{then}\,\eta(\mathsf{tt})\,\mathrm{else}\,y
  \end{array}
\end{displaymath}
Examining the morphisms so defined in $\Fam(\LatGal)$, we can see
that, in the $\Set$ component, they are both exactly the normal
boolean-or operation. However, they have different approximation
behaviour, reflecting the different ways that they examine their
inputs. Let us write $\top,\bot$ for the elements of the approximation
lattice at each point of $\Lift(\Bool)$, then applying the reverse
derivative at the point $(\mathsf{tt},\mathsf{tt})$ to the tangent
$\top$ reveals which of the inputs contributed to the output for each
function:
\begin{displaymath}
  \begin{array}{lcl}
    (\partial \mathrm{strictOr})_r(\mathsf{tt},\mathsf{tt})(\top) &=& (\top, \top)  \\
    (\partial \mathrm{shortCircuitOr})_r(\mathsf{tt},\mathsf{tt})(\top) &=& (\top, \bot)
  \end{array}
\end{displaymath}
In comparison to the categories $\CM$ and $\Lposet$ from
\secref{approx-as-tangents}, we have retained the usage information in
the forward and reverse tangents, but we also accurately model
totality of the functions. That is, the constantly $\bot$ function is
also present in both $\CM$ and $\Lposet$, but is not expressible in
$\Fam(\LatGal)$.

An analogue of the $\mathrm{parallelOr}$ function from
\exref{parallel-or} is not definable in $\Fam(\LatGal)$. We would have
to have
$(\partial \mathrm{parallelOr})_f(\mathsf{tt},\mathsf{tt})(\top,\bot)
= (\partial \mathrm{parallelOr})_f(\mathsf{tt},\mathsf{tt})(\bot,\top)
= \top$ to reflect the desired property that either of the inputs
being $\mathsf{tt}$ is enough to determine the output. We also must
have
$(\partial \mathrm{parallelOr})_f(\mathsf{tt},\mathsf{tt})(\bot,\bot)
= \bot$, to reflect the fact that we will get no information in the
output if we required that neither of the inputs is examined. However,
this means that
$(\partial \mathrm{parallelOr})_f(\mathsf{tt},\mathsf{tt})$ will not
presere meets because $(\top,\bot) \sqcap (\bot,\top) = (\bot,\bot)$
but $\top \neq \bot$.

An analogue of the $\mathrm{gustave}$ function from
\exref{parallel-or} is definable in $\Fam(\LatGal)$, but not using the
lifting monad structure as we could for $\mathrm{strictOr}$ and
$\mathrm{shortCircuitOr}$.

\begin{remark}
  \label{rem:further-structure}
  These examples highlight a potential criticism of $\Fam(\LatGal)$ as
  a category for modelling \GPS. For $\mathrm{shortCircuitOr}$, we had
  $(\partial \mathrm{shortCircuitOr})_r(\mathsf{tt},\mathsf{tt})(\top)
  = (\top, \bot)$, indicating that the second argument was not needed
  for computing the output. However, there is no way, in
  $\Fam(\LatGal)$, of turning this into a rigorous statement that the
  $\Set$-component of this morphism does not actually depend on its
  second argument. We conjecture that this can be rectified by
  requiring some kind of additional structure on each object
  $(X, \partial X)$ consisting of a map
  $\Pi_{x : X}.\partial X(x) \to \mathcal{P}(X)$, where $\mathcal{P}$
  is the powerset, which identifies for each $x$ which elements are
  indistinguishable from $x$ at this level of approximation. One would
  also presumably have to require additional conditions for this to
  respect the lattice structure and be preserved by
  morphisms\footnote{This additional structure is reminiscent of the
    additional structure on \emph{directed containers} defined by
    \citet{ahman-uustalu}. They require a map
    $\Pi_{x : X}.\partial X(x) \to X$ picking out a specific $X$
    ``jumped to'' by some change $\delta x : \partial X(x)$ at $x$. In
    our proposed setup, we follow the approximation theme of \GPS by
    having a \emph{set} of things that could be taken instead of the
    original $x$. Objects of $\Fam(\LatGal)$ are ``directed'' in this
    sense because the approximation map can pick out the element of
    the original poset that was approximating $x$.}.
\end{remark}

The $\Lift$ monad provides a controllable way of adding
presence/absence approximation points to composite data, and its monad
structure makes explicit in the program structure exactly how such
approximations are propagated through computations. The fact that
there are different choices of this kind of approximation tracking
provides freedom to the language implementor to decide what
information is worth tracking. The Galois slicing implementations
discussed in \citet{perera12a} and \citet{ricciotti17}, for example,
bake-in an approximation point at every composite type constructor. We
will see in \secref{cbn-translation} that this choice can be
systematised in our setting by considering a monadic CBN translation
to uniformly add $\Lift$ approximation points to composite data types.

\subsubsection{An Approximation Object and the Tagging Monad}

The $\Lift$ monad provides a way of tagging first-order data with
presence and absence information. The object $\Lift(1)$, the lifting
of the terminal object in $\Fam(\LatGal)$, yields an object that
consists purely of presence/absence information:
$\mathbb{A} = (1, \lambda\_.\,\{\top,\bot\})$. This object is the
carrier of a commutative monoid in $\Fam(\LatGal)$, where the forward
maps take the meet (both the inputs are required for the output to be
present) and the backwards maps duplicate.% \bob{Is it also a comonoid?}

Since $\mathbb{A}$ is a monoid, we can define the writer monad
$T(X) = \mathbb{A} \times X$ in $\Fam(\LatGal)$ which ``tags'' $X$
values with approximation information. This is similar to the $\Lift$
monad in that it adds approximation information to an object. On
discrete objects it agrees with the lifting:
$T(\Disc(X)) \cong L(\Disc(X))$. However, on composite data, the two
monads give different approximation lattices. Let $A$ and $B$ be sets.
Then $T(T(\Disc(A)) \times T(T(\Disc(B))))$ has approximation lattices
at $(a,b)$ that are always isomorphic to $\{\top,\bot\}^3$. The
corresponding $\Lift(\Lift(\Disc(A)) \times \Lift(\Disc(B)))$ object's
approximation lattice at $(a,b)$ is always isomorphic to
$(\{\top,\bot\}^2)_\bot$.
In terms of usage tracking, the object using the $\Lift$ monad is more
appealing. The approximation lattice resulting from the use of the $T$
monad contains apparently nonsensical elements corresponding to
``using'' one or other components of the product \emph{without using
  the product itself}. (This arises from the fact that we can project
the $X$ out of $\mathbb{A} \times X$ without touching the
$\mathbb{A}$.)


This example shows that we have to be careful about how we choose the
interpretation of approximable sets in $\Fam(\LatGal)$, and again
highlights the point we made in \remref{further-structure} that
perhaps $\Fam(\LatGal)$ does not have quite enough structure to
determine ``sensible'' approximation information. On the other hand,
the use of the $T$ monad does have the advantage that the
approximation lattices built from discrete sets, products, and
coproducts, are always Boolean lattices, meaning that we can take
complements of usage information. The ability to take boolean
complements of approximations has been used by \citet{perera22} to
compute \emph{related} outputs, as we discuss in
\secref{related-work:galois-slicing}.

\subsubsection{Approximating Numbers by Intervals}

So far, the approximation lattices we have looked at in
$\Fam(\LatGal)$ have only consisted of those constructed from finite
products and lifting, and only track binary usage/non-usage
information. \exref{intervals-and-maxima-elements} shows how we can go
beyond this to get more ``quantitative'' approximation
information. Let the object of reals with interval approximations in
$\Fam(\LatGal)$ be
$(\mathbb{R}, \lambda x.\,\{[l,u] \mid l \leq x \leq u\} \cup
\{\bot\})$ where the lattices of intervals are reverse ordered by
inclusion with $\bot$ at the bottom. Then, following the examples in
\exref{intervals-and-maxima-elements} and \exref{stable-functions}, we
can define addition, negation, and scaling by $r \geq 0$:
\begin{displaymath}
  \begin{array}{lcl}
    \mathrm{add}&=&(\begin{array}[t]{@{}l}
      \lambda (x_1,x_2).\,x_1+x_2, \\
      \lambda (x_1,x_2)\,([l_1,u_1],[l_2,u_2]).\,[(l_1+x_2) \sqcap (l_2+x_1),(u_1+x_2) \sqcup (u_2+x_1)] \\
      \lambda (x_1,x_2)\,[l,u].\,([l-x_2,u-x_2],[l-x_1,u-x_1]))
    \end{array}\\
    \mathrm{neg}&=&(\lambda x.\,-x, \lambda x\,[l,u].\,[-u,-l], \lambda x\,[l,u].\,[-u,-l]) \\
    \mathrm{scale}(r)&=&(\lambda x.\,rx, \lambda x\,[l,u].\,[rl,ru], \lambda x\,[l,u].\,\mathrm{if}\,r=0\,\mathrm{then}\,\bot\,\mathrm{else}\,[\frac{l}{r},\frac{u}{r}])
  \end{array}
\end{displaymath}
Scaling by negative numbers is also possible with swapping of lower
and upper bounds, as is multiplication.

\subsection{Summary}

We have constructed the category
$\Fam(\MeetSLat \times \JoinSLat^\op)$, which is a cartesian closed
category with all coproducts, enough to interpret the total
higher-order language we define in the next section with primitive
types and operations defined in $\Fam(\LatGal)$. However, we are not
guaranteed by construction that at first-order type, the
interpretations are in fact Galois connections. We will rectify this
in \secref{definability} using a logical relations construction.

As we did in \secref{approx-as-tangents}, we end the section with a
conjecture relating our categories to Tangent categories.

\begin{conjecture}
  (1) The category $\Fam(\MeetSLat)$ is a Tangent category, with the
  tangent bundle
  $T(X, \partial X) = (\Sigma_{x : X}. \partial X(x), \lambda (x,
  \delta x).\, \downset{\delta x})$. (2) The category $\Fam(\LatGal)$
  is a reverse Tangent category with the analogous definition of
  tangent bundle.
\end{conjecture}

Comparing this conjecture to the previous one, we can see that the
difference between $\CM$, $\Lposet$ and $\Fam(\MeetSLat)$,
$\Fam(\LatGal)$ is that the latter have a separation of points from
tangents, somewhat analogous to the situation with manifolds. Also
note that, if this conjecture holds, then contrary to the $\Fam(\FinVect)$ representation of
manifolds and differentiable maps, we do
not throw away information about higher derivatives. It is retained in
the order structure of the tangent fibres.
