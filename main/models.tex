\section{Models of \GPS for a Total Language}
\label{sec:models-of-total-gps}

The previous section concluded that L-posets and stable functions give
a model of \GPS analogous to manifolds and smooth functions. However,
we noted a conceptual shortcoming of this model, for the purposes of
modelling total computations, that proper values and their
approximations live in the same category. In this section, we propose
a model for total \GPS based on the \emph{Category of Families}
construction. This construction, and the more general Grothendieck
construction, has been previously used by V{\'a}k{\'a}r and
collaborators \bob{CITE} to model automatic differentiation for
higher-order programs on the reals. We reuse some of their results,
and discuss the commonalities as we go.

\subsection{The Category of Families Construction}
\label{sec:models-of-total-gps:fam}

L-posets are partially ordered sets where every principal downset
$\downset{x}$ is a bounded lattice of approximations/tangents. As we
explained in \secref{stab}, the shortcoming of this setup is that
proper values and their approximations live in the same set. We fix
this by changing our model to one where we have sets $X$ of values,
and for each $x \in X$, a bounded lattice $\partial X(x)$ of
approximations of $x$. This construction is an instance of the general
\emph{Category of Families} construction:

\begin{definition}
  Let $\cat{C}$ be a category. The \emph{Category of Families} over
  $\cat{C}$, $\Fam(\cat{C})$, has as objects pairs $(X, \partial X)$,
  where $X$ is a set and $\partial X : X \to \cat{C}$ is an
  $X$-indexed family of objects in $\cat{C}$. A morphism
  $f : (X, \partial X) \to (Y, \partial Y)$ consists of a pair of a
  function $f : X \to Y$ and a family of morphisms of $\cat{C}$,
  $\partial f : \Pi_{x : X}.\,\cat{C}(\partial X(x), \partial Y(f\,x))$.
\end{definition}

The reason for choosing the $\Fam$ construction is that composition in
this category is an abstract version of the chain rule that we have
seen in \remref{chain-rule}, \remref{chain-rule-cm}, and
\remref{chain-rule-stable}. Composition $f \circ g $ of morphisms
$f : (Y, \partial Y) \to (Z, \partial Z)$ and
$g : (X, \partial X) \to (Y, \partial Y)$ in this category is given by
normal function composition on the set components, and
$\partial (f \circ g)(x) = \partial f(f, x) \circ \partial g(x)$,
where the latter composition is in $\cat{C}$.

The fact that morphisms in $\Fam(C)$ compose according to a chain rule
means that the categories we have seen so far embed into $\Fam(C)$ for
appropriate $C$. If we let $\FinVect$ be the category of finite
dimensional real vector spaces, then:

\begin{proposition}
  \label{prop:embed-manifolds}
  There is a faithful functor $\Man \to \Fam(\FinVect)$ that sends a
  manifold $M$ to $(M, \lambda x. T_x(M))$, and each smooth function
  $f$ to $(f, f_*)$, the pair of $f$ and its forward derivative.
\end{proposition}

As similar result is given by \citet{cruttwell2022}, where Euclidean
spaces $\RR^n$ and smooth functions are embedded into a category of
lenses (the ``simply typed'' version of the $\Fam$ construction). As
in \citet{vákár22}, the idea is to formally separate functions on
points and their forward/reverse tangent maps for the purposes of
implementation of automatic differentiation. In the case of smooth
maps, this process throws away information on higher derivatives by
turning smooth maps into pairs of plain functions and linear
functions. We conjecture at the end of this section that the analogous
construction in our partially ordered setting does not.

For the categories $\CM$ and $\Lposet$, we pick the appropriate
categories of partial orders and monotone maps:

\begin{definition}
  \item
  \begin{enumerate}
  \item The category $\LatGal$ has bounded lattices as objects and
    Galois connections as morphisms, with the right adjunct going in
    the ``forward'' direction.
  \item The category $\MeetSLat$ has meet-semi-lattices with top as
    objects and monotone finite meet preserving functions as
    morphisms.
  \item The category $\JoinSLat$ has join-semi-lattices with bottom as
    objects and monotone finite join preserving functions as
    morphisms.
  \end{enumerate}
\end{definition}

We can get an analogous result to \propref{embed-manifolds} for L-posets:

\begin{proposition}
  \label{prop:embed-stable}
  There is a faithful functor $\Lposet \to \Fam(\LatGal)$ that maps an
  L-poset $X$ to $(X, \lambda x. \downset{x})$ and stable functions
  $f$ to $(f, \lambda x. (f_x, f^*_x))$. Likewise, there is a faithful
  functor $\CM \to \Fam(\MeetSLat)$.
\end{proposition}

Despite its similarity, this proposition has a lesser status than
\propref{embed-manifolds} because it is not clear that the category
$\Lposet$ (or $\CM$) is a canonical definition of approximable sets
and functions with approximation derivatives, as we discussed at the
end of \secref{stab}.
% We could restrict $\Lposet$ to morphisms that preserve maximal
% elements and take L-posets $X$ to
% $(\mathrm{Max}(M), \lambda x. \downset{x})$.
Our current hypothesis is that $\Fam(\LatGal)$, where values and their
approximations are separated by construction, is the natural model of
semantic \GPS in a total setting. We now investigate some categorical
properties of this category, with a view to modelling a higher-order
total programming language in \secref{language}.

\subsection{Categorical Properties of $\Fam(\cat{C})$}

\subsubsection{Coproducts and Products}

The categories $\Fam(\cat{C})$ are the free coproduct completions of
categories $\cat{C}$, so they are guaranteed to have all coproducts:

\begin{proposition}
  For any $\cat{C}$, $\Fam(\cat{C})$ has all coproducts, which can be
  given on objects by:
  \begin{displaymath}
    \coprod_i (X_i, \partial X_i) = (\coprod_i X_i, \lambda (i, x_i).\, \partial X_i(x))
  \end{displaymath}
  Coproducts in $\Fam(\cat{C})$ are \emph{extensive}
  \cite[Proposition 2.4]{carboni-lack-walters93}.
\end{proposition}

For $\Fam(\cat{C})$ to have finite products, we need $\cat{C}$ to have
finite products:

\begin{proposition}
  If $\cat{C}$ has finite products, then so does $\Fam(\cat{C})$. On
  objects, binary products can be defined by:
  \begin{displaymath}
    (X, \partial X) \times (Y, \partial Y) = (X \times Y, \lambda (x, y). \partial X(x) \times \partial Y(y))
  \end{displaymath}
  Since $\Fam(\cat{C})$ is extensive, products and coproducts
  distribute.
\end{proposition}

Using the infinitary coproducts and finite products, we can construct
a wide range of other useful semantic models of datatypes in
$\Fam(\cat{C})$. For example lists can be constructed as the coproduct
\begin{displaymath}
  \mathrm{List}\,X = \coprod_{n \in \mathbb{N}} X^n
\end{displaymath}
where $X^0 = 1$ (the terminal object) and $X^{n+1} = X \times X^n$.

Our category of interest, $\Fam(\LatGal)$ has coproducts and finite
products, because $\LatGal$ has products. As we shall see below, these
are also coproducts, which is essential to obtaining cartesian
closure.

\subsubsection{Cartesian Closure}

For cartesian closure of categories of families, we rely on the
following theorem of \citet{nunes2023}, specialised from their setting
with the general Grothendieck construction to $\Fam(\cat{C})$. This
relies on the definition of \emph{biproducts}, which we discuss below
in \secref{biproducts}.

\begin{theorem}[\cite{nunes2023}]
  \label{thm:fam-closed}
  \AGDA.  If $\cat{C}$ has biproducts and all products, then
  $\Fam(\cat{C})$ is cartesian closed\footnote{More precisely, if
    $\cat{C}$ has coproducts then we have a monoidal product on
    $\Fam(\cat{C})$ which is closed by this construction. When these
    coproducts are in fact biproducts, we get cartesian closure.}. On
  objects, the internal hom can be given by:
  \begin{displaymath}
    (X, \partial X) \to (Y, \partial Y) = (\Pi_{x : X}. \Sigma_{y : Y}. \cat{C}(\partial X(x), \partial Y(y)), \lambda f. \Pi_{x : X}. \partial Y(\pi_1(f\, x)))
  \end{displaymath}
\end{theorem}

Before we explore the hypotheses of this theorem, we give some
intuition on how it relates to differentiation. The Set-component of
$(X, \partial X) \to (Y, \partial Y)$ consists of exactly the
morphisms of $\Fam(\cat{C})$, rephrased into a single object. When
$\cat{C} = \FinVect$, these are functions with an associated linear
map at every point, and when $\cat{C} = \LatGal$, these are functions
with an associated Galois connection at every point. A tangent to a
function is then defined to be a mapping from points in the domain to
tangents in the codomain along the function.

Unfortunately, neither of the categories $\LatGal$ or $\FinVect$
satisfy the hypotheses of this theorem, because neither of them have
infinite products. We will consider ways to rectify this below in
\secref{fixing-completeness}.

\begin{remark}
  \label{rem:hermida-exponentials}
  There is another construction of internal homs on $\Fam(\cat{C})$
  arising from the use of fibrations for categorical logical relations
  proofs, due to \citet{hermida99} (Corollary 4.12). If we assume that
  $\cat{C}$ is itself cartesian closed and has all products, then we
  could construct an internal hom as:
  \begin{displaymath}
    (X, \partial X) \to (Y, \partial Y) = (X \to Y, \lambda f. \Pi_{x : X}.\,\partial X(x) \to \partial Y(f\,x))
  \end{displaymath}
  However, for the purposes of modelling differentiable programs, this
  is fatally flawed in that neither of $\LatGal$ nor $\FinVect$ are
  cartesian closed, and there is no way of making them so without
  losing the property of having biproducts. We will use Hermida's
  construction in our definability proof in \secref{definability},
  where we use a logical relations argument to show that every
  morphism definable in the higher order language is also first-order
  definable.
\end{remark}

\subsubsection{CMon-Categories and Biproducts}
\label{sec:biproducts}

Loosely stated, biproducts are objects that are both products and
coproducts. They can be defined in any category, as shown by
\citet{karvonen18}, but for our purposes it will be more convenient to
use the shorter definition in categories enriched in commutative
monoids:
\begin{definition}
  A category $\cat{C}$ is enriched in $\CMon$, the category of
  commutative monoids, if every homset $\cat{C}(X,Y)$ is a commutative
  monoid with $(+,0)$ and composition is bilinear:
  \begin{salign*}
    f \comp \zero = f = \zero \comp f
  \end{salign*}
  \begin{salign*}
    (f + g) \comp h &= (f \comp h) + (g \comp h) \\
    h \comp (f + g) &= (h \comp f) + (h \comp g)
  \end{salign*}
\end{definition}
In any $\CMon$-category we can define what it means to be the
biproduct of two objects:
\begin{definition}
  In a $\CMon$-category a biproduct is an object $X \biprod Y$
  together with morphisms

  \begin{center}
    \begin{tikzcd}
      X \arrow[r, "\biinj_X", shift left] &
      X \biprod Y \arrow[l, "\biproj_X", shift left] \arrow[r, "\biproj_Y"', shift right] &
      Y \arrow[l, "\biinj_Y"', shift right]
    \end{tikzcd}
  \end{center}

  \noindent satisfying

  \vspace{-3mm}
  \begin{minipage}[t]{0.45\textwidth}
    \begin{center}
      \begin{salign*}
        \biproj_X \comp \biinj_X &= \id_X \\
        \biproj_Y \comp \biinj_X &= \zero_{X,Y}
      \end{salign*}
    \end{center}
  \end{minipage}%
  \begin{minipage}[t]{0.45\textwidth}
    \begin{center}
      \begin{salign*}
        \biproj_Y \comp \biinj_Y &= \id_Y \\
        \biproj_X \comp \biinj_Y &= \zero_{Y,X}
      \end{salign*}
    \end{center}
  \end{minipage}

  \begin{salign*}
    (\biinj_X \comp \biproj_X) + (\biinj_Y \comp \biproj_Y) &= \id_{X \biprod Y}
  \end{salign*}
  A zero object is an object that is both initial and terminal.
\end{definition}
As the name suggests, biproducts in a category are both products and
coproducts:
\begin{proposition}
  \item
  \begin{enumerate}
  \item A $\CMon$-category that has biproducts $X \oplus Y$ for all
    $X$ and $Y$ also has products and coproducts with
    $X \times Y = X + Y = X \oplus Y$.
  \item A $\CMon$-category with (co)products also has biproducts, and
    any initial or terminal object is a zero object.
  \end{enumerate}
\end{proposition}

\begin{example}
  The following categories are $\CMon$-enriched and have finite
  products, and hence biproducts:
  \begin{enumerate}
  \item In $\FinVect$, morphisms are linear maps and so can be added
    and have a zero map. Finite products are given by cartesian
    products of the underlying sets, with the vector operations
    defined pointwise.
  \item In $\LatGal$, right adjoints are summed using meets and left
    adjoints are summed using joins. The zero maps are given by the
    constantly $\top$ and constrantly $\bot$ functions
    respectively. Products are given by the cartesian product of the
    underlying set.
  \item The $\LatGal$ example splits into two separate categories
    $\MeetSLat$ and $\JoinSLat$ of meet semi-lattices and join
    semi-lattices respectively, with their respective homomorphisms as
    the morphisms. As for $\LatGal$, these categories are
    $\CMon$-enriched and have finite products, and so finite
    biproducts.
  \end{enumerate}
\end{example}

The following proposition shows why the alternative construction of
exponentials in $\Fam(C)$ described in \remref{hermida-exponentials}
is not suitable:
\begin{proposition}
  A $\CMon$-category that is cartesian closed has trivial homsets:
  $\cat{C}(X,Y) \cong 1$.
\end{proposition}

\begin{proof}
  $\cat{C}(X, Y) \cong \cat{C}(0 + X,Y) \cong \cat{C}(0 \times X,Y)
  \cong \cat{C}(0,X \to Y) \cong 1$.
\end{proof}

\subsubsection{Discrete Completeness}
\label{sec:fixing-completeness}

The second hypothesis of \thmref{fam-closed} is that the category
$\cat{C}$ has all (i.e., infinite) products. This is required to
gather together tangents for all of the points in the domain of the
function. Unfortunately, neither $\FinVect$ nor $\LatGal$ is complete
in this sense.

In the case of $\FinVect$, the solution is to expand to the category
of all vector spaces $\Vect$, where infinite direct products
exist. Note that these infinite products are not biproducts because
the vector space operations themselves are finitary. This is the
solution that \cite{vakar2021} uses for the semantics of forward
($\Fam(\Vect)$) and reverse ($\Fam(\Vect^\op)$) automatic
differentiation for higher order programs. Since the forward and
reverse derivatives of a smooth map intrinsically defined,
\citet{vakar2021}'s correctness theorem shows that, for programs with
first-order type, the interpretation in $\Fam(\Vect)$ correctly yields
the forward derivative of the defined function on the reals (and
reverse derivative for $\Fam(\Vect^\op)$.

For $\LatGal$ we could expand to the category of complete lattices and
Galois connections between them. From a classical mathematical point
of view, this would give a model of \GPS that would be suitable for
reasoning about programs' behaviour and their forward and backward
approximations. However, in terms of building an executable model
inside the Agda proof assistant, and with an eye toward implementation
strategies, we seek a finitary solution. (Note that the solution of
moving to complete lattices is very different to moving to arbitrary
dimension vector spaces: in the former we have infinitary operations,
while the latter still has only finitary operations.)

We will avoid the need for infinitary operations by separating the
forward and backward parts of the Galois connections to act
independently by moving to the product category
$\MeetSLat \times \JoinSLat^\op$. Objects in this category consist of
\emph{separate} meet- and join-semi-lattices and potentially unrelated
forward meet-preserving and backward join-preserving maps. We first
check that this category satisfies the hypotheses of
\thmref{fam-closed}:

\begin{proposition}
  $\MeetSLat \times \JoinSLat^\op$ has biproducts and all products.
\end{proposition}

\begin{proof}
  $\MeetSLat$ and $\JoinSLat$ are both $\CMon$-enriched and have
  finite products, as noted above. The opposite of a category with
  biproducts also has biproducts (by swapping the injections $i$ and
  projections $p$), and products of categories with biproducts also
  have biproducts pointwise. Hence $\MeetSLat \times \JoinSLat^\op$
  has biproducts.

  $\MeetSLat$ has all products, indeed all limits, because it is the
  category of algebras for a Lawvere theory. Similarly, $\JoinSLat$
  has all coproducts, indeed all colimits, for the same reason. Note
  that these are very different constructions: elements of a product
  of meet-semi-lattices consist of (possibly infinite) tuples of
  elements, while elements of a coproduct of join-semilattices consist
  of \emph{finite} formal joins of elements quotiented by the
  join-semi-lattice equations. Since $\JoinSLat$ has all coproducts,
  $\JoinSLat^\op$ has all products, and so
  $\MeetSLat \times \JoinSLat^\op$ has all products, as required.
\end{proof}

\begin{corollary}
  $\Fam(\MeetSLat \times \JoinSLat^\op)$ is cartesian closed and has
  all coproducts.
\end{corollary}

This corollary means that, assuming a sensible intepretation of
primitive types and operations, we can use
$\Fam(\MeetSLat \times \JoinSLat^\op)$ to interpret the higher-order
language we describe in the next section. We still regard the category
$\Fam(\LatGal)$ as the reference model of approximable sets with
forward and backward approximation maps; the category
$\Fam(\MeetSLat \times \JoinSLat^\op)$ is a technical device to carry
out the interpretation of higher-order programs. To get
interpretations of first-order types and primitive operations, we can
embed $\Fam(\LatGal)$ into $\Fam(\MeetSLat \times \JoinSLat^\op)$:

\begin{proposition}
  The functor
  $F : \Fam(\LatGal) \to \Fam(\MeetSLat \times \JoinSLat^\op)$ defined
  on objects as
  $F(X, \partial X) = (X, \lambda x. (\partial X(x), \partial
  X(x)))$. This functor is faithful and preserves coproducts and
  finite products.
\end{proposition}

With this embedding functor, we can see that the interpretation of
first-order types will be the same up to isormorphism in
$\Fam(\LatGal)$ and $\Fam(\MeetSLat \times \JoinSLat^\op)$, as long as
we interpret the base types as objects in $\Fam(\LatGal)$. At
higher-order, however, the meet-semi-lattice and join-semi-lattice
sides of the interpretation will diverge, and it is no longer clear
that the interpretation of programs using higher-order functions
internally will result in Galois connections. In \secref{definability}
we will see that every program with first-order type (even if it uses
higher-order functions internally) does in fact have an interpretation
definable in $\Fam(\LatGal)$.

% $\GLR(\Fam(\MeetSLat \times \JoinSLat^\op), F)$ fibred over
% $\Fam(\MeetSLat \times \JoinSLat^\op)$ with a full and faithful functor $\Fam(\LatGal) \to \GLR(\Fam(\MeetSLat \times \JoinSLat^\op), F)$.

% \bob{In the conclusion, we have to admit to the possibility that
%   $\Fam(CLatGal)$ might be sensible. We could have also used
%   PShCMon(LatGal) and there is PShCMon(GraphLang) for
%   normalisation. We would have different problems with proving
%   correctness however.}

\subsection{Semantic Galois Slicing in $\Fam(\LatGal)$}
\label{sec:semantic-gps}

The above discussion has been somewhat abstract, so we now consider
some examples in the category $\Fam(\LatGal)$ and how they relate to
\GPS. We highlight how

\subsubsection{Lifting Monad}

\bob{Move the discussion in \texttt{first-order.tex} to here, but
  construct the lifting monad in a more elementary way}

\subsubsection{An Approximation Object and the Tagged Data Monad}

\bob{flesh this out}

The object $\mathbb{A} = (1, \lambda \_.\, \{\top,\bot\})$ represents
a basic object for measuring approximations in terms of presence and
absence. (If all we care about is presence or absence in our
approximations, we could see this object as the analogue of the real
numbers for normal differentiation.)

This object is a commutative monoid in $\Fam(\LatGal)$, and so we can
construct the writer monad $T(X) = \mathbb{A} \times X$ in
$\Fam(\LatGal)$.

On objects $X$ with trivial approximations, we have $L(X) \cong T(X)$,
but for more structured objects we get usage information at every level.

This is a bit like using the writer monad and Moggi's CBN translation
as a basic way of doing provenance tracking, but here we can track
backwards and forwards.

\subsubsection{Approximating Numbers by Intervals}

\bob{Describe how \exref{intervals-and-maxima-elements} fits into
  $\Fam(\LatGal)$}

\subsection{Summary}

We have constructed the category
$\Fam(\MeetSLat \times \JoinSLat^\op)$, which is a cartesian closed
category with all coproducts, ienough to interpret the total
higher-order language we define in the next section with primitive
types and operations defined in $\Fam(\LatGal)$. However, we are not
guaranteed by construction that at first-order type, the
interpretations are in fact Galois connections. We will rectify this
in \secref{definability} using a logical relations construction.

As we did in \secref{approx-as-tangents}, we end the section with a
conjecture relating our categories to Tangent categories.

\begin{conjecture}
  (1) The category $\Fam(\MeetSLat)$ is a Tangent category, with the
  tangent bundle
  $T(X, \partial X) = (\Sigma_{x : X}. \partial X(x), \lambda (x,
  \delta x).\, \downset{\delta x})$. (2) The category $\Fam(\LatGal)$
  is a reverse Tangent category with the analogous definition of
  tangent bundle.
\end{conjecture}

Comparing this conjecture to the previous one, we can see that the
difference between $\CM$, $\Lposet$ and $\Fam(\MeetSLat)$,
$\Fam(\LatGal)$ is that the latter have a separation of points from
tangents, somewhat analogous to the situation with manifolds. Also
note that, contrary to the $\Fam(\FinVect)$ representation of
manifolds and differentiable maps, (if this conjecture holds) we do
not throw away information about higher derivatives. It is retained in
the order structure of the tangent fibres.
