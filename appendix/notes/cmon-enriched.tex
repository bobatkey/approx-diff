\subsection{$\CMon$-enriched categories}
\label{sec:cmon-enriched}

\begin{definition}[Commutative monoid]
A \emph{commutative monoid} $X = (X, \bullet, \varepsilon)$ is a set $X$ equipped with distinguished element
$\varepsilon \in X$ called the \emph{unit} and associative binary operation $\bullet: X^2 \to X$ satisfying
$\varepsilon \bullet x = x$ and $x \bullet \varepsilon = x$ for any $x \in X$.
\end{definition}

A commutative monoid homomorphism from $X$ to $Y$ is any function $f: X \to Y$ preserving $\varepsilon$ and
$\bullet$.

\subsubsection{Category of commutative monoids}

\begin{definition}[Category $\CMon$]
Define $\CMon$ to be the category which has as objects $X$ all commutative monoids and as morphisms $f: X \to
Y$ all commutative monoid homomorphisms.
\end{definition}

$\CMon$ is monoidal closed, with the trivial one-element monoid $\One$ as terminal object and monoidal product
$A \tensor B$ given by the Cartesian product $A \times B$. $\CMon$ is complete and cocomplete, inheriting all
limits and colimits from $\Set$.

\subsubsection{Example of commutative monoid}

\begin{definition}[Bounded semilattice]
\label{def:cmon-enriched:bounded-semilattice}
A \emph{bounded semilattice} $X = (X, \bullet, \varepsilon)$ is a commutative monoid where $\bullet$ is
\emph{idempotent}, i.e.~satisfies $x \bullet x = x$.
\end{definition}

\noindent The idempotence of $\bullet$ (together with commutativity and associativity) induces a partial order
$\le_\bullet$ on $X$, with $x \le_{\bullet} y \iff x \bullet y = x$. With respect to this partial order, $x
\bullet y$ is the greatest lower bound (meet) of $x$ and $y$ and $\varepsilon$ is the top element; with
respect to the opposite order, $x \bullet y$ is the least upper bound (join) of $x$ and $y$ and $\varepsilon$
is the bottom element. This therefore provides an algebraic characterisation of the usual (dual)
order-theoretic notions of bounded meet semilattices $(X, \meet, \top)$ and bounded join semilattices $(X,
\join, \bot)$.

\begin{definition}[Category $\SemiLat$]
Define $\SemiLat$ to be the category which has as objects all bounded semilattices and as morphisms all
bounded semilattice homomorphisms.
\end{definition}

\subsubsection{$\CMon$-enriched category}

Recall that if a category $\cat{C}$ is $\CMon$-enriched, then:
\begin{enumerate}
\item Every hom-object $\Hom{\cat{C}}{X}{Y}$ is a commutative monoid of morphisms; we write $\zero_{X,Y}$
(zero morphism) for the unit, and $+_{X,Y}$ (addition of morphisms) for the binary operation (omitting the
indices where implied by the context).
\item Composition is \emph{bilinear}, i.e.~given by a family of morphisms $\Hom{\cat{C}}{Y}{Z} \tensor
\Hom{\cat{C}}{X}{Y} \to \Hom{\cat{C}}{X}{Z}$ in $\CMon$ that preserve the additive structure in
$\Hom{\cat{C}}{Y}{Z}$ and $\Hom{\cat{C}}{X}{Y}$ separately:

\begin{salign*}
f \comp \zero = f = \zero \comp f
\end{salign*}
\begin{salign*}
(f + g) \comp h &= (f \comp h) + (g \comp h) \\
h \comp (f + g) &= (h \comp f) + (h \comp g)
\end{salign*}
\end{enumerate}

Because $\CMon$ is monoidal closed, it is enriched over itself; every hom-object $\Hom{\CMon}{X}{Y}$ is a
commutative monoid with:

\begin{enumerate}
\item unit $0_{X,Y} = \const(\varepsilon_Y)$, the constant homomorphism sending every element of $X$ to
$\varepsilon_Y$;
\item binary operation $+_{X,Y}$ given by pointwise addition of homomorphisms $(f + g)(x) = f(x) + g(x)$.
\end{enumerate}

Suppose $F, G: \cat{D} \to \cat{C}$ are functors where $\cat{C}$ is $\CMon$-enriched. Then the hom-object
$\Hom{\Func{\cat{C}}{\cat{D}}}{F}{G}$ of natural transformations between $F$ and $G$ has a zero $0_{F,G}$ and
addition $\eta + \mu$ given component-wise as $(0_{F,G})_X = 0_{F(X),G(X)}$ and $(\eta + \mu)_X = \eta_X +
\mu_X$.

\begin{proposition}
If $C$ is $\CMon$-enriched then any functor category $\Func{\cat{D}}{\cat{C}}$ is $\CMon$-enriched.
\end{proposition}

\subsubsection{$\CMon$-enriched presheaves}

\begin{definition}[$\mathscr{V}$-enriched functor]
\label{def:cmon-enriched:enriched-functor}
For any monoidal category $(\mathscr{V}, \tensor, I)$ and $\mathscr{V}$-enriched categories $\cat{C}$ and
$\cat{D}$, a \emph{$\mathscr{V}$-enriched} functor $F: \cat{C} \to \cat{D}$ on morphisms is a map sending
every object $X$ of $\cat{C}$ to an object $F(X)$ of $\cat{D}$, plus a family of morphisms $F_{X,Y}:
\Hom{\cat{C}}{X}{Y} \to \Hom{\cat{D}}{F(X)}{F(Y)}$ between hom-objects in $\mathscr{V}$ making the
following commute:

\begin{center}
\begin{tikzcd}[column sep=1.8cm]
I \arrow[d, equals] \arrow[r, "\id_X"] & \Hom{\cat{C}}{X}{X} \arrow[d, "F_{X,X}"] \\
I \arrow[r, "\id_{F(X)}"'] & \Hom{\cat{D}}{F(X)}{F(X)}
\end{tikzcd}
\hspace{5mm}
\begin{tikzcd}[column sep=2.6cm]
   \Hom{\cat{C}}{Y}{Z} \tensor \Hom{\cat{C}}{X}{Y} \arrow[r, "\comp_{X,Y,Z}"] \arrow[d, "F_{Y,Z} \tensor F_{X,Y}"']
   & \Hom{\cat{C}}{X}{Z} \arrow[d, "F_{X,Z}"] \\
   \Hom{\cat{D}}{F(Y)}{F(Z)} \tensor \Hom{\cat{D}}{F(X)}{F(Y)} \arrow[r, "\comp_{F(X),F(Y),F(Z)}"'] &
   \Hom{\cat{D}}{F(X)}{F(Z)}
\end{tikzcd}
\end{center}
\end{definition}

\noindent These generalise the usual functor laws to a setting where the hom-objects are not necessarily sets
and thus do not support an elementwise formulation, although here we are only interested in the situation
where $\mathscr{V}$ is concrete.

\begin{definition}[Category of enriched functors]
Write $\Func{\cat{C}}{\cat{D}}_{\mathscr{V}}$ for the category of $\mathscr{V}$-enriched functors from
$\cat{C}$ to $\cat{D}$.
\end{definition}

In the specific setting of a $\CMon$-enriched presheaf $F: \cat{C}^\op \to \CMon$, enrichment means that $F$
preserves the additive structure of morphisms:
\begin{itemize}
\item $F(0_{X,Y}) = 0_{F(Y),F(X)}$
\item $F(f + g) = F(f) + F(g)$
\end{itemize}
It then follows automatically that $F$ also preserves bilinear composition, i.e.:
\begin{itemize}
\item $F(f) \comp F(0) = F(f) = F(0) \comp F(f)$
\item $F(f + g) \comp F(h) = F(h \comp f) + F(h \comp g)$
\item $F(h) \comp F(f + g) = F(f \comp h) + F(g \comp h)$
\end{itemize}
in addition to the usual enriched functor properties from \defref{cmon-enriched:enriched-functor}.

\subsubsection{$\CMon$-enriched Yoneda embedding}

Suppose $\cat{C}$ a small category. The usual \emph{Yoneda embedding} for $\cat{C}$ is the functor $\Yoneda:
\cat{C} \to \Func{\cat{C}^\op}{\Set}$ defined as follows:
\begin{itemize}
\item on objects: $\Yoneda(X) = \Hom{\cat{C}}{-}{X}: \cat{C}^\op \to \Set$, the contravariant hom-functor
sending:
   \begin{itemize}
   \item any object $Y$ in $\cat{C}$ to the hom-set $\Hom{\cat{C}}{Y}{X}$;
   \item any morphism $f: Y \to Z$ in $\cat{C}$ to the function $\Hom{\cat{C}}{f}{X} = (- \comp f):
   \Hom{\cat{C}}{Z}{X} \to \Hom{\cat{C}}{Y}{X}$.
   \end{itemize}
\item on morphisms: for any $f: X \to Y$ in $\cat{C}$, $\Yoneda(f)$ is the natural transformation where
$\Yoneda(f)_Z = (f \comp -): \Hom{\cat{C}}{Z}{X} \to \Hom{\cat{C}}{Z}{Y}$.
\end{itemize}

\noindent Now suppose $\cat{C}$ is $\mathscr{V}$-enriched with $\mathscr{V}$ monoidal closed; then the
hom-functor $\Hom{\cat{C}}{-}{X}: \cat{C}^\op \to \mathscr{V}$ is itself $\mathscr{V}$-enriched, sending
objects $Y$ to the hom-object $\Hom{\cat{C}}{Y}{X}$ in $\mathscr{V}$, and providing a family of morphisms
$\Hom{\cat{C}}{-}{X}_{Y,Z} \to \Hom{\mathscr{V}}{\Hom{\cat{C}}{Y}{X}}{\Hom{\cat{C}}{Z}{X}}$ for any $X, Y$ in
$\cat{C}$. The following diagram commutes because precomposition with the identity is the identity on
morphisms:

\begin{center}
\begin{tikzcd}[column sep=1.8cm]
I \arrow[d, equals] \arrow[r, "\id_X"] & \Hom{\cat{C}}{X}{X} \arrow[d, "\Hom{\cat{C}}{-}{W}_{X,X}"] \\
I \arrow[r, "\id_{\Hom{\cat{C}}{X}{W}}"'] & \Hom{\mathscr{V}}{\Hom{\cat{C}}{X}{W}}{\Hom{\cat{C}}{X}{W}}\
\end{tikzcd}
\end{center}

\noindent and the following commutes by the associativity of composition:

\begin{center}
\begin{tikzcd}[column sep=2.7cm]
   \Hom{\cat{C}}{Z}{Y} \tensor \Hom{\cat{C}}{Y}{X} \arrow[r, "\comp_{X,Y,Z}"] \arrow[d,
   "\Hom{\cat{C}}{-}{W}_{Y,Z}\,\tensor\,\Hom{\cat{C}}{-}{W}_{X,Y}"'] & \Hom{\cat{C}}{Z}{X} \arrow[d,
   "\Hom{\cat{C}}{-}{W}_{X,Z}"] \\
   \Hom{\mathscr{V}}{\Hom{\cat{C}}{Y}{W}}{\Hom{\cat{C}}{Z}{W}} \tensor
   \Hom{\mathscr{V}}{\Hom{\cat{C}}{X}{W}}{\Hom{\cat{C}}{Y}{W}} \arrow[r,
   "\comp_{\Hom{\cat{C}}{X}{W},\Hom{\cat{C}}{Y}{W},\Hom{\cat{C}}{Z}{W}}"'] &
   \Hom{\mathscr{V}}{\Hom{\cat{C}}{X}{W}}{\Hom{\cat{C}}{Z}{W}}
\end{tikzcd}
\end{center}

\noindent There exists an enriched version of the Yoneda embedding, namely a functor $\Yoneda_{\mathscr{V}}:
\cat{C} \to \Func{\cat{C}^\op}{\mathscr{V}}_{\mathscr{V}}$, which is itself $\mathscr{V}$-enriched, sending
objects $X$ to the $\mathscr{V}$-enriched hom-functor $\Yoneda_{\mathscr{V}}(X) = \Hom{\cat{C}}{-}{X}:
\cat{C}^\op \to \mathscr{V}$ and morphisms $f: X \to Y$ in $\cat{C}$ to $\mathscr{V}$-natural transformations
$\Yoneda_{\mathscr{V}}(f): \Hom{\cat{C}}{-}{X} \naturalto \Hom{\cat{C}}{-}{Y}$ acting at each component as
post-composition with $f$.

The enriched Yoneda lemma says that for any $\mathscr{V}$-enriched presheaf $F$ and any object $X \in
\cat{C}$, the hom-object of $\mathscr{V}$-natural transformations (which also lives in $\mathscr{V}$) between
the $\mathscr{V}$-enriched presheaf $\Hom{\cat{C}}{-}{X}$ and $F$ is isomorphic to $F(X)$:

\begin{lemma}
Suppose $\cat{C}$ is $\mathscr{V}$-enriched with $\mathscr{V}$ monoidal closed. For any $F: \cat{C}^\op \to
\mathscr{V}$ and any object $X$ in $\cat{C}$:
\[\Hom{\Func{\cat{C}^\op}{\mathscr{V}}_{\mathscr{V}}}{\Hom{\cat{C}}{-}{X}}{F} \iso F(X)\]
\end{lemma}

% First we note that the hom-functor $\Hom{C}{-}{X}: C \to \CMon$ is $\CMon$-enriched because composition is
% bilinear and so $(- \comp f)$ preserves the additive structure of morphisms.
