\subsection{Automatic differentiation}

Write $\tangents_x(\RR^n)$ for the tangent space at a point $x \in \RR^n$. Then the \emph{forward derivative}
(tangent map or pushforward) $\pushf{f}_x$ of a differentiable function $f: \RR^m \to \RR^n$ at $x \in \RR^m$
is the unique linear map $\tangents_x(\RR^m) \linearto \tangents_{f(x)}(\RR^n)$ given by the Jacobean matrix
of $f$ at $x$. Since $\RR^n$ is flat, $\tangents_x(\RR^n)$ is naturally isomorphic to $\RR^n$, so in fact
$\pushf{f}_x$ is a linear map $\RR^m \linearto \RR^n$. The \emph{backward derivative} (cotangent map or
pullback) $\pullf{f}_x$ is the unique linear map $\RR^n \linearto \RR^m$ given by the transpose (adjoint) of
the Jacobean matrix of $f$ at $x$.

\subsubsection{Chain rule}

Derivatives respect composition. Suppose $f: \RR^m \to \RR^n$ and $g: \RR^n \to \RR^k$. For any $x \in \RR^m$
we have:

\begin{itemize}
\item $\pushf{(g \comp f)}_x = \pushf{g}_{f(x)} \comp \pushf{f}_x: \RR^m \to \RR^k$
\item $\pullf{(g \comp f)}_x = \pullf{f}_{x} \comp \pullf{g}_{f(x)}: \RR^k \to \RR^m$
\end{itemize}

\begin{definition}[Forward mode automatic differentiation functor]
Define the functor $\tangents_*$ which sends every vector space $\RR^n$ to itself and every differentiable map
$f: \RR^m \to \RR^n$ to the function $\tangents_*(f) = x \mapsto (f(x), \pushf{f}_x): \RR^m \to \RR^n \times
(\RR^m \linearto \RR^n)$, associating to every point its image in $f$ and the forward derivative of $f$ at
that point.
\end{definition}

For any map $f: \RR^m \to \RR^n \times (\RR^m \linearto \RR^n)$ write $f_1$ for $\pi_1 \comp f$ and $f_2$ for
$\pi_2 \comp f$.

\begin{definition}[$\Diff$]
\label{def:auto-diff:Diff}
Define $\Diff_*$ to be the category where the objects are all finite vector spaces $\RR^n$ and the morphisms
are all maps $f: \RR^m \to \RR^n \times (\RR^m \linearto \RR^n)$ with $f_1$ differentiable. Composition in
$\Diff_*$ is given by:
\begin{align*}
(g \comp f)_1(x) &= f_1(g_1(x)) \\
(g \comp f)_2(x) &= g_2(f_1(x)) \comp f_2(x)
\end{align*}
\end{definition}

We can verify that $\tangents_*$ is indeed a functor, regardless of whether the linear maps $\RR^m \linearto
\RR^n$ happen to be derivatives. But if $f_2(x)$ is the derivative of $f_1$ at $x$ for any $x \in \RR^m$ and
similarly for $g$ then the derivatives compose according to the forward chain rule and $(g \comp f)_2(x)$ is
the derivative of $(g \comp f)_1$ at $x$.

\begin{definition}[Reverse mode automatic differentiation functor]
Define the functor $\tangents^*$ which also sends every vector space $\RR^n$ to itself but which sends every
differentiable map $f: \RR^m \to \RR^n$ to the function $\tangents^*(f) = x \mapsto (f(x), \pullf{f}_x): \RR^m
\to \RR^n \times (\RR^n \to \RR^m)$.
\end{definition}

\noindent and again we can define a corresponding category $\Diff^*$ of such maps where composition of the
cotangent maps respects the (backward) chain rule.

\subsubsection{Automatic differentiation via category of families}

Recall that the category $\Fam(\FinVect)$ is the Grothendieck construction $\Grothendieck{X}\Fam(X,
\FinVect)$. The functor $\Diff_* \hookrightarrow \Fam(\FinVect)$ which send objects $\RR^n$ to constant
families of tangent spaces $(\RR^n, \{\RR^n\}_{x \in \RR^n})$ and morphisms $f: \RR^m \to \RR^n \times (\RR^m
\linearto \RR^n)$ to pairs of differentiable functions $f_1: \RR^m \to \RR^n$ and constant families of linear
maps $f_2: \{\delta f_x: \RR^m \linearto \RR^n\}_{x \in \RR^m}$ exhibits $\Diff$ as equivalent to a full
subcategory of $\Fam(\FinVect)$.

For any $(f, \partial f): (\RR^m, \{\RR^m\}_{x \in \RR^m}) \to (\RR^n, \{\RR^n\}_{x \in \RR^n})$ and $(g,
\partial g): (\RR^n, \{\RR^n\}_{x \in \RR^n}) \to (\RR^k, \{\RR^k\}_{x \in \RR^k})$ in the image of that
functor, we have that for any $x \in \RR^m$:
\begin{itemize}
\item $\partial f_x: \RR^m \to \RR^n$;
\item $\reindex{\partial g}{f}_x = \partial g_{f(x)}: \RR^n \to \RR^k$
\end{itemize}

\noindent with the composition $(g, \partial g) \comp (f, \partial f) = (g \comp f, \reindex{\partial g}{f}
\comp \partial f): (\RR^m, \{\RR^m\}_{x \in \RR^m}) \to (\RR^k, \{\RR^k\}_{x \in \RR^k})$ corresponding
exactly to the chain rule, since $(\reindex{\partial g}{f} \comp \partial f)_x = {\partial g}_{f(x)} \comp
{\partial f}_x$.
